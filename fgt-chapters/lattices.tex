\begin{df}
Let $V=\R^n$. A \textbf{lattice} $\Ga$ in $V$ is a subgroup of $V$ which is a free $\Z$-module with $\Z$-basis $\{v_1,\ldots, v_n\}$ which is linearly independent over $\R$. $\Ga$ is complete if $m=n$.
\end{df}
For example, $\Z+\Z i\subeq\C\cong \R^2$ is a complete lattice, but $\Z[(1,0),(-\sqrt 3,0)]$ is not a lattice.

\begin{pr}
Suppose $\Ga\subeq \Ga'\subeq V$ are subgroups, such that $[\Ga:\Ga']=m<\iy$. 
\begin{enumerate}
\item
$\Ga'$ is a lattice iff $\Ga$ is a lattice.
\item
$\Ga'$ is a complete lattice iff $\Ga$ is a complete lattice.
\end{enumerate}
\end{pr}
\begin{proof}
It suffices to prove part 2 (just replace $V$ by $\spn(\Ga)$).
Since $\Ga/\Ga'$ is killed by $m$,
\[
m\Ga\subeq \Ga'\subeq \Ga.
\]
Now $\spn(m\Ga)\subeq \spn(\Ga')\subeq \spn(\Ga)$ so equality must hold.

For the forward direction, note $\Ga$ is generated by $\Ga'$ (finitely generated) and representatives of $\Ga/\Ga'$ (a finite set). $\Ga$ is a finitely generated abelian group; since $V$ is torsion-free so is $\Ga$; $\Ga$ must be free. But $\rank_{\Z}(m\Ga)\le\rank_{\Z}(\Ga')\le \rank_{\Z}(\Ga)$ so these all have rank $n$. $\Ga'$ generates $V$, so $\Ga$ does as well. (A $\Z$-basis $\{v_1,\ldots, v_n\}$ of $\Ga$ is a $\R$-basis.)

For the reverse direction, note $m\Ga$ is a complete lattice, so from the forward assertion. $\Ga'\subeq m\Ga$ is also a complete lattice.
\end{proof}
\begin{df}
For $\Ga\subeq V$, define a \textbf{fundamental mesh} (parallelopiped) $\cal F(V)$ to be
\[
\set{\sum_{i=1}^m a_iv_i}{0\le a_i\le 1}.
\] 
\end{df}
The volume is independent of the choice of $\Z$-basis of $\Ga$.
%ex. standard in \R^2.
Define $\vol(\Ga)=\vol(\cal F(\Ga))$ with respect to the measure given on $V$.
\begin{pr}\label{bounded-subset-add}
Let $\Ga$ be a lattice of $V$. Then $\Ga$ is complete iff there exists a bounded subset $M$ such that $M+\Ga= V$ bounded.
\end{pr}
\begin{proof}
For the forwards direction, take $M$ to be the fundamental mesh. For the reverse direction, suppose \bwoc{} that $\spn_{\R}(\Ga)\sub V$. Extend the $\Z$-basis of $\Ga$ to a $\R$ basis $\{v_1,\ldots,v_n\}$ of $V$ ($m<n$). The elements of $M$ has bounded $n$th coordinate with respect to $\{v_1,\ldots, v_n\}$. Elements of $\Ga$ have $n$th coordinate 0. Hence $\Ga+M\ne V$.
\end{proof}
\begin{pr}
Let $\Ga\subeq V$ be a subgroup. Then $\Ga$ is a lattice iff $\Ga$ is discrete in $V$.
\end{pr}
\begin{proof}
We may assume $\spn_{\R}(\Ga)=V$. (Replace $V$ with $\spn_{\R}(\Ga)$.)

For the reverse direction, $\Ga$ contains a $\R$-basis $\{v_1,\ldots,v_n\}$ for $V$. Then $\Ga'=\Z[v_1,\ldots,v_n]\subeq \Ga$. Then $\Ga'$ is a complete lattice. By Proposition~\ref{bounded-subset-add}, %it is enough to see $\Ga'\subeq \Ga$ has finite index. 
there exists bounded $M\subeq V$ such that $\Ga'+M=V$. 

We claim that $\Ga\cap M\to \Ga/\Ga'$ is onto. The LHS is a finite set, since $\Ga$ is discrete. Note $\ga\in \Ga$ means $\ga=\ga'+m, \ga'\in \Ga',m\in M$. Then $\ga-\ga'=m\in P\cap M$. %\mapsto \ga
The previous proposition says $\Ga$ is a complete lattice.

Forward direction omitted. %If $\Ga$ is not discrete, find sequence converging to 0. If $P=\Z[v_1,\ldots, v_n]$ The volume of the parallelopiped $(v_1,\ldots,v_n,x_i)$ goes to 0, but this cannot happen since $x_i$ is a $\Z$-linear combination of $(v_1,\ldots, v_n)$. vol=Vol x det(Z matrix)
%Or: fund domain has fin many lattice pts in it.
%|\Ga\cap \rc 2\cal F(V)|\le 1
%\ga_1,\ga_2 \sum a_iv_i, \sum b_iv_i in
%\ga_1-\ga_2=\sum(a_i-b_i) v_i, 0\le a_i-b_i\le \rc2 so =0
\end{proof}