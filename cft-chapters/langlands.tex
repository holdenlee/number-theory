\textbf{Warning:} This section is written to be ``roughly" true; I'm not sure of the precise correct statements.

We rephrase global class field theory in the form that generalizes under the Langlands program (for an introduction to the Langlands program, see Section~\ref{sec:intro-langlands}).
\begin{thm}[Rephrase of GCFT]\llabel{thm:rephrase-gcft}
There is a bijection between continuous homomorphisms $\chi:\A_K^{\times}/\ol{K^{\times}(K^{\times}_{\iy})^0}\to \C^{\times}$ and continuous homomorphisms $f:G(\ol K/K)\to \GL_1(\C)$, given by the following.
\begin{align*}
\{
\chi:\A_K^{\times}/\ol{K^{\times}(K^{\times}_{\iy})^0}\to \C^{\times}
\}
&\leftrightarrow
\{f:G(\ol K/K)\to \GL_1(\C)\} \\
\chi&\mapsto \chi\circ \phi_K^{-1}
\end{align*}
\end{thm}
\begin{proof}
From Theorem~\ref{thm:gcft-topology}, the Artin map gives a topological isomorphism $ \A_K^{\times}/\ol{K^{\times}(K^{\times}_{\iy})^0}\to G(K\abe/K)$. It remains to note that any function $G(\ol K/K)\to \GL_1(\C)$ factors through $G(\ol K/K)\abe=G(K\abe/K)$, since $\GL_1(\C)$ is abelian.
\end{proof}
The homomorphisms $\chi:\A_K^{\times}/\ol{K^{\times}(K^{\times}_{\iy})^0}\to \C$ are ``automorphic functions" on $\GL_1(\A_K)$, also known as Hecke characters, and the homomorphisms $f:G(\ol K/K)\to \GL_1(\C)$ are 1-dimensional ``Galois representations." Our correspondence is unsatisfactory, however, because we would like to get all continuous homomorphisms $\A_K^{\times}/K^{\times}\to \C^{\times}$, not just those factoring through $\A_K^{\times}/\ol{K^{\times}(K^{\times}_{\iy})^0}$. Since $G(K\abe/K)$ has the profinite topology, %and \C has no small subgroups
any continuous homomorphism $G(\ol K/K)\to \GL_1(\C)$ must have finite image, while functions $\A_K^{\times}/K^{\times}\to \C^{\times}$ can have infinite image. To remedy this, we introduce functions $G(\ol K/K)\to \GL_1(\C)$ with infinite image (no longer continuous under the complex topology). 
%For example, in the case $K=\Q$, we can consider the following map.
%\begin{ex}\llabel{ex:chi-l}
%Let $\ell$ be a prime of $\Q$. Define the map $\chi_{\ell}$ by
%\[
%\chi_{\ell}:\xymatrix{
%G(\ol{\Q}/\Q) \ar@{->>}[r] &
%G(\Q\abe/\Q)=G(\Q(\ze_{\iy})/\Q)\ar[r]^-{\cong} &
%\wh{\Z}^{\times}=\prod_p\Z_p^{\times} \ar@{->>}[r] &
%\GL_1(\Z_{\ell}).
%}
%\]
%\end{ex}
%Note there is a noncanonical field isomorphism $\ol{\Q_{\ell}}\cong \C$, so we can think of $\GL_1(\Z_{\ell})$ as being ``inside" $\GL_1(\C)$. %Fixing an isomorphism
%\begin{df}
%\begin{enumerate}
%\item
%%Let $v$ be an archimedean place.
%We say a function $\pi:\A_K^{\times}/K^{\times}\to \C$ is \textbf{algebraic} at $\iy$ if for all complex places $v$, $\pi(i_v(x))=x^m\ol x^n$ for some $m,n\in \Z$, for all real places $v$, $\pi(i_v(x))=\sign(x)^{m}x^n$ for some $m\in \{0,1\}$ and $n\in \Z$, and for all nonarchimedean places $\pi(i_v(x))=1$.
%\item
%We say a function $f:G(\ol K/K)\to \GL_1(\Q_{\ell})$ is \textbf{algebraic} at $\ell$ if it is in the form $\prod_{v\mid \ell} f_v$ where $f_v$ is the composition of the projection $G(\ol K/K)\to D_v(\ol K/K)\cong G(\ol{K_v}/K_v)$ and a continuous homomorphism $G(\ol{K_v}/K_v)\to \GL_1(\Q_{\ell})$.
%%Let $\ell$ be a prime of $\Q$, and let $v\mid \ell$ be a place of $K$. We say a function $G(\ol K/K)\to \GL_1(\Q_{\ell})$ is \textbf{algebraic} at $v$ if 
%\end{enumerate}
%\end{df}
%We have the following ``addendum" to Theorem~\ref{thm:rephrase-gcft}.
%\begin{thm}\llabel{thm:gcft-char2}
%Fixing a field homomorphism $\ol{\Q_{\ell}}\cong \C$, 
%there is an bijection between continuous homomorphisms $\pi:\A_K^{\times}/K^{\times}\to \C$ algebraic at $\iy$ and continuous homomorphisms $f:G(\ol K/K)\to \GL_1(\Q_{\ell})$ algebraic at $\ell$.
%\end{thm}
%%I don't think this is straightforward to prove... need to show that a place v|l with ramification degree e corresponds to e characters, i.e. we need the Z_p-rank of G(K_v^ab/K_v) to be e. (there are no nontrivial hom's from pro-l group to pro-p group)
%\begin{ex}
%In the case $K=\Q$, it's enough to introduce one new character. Let $\ad:\A_K^{\times}\to \C^{\times}$ denote the map $\ab{\mathbf x}=\prod_{v\in V_K} |x_v|_v$, and define $\chi_{\ell}$ as in Example~\ref{ex:chi-l}.
%
%Putting Theorem~\ref{thm:rephrase-gcft} and~\ref{thm:gcft-char2} together, a continuous homomorphism $\pi:\A_{\Q}^{\times}/\Q^{\times}\to \C^{\times}$ algebraic at $\iy$ corresponds to the product of a continuous homomorphism $f:G(\ol{\Q}/\Q)\to \GL_1(\ol{\Q_{\ell}})$ algebraic at $\ell$, and a homomorphism  $f:G(\ol{\Q}/\Q)\to \GL_1(\Q_{\ell})$ with finite image, via the map
%%
%%and continuous homomorphisms $G(\ol{\Q}/\Q)\to \GL_1(\Q_{\ell})$. All continuous functions $\pi:\A_K^{\times}/K^{\times}\to \C^{\times}$ are in the form $\ad^n\cdot \chi$ where $\chi$ is a finite character (i.e. a continuous homomorphism $\A_{\Q}/\ol{\Q^{\times}\R_0^{\times}\to \C^{\times}}$), and $n\in \Z$. The isomorphism is given by
%\begin{align*}
%%\{
%%\pi:\A_{\Q}^{\times}/\Q^{\times}\to \C^{\times}
%%\}
%%&\leftrightarrow
%%\{f:G(\ol{\Q}/\Q)\to \GL_1(\Q_{\ell})\} \\
%\pi=\ad^n\cdot \chi&\leftrightarrow \chi_{\ell}^n
%\cdot (\chi\circ \phi_K^{-1}).
%\end{align*}
%\end{ex}
For simplicity, we just consider the case of $\Q$.
\begin{ex}
We say a function $\pi:\A_{\Q}^{\times}/\Q^{\times}\to \C$ is \textbf{algebraic} at $\iy$ if $\pi(i_{\R}(x))=\sign(x)^m|x|^n$ for some $m\in \{0,1\}$ and $n\in \Z$. We say a function $f:G(\ol {\Q}/\Q)\to \GL_1(\Q_{\ell})$ is \textbf{algebraic} at $\ell$ if it is the composition of the projection $G(\ol {\Q}/\Q)\to D_{\ell}(\ol{\Q}/\Q)\cong G(\ol{\Q_{\ell}}/\Q_{\ell})$ and a continuous homomorphism $G(\ol{\Q_{\ell}}/\Q_{\ell})\to \GL_1(\Q_{\ell})$.

Let $\ell$ be a prime of $\Q$. Let $\ad:\A_{\Q}^{\times}/\Q^{\times}\to \C^{\times}$ denote the map $\ab{\mathbf x}=\prod_{v\in V_{\Q}} |x_v|_v$, and define $\chi_{\ell}$ by
\[
\chi_{\ell}:\xymatrix{
G(\ol{\Q}/\Q) \ar@{->>}[r] &
G(\Q\abe/\Q)=G(\Q(\ze_{\iy})/\Q)\ar[r]^-{\cong} &
\wh{\Z}^{\times}=\prod_p\Z_p^{\times} \ar@{->>}[r] &
\GL_1(\Z_{\ell}).
}
\]
(Note there is a noncanonical field isomorphism $\ol{\Q_{\ell}}\cong \C$, so we can think of $\GL_1(\Z_{\ell})$ as being ``inside" $\GL_1(\C)$.)

A continuous homomorphism $\pi:\A_K^{\times}/K^{\times}\to \C^{\times}$ algebraic at $\iy$ corresponds to the product of a continuous homomorphism $f:G(\ol{\Q}/\Q)\to \GL_1({\Q_{\ell}})$ algebraic at $\ell$, and a homomorphism  $f:G(\ol{\Q}/\Q)\to \GL_1(\ol{\Q_{\ell}})$ with finite image, via the map
\begin{align*}
\pi=\ad^n\cdot \chi&\leftrightarrow \chi_{\ell}^n
\cdot (\chi\circ \phi_K^{-1}).
\end{align*}
\end{ex}
