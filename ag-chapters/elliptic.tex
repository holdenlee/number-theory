\chapter{Elliptic curves}
Why are we interested in elliptic curves over different fields? We are interested in elliptic curves over $\Q$ because much of number theory is concerned with solving equations over $\Q$; we are interested in elliptic curves over finite fields $k$ because of its applications to cryptography, factoring, and primality proving.

On the other hand, even if we just wanted information about an elliptic curve $E$ over $\Q$, it helps to investigate it over other fields. $\Q$ is a difficult field to work with because it is a {\it global} field. So we make two reductions.
\begin{enumerate}
\item Consider $E$ over the local fields $\Q_p$ (and $\C$!). Then combine this information to get information on $E(\Q)$.
\item Consider $E$ over finite fields $\F_p=\Q_p/p\Q_p$. Use this to get information about $E$ over local fields $\Q_p$.
\end{enumerate}
We look at item 2 more closely. Suppose $K$ is a local field; let $k$ be its residue field. 
Note that when we reduce an elliptic curve modulo $p$, some points will get sent to 0; this is the kernel of reduction $E_1(K)$. Thus we get a exact sequence
\[
\xymatrix{
0\ar[r]& E_1(K)\ar[r]&  E_0(K)\ar[r]&  \wt{E}_{\text{ns}}(k)\ar[r]&  0\\%criteria and consequences for good/bad reduction.
%nonsingular points form a group.
& \wh{E}(\mm)\ar@{=}[u] &  & &}
\]
How can we study $E_0(K)$? We know these are points whose coordinates are in the maximal ideal $\mm$ associated to $K$ (so get sent to 0 upon reduction). 
%To go from elliptic curves over finite fields to elliptic curves over local fields, we use the following exact sequences.
To get these points we take a uniformizer $z\in \mm$, and write the coordinates of a point in $E_0(K)$ as power series in $z$. Thus we investigate the elliptic curve over a {\it ring of formal power series}, and the group law becomes a group law for power series, called a {\it formal group law} (see Section BLAH). We then get to $E(K)$ from the exact sequence $0\to E_0(K)\to E(K)\to E(K)/E_0(K)\to 0$; fortunately, $E(K)/E_0(K)$ is finite.

To get from $E$ over local fields to $E$ over global fields, we would like to use the Hasse principle as in quadratic forms. However, this fails. There is, however, a way of measuring the failure of the Hasse principle using the Selmer and Shafarevich-Tate groups (see Silverman, X).
\subsection{Elliptic curves over finite fields}
Consider the finite field $\F_q$ where $q=p^n$. 
We know that $E(\F_q)$ is a finite abelian group. The main questions we seek to answer are the following.
\begin{enumerate}
\item
How big is it?
\item
What is its structure? In particular, what is its $p$-torsion?
\item
How does it vary as $q$ varies over all powers of a prime $p$?
\end{enumerate}
We will answer these questions in this chapter.
\begin{enumerate}
\item
By Hasse's Theorem, $|E(\F_q)-q-1|\le 2\sqrt q$.
\item
$E[p^r]$ (over $\ol{\F_p}$) is either 0 or $\Z/p^r\Z$; we give criterion for these to happen. (Which case occurs, is called the Hasse invariant.)
\item
Let $t_n=|E(\F_{p^n})|$. Then the $t_n$ are coefficients of a nice generating function. This follows from the Weil conjectures for elliptic curves.
\end{enumerate}
\subsection{Hasse's Theorem}
Let $E$ be the curve $y^2+a_1xy+a_3y=x^3+a_2x^2+a_4x+a_6$. A quadratic function $ay^2+by+c$ in a finite field attains about half of the values, so given a value of $x$, there is about a $\rc2$ chance that the equation is solvable in $y$. When it is solvable, there will usually be 2 solution. Thus we expect the number of solutions to be close to $p$. Including the point at infinite, the expected number becomes $p+1$. Hasse's Theorem tells us that the number of solutions is not too far from $p+1$.
\begin{thm}
Let $E/\F_q$ be an elliptic curve. Then
\[
||E(\F_q)|-q-1|\le 2\sqrt q.
\]
\end{thm}
\begin{proof}
The idea is to count the number of points of $E(\ol{\F_q})$ in $E(\F_q)$ by viewing $E(\F_q)$ as the kernel of $1-\phi_q$, where $\phi_q$ is the Frobenius map. The kernel equals the degree of the map. We know the degree of $1$ and $\phi_q$; to get the degree of $1-\phi_q$ we use the fact that $\deg$ is a quadratic form, and a version of the Cauchy-Schwarz inequality.

Let $\phi_q:E\to E$ be the $q$th power Frobenius morphism. 
Since $\F_q$ consists of exactly the solutions to $x^q=x$, we have
\[
\ol{\F_q}^{\ph_q}=\F_q,
\]
i.e. the fixed field of $\ol{\F_q}$ under $\phi_q$ is $\F_q$. Hence $P\in E(\F_q)$ iff $\phi_q(P)=P$, iff $P\in \ker(1-\phi_q)$. We have
\[
|E(K)|=|\ker(1-\phi_q)|=\deg(1-\phi).
\] 
The latter from CITE. SEPARABLE. Now $\deg$ is a positive definite quadratic form. We use the following.
\begin{lem}[Cauchy-Schwarz inequality for groups]
Let $A$ be an abelian group and $d:A\to \Z$ be a positive definite quadratic form. Then for all $\psi,\phi\in A$,
\[
|d(\psi-\phi)-d(\phi)-d(\psi)|\le 2\sqrt{d(\phi)d(\psi)}
\]
\end{lem}
\begin{proof}
The proof is similar to the proof for the ordinary Cauchy-Schwarz inequality. Since $d$ is a quadratic form,
\[
B(\psi,\phi)=d(\psi-\phi)-d(\psi)-d(\phi)
\]
is bilinear. Since $d$ is positive definite,
\[
0\le d(m\psi-n\phi)=m^2d(\psi)+mnB(\psi,\phi)+n^2d(\phi).
%-\rc2L(m\psi-n\psi,m\psi-n\psi)=
%(d(m\psi)+d(n\phi)+L(\psi,\phi))^2
\]
The RHS is quadratic in $m$ and $n$, hence obtains minimum at $\fc mn=-\fc{B(\psi,\phi)}{2d(\psi)}$. So take $m=-B(\psi,\phi)$ and $n=2d(\psi)$. We get 
\[
0\le d(\psi)[4d(\psi)d(\phi)-L(\psi,\phi)^2].
\]
When $\psi\ne 0$, we get $4d(\psi)d(\phi)\ge L(\psi,\phi)^2$, from which the desired inequality follows from taking square roots.
For $\psi=0$ the result is obvious.
\end{proof}
Since $\deg \phi_q=q$, the Cauchy-Schwarz inequality gives
\[
||E(\F_q)|-1-q|=| \deg(1-\phi)-\deg(1)-\deg(\phi)|\le 2\sqrt q
\]
as needed.
\end{proof}
Application to character sums (Silverman, p. 132).
%Given a value for $x$, there are 
\chapter{Elliptic curves over $\C$}
\section{Elliptic curves and lattices}
%We show that elliptic curves correspond to lattices in $\C$ by parametrization using the Weierstrass $\wp$-function.
We show that the Weierstrass $\wp$-function can parametrize elliptic curves.
\begin{thm}
Let $\La$ be a lattice. Then 
\[E:y^2=4x^3-g_2x-g_3\] is an elliptic curve (i.e. $\De(\Ga):=g_2^m-27g_3^2\ne 0$ and $4x^3-g_2x-g_3=0$ has distinct roots).

The map
\begin{align*}
\phi:\C/\La&\to E\\
z&\mapsto [\wp(z),\wp'(z),1]
\end{align*}
is a complex analytic isomorphism of complex Lie groups.
\end{thm}

Now we show that every elliptic curve over $\C$ arises in this way, and moreover, the maps between the lattices correspond to maps between the curves.
\begin{thm}
There is an equivalence of categories between the following categories.
\begin{enumerate}
\item
Objects: Elliptic curves over $\C$.\\
Maps: Isogenies.
\item
Objects: Lattices $\La\sub \C$ up to homothety.\\
Maps: The maps between $\La_1$ and $\La_2$ are $\set{\al\in \C}{\al\La_1\subeq \La_2}$.
\end{enumerate}
\end{thm}
\begin{thm}
Let $E/\C$ be an elliptic curve, and let $\La:=\an{\om_1,\om_2}$ be its associated lattice.
\begin{enumerate}
\item
If $\fc{\om_1}{\om_2}$ is not quadratic over $\Q$, then
\[
\End(E)=\Z.
\]
\item 
If $\fc{\om_1}{\om_2}$ is quadratic over $\Q$, then
$\End(E)$ is isomorphic to an order in an imaginary quadratic extension, i.e. $E$ has complex multiplication.
\end{enumerate}
\end{thm}
\section{Moduli spaces of elliptic curves}
(Eventually move to its own chapter)
We know elliptic curves correspond to lattices. The space of lattices can be considered as the upper half place quotiented out by the action of $\SL_2(\Z)$, $\cal H/\Ga(1)$, which is a Riemann surface, called the \textbf{moduli space}. This surface can be mapped to an algebraic curve using the modular function $j$ (this algebraic curve is obtained because $j$ satisfies the modular equation REF). Thus, to study the space of all elliptic curves, we can just study the resulting curve. 

In fact we look at elliptic curves with additional structure.
For instance, to study the torsion points of an elliptic curve we could consider the set of pairs $(E,P)$ where $E$ is an elliptic curve and $P$ is a point of order $N$. More possibilities are given below.

An example of a theorem proved with using moduli spaces is the following.
\begin{thm}[Mazur]
Let $E$ be an elliptic curve defined over $\Q$. Then $E_{tors}(\Q)$ is isomorphic to one of the following 15 groups.
\begin{itemize}
\item
$\Z/n\Z$ for $1\le n\le 10$ or $n=12$.
\item
$\Z/2\Z\times \Z/m\Z$, $m=2,4,6,8$.
\end{itemize}
\end{thm}
Essentially, consider the moduli space associated to $(E,C)$ for other subgroups, and show they have no rational points.

Also Fermat's last theorem...

A summary.

\begin{tabular}{|c|c|c|c|}
\hline
Enhanced elliptic curve & Equivalence relation &Consider $\cal H$ mod...&Moduli space\\
\hline
$(E,C)$, $C$ subgroup of order $N$ & & $\Ga_0(N)$ & $S_0(N)$\\
\hline$(E,Q)$, $Q$ point of order $N$ & & $\Ga_1(N)$ & $S_1(N)$\\
\hline$(E,(P,Q))$, $P,Q$ points of order $N$& &$\Ga(N)$& $S(N)$\\
\hline
\end{tabular}
