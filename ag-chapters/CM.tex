\chapter{Complex multiplication}\llabel{ch:CM}
\index{complex multiplication}
In this chapter, we combine class field theory with the theory of elliptic curves, first to characterize the maximal abelian extension of $K$, then to illustrate the relationships in Section~\ref{ch:cft-app}.\ref{sec:intro-langlands} for CM elliptic curves. We will assume basic facts about elliptic curves (for an introduction see Silverman~\cite[Chapter III]{Si86}).

We know that every elliptic curve over $\C$ has endomorphism ring either equal to $\Z$ or a quadratic order. In the second case, the elliptic curve is said to have \textbf{complex multiplication}. This gives the elliptic curve a lot more structure. On one hand, it is useful algebraically---as we will see, torsion points of a CM elliptic curve give abelian extensions of imaginary quadratic fields. In general, because of the added structure, much more is known about CM elliptic curves than other elliptic curves, and they can act as a kind of ``testing ground" or ``first case" of general conjectures.

On the other hand, CM elliptic curves have practical uses---for instance, if we take an CM elliptic curve corresponding to a specific endomorphism ring, we can easily compute its order. Hence we can generate an elliptic curve with near-prime order, useful in cryptography. This is much more efficient than generating random elliptic curves and using Schoof's algorithm to find their orders. 

There are several big theorems about complex multiplication. In Section~\ref{sec:cm-C}, we specialize our knowledge about the relationship between elliptic curves over $\C$ and complex tori to CM elliptic curves and build a toolbox of basic facts. However, since we are interested in number theory, we want to take curves defined over $\C$ and define them over $\ol{\Q}$ instead---which we do in Section~\ref{sec:cm-Q}. Once we have these basics, we can then prove the big theorems.

We suppose $E$ has CM by a quadratic order $\sO\sub K$ (i.e. $\End(E)\cong \sO$), where $K$ is a quadratic extension of $\Q$. Then the following hold.
\begin{enumerate}
\item
The $j$-invariant $j(E)$ generates the {\it ring class field} of $\sO$ over $K$. In particular, if $\sO=\sO_K$, then $j(E)$ generates the {\it Hilbert class field} of $K$, the maximal unramified abelian extension (Theorem~\ref{thm:j-generates-hilbert}):
\[
K(j(E))=H_K.
\]
\item
If $E$ is defined over $H_K$, and we adjoin certain functions of torsion points of $E$, then we get the {\it maximal abelian extension} of $K$ (Theorem~\ref{thm:max-abe-ext-K}):
\[
K(j(E),h(E\tors))=K\abe.
\]
Compare this with the Kronecker-Weber Theorem, which says the maximal abelian extension of $\Q$ is generated by roots of unity (torsion points of $\ol{\Q}^{\times}$).
\item
$j(E)$ is moreover an {\it algebraic integer} (We omit this; see Silverman AT,~\cite[II.6]{Si94}.) %(Theorem~\ref{thm:j-alg-int}). 
\item
The action of the idele class group sending $K/\ma$ to $K/\mx^{-1}\ma$ corresponds to the Galois action on the corresponding elliptic curves, where the Galois action is given by the Frobenius element of $\si$. This is the Main Theorem of Complex Multiplication~\ref{thm:mt-cm}, and plays an important part in taking moduli spaces initially defined only over $\C$ and defining them over algebraic number fields.
%There is an {\it analytic} isomorphism between $K/\ma$ (a ``lattice" in $K$) and the corresponding elliptic curve
\item
The $L$-series of a CM elliptic curve is particularly easy to understand, because it is a product of 2 Hecke $L$-series (Theorem~\ref{thm:cmec-l}). 
%(I'm not sure I like this "reason.") This is because the associated Galois representation is abelian so ``decomposes" into two abelian subrepresentations.
\end{enumerate}

Two ``big ideas" we'll consistently see are the following.
\begin{enumerate}
\item We expect abelian extensions because for CM elliptic curves (with endomorphism ring $\sO_K$, say), the image of the map $G(L/H_K)\hra \Aut(E[m])$ commutes with $\sO_K$, not just $\Z$ and hence must be abelian, with appropriate $L$.
\item We can use torsion points $E[m]$ to ``keep book" on the action of Frobenius, in the same way that we used the roots of unity $\mu_m$ to keep book on the action of Frobenius on $G(\Q(\mu_m)/\Q)$.
\end{enumerate}
\section{Elliptic curves over $\C$}
The following theorem helps us understand elliptic curves over $\C$.
\begin{thm}\llabel{thm:lattice-ec-eoc}
Let $g_2(\La)=60G_4(\La)$ and $g_3(\La)=140G_6(\La)$, where $G_n$ is the Eisenstein series.
Let $\La$ be a lattice in $\C$ and $\wp$ be the associated Weierstrass $\wp$-function.

There is a complex analytic isomorphism between the complex torus $\C/\La$ and the elliptic curve over $\C$,
\[
y^2=4x^3-g_2(\La)x-g_3(\La)
\]
given by
\[
\Phi(z)=(\wp(z),\wp'(z)).
\]

The map $\Phi$ gives an equivalence of categories between the following.
\begin{enumerate}
\item
Objects: Complex tori $\C/\La$, where $\La$ is a lattice in $\C$.\\
Maps: Multiplication-by-$\al$ $\C/\La_1\to \C/\La_2$ where $\al\La_1\subeq \La_2$.
\item
Objects: Elliptic curves over $\C$.\\
Maps: Isogenies.
\end{enumerate}
\end{thm}
\begin{proof}
Silverman~\cite[VI.5.1.1, 5.3]{Si86}
\end{proof}
The endomorphism ring of a lattice $\La\sub \C$ is either $\Z$ or an imaginary quadratic order, so the same is true of an elliptic curve $E$ over $\C$. If the endomorphism ring is a quadratic order $\sO$, we say $E$ has \textbf{complex multiplication} by $\sO$.
\section{Complex multiplication over $\C$}\llabel{sec:cm-C}
%\circlearrowright=\cir
%We've associated each elliptic curve with a lattice, and shown that that isogenies correspond to maps on the lattice. 
\subsection{Embedding the endomorphism ring}
We know the endomorphism ring $\End(E)$ of a CM elliptic curve corresponds to a quadratic order $\sO$ but since any quadratic order has conjugation as an isomorphism, we need to specify a way to embed $\End(E)$ into $\C$. 
\begin{ex}\llabel{ex:which-i}
Consider the curve $E:y^2=x^3+x$. We note that the endomorphisms
\begin{align*}
\phi_1(x,y)&=(-x,iy)\\
\phi_2(x,y)&=(-x,-iy)
\end{align*}
both square to $-1$. Which one should we call $[i]$, multiplication by $i$?
\end{ex}
Fortunately, we have a way of embedding $\End(\La)$ into $\C$, where $\La$ is the lattice corresponding to $E$, because $\La$ itself is in $\C$. This to give a canonical way of embedding $\End(E)$ into $\C$.

%Restrict our attention to elliptic curves with specified endomorphism ring congruent to the quadratic order $\sO$. There are two possible isomorphisms $R\cong \End(E)$; we need to fix one so that we get commutativity. This we do with the invariant differential.
\begin{pr}\llabel{pr:normalize-cmec}
Let $E/\C$ be a CM elliptic curve with complex multiplication by $\sO$. There is a unique isomorphism $[\cdot]:\sO\xra{\cong}\End(E)$ 
satisfying either of the following equivalent conditions.
\begin{enumerate}
\item
$[\al]$ is the unique morphism making the following diagram commute, where the top map is multiplication by $\al$.
\[\xymatrix{
\C/\La\ar[r]^{m_{\al}}\ar[d]^{\Phi} & \C/\La\ar[d]^{\Phi}\\
E_{\La}\ar[r]^{[\al]}& E_{\La}
}\]
\item For any invariant differential $\om\in \Om_E$, $[\al]^*\om=\al\om$.
\end{enumerate}
Moreover, we have the following.
\begin{enumerate}
\item[3.]
Define $[\cdot]_1$ and $[\cdot]_2$ for elliptic curves $E_1$ and $E_2$. For any morphism $\phi:E_1\to E_2$,
\[
\phi\circ [\al]_{1}=[\al]_{2}\circ \phi.
\]
In other words, multiplication by $\al$ commutes with all morphisms.
\item[4.] For any $\si\in \Aut(\C)$,
\[
[\al]_E^{\si}=[\si(\al)]_{\si(E)},
\]
i.e. it commutes with Galois action.
\end{enumerate}
\end{pr}
The pair $(E,[\cdot])$ is called a \textbf{normalized} elliptic curve. After we prove this proposition, we will assume all CM elliptic curves are normalized.
\begin{proof}
The uniqueness and existence of $[\al]$ satisfying item 1 follows directly from the equivalence of categories (Theorem~\ref{thm:lattice-ec-eoc}).

Define $[\al]$ as in item 1.
%By Proposition REF, the space of invariant differentials is 1-dimensional. 
For any invariant differential $\om$ on $E_{\La}$, since $\Phi$ is an analytic isomorphism, we can consider its pullback to $\C/\La$; it will be $c\,dz$ for some $c$ (The space of invariant differentials on $\C/\La$ is 1-dimensional.) Clearly, $m_{\al}^*(c\,dz)=c\,d(\al z)=\al c\,dz$. Transferring this to the bottom row of the commutative diagram gives $[\al]^*\om=\al\om$. 
For uniqueness, note the map
\begin{align}
\Hom(E_1,E_2)&\hra \Hom(\Om_{E_2},\Om_{E_1})\llabel{eq:ec-diff-inj}
\\
\nonumber\phi&\to \phi^*
\end{align}
is injective when all isogenies $E_1\to E_2$ are separable (in particular, in characteristic 0), i.e. the action of an isogeny of elliptic curves on an invariant differential completely determines the morphism. Taking $E_1=E_2$ and considering the preimage of multiplication-by-$\al$ gives uniqueness in item 2.

A simple diagram chase shows that $(\phi \circ [\al]_1)^*$ and $([\al]_2\circ \phi)^*$ act the same way on $\om\in \Om_{E_2}$. Then~(\ref{eq:ec-diff-inj}) gives item 3.

The proof of item 4 is similar.
\end{proof}
\begin{ex}
The definition using differentials is useful for calculations. 
Revisiting the above Example~\ref{ex:which-i}, we see that we should let
\begin{align*}
[i](x,y)&=(-x,iy).
\end{align*}
Indeed, defining $[i]$ in this way, we check that
\[
[i]^*\fc{dx}y=\fc{d(-x)}{iy}=i\fc{dx}y.
\]
\end{ex}
\subsection{The class group parameterizes elliptic curves}
Let $K$ be an imaginary quadratic field and $\sO$ an order inside $K$.
\begin{df}
Let $L$ be a field. Define
\begin{align*}
\text{Ell}_L(\sO)&=\{\text{elliptic curves $E/L$ with $\End(E)\cong \sO$}\}\\
\Ell_{L}(\sO)&=\fc{\{\text{elliptic curves $E/L$ with $\End(E)\cong \sO$}\}}{\text{isomorphism over }L},
\end{align*}
i.e. $\Ell_L(\sO)$ is the set of elliptic curves over $L$ whose endomorphism ring is $\sO$. If we omit $L$, we assume $L=\C$.
\end{df}
If $E\in \text{Ell}(\sO)$, then its corresponding lattice $\La$ must be homothetic to a fractional ideal of $\sO$: indeed, we can scale the lattice so that $1\in \La$; then $\sO\subeq \La$ so $\La\subeq K$; since it is a lattice it must be a fractional $\sO$-ideal. %We may assume
%, so correspond to an ideal of $\sO$. 
Now note an $\sO$-ideal $\ma$ has endomorphism ring $\sO$ iff $\ma$ is a {\it proper} ideal (see Definition~\ref{quadratic-forms}.\ref{df:proper-ideal}).\footnote{When $R=\sO_K$, all ideals are proper, so this distinction is not important. The reader unfamiliar with non-maximal orders can take $R=\sO_K$ throughout.} Hence we get a correspondence between isomorphism classes of elliptic curves $[E]\in \Ell(\sO)$ and proper $\sO$-ideals up to homothety.
%Two elliptic curves are isomorphic iff they correspond to homothetic fractional ideals, and 
However, two fractional ideals $\ma$ and $\mb$ are homothetic iff $\la\ma=\mb$ for some $\la$, i.e. iff they are equivalent in the class group. Thus the class group of $\sO$ parameterizes all isomorphism classes of elliptic curves with endomorphism ring $\sO$. This is summarized in the following.
\[
\Ell(\sO)=\fc{\{\text{elliptic curves $E/\C$ with $\End(E)\cong \sO$}\}}{\text{isomorphism over }\C}
=\fc{\{\text{proper fractional $\sO$-ideal}\}}{\text{principal $\sO$-ideals}}=\Cl(\sO).
\]
We state this as a theorem.
\begin{thm}\llabel{thm:ell=cl}
We have a bijection
\[
\Ell(\sO)\cong \Cl(\sO)
\]
where $[E]\in \Ell(\sO)$ is sent to a $[\ma]$, where $\ma$ is a fractional ideal homothetic to the lattice corresponding to $E$. 
\end{thm}
We get much more than this, however. $\Ell(\sO)$ is a priori just a set; however, $\Cl(\sO)$ is a {\it group}. 
We can define the action of $I(\sO)$ on $\text{Ell}(\sO)$ since $I(\sO)$ acts on lattices. This action will descend to an action of $\Cl(\sO)$ on $\Ell(\sO)$, since isomorphic elliptic curves correspond to equivalent ideals.
%: because the class group acts naturally on itself, it acts on isomorphism classes of elliptic curves.
%The class group acts naturally on itself, it acts on isomorphism classes of elliptic curves.
%Define
%\[
%Ell(R)=\fc{\{\text{elliptic curves $E/\C$ with $\End(E)\cong R$}\}}{\text{isomorphism over }\C}.
%\]
%Silverman 2.1.2.
\begin{thm}
There is a group action of $\Id(\sO)$ on $\text{Ell}(\sO)$ given by
\[
\ma E_{\La}=E_{\ma^{-1}\La}
\]
where $E_{\La}$ denotes the elliptic curve corresponding to the lattice $\La$.

This descends to a simply transitive group action of $\Cl(\sO)$ on $\Ell(\sO)$.
\end{thm}
\begin{proof}
Just check that if $\La$ has endomorphism ring $\sO$, then so does the lattice $\ma^{-1}\La$. (Note that $\mb L$ is defined by $\set{s\al}{s\in \mb, \al\in L}$.)

For the second part, note that $E_{\La}\cong \ma E=E_{\ma^{-1}\La}$ iff $\La$ and $\ma^{-1}\La$ are homothetic, i.e. $\ma$ is principal.
\end{proof}
\index{torsor}
\begin{rem}
Another way of saying that $\Cl(\sO)$ acts simply transitively on $\Ell(\sO)$ is that $\Ell(\sO)$ is a \textbf{torsor} or \textbf{principal homogeneous space} for $\Cl(\sO)$.
\end{rem}
This action will be fundamental to our understanding of CM elliptic curves. Later on we will relate this to the Galois action. The interplay between these two actions is the source for much of the richness of CM theory.

\subsection{Ideals define maps}
For any $n\in \Z$ and any elliptic curve $E$, $n$ defines the multiplication by $n$ map $[n]:E\to E$. When $E$ has CM, we saw in Theorem~\ref{pr:normalize-cmec} that $\al\in \sO$ defines (canonically) the multiplication by $\al$ map $[\al]:E\to E$. We now extend this to {\it ideals}: if $\ma$ is a proper $\sO$-ideal, $\ma$ determines a ``multiplication by $\ma$" map. The only difference is that $[\ma]$ is now a map $E\to \ma E$.
\begin{df}
Let $E\in \text{Ell}(\sO)$ correspond to the lattice $\La$. Let $\ma$ be a proper integral ideal of $\sO$. We have $\ma R\subeq R$, so $\ma$ determines a map $\C/\La\to \C/\ma^{-1}\La$, sending $z\mapsto z$. 
Define the multiplication by $\ma$-map as the corresponding map on elliptic curves 
\[[\ma]:E\to E_{\ma^{-1}\La}=\ma E.\]
\end{df}
\begin{pr}\llabel{pr:E[a]}
%(Does this need to be over $\sO_K$?)
\fixme{Do we need $R=\sO_K$?}
Let $E\in \text{Ell}(\sO_K)$. We have the following.
\begin{enumerate}
\item
The kernel of $[\ma]$ (the ``$\ma$-torsion points") is
\[
E[\ma]:=\set{P\in E}{[\al]P=0\text{ for all }\al\in \ma}\cong \sO_K/\ma.
\]
\item
The degree of $[\ma]$ is
\[\deg([\ma])=|E[\ma]|=\fN(\ma),\]
and in particular, $\deg([\al])=|E[\al]|=\nm_{K/\Q}(\al)$.
\end{enumerate}
\end{pr}
\begin{proof}
Silverman AT~\cite[pg. 102-3]{Si94}.
\end{proof}
%Define the group of $\ma$-torsion points of $E$ by
%\[
%E[\ma]=\set{P\in E}{[\al]P=0\text{ for all }\al\in \ma}.
%\]
\section{Defining CM elliptic curves over $\ol{\Q}$}
\llabel{sec:cm-Q}
We show that we do not lose anything if we just consider elliptic curves over $\ol{\Q}$ instead of over $\C$. To do this, we look at the $j$-invariants.
%$\Ell_{\ol{\Q}}(R)$ instead
\begin{pr}\llabel{pr:j-alg}
Suppose $E$ is an elliptic curve with CM by a quadratic order $\sO$.
Then $j(E)\in \ol{\Q}$, i.e. $j(E)$ is algebraic.
\end{pr}
\begin{proof}
Let $\si$ be any automorphism of $\C$ over $\Q$. We look at how $\si$ acts on $j(E)$.

Note that $E^{\si}$ is defined by taking any equation for $E$ and operating on all the coefficients of $E$ by $\si$, so $\si(j(E))=j(E^{\si})$.

First note that $ \End(E)\cong \End(E^{\si})$ by the map $\phi\mapsto \phi^{\si}$. Hence $\End(\si(E))=\sO$ as well. But $\Cl(\sO)$ is finite, and as $|\Cl(\sO)|=|\Ell(\sO)|$ (Theorem~\ref{thm:ell=cl}) we see that the $E^{\si}$ lie in finitely many isomorphism classes. Because isomorphic elliptic curves have the same $j$-invariant, there are a finite number of possibilities for $j(E^{\si})$.

As $\set{\si(j(E))}{\si\in \Aut(\C)}$ is finite, $j(E)$ must be algebraic.
\end{proof}
This allows us to prove the following.
\begin{thm}\llabel{thm:ell-c-q}
We have
\[
\Ell_{\C}(\sO)\cong \Ell_{\ol{\Q}}(\sO).
\]
\end{thm}
\begin{proof}
We use the following properties of the $j$-invariant. (\cite[III.1.4]{Si86})
\begin{enumerate}
\item
For every $j\in K$, there exists an elliptic curve $E/K$ with $j(E)=j$.
\item
Let $K$ be an algebraically closed field and $E_1$, $E_2$ be elliptic curves defined over $K$. Then $E_1\cong E_2$ over $K$ iff $j(E_1)=j(E_2)$. (The backwards direction does not necessarily hold if $K$ is not algebraically closed.)
\end{enumerate}
We show that the map
\begin{equation}\llabel{eq:ellqc}
\Ell_{\ol{\Q}}(\sO)\to \Ell_{\C}(\sO)
\end{equation}
is an isomorphism (of sets, in fact, of $\Cl(\sO)$-modules). The map is well-defined, because any automorphism over $\ol{\Q}$ is an automorphism over $\C$.

By Lemma~\ref{pr:j-alg}, if $[E]\in \Ell_{\C}(\sO)$ then $j(E)\in \ol{\Q}$.
By item 1, there exists an elliptic curve $E'$ defined over $\ol{\Q}$ with $j(E')=j(E)$. Then $E'$ is isomorphic to $E$ over $\C$. Thus the map~(\ref{eq:ellqc}) above is surjective. It is injective because if $E,E'$ are defined over $\ol{\Q}$ and isomorphic over $\C$, then item 2 says $j(E)=j(E')$; and the other direction of item 2 says that $E\cong E'$ over $\ol{\Q}$.
\end{proof}
It is also important to know what fields we can define elliptic curves and isogenies over.
\begin{pr}
Suppose $E$ is an elliptic curve with CM by $\sO\sub K$, where $K$ is an imaginary quadratic field.
\begin{enumerate}
\item
If $E$ is defined over $L$ then endomorphisms of $E$ can be defined over $LK$.
\item
If $E_1,E_2$ are defined over $L$ then there exists a finite extension $M/L$, so that every isogeny $E_1\to E_2$ is defined over $M$.
\end{enumerate}
\end{pr}
\begin{proof}
For item 1, note that all endomorphisms are in the form $[\al]$ and use Proposition~\ref{pr:normalize-cmec}(4). 

For item 2, first we claim that any isogeny $\phi$ is defined over a finite extension of $L$. For any $\si\in \Aut(\C)$ fixing $L$, $\phi^{\si}$ is a map $E_1\to E_2$ having the same degree as $\phi$.
Any isogeny is determined by its kernel, up to automorphism of $E_1$ and $E_2$. As $E_1$ has a finite number of subgroups of given index and $\deg(\phi)=\ker(\phi)$, there are finitely many isogenies of a given degree. Hence $\set{\phi^{\si}}{\si \in G(\C/L)}$ is finite, showing $\phi$ is defined over a finite extension of $L$.

Now $\Hom(E_1,E_2)$ is a finitely generated group, so we can take the field of definition for a finite set of generators.
\end{proof}
\section{Hilbert class field}
\subsection{Motivation: Class field theory for $\Q(\ze_n)$ and Kronecker-Weber}\llabel{sec:kw-cm}
\subsubsection{The case of $\Q$}
First we give some motivation for the next two sections by making an analogy with class field theory for $\Q(\ze_n)$.
We can think of $\mu_n$, the $n$th roots of unity, as the analogue of $E[n]$: $\mu_n$ are the $n$-torsion points of the group variety $\ol{\Q}^{\times}$ under multiplication, and $E[n]$ are the $n$-torsion points of an elliptic curve. To emphasize this analogy, we write $K^{\times}[n]$ to denote the $n$th roots of unity in $\ol K$.

Recall how we established class field theory for $\Q(\ze_n)$: given a prime $p$, we want to find $(p,\Q(\ze_n)/\Q)$. To do this we looked at the action of $(p,\Q(\ze_n)/\Q)$ on $\Q^{\times}[n]=\mu_n$, by taking everything modulo $p$. We know by definition of $(p,\Q(\ze_n)/\Q)$ how it must act on the residue field extension $l/\F_p$ and hence on $\F_p^{\times}[n]$. Suppose $p\nmid n$. Because the maps
\begin{align}
\nonumber\Q^{\times}[n]&\hra \F_p^{\times}[n]\\
\End(\Q^{\times}[n])&\hra  \End(\F_p^{\times}[n])\llabel{eq:end-inj}
\end{align}
are injective (the first is because $p\nmid n$ and the second is a direct consequence of the first), once we know how $(p,\Q(\ze_n)/\Q)$ acts on $\F_p^{\times}[n]$, we know it acts on $\Q_p^{\times}[n]$, so we know exactly what automorphism it is:
\[
(p,\Q(\ze_n)/\Q)(\ze_n)=\ze_n^p.
\]
In particular, since $\ze_n$ is a $n$-torsion point (i.e. $\ze_n^n=1$) this only depends on $p\pmod n$. Hence we get the Artin map $\psi_{\Q(\ze_n)/\Q}$ factoring through the modulus $\iy n$:\footnote{The $\iy$ is a technical detail coming from the fact that $\Q$ is totally real.}
\[
\psi_{\Q(\ze_n)/\Q}:I_{\Q}/I_{\Q}(1,n\iy)\xrc G(\Q(\ze_n)/\Q).
\]
Finally, since every modulus divides $\iy n$ for some $n$, we get the Kronecker-Weber Theorem
\[
\Q\abe=\Q(\ze_{\iy})=\Q(\Q^{\times}[\iy]).
\]
%(Also mention how parameterized by exponential.)

In summary, we found the ray class groups and thus the maximal abelian extension by looking at how $(p,\Q(\ze_n)/\Q)$ acted on $\Q^{\times}[n]$:
\begin{equation}\llabel{thm:max-ab-Q}
\xymatrix{
&\Q^{\times}[n] \ar@{.>}[r]^{\wt{\bullet}}_{\text{reduction}}
\ar@{}[d]|-{\textstyle{\cir}} & \F_p^{\times}[n]\ar@{}[d]|-{\textstyle\cir}\\
I_{\Q}/P_{\Q}(1,n\iy)\ar[r]^{\psi_{\Q(\ze_n)/\Q}} & G(\Q(\ze_n)/\Q)\ar@{.>}[r]^{\wt{\bullet}} & G(\F_p(\ze_n)/\F_p).
}
\end{equation}
\subsubsection{The case of $K$}
One big difference when we're working over an imaginary quadratic field $K$ is that while we had $\Cl_{\Q}=1$, we have $\Cl_K$ is nontrivial in general. This corresponds to the fact that there is only 1 nonisomorphic ``version" of $\G_m(\Q)=\Q^{\times}$, but multiple elliptic curves with endomorphism ring by the same order $\sO$. Hence $G(K\abe /K)$ no longer operates on the same elliptic curve. Instead we have to analyze it in two steps.
\begin{enumerate}
\item
Consider the action of $G(H_K/K)$ on $\Ell_{\ol{\Q}}(\sO)$, i.e. equivalence classes of elliptic curves with CM by $\sO$.
\item
Consider the action of $G(K\abe/H_K)$ on the torsion points $E\tors$ of a single elliptic curve. 
\end{enumerate}
In both cases, we will understand the action by looking at how the Frobenius elements of the Galois groups act.

\subsubsection{The case of $K$: Part 1}
We have two natural actions on the set of elliptic curve $\Ell_{\ol{\Q}}(\sO_K)$, namely the action of $G(\ol K/K)$ and $\Cl(\sO_K)$. Our first task is to relate these, i.e. find a dotted map that preserves the action on $\Ell_{\ol{\Q}}(\sO_K)$:
\begin{equation}\llabel{eq:G-cl-compat}
\xymatrix{
&\Ell_{\ol{\Q}}(\sO_K)\ar@{}[ld]|-{\textstyle{\cir}}\ar@{}[rd]|-{\textstyle{\cir}}&\\
G(\ol K/K)\ar@{.>}[rr] &&\Cl(\sO_K).
}
\end{equation}
We'll see that this map factors through $G(L/K)$ where $L=K(j(E))$. 
We have a map $\psi_{L/K}:I_{K}^{\mf}/P_{K}(1,\mf)\to G(L/K)$; we show that $\mf=1$ and the composition of the two maps is an isomorphism, and that in fact we have
\beq{eq:G-cl-compat2}
\xymatrix{
&&\Ell_{\ol{\Q}}(\sO_K)\ar@{}[ld]|-{\textstyle{\cir}}\ar@{}[rd]|-{\textstyle{\cir}}&\\
I_K/P_K\ar[r]^{\Psi_{L/K}}\ar@/_1pc/@{.>}[rrr]_{\ma\mapsto [\ma]}&G(L/K)\ar[rr] &&\Cl(\sO_K).
}
\eeq
We establish~\eqref{eq:G-cl-compat2} by looking at the reduction of the elliptic curves modulo some $\mP$.

Since $G(H_K/K)\cong \Cl(\sO_K)$ this will show that $L=H_K$, the Hilbert class field of $K$.
\subsubsection{The case of $K$: Part 2}
We can now do the same thing we did with $\Q$, use the torsion points of elliptic curves to find the ray class fields and the maximal abelian extensions. We can't work directly over $K$ because $\Cl_K$ is nonzero, but if we imitate the argument (with some modifications) over $\Q$ for $H_K$ we will get the ray class fields of $K$. We let $L_n=K(j(E),h(E[n]))$ where $h$ is a Weber function (to be defined).

Let $l_n$, $l$ be the residue fields of $L_n$ and $H_K$ modulo some prime. 
We show $L_n$ is the ray class field for $(n)$ by constructing the diagram
%We are going to do the same thing for torsion points of elliptic curves and find that
\beq{eq:rcf-k}
\xymatrix{
&E[n] \ar@{.>}[r]^{\wt{\bullet}}_{\text{reduction}}
\ar@{}[d]|-{\textstyle{\cir}} & \wt{E}[n]\ar@{}[d]|-{\textstyle\cir}\\
\nm_{H_K/K}(I_{H_K}^n)/P_{K}(1,n)\ar[r]^-{\psi_{L_n/K}}_{\cong} & G(L_n/H_K)\ar@{.>}[r]^{\wt{\bullet}} & G(l_n/l).
}
\eeq
%{thm:max-ab-Q}

%Class field theory then tells us that the ray class group of $G(\Q(\ze_n)/\Q)$ is 
%we get that the Artin map $\psi_{\Q(\ze_n)/\Q}$ factors
We now carry out these two parts.
\subsection{The Galois group and class group act compatibly}
We establish the map in~(\ref{eq:G-cl-compat}).
\begin{thm}\llabel{thm:map-G-cl}
There exists a map $F:G(\ol K/K)\to \Cl(\sO_K)$ such that for {\it any} elliptic curve $E$,
\[
[E^{\si}]=F(\si)E.
\]
This map factors through $G(K\abe/K)$.
\end{thm}
As a reminder, the action of $\Cl(\sO_K)$ on $\Ell_{\ol{\Q}}(\sO_K)$ is such that if $E=E_{\La}$, then $F(\si)E=E_{F(\si)^{-1}\La}$. Theorem~\ref{thm:map-G-cl} expresses a deep relationship because the left-hand side expresses an algebraic action, while the right-hand side expresses an analytic action, as it is defined on lattices and the map between $E$ and $\C/\La$ is inherently analytic.

Proving this theorem essentially boils down to showing the Galois action commutes with the action on $\Cl(\sO_K)$.
\begin{pr}
For all $E$,
\[
\si([\ma] [E])=[\si(\ma)][\si(E)].
\]
\end{pr}
%\fixme{MOTIVATION??}
\begin{proof}
Suppose $E$ corresponds to $\La$, i.e. $E\cong \C/\La\C$. Then we have the exact sequence
\[
0\to \La\to \C\to E\to 0.
\]
Then $\ma E$ corresponds to $\ma^{-1}\La$. Take a resolution for $\ma$:
\[
R^m\xra{A} R^n\to \ma\to 0.
\]
Take a ``Hom product" and use the Snake Lemma. See~\cite[II.2.5]{Si94}.
%this CRAZY HOM PRODUCT and use SNAKE LEMMA.
\end{proof}
\begin{proof}[Proof of Theorem~\ref{thm:map-G-cl}]
See~\cite[II.2.4]{Si94}.
\end{proof}
\index{Hilbert class field of imaginary quadratic field}
\subsection{Hilbert class field}
Before we proceed with finding the Hilbert class field, we need to show injectivity of the reduction map like in~(\ref{eq:end-inj}).
\begin{thm}\llabel{thm:ec-hom-red}
Suppose $E_1$ and $E_2$ are elliptic curves defined over $L$ with good reduction at $\mP$. Then the reduction map
\[
\Hom(E_1,E_2)\to \Hom(\wt{E_1},\wt{E_2})
\]
is injective and preserves degrees.
\end{thm}
\begin{proof}
See Silverman AT \cite[pg. 124]{Si94} (Also see Silverman's errata).
\end{proof}

\fixme{Eventually rewrite this to work for all orders.}
The main theorem of this section is the following.
\begin{thm}[$j(E)$ generates the Hilbert class field]\llabel{thm:j-generates-hilbert}
Let $E$ be an elliptic curve with CM by $\sO_K$. Then
\begin{enumerate}
\item $K(j(E))=H_K$, the Hilbert class field of $K$.
\item $G(\ol K/K)$ acts transitively on the isomorphism classes of 
%$j$ invariants of 
curves in $\Ell(\sO_K)$.
%if $\{E_i\}$ are a set of representatives for the elements in $\Ell(\sO_K)$, then 
%\[
%\{j(E_i)\}=\set{\si(j(E_1))}{\si\in G(\ol K/K)}.
%\]
\item For any ideal $\ma\in I_K$,
\[
%j(E)^{\psi_{H_K/K}(\ma)}=j([\ma]E).
[E^{\psi_{H_K/K}(\ma)}]=[\ma][E].
\]
In particular, the action of Frobenius on the $j$-invariant is given by operating by $[\mfp]$ on the elliptic curve:
\[
%j(E)^{\Frob_{L/K}(\mfp)}=j(\mfp E).
[E^{(\mfp,H_K/K)}]=[\mfp][E].
\]
\end{enumerate}
\end{thm}
%Remark about $\ma^{-1}$ in action.
\begin{proof}
\noindent{\underline{Step 1:}}
First we show the following: There exists a finite set of primes $S$ of $\Z$ such that for any $p\nin S$ that splits completely in $K$, $p=\mfp\ol{\mfp}$, we have
\[
F((\mfp,L/K))=[\mfp]\in \Cl(\sO_K).
\]
This will show the dotted map in~(\ref{eq:G-cl-compat2}) is the identity for a large number of primes $\mfp$.

We have the map $[\mfp]:E\to \mfp E$. We show that this is ``like" the $p$th power Frobenius map. To do this, we show that it is inseparable of degree $p$ (this is why we needed $p$ to be split)\footnote{If $\mfp$ is not split, one can still show the map is inseparable of degree $p^2$, with some more work.}, %(according to Serre in the short CM article in CF.
and then look at the $j$-invariants of the reduced maps modulo $\mfp$.

As $\Ell_{\ol{\Q}}(\sO_K)=\Ell_{\C}(\sO_K)$ is finite, we can find a finite extension $L/K$ and representatives $E_1,\ldots, E_h$ of classes in $\Ell_{\C}(\sO_K)$, that are defined over $L$. Let $S$ be a set of primes containing the primes that satisfy one of the following conditions.
\begin{enumerate}
\item
$p$ ramifies in $L$. (Primes that ramify always cause trouble.)
\item
$E$ or some $E_i$ has bad reduction at some prime of $L$ lying over $p$.
\item
$v_p(\nm_{L/\Q}(j(E_i)-j(E_k)))\ne 0$ for some $i\ne k$. (This allows us to know what equivalence class an elliptic curve lies in, just by looking at its reduction modulo $p$.)
\end{enumerate}
Let $\La$ be the lattice such that $E(\C)\cong \C/\La$, and let $\ma$ be an integral ideal relatively prime to $\mfp$ such that $\ma\mfp=(\al)$ is principal (This exists by Corollary~\ref{factorization}.\ref{uf-dedekind-cor}). By the equivalence of categories~\ref{thm:lattice-ec-eoc}, the following maps on complex tori correspond to isogenies of elliptic curves:
\[
\xymatrix{
\C/\La \ar[r]^-i \ar[d]^{\Phi}_{\cong} &
\C/\mfp^{-1}\La \ar[r]^-i \ar[d]^{\Phi}_{\cong} &
\C/\mfp^{-1}\ma^{-1}\La \ar[r]^-{[\al]}_-{\cong} \ar[d]^{\Phi}_{\cong}&
\C/\La \ar[d]^{\Phi}_{\cong}\\
E\ar[r]^{\phi_1}&
\mfp E\ar[r]^{\phi_2}&
\ma \mfp E\ar[r]^-{\phi_3}_-{\cong}&
E
}
\]
Let the composition of the top maps be $f$ and the composition of the bottom maps be $g$.

Let $\om$ be an invariant differential on $E$. Then $\om'=\Phi^*\om$ is an invariant differential on $\C/\La$. It is in the form $c\,dz$. The composition of the top maps is just multiplication by $\al$, so $f^*\om'=\al\om'$. By commutativity, we get $g^*\om=\al\om$ as well.

Let $p\nin S$ and $\mP\mid \mfp\mid p$ in $L$, $K$, $\Q$,  respectively. 
Since $E$ has good reduction at $\mP$, we can reduce the elliptic curves and maps modulo $\mP$ to get
\[
\wt{g}^* \wt{\om}=\wt{\al}\wt{\om}=0
\]
since $\mP\mid \al$. By a criterion for separability ($g$ is separable iff $g^*$ does not act as 0 on $\Om_E$), $\wt g$ is inseparable. Now
\begin{align*}
\deg(\phi_1)&=\fN\mfp=p,\\
\deg(\phi_2)&=\fN\ma\perp p,\\
\deg(\phi_3)&=1.
\end{align*}
An inseparable map must have degree divisible by $p$, and the composition of separable maps is separable, so $\wt{\phi_1}$ must be inseparable.

Any inseparable map factors through the Frobenius map:
\beq{pE-factors-frob}
\xymatrix{
\wt E\ar[r] \ar[rd]_{\wt{\phi_1}}\ar[r]_{\phi_{p}}& \wt E^{(p)}\ar[d]_{\cong}^{\ep}\\
&\wt {\mfp E}.
}
\eeq
We have $p\deg(\ep)=\deg(\phi_p)\deg(\ep)=\deg(\wt{\phi_1})=p$ so $\deg(\ep)=1$. This shows $\ep$ is an isomorphism.

Thus we have 
\[
\wt{\mfp E}\cong \wt{E}^{(p)}.
\]
Now by definition of the Frobenius element (it is the $p$th power map modulo $\mP$), we have $j(\wt E^{(p)})=j(\wt E)^p=j(E)^{(\mfp,L/K)}$ modulo $\mP$. Putting everything together,
\[
j(\mfp E)\equiv j(\wt E^{(p)})\equiv j(E^{(\mfp,L/K)})\pmod{\mP}.
\]
But we chose $p$ so that nonisomorphic curves have $j$-invariants that are not congruent modulo $p$ (item 3). Therefore, $\mfp E\cong E^{(\mfp,L/K)}$. This shows that the action of $\mfp$ is the same as the action of $(\mfp,L/K)$, i.e. $F((\mfp,L/K))=[\mfp]$.\\

\step{2}
We show that $F:G(\ol K/K)\to \Cl(\sO_K)$ has kernel equal to $G(\ol K/K(j(E)))$, and so factors through $G(K(j(E))/K)\hra \Cl(\sO_K)$. Indeed,
\begin{align*}
\ker(F)&=\set{\si}{F(\si)E=E}\\
&=\set{\si}{E^{\si}=E}&\text{definition of }\si\\
&=\set{\si}{j(E)^{\si}=j(E)}&j \text{ parameterizes isomorphism classes}\\
&=G(\ol K/K(j(E))).
\end{align*}
We let $L=K(j(E))$.\\

\step{3} Let $\mf$ be the conductor of $L/K$. 
We extend Step 1 to all ideals $\ma$: for all $\ma$ we have
\[
F((\ma,L/K))=[\ma]\in \Cl(\sO_K);
\]
in other words $\mf=1$ and the following composition is the identity map.
\beq{eq:F(a)=[a]}
\xymatrix{
I_K/P_K\ar[r]^{\psi_{L/K}}\ar@/_2pc/@{.>}[rr]_{\Id}^{\cong}&G(L/K)\ar@{^(->}[r]^F &\Cl(\sO_K).
}
\eeq
Given $\ma\in I_K^{\mf}$, there are infinitely many $\mfp\in I_K^{\mf}$ in the same class as $\ma$ with degree 1 by Corollary~\ref{ch:cft-app}.\ref{cor:chebotarev-resfield1}. 
%Indeed, by the proof of Dirichlet's theorem, $\sum_{\mfp\sim\ma} \rc{\fN \mfp}$ diverges, while $\sum_{f(\mfp)>1}\rc{\fN\mfp}$ converges, so there must be infinitely many primes $\mfp\sim \ma$ with $f(\mfp)=1$. 
Choose such a prime $\mfp$, that does not divide a prime in $S$. 
%This argument due to http://math.stackexchange.com/questions/27308/does-every-ideal-class-contains-a-prime-ideal-that-splits
Note $\ma,\mfp$ differ by an ideal in $P_K(1,\mf)$ so they have the same image by the Artin symbol. Step 1 shows that
\[
F((\ma,L/K))=F((\mfp,L/K))\stackrel{\text{Step 1}}=[\mfp]=[\ma].
\]
In particular, for any principal ideal $(\al)\in I_K^{\mf}$, we have $F(((\al),L/K))=1$. However, by definition the conductor is the smallest $\mfp$ such that $\al\equiv 1\pmod{\mf}$ implies $((\al),L/K)=1$, so we must have $\mf=(1)$.\footnote{
Technically, we only have $((\al),L/K)=1$ for $(\al)\perp\mf$, and a priori $((\al),L/K)$ is not defined for $(\al)\perp \mf$. (We don't know $\mf=1$ yet.) The proper way to conclude $\mf=(1)$ is transfer the problem over to ideles: We know $\psi_{L/K}(P_K^{\mf})=1$, so $\phi_{L/K}(K^{\times}\mathbb U_K^{\mf})=1$. By $\I_K^{\mf}/K(1,\mf)\mathbb U_K(1,\mf)\cong \I_K/K^{\times}\mathbb U_K(1,\mf)$ we conclude that $\phi_{L/K}(K^{\times}\mathbb U_K)=1$. Hence $\mf =1$.
%So technically we should transfer the problem over to ideles, where we don't have this problem, and then conclude from there that the group corresponding to $L/K$ is exactly that of the Hilbert class field. \fixme{Fix this! Probably add a lemma on the conductor in the intro to cft chapter.}
}
Thus the map $F:I_K^{\mf}/P_K(1,\mf)\to G(L/K)$ we had originally is actually just $F:I_K/P_K\to G(L/K)$, and we get~(\ref{eq:F(a)=[a]}).\\

\step{4} 
Since the conductor is divisible by exactly the ramifying primes, $L/K$ is unramified, and $L\subeq H_K$. On the other hand, the map $F\circ \psi_{L/K}:I_K/P_K\to \Cl(\sO_K)$ is an isomorphism because $F\circ \psi_{L/K}$ is just the identity map. This gives $[L:K]=|\Cl(\sO_K)|=[H_K:K]$. Hence $L=H_K$. This shows item 1.\\

\step{5}
Item 3 now follows immediately, since we already showed $E^{\psi_{L/K}(\ma)}=[\ma]E$ and we now know $L=H_K$. Item 2 follows since the fact that the composition in~(\ref{eq:F(a)=[a]}) is an isomorphism means the map $F:G(L/K)\to \Cl(\sO_K)$ is surjective. Since $F$ transfers the action of $G(L/K)$ on $\Ell_{\ol{\Q}}(\sO_K)$ to $\Cl(\sO_K)$, and $\Cl(\sO_K)$ acts simply transitively on $\Ell_{\ol{\Q}}(\sO_K)$, we get that the same is true for $G(L/K)$. %We can of course transfer this statement to the $j$-invariants.
\end{proof}
%\begin{rem}
%In step 1, we used the fact that for $\mfp$ satisfying certain conditions, $\wt{\mfp E}\cong \wt{E}^(p)$. 
%Consider the special case where $\mfp E\cong \wt E^{(p)}$.
%\end{rem}
\section{Maximal abelian extension}
\index{maximal abelian extension of imaginary quadratic field}
We next carry out part 2 of our outline in Section~\ref{sec:kw-cm}. We construct the ray class fields for $K$, then take their compositum to get the maximal abelian extension.
\begin{df}
Suppose $E$ has CM by an order in $K$, and $E$ is defined over $H_K$.
A \textbf{Weber function} is an isomorphism $h:E/\Aut(1)\to \Pj^1$ defined over $H_K$. (So if $f:E\to E'$ is an automorphism, then $h(P)=h(f(P))$.)
\end{df}
We can always fix a concrete Weber function.
\begin{ex}
The simplest Weber function is the following. If $E$ has the form
\[
y^2=x^3+Ax+B,\quad A,B\in H_K,
\]
then take \[h(P)=\begin{cases}
x,&AB\ne 0\\
x^2,&B=0\\
x^3,&C=0.
\end{cases}
\]
In the 3 cases, respectively, $\Aut(E)$ is $1$, $\Z/2$ or $\Z/4$, and $\Z/3$ or $\Z/6$. %need to check this.

We can define a Weber function that is ``model independent," i.e. doesn't change under if we change to an isomorphic elliptic curve, by
\[
h(f(z))=\begin{cases}
\fc{g_2(\La)g_3(\La)}{\De(\La)}\wp(z,\La),&j(E)\ne 0,1728\\
\fc{g_2(\La)^2}{\De(\La)}\wp(z,\La)^2,&j(E)=1728\\
\fc{g_3(\La)}{\De(\La)}\wp(z,\La)^3,&j(E)=0.
\end{cases}
\]
This is because the expressions have ``weight 0."
%\fixme{mention this is because the expressions have ``weight 0."}
\end{ex}
The importance of the Weber function is given below. It would not be true if $h(P)$ were just defined as $h(x,y)=x$.
\begin{lem}\llabel{lem:K-ab-ext}
Let $E$ be an elliptic curve with CM by $\sO$. 
\begin{enumerate}
\item
The extension $K(j(E),E\tors)/K(j(E))$ is abelian.
\item
The extension $K(j(E),h(E\tors))/K$ is abelian.
\end{enumerate}
\end{lem}
The first statement is important because it tells us $G(\ol K/K(j(E)))$ acts in an abelian way on $E\tors$. Thus the ``Galois representation" of the Galois group on $E\tors$ is abelian. Thus, as we will see, it will decompose into two Gr\"ossencharacters.
\begin{proof}
%Since $E[m]=\Z/m\times \Z/m$, 
We have an injective map $G(K(j(E),E[m])/K(j(E)))\hra \Aut(E[m])$.\footnote{Since $E[m]=\Z/m\times \Z/m$, if we choose a basis for $E[m]$, we have $\Aut(E[m])\cong \GL_2(\Z/m)$, so we have a Galois representation.} Now, the image of $G$ in $\Aut(E[m])$ commutes with $\sO_K$, so is contained in 
\[\Aut_{\sO_K/m\sO_K}(E[m])\cong \Aut_{\sO_K/m\sO_K}(\sO_K/m\sO_K)\cong (\sO_K/m\sO_K)^{\times}
\]
which is abelian.

%For the second, we know $K(j(E),h(E\tors))/K(j(E))$ is abelian. We need t
For the second, suppose $\si,\tau\in G(K(j(E),h(E\tors))/K)$. We show that $\si\tau=\tau\si$. Since $K(j(E))/K$ is abelian, $\si\tau\si^{-1}\tau^{-1}$ fixes $j(E)$. %Suppose $P\in E\tors$. Then Now $\si(\tau(P))\in \si(\tau(E))$ and $\tau(\si(P))\in \tau(\si(E))$. However, $E':=\si(\tau(E))=\tau(\si(E))$ because the Galois action factors through $G(K\abe/K)$ (Theorem~\ref{thm:map-G-cl}). 
Now 
$\si\tau\si^{-1}\tau^{-1}$ gives an automorphism of $E'=\tau\si(E)$ because 
\[(\si\tau\si^{-1}\tau^{-1})\tau\si(E)=\si\tau(E)\cong \tau\si(E),\]
as the Galois action factors through $G(K\abe/K)$ and hence is abelian (Theorem~\ref{thm:map-G-cl}) (alternatively, because $\si\tau\si^{-1}\tau^{-1}$ fixes $j(E)$). As $E$ is defined over $H_K$, we actually have equality.

Since $h$ is invariant under automorphism, for any $P\in E\tors$,
\[
h(P)=h(\si\tau\si^{-1}\tau^{-1} P)=\si\tau\si^{-1}\tau^{-1} h(P).
\]
(We know $h$ is defined over $H_K$ and $\si\tau\si^{-1}\tau^{-1}$ fixes $H_K=K(j(E))$.) Hence $\si\tau\si^{-1}\tau^{-1}$ fixes $h(E\tors)$ as well, and $\si\tau\si^{-1}\tau^{-1}=1$.
%As the map $G(K(j(E),h(E\tors))/K)\to G(K(j(E))/K)\times G(K(h(E\tors))/K)$ is injective, we get $\si\tau\si^{-1}\tau^{-1}=1$, as needed.
\end{proof}
\begin{thm}\llabel{thm:max-abe-ext-K}
Suppose $K$ is a quadratic imaginary field and $E$ has CM by $\sO_K$.
\begin{enumerate}
\item
For an integral ideal $\ma$ of $\sO_K$, $L_{\ma}:=H_K(h(E[\ma]))=K(j(E),h(E[\ma]))$ is the ray class field of $K$ modulo $\ma$. 
\item
The maximal abelian extension of $K$ is
\[
K(j(E),h(E\tors)).
\]
\end{enumerate}
\end{thm}
\begin{proof}
\step{1}
We need the following lemma.
\begin{lem}\llabel{lem:comm-in-image}
Suppose $E$ is an elliptic curve defined over $L$ with CM by $\sO_K$, and has good reduction at $\mP$. Let $\wt{E}$ be the reduction modulo $\mP$. Let $\theta:\End(E)\to \End(\wt E)$ be the reduction map on endomorphisms. Then for any $\ga\in \End(\wt{E})$,
\[
\ga\in \im(\te)\iff \ga\text{ commutes with every element in }\im(\te).
\]
\end{lem}
\begin{proof}
Since $E$ has good reduction, the map $\End(E)\hra \End(\wt E)$ in injective. Consider 2 cases.
\begin{enumerate}
\item
$\End(\wt E)$ is a quadratic order. Then $\End(E)=\End(\wt E)$ (as $\End(E)$ is a maximal order) so this case is clear.
\item
$\End(\wt E)$ is an order in a quaternion algebra. Then $\End(E)\ot \Q$ is its own centralizer in the quaternion algebra $\End(\wt E)\ot \Q$, by the Double Centralizer Theorem~\ref{galois-cohomology-ch}.\ref{thm:dct-gen}.
%following general theorem from noncommutative algebra.
%\begin{thm}[Double centralizer theorem]
%Let $B$ be a simple $k$-subalgebra of a central simple $k$-algebra $A$. Then the centralizer $C=C(B)$ is simple, 
%\[[B:k][C:k]=[A:k],\]
%and $C(C(B))=C$
%\end{thm}
%\begin{proof}
%See Milne~\cite{Mi08}, Theorem IV.3.1 (pg. 129).
%\end{proof}
\end{enumerate}
\end{proof}

\step{2} We show that in general, we can lift the Frobenius map.
\begin{pr}\llabel{pr:ec-lift-frob}
Suppose $E$ has CM by $\sO_K$ and is defined over $H_K$. Let $\mP\mid \mfp\mid p$ in $H_K$, $K$, $\Q$, respectively, with $\mfp$ having degree 1 and $p\nin S$, $S$ being defined as in the proof of Theorem~\ref{thm:j-generates-hilbert}. Then the $p$th power Frobenius map can be lifted to a map on $E$, i.e. there is $\la$ making the following commute:
\[
\xymatrix{
E\ar[r]^-{\la} \ar[d] &E^{(\mfp,H_K/K)}\ar[d]\\
\wt E\ar[r]^{\wt{\la}=\phi_p} & \wt E^{(p)}.
}
\]
Moreover, if $E$ corresponds to the complex torus $\C/\La$, then up to isomorphism, $\la$ corresponds to the map $\C/\La\to \C/\mfp^{-1}\La$. (Recall that $E^{(\mfp,H_K/K)}\cong \mfp E$ by Theorem~\ref{thm:j-generates-hilbert}.)
\end{pr}
\begin{proof}
We need to show $\phi_p$ is the reduction of some map; we do this by first reducing the problem to showing a certain endomorphism is in the image of $\te$ and then showing the conditions of the previous lemma hold.

Again we use~(\ref{pE-factors-frob}): $\wt{\phi_1}:\wt E\to \wt{\mfp E}$ is ``like" the Frobenius map. We know $\wt{\phi_1}$ is the reduction of a map, namely the map $\phi_1:E\to \mfp E$. Now note $\wt{\mfp E}\cong \wt{E^{(\mfp,L/K)}}=\wt{E}^{(p)}$, the first from Thm~\ref{thm:j-generates-hilbert} and the second from definition of the Frobenius element.

Let $\si=(\mfp,L/K)$. It remains to show that $\ep:\wt{E^{\si}}\to \wt{\mfp E}\cong \wt{E^{\si}}$ is the reduction of a map $\ep'$, because then ${\ep'}^{-1}\circ \phi_1$ will be the desired map. Let $\wt{[\al]}\in \Aut(\wt{E^{\si}})$ be the reduction of a map $[\al]$. 
To show $\ep$ commutes with $[\al]$, we consider $\wt{\phi_1}=\ep\circ \phi_p$, and consider how $[\al]$ ``commutes" with $\wt{\phi_1}$ and $\phi_p$.
\begin{enumerate}
\item
$\wt{\phi_1}$: By normalization (Proposition~\ref{pr:normalize-cmec}(3)), we know
\[
\phi_1\circ [\al]_E=[\al]_{E^{\si}}\circ \phi_1.
\]
\item
$\phi_p$: Note that for any morphism of varieties $f:V\to W$ over a field of characteristic $p$, the following commutes, where $\phi_V,\phi_W$ are the $p$th power Frobenius maps on $V$ and $W$:
\[
\xymatrix{
V\ar[r]^f \ar[d]^{\phi_V}& W\ar[d]\ar[d]^{\phi_W}\\
V^{(p)}\ar[r]^{f^{\si}} & W^{(p)}
}\quad \phi_W\circ f=f^{\si}\circ \phi_V.
\]
Applying this to $[\al]_E$,
\[
\phi_p\circ \wt{[\al]_E}=\wt{[\al]_E^{\si}}\circ \phi_p=\wt{[\al]_{E^{\si}}}\circ \phi_p,
\]
where in the last step we used Theorem~\ref{pr:normalize-cmec}(4), noting $\si(\al)=\al$ since $\al \in K$ and $\si\in G(H_K/K)$.
\end{enumerate}
Hence
\[
\wt{[\al]_{E^{\si}}}\circ \underbrace{\ep\circ \phi_p}_{\phi_1}\stackrel{1}=
\ep\circ \phi_p\circ \wt{[\al]_E}
\stackrel{2}=
\ep\circ \wt{[\al]_{E^{\si}}}\circ \phi_p.
\]
Cancelling $\phi_p$ gives $\wt{[\al]_{E^{\si}}}\circ \ep=\ep\circ \wt{[\al]_{E^{\si}}}$, so Lemma~\ref{lem:comm-in-image} shows $\ep$ is the reduction of some $\ep'$, as needed.

To finish, note that $\phi_1$ does indeed correspond to $\C/\La\to \C/\mfp^{-1}\La$. Hence $\la$ corresponds to $\C/\La\to \C/\mfp^{-1}\La$, up to some automorphism.
\end{proof}

\step{3} When $(\mfp,H_K/K)=1$, $\la$ is just an endomorphism of $E$, hence equals $[\al]$ for some $\al$. In fact, the following proposition shows it is $[\pi]$ for some $\pi$ generating $\mfp$, so that multiplication by $\pi$ corresponds to the $p$th power Frobenius in the reduction.
\begin{pr}\llabel{pr:pi-is-frob}
Suppose $E$ has CM by $\sO_K$ and is defined over $H_K$. For all but finitely many degree 1 prime ideals $\mfp$ with $(\mfp,H_K/K)=1$ (equivalently, such that $\mfp$ is principal), there exists a unique $\pi$ such that $\mfp=(\pi)$ and the following commutes.
\[
\xymatrix{
E\ar[r]^{[\pi]} \ar[d] &E \ar[d]\\
\wt E\ar[r]^{\phi_p} & \wt E.
}
\]
\end{pr}
\begin{proof}
Since $(\mfp,H_K/K)=1$, Proposition~\ref{pr:ec-lift-frob} gives a diagram
\[
\xymatrix{
E\ar[r]^{\la} \ar[d] &E \ar[d]\\
\wt E\ar[r]^{\phi_p} & \wt E.
}
\]
for some $\la$. We know $\la$ is in the form $[\pi]$, and show $\pi$ satisfies the desired conditions. We have by Proposition~\ref{pr:E[a]} that
\[
\nm_{K/\Q}(\pi)=\deg([\pi])=\deg(\phi)=p=\fN\mfp
\]
so either $(\pi)=\mfp$ or $(\pi)=\ol{\mfp}$. As always, when we're deciding between conjugates, normalization comes to the rescue. Take $\om\in \Om_E$ whose reduction modulo $\mP$ is nonzero. Normalization says that $[\pi]^*\om=\pi \om$ so
\[
\wt{\pi}\wt{\om}=\wt{[\pi]}^*\wt{\om}=\phi_p^*\wt{\om}=0,
\]
the last step since the Frobenius map is inseparable. We get $\mP\mid \pi$, forcing $(\pi)=\mfp$.

For uniqueness, note the map
\[
\xymatrix{
\sO_K\ar[r]^-{[\cdot]}_-{\cong}& \End(E)\ar[r]^{\wt{E}} &\End(\wt E)
}
\]
is injective for $E$ having good reduction at $\mP$ (Theorem~\ref{thm:ec-hom-red}).
\end{proof}

\step{4} Consider~(\ref{eq:rcf-k}). We need to show that $P_K(1,\ma)$ is exactly the kernel of the Artin map $\psi_{L_{\ma}/K}$. 
Note that $P_K(1,\ma)$ and $\ker(\psi_{L_{\ma}/K})$ are both subgroups of $P_K^{\ma}=\ker(\psi_{H_K/K})=\ker(\psi_{L_{\ma}/K}(\bullet)|_{H_K})$. It suffices to show that for all but finitely many primes $\mfp$ of degree 1 such that $(\mfp,H_K/K)=1$, we have $\mfp\in P_K(1,\ma)$ iff $\mfp\in \ker(\psi_{L_{\ma}/K})$.

Let $\mfp$ satisfy the conditions of Proposition~\ref{pr:pi-is-frob}. Since the reduction of $\psi_{L/K}(\mfp)$ is the Frobenius map, we get that $\psi_{L/K}(\mfp)=[\pi]$, for some $\pi$ such that $(\pi)=\mfp$.\footnote{Note the analogy with the cyclotomic case. $\psi_{L/K}(\mfp)$ acts on torsion points as $[\pi]$, just as in the cyclotomic case it acted as the $p$th power map, that corresponds to $[p]$ if we consider the natural map $\Z\to \End(\Q(\ze_n))$.} Since $(\mfp,H_K/K)=1$, we have the commutative diagram
\beq{eq:pi-is-frob}
\xymatrix{
E\ar[r]^{\psi_{L/K}(\mfp)=[\pi]} \ar[d] &E \ar[d]\\
\wt E\ar[r]^{\phi_p} & \wt E.
}
\eeq

We have the following string of equivalences, for all but finitely many degree 1 primes $\mfp$ with $(\mfp,H_K/K)=1$,
\begin{enumerate}
\item
$\mfp\in P_K(1,\ma)$.
\item
$\mfp=(\pi)$ where $\pi=u\al$ where $u$ is a unit and $\al\equiv 1\pmod{\ma}$.
\item For all $\ma$-torsion points $P\in E[\ma]$, 
$h([\pi]P)=h(P)$.
\item[$3'$.] For all $\ma$-torsion points $P\in \wt E[\ma]$, 
$\wt h(\wt{[\pi]}\wt P)=\wt h(\wt P)$.
\item
$(\mfp,L_{\ma}/K)$ fixes $h(E[\ma])$.
\item
$\mfp\in \ker(\psi_{L_{\ma}/K})$.
\end{enumerate}
(1)$\iff$(2) is clear.

For $(2)\implies (3)$, note that for all $\ma$ torsion points $P\in E[\ma]$, %since $h$ is $\Aut(E)$-invariant and $[u]$ is an automorphism,
\begin{align*}
%h(P)^{(\mfp,L/K)}&= h(P^{(\mfp,L/K)})&\text{$(\mfp,L/K)$ fixes $H_K$ and $E$ defined over $H_K$}\\
%&=
h([\pi]P)
&=h([u][\al]P)\\
&=h([\al]P)&\text{$h$ is $\Aut(E)$-invariant}\\
&=h(P)&\al\equiv 1\pmod{\ma}\text{ and }P\in E[\ma].
\end{align*}
Note it is important that $h$ be $\Aut(E)$-invariant.

For $(3')\implies (2)$, let $P\in E[\ma]$ be a torsion point. By~\cite[VII.3.1b]{Si86}, $E[\ma]\hra \wt{E}[\ma]$ is injective for $\mfp\nmid \ma$ and $E$ with good reduction at $\mfp$. Since $h$ is an isomorphism (in particular, an injection) $E/\Aut(E)\to \Pj^1$, we get that $[\pi]P= [u]P$ for some $[u]\in \Aut(E)$. 
But $E[\ma]\cong \sO_K/\ma$, so we can choose $u$ such that $\pi\equiv u\pmod{\ma}$. Then there exists $\al$ such that $\pi=u\al$, with $\al\equiv 1\pmod{\ma}$.

For $(3)\implies (4)$, we calculate the action of $(\mfp,L/K)$ on a torsion point $P\in E[\ma]$, in the reduced curve:
\[
\wt{P^{(\mfp,L/K)}}=\phi_p(\wt P)=\wt{[\pi]P},
\]
the second equality from Proposition~\ref{pr:pi-is-frob}. This allows us to understand the action on the nonreduced curve, since $E[\ma]\hra\wt E[\ma]$ is injective for $\mfp\nmid \ma$ and $\mfp$ of good reduction. We get
\[
P^{(\mfp,L/K)}=[\pi]P.
\]
Thus (3) implies 
\begin{align*}
h(P)^{(\mfp,L/K)}&= h(P^{(\mfp,L/K)})&\text{$(\mfp,L/K)$ fixes $H_K$ and $E$ defined over $H_K$}\\
&=h([\pi]T)\\
&=h(T)&\text{by (3)}.
\end{align*}

Now we prove $(4)\implies(3')$. Let $\si\in G(\ol K/K)$ be an automorphism such that $\si|_{K\abe}=(\mfp,K\abe/K)$. Then for any $P\in E[\ma]$,
\[
\wt{h}(\wt{[\pi]}\wt P)\stackrel{\eqref{eq:pi-is-frob}}{=}\wt{h}(\phi(\wt{P}))=\wt{h(P^{\si})}=\wt{h(P)}^{\si}=\wt h(\wt P),
\]
the last two equalities since $\si|_H=1$, $h$ is defined over $H$, and $\si|_{L_{\ma}}$ fixes $h(E[\ma])$ by assumption. Thus $(3')$ holds. %\fixme{Reduced vs. nonreduced} 

Now (4)$\iff$(5) comes from the fact that $(\mfp,L_{\ma},K)$ already fixes $K(j(E))$, so to fix $L_{\ma}$ it only needs to fix $h(E[\ma])$.\\
%
%%We know $\mfp\in P_K(1,\ma)\subeq P_K$ so $(\mfp,H_K/K)=(\mfp,L_n/K)|_{H_K}=1$. 
%It remains show that $(\mfp,L_n/K)$ fixes $h(E\tors)$; then it will fix $H_K(h(E\tors))$, as needed.
%
%Let $\mfp=(\al)\in P_K(1,\ma)$. We want to get information about $\psi_{L/K}(\mfp)$ from knowing that its reduction is the Frobenius map. This is~(\ref{pr:pi-is-frob})! \fixme{$\psi_{L/K}(\mfp)$ acts on torsion points as $[\pi]$, just as in the cyclotomic case it acted as the $p$th power map, that corresponds to $(p)$ if we consider the natural map $\Z\to \End(\Q(\ze_n))$.} We get, for all but finitely many $\mfp$,
%\beq{eq:pi-is-frob}
%\xymatrix{
%E\ar[r]^{[\pi]} \ar[d] &E \ar[d]\\
%\wt E\ar[r]^{\phi_p} & \wt E.
%}
%\eeq
%where $(\pi)=(\al)$, i.e. $\pi=u\al$ for some unit $u$. 
%
%We calculate the action of $(\mfp,L/K)$ on a torsion point $P\in E[\ma]$, in the reduced curve:
%\[
%\wt{P^{(\mfp,L/K)}}=\phi_p(\wt P)=\wt{[\pi]P},
%\]
%the last from Proposition~\ref{pr:pi-is-frob}. This allows us to understand the action on the nonreduced curve, since $E[\ma]\hra\wt E[\ma]$ is injective (conditions?). We get
%\[
%P^{(\mfp,L/K)}=[\pi]P
%\]
%We get that
%\begin{align*}
%h(P)^{(\mfp,L/K)}&= h(P^{(\mfp,L/K)})&\text{$(\mfp,L/K)$ fixes $H_K$ and $E$ defined over $H_K$}\\
%&=h([\pi]T)\\
%&=h([u][\al]T)&\text{$h$ is $\Aut(E)$-invariant}\\
%&=h(T)&\al\equiv 1\pmod{\ma}\text{ and }P\in E[\ma].
%\end{align*}
%Note it is important that $h$ be $\Aut(E)$-invariant.\\
%\step{6}
%We show the other direction, $P_K(1,\ma)\supeq \ker(\psi_{L_n/K})$. This is basically the backwards argument. Take $\mfp\in \ker(\psi_{L_n/K})$ of degree 1...; we get $(\ma,H_K/K)=(\ma,L_n/K)|_{H_K}=1$. 
%As in~(\ref{eq:pi-is-frob}), we get $\pi$ such that $\wt{[\pi]}=\phi_p$.
%
%Pretty darn straightforward; See Silv. p.137.\\

\step{7}
The maximal abelian extension is the union of the all ray class fields. Note every $\mc$ divides $n$ for some $n$ so we can just restrict to ray class fields corresponding to $(n)$ for some $n\in \N$:
\[
K\abe=\bigcup_{n}K(j(E),h(E[n]))=K(j(E),h(E\tors)).
\]
\end{proof}
%\section{$j(E)$ is an algebraic integer}
%Omitted for now.
\section{The Main Theorem of Complex Multiplication}
\index{main theorem of complex multiplication}
Given $\si\in \Aut(\C/K)$, consider the map $\si: E(\C)\to E^{\si}(\C)$. We would like to know how this map acts on torsion points. This is since to get Galois representations of elliptic curves, we look at how $\si$ acts on torsion points---often specializing to torsion points that are a power of a prime.

Because we are considering CM elliptic curves, we can identify the torsion points with $K/\ma$, for some ideal $\ma$. Namely, given an analytic isomorphism $f:\C/\ma\xra{\cong} E(\C)$, we can restrict it to $K/\ma$ to get
\[
f|_{K/\ma}:K/\ma\xrc E\tors \hra E(\C).
\]

The main theorem of complex multiplication tells us we can transfer the map $\si:E(\C)\to E^{\si}(\C)$ via an {\it analytic isomorphism} to a multiplication-by-an-idele map $[\mathbf x^{-1}]:K/\ma\to K/\mathbf x^{-1}\ma$, where $\mathbf x$ and $\si$ are related in terms of the Artin map (to be made precise).

%In some sense, we have been building up to this theorem: 
%%The map $E\to E^{(\mfp,
%From Theorem~\ref{j-generates-hilbert}, we know that $(\mfp,L/K)$ sends an elliptic curve associated with to lattice $\La$, to a an elliptic curve associated to the lattice $\mfp^{-1}\La$; we can think of this as the case where $\si=(\mfp,L/K)$ and $\mathbf x$ corresponds to the ideal $\mfp$. 
%Thinking of $K\ma\cong E\tors$, we have a commutative diagram
%\[
%\xymatrix{
%K/\ma\ar[r]^i\ar[d]^{\Phi} & K/\mfp^{-1}\ma\\
%E\ar[r]^{(\mfp,L/K)} &E^{\si}
%}
%\]
%where the top map is just the inclusion.
%
%More generally, we want to consider the map $K/\ma\to K/\mathbf a^{-1}\ma$ for 
%
%of ideles on lattices, not just the action of ideals.
\begin{df}
Let $\mathbf x=\prod_{\mfp\in V_K^0} \mfp^{m(\mfp)}\prod_{v\in V_K^{\iy}} v^{m(v)}\in \I_K$ be an idele.
Let $\ma$ be an ideal, and define $\mathbf x\ma$ by
\[
\mathbf x\ma=p(\mathbf x)\ma = \pa{\prod_{\mfp\in V_K} \mfp^{m(\mfp)}}\ma.
\]
Define the map
\beq{eq:mult-idele-map}
[\mathbf x]:K/\ma\to K/\mathbf x\ma
\eeq
as follows. Note $K/\ma\cong \prod_{\mfp} K_{\mfp}/\ma K_{\mfp}$ by the Chinese Remainder Theorem, %or an easy generalization of
where $x$ is just identified with its images in the $K_{\mfp}/\ma K_{\mfp}$: $(x_{\mfp})_{\mfp\in V_K^0}$. Then~(\ref{eq:mult-idele-map}) sends
\beq{eq:mult-by-idele}
(a_{\mfp})\mapsto (x_{\mfp}a_{\mfp}) \text{ where }\mathbf x=(x_{\mfp}).
\eeq
\end{df}
\begin{thm}[Main Theorem of Complex Multiplication]\llabel{thm:mt-cm}
Suppose $E$ is an elliptic curve with CM by $\sO_K$. Let $\si\in \Aut(\C/K)$ and $\mathbf x\in \I_K$ be such that
\[
\si|_{K\abe}=\phi_{K}(\mathbf x).
\]
Fix an analytic isomorphism $f:\C/\ma\xra{\cong} E(\C)$. Then there exists a unique analytic isomorphism $f':K/\mx^{-1}\ma\to E^{\si}(\C)$ such that the following commutes:
\[
\xymatrix{
K/\ma \ar[r]^{\mx^{-1}}\ar[d]^f & K/\mx^{-1}\ma\ar@{->}[d]^{f'}\\
E(\C)\ar[r]^{\si} & E^{\si}(\C).
}
\]
\end{thm}
\begin{rem}
The map~(\ref{eq:mult-by-idele}) can be a bit weird to think about: For instance, consider the simpler case $K=\Q$, $\ma=\Z$. Take the idele $\mx$ with 1's everywhere except $x_5=2$. Then $[\mx]$ sends $\rc 2\mapsto \rc 2,\rc3\to \rc 3, \rc 7\to \rc 7$ and so forth but sends $\rc 5\to \fc 25$. So it is surprising that $\mx^{-1}:K/\ma\to K/\mx^{-1}\ma$ can be related analytically to $E(\C)\to E^{\si}(\C)$. %can be related to an map of varieties over $\C$
\end{rem}
Compare this theorem to Proposition~\ref{pr:pi-is-frob}. Rather tan just dealing with the Frobenius element of a prime, we deal with the Artin map of an idele.
\begin{proof}
Note uniqueness follows from the fact that topologically, the closure of $K/\mx^{-1}\ma$ is $\C/\mx^{-1}\ma$, and any continuous function is determined by its values on a dense set.\\

First we prove this for $E$ defined over $\Q(j(E))$ and $\ma$ integral. We do this in 2 steps.
\step{1} Approximate $\si$ by a field automorphism $\la$ that is the Frobenius element of a prime $\mfp$. (The Frobenius element is something much more concrete to work with than the abstract Artin map of an idele.) We will take better and better approximations, which determine the action on $E[m]$ for larger and larger $m$, and take an inverse limit.

So let $L'_m$ be the Galois closure of $K(j(E),E[m])/K$. By Corollary~\ref{ch:cht-app}.\ref{chebotarev-resfield1}, there are infinitely many primes with $\mP\mid \mfp$ in $K$ and $L$ such that
\[
(\mP,L/K)=\si|_{L'_m},\quad \fN(\mfp)=1.
\]
%Each $\mfp$ is divisible by $|\an{\si|_{L'_m}}|$ primes, ranging over exactly the elements of the conjugacy class (Lemma~\ref{intro-cft}.\ref{lem:frob-lem}), so we can take $\mP\mid \mfp$ with $(\mP,L'_m/K)=\si|_{L'_m}$.  
We can furthermore choose $\mfp$ satisfying the following, because each condition excludes only finitely many primes.
\begin{enumerate}
\item
$\mfp$ is unramified in $L'_m$.
\item
$\mfp\nin S$, where $S$ is defined as in the proof of Theorem~\ref{thm:j-generates-hilbert}.
\item
$\mfp\nmid m$.
\end{enumerate}
By Proposition~\ref{pr:ec-lift-frob}, there exists a map $\la:E\to E^{\si}$ that reduces to $\phi_p$ modulo $\mP$. On $\wt{E}[m]$, both $\la$ and $\si$ act as $\phi_p$. 
Because $\mP\nmid m$ by item 3, the reduction map modulo $\mP$,  $E[m]\to \wt E[m]$, is injective. Hence $\la$ and $\si$ act the same on $E[m]$:
%, and the following commutes: 
%Now $\la|_{L_m}=(\mP,L_m/K)=\si|_{L_m}$ acts as the $p$th power Frobenius map on $\wt E[m]$, so $\la
%\[
%\xymatrix{
%E[m]\ar[r]^{\la}\ar@{^(->}[d] & E^{\si}[m]\ar@{^(->}[d]\\
%E(\C)\ar[r]^{\si} E^{\si}(\C).
%}
%\]
\beq{eq:la=si}
\la|_{E[m]}=\si|_{E[m]}:E[m]\to E^{\si}[m].
\eeq
But we know how the map $\la$ acts: 
Proposition~\ref{pr:ec-lift-frob} tells us that 
%by Theorem~\ref{thm:j-generates-hilbert}, $[E^{\si}]=[\mfp]*[E]$, and 
the map $\la: E\to E^{\si}$ corresponds to the map on complex tori $i:\C/\ma\to \C/\mfp^{-1}\ma$.\footnote{The map $\si$ and $\mx^{-1}$ appearing in the theorem statement are bijections, while $\la$ and $i$ are not. This is okay, though, because we only use $\la,i$ to approximate $\si$ on $m$-torsion, and $\la,i$ are injective on $m$-torsion, since $\mP\nmid m$.} Hence we have the commutative diagram
\beq{eq:mt-cm-1}
\xymatrix{
\C/\ma \ar[r]^{i}\ar[d]^{f} & \C/\mfp^{-1}\ma\ar[d]^{f''}\\
E(\C)\ar[r]^{\la} &E^{\si}(\C)
}
\eeq
for some analytic isomorphism $f''$.

%But we're given a map $\C/\ma\to \C/\mx^{-1}\ma$. Note we can relate $\mx$ and $\mfp$ as follows. Let $\pi$ be the uniformizer of $K_{\mfp}$, let $i_{\mfp}:K_{\mfp}\to \I_K$ be the inclusion map.
\step{2}
%Note that
%\[
%\C/\mfp^{-1}\ma \cong E^{\si}(\C)\cong \C/\mx^{-1}\ma,
%\]
%already tells $\mfp^{-1}\ma$ and $\mx^{-1}\ma$ must be homothetic ideals, in other words, $(\mx)=(\al)\mfp$ for some principal ideal $(\al)$. We get an even stronger condition, though, from the fact that
By Theorem~\ref{thm:max-abe-ext-K}, the ray class group modulo $m$ is $K_m=K(j(E),h(E[m]))$. Note $K_m\subeq L_m'$. 
Now by assumption, $\mfp$ was chosen so that the images of $\mfp$ and $\mx$ under the Artin map both project to $\si|_{K_m}$:
\[
%\psi_{L'_m/K}((\mx))=
\phi_{K_m/K}(\mx)=\si|_{K_m} =\psi_{K_m/K}(\mfp)=\phi_{K_m/K}(i_{\mfp}(\pi))
\]
where $\psi$, $\phi$ denote the Artin map on ideals and on ideles, respectively, and $\pi$ is the uniformizer of $\mfp$ in $K_{\mfp}$. We have
\[
\ker\psi_{K_m/K}=K^{\times}\mathbb U_K(1,m).
%P_K(1,m).
\]
(See Definition~\ref{intro-cft}.\ref{df:more-idele-dfs} for notation.) This follows from the definition of the ray class field and from the correspondence between ray class groups in Definition~\ref{intro-cft}.\ref{df:ray-class-field} and idele class groups in Example~\ref{intro-cft}.\ref{ex:class-group-idele-quotient}. 
We have $\mx\in i_{\mfp}(\pi)\ker \phi_{K_m/K}$, giving
%Hence, since the kernel of $(\bullet):\I_K\to I_K$ is exactly the units $\prod_{v}U_v$, we can write
\[
\mx=\al\cdot  i_{\mfp}(\pi)\cdot  \mathbf u,\quad \al\in K^{\times},\quad \mathbf u\in \mathbb U_K(1,m).
\]

%where $\al\in K^{\times}$ and $u\in \I_K(1,m)$ is a unit at every component.
% $i_{\mfp}(\pi)$ denote the element of $\I_K$ with $\pi$ at 

We now compose~(\ref{eq:mt-cm-1}) with the homothety $\al^{-1}$, and note $(\mx)=(\al)\mfp$, to get the desired map $\C/\mx^{-1}\ma\to E^{\si}(\C)$:
\beq{eq:mt-cm-2}
\xymatrix{
\C/\ma \ar[r]^i\ar[d]^f & \C/\mfp^{-1}\ma\ar[r]^{\al^{-1}} \ar[d]^{f''} & \C/\mx^{-1}\ma  \ar[ld]^{f'_m}\\
E(\C) \ar[r]^{\la} & E^{\si}(\C) & 
}
\eeq
Here, $f'_m(z):=f''(\al z)$.

This isn't quite what we want yet, though, because the top row is the map $\al^{-1}$ rather than the map $\mx^{-1}$. We need to show that for $m$-torsion points, $\al^{-1}$ acts the same as $\mx^{-1}$. Then we would have
\[
\si(f(t))=\la(f(t))=f'_m(\al^{-1}t)=f'_m(\mx^{-1}t),\quad t\in m^{-1}\ma/\ma.
\]
The first equality is since $\si$, $\la$ were by construction the same on $E[m]$ (\ref{eq:la=si}), so $\si\circ f$ and $\la\circ f$ are the same on $m^{-1}\ma/\ma$. The second is by commutativity of~(\ref{eq:mt-cm-2}).

To show the third equality, we note that
\begin{align*}
&f'_m(\al^{-1}t)=f'_m(\mx^{-1}t)&\text{for all } t\in m^{-1}\ma/\ma\\
(f'_m\text{ bijective})\quad \iff& \al^{-1}t-\mx^{-1}t\in \ma &\text{for all } t\in m^{-1}\ma\\
\iff&\al^{-1}t_{\mq}-x_q^{-1}t_{\mq} \in \ma_{\mq}&\text{for all } t\in m^{-1}\ma,\,\mq\\
\text{(multiplying by $x_{\mq}=\al [i_{\mfp}(\pi)]_{\mq} u_{\mq}$)} \quad \iff 
& [i_{\mfp}(\pi)]_{\mq} u_{\mq}t-t\in \ma_{\mq} &\text{for all }t\in m^{-1}\ma_{\mq}\\
\iff& ([i_{\mfp}(\pi)]_{\mq} u_{\mq}-1)\ma_{\mq}\subeq m\ma_{\mq}\\
u_{\mq}\in \mathbb U_K(1,m)  \quad \iff &
([i_{\mfp}(\pi)]_{\mq}-1)\ma_{\mq}\subeq m\ma_{\mq}.
\end{align*}
Consider 2 cases.
\begin{enumerate}
\item $\mq\ne \mfp$. In this case, $[i_{\mfp}(\pi)]_{\mq}=1$, so this is trivial.
\item $\mq= \mfp$: $[i_{\mfp}(\pi)]_{\mfp}=\pi$, and $\pi-1$ is a unit. By assumption $\mfp\nmid m$. hence $(\pi-1)\ma=\ma=m\ma$.
\end{enumerate}

\step{3} 
We now show that the maps $f'_m$ are all actually the same for $m\ge 3$. Indeed, $f'_m|_{E[m]}=f'_{mn}|_{E[m]}$ by construction, so $f'_m,f'_{mn}$ differ by an automorphism that fixes $E[m]$. This automorphism must be $[\ze]$ for some element of norm 1 in $K$, and $f'_m=[\ze]\circ f'_{mn}$. Since $f'_m,f'_{mn}$ are isomorphisms, this says
\[
E[m]\subeq \ker [1-\ze]
\]
The only possibilities are $\ze$ a 4th or 6th root of unity, and if $\ze\ne 1$, then $[1-\ze]$ has norm at most 4. So for $m\ge 3$, $\ze=1$, and $f'_m=f'_{mn}$.\\

\step{4}
Finally, we show the theorem holds for general $E/L$. Any elliptic curve $E$ has a model $E'$ defined over $M'=\Q(j(E))$, corresponding to a complex torus $\C/\ma'$ with $\ma'$ an integral ideal (see the left face below). Let $E\to E'$ be an isomorphism and $K/\ma\to K/\ma'$ be the corresponding map on torsion. Then the existence of $f'_{E'}$ for $E'/L$ gives the existence of $f'_{E}$ for $E/L$, by choosing $f'_{E}$ to make the below diagram commute.
\[
\xymatrix{
%line 1
K/\ma\ar[rr]^{\mx^{-1}}\ar[rd]^{\cong}
\ar[dd]_{f_E} & & K/\mx^{-1}\ma \ar@{.>}[dd]^{f_E'}\ar[rd]^{\cong} & \\
%line 2
& K/\ma' \ar[rr]^{\mx^{-1}} \ar[dd]^{f_{E'}} & & K/\mx^{-1}\ma'\ar[dd]^{f'_{E'}}\\
%line 3
E(\C) \ar[rr]^{\si} \ar[rd]& & 
E^{\si}(\C)\ar[rd]^{\cong} &\\
%line 4
& E'(\C)\ar[rr] & & E'^{\si}(\C).
}
\]
%We now show the maps $f'_m|_{L_m}:m^{-1}\ma/\ma\to m^{-1}\mx^{-1}\ma/\ma$ fit together compatibly, so that we can define $f':K/\ma\to K/\mx^{-1}\ma$ by
%\[
%f':=\varprojlim_m f'_m.
%\]
%By construction, $f'_m$ and $f'_{nm}$ act the same on 
\end{proof}

\subsection{The associated Gr\"ossencharacter}
\index{Gr\"ossencharacter of elliptic curve}
The Main Theorem involved 2 different elliptic curves, and 2 different analytic isomorphisms.
%analytic isomorphisms. 
In the special case that $\si$ fixes $E$, the curves will be the same, and by nudging the map upstairs by a constant depending on $\mathbf x$, we can restate the theorem using a consistent choice of $f$. (Compare to how we specialized from Proposition~\ref{pr:ec-lift-frob} to~\ref{pr:pi-is-frob}.) 
%We restate it in a way that involves a consistent choice of $f$ throughout. Then 
The action of $\phi_L(\mathbf x)$ on the elliptic curve will ``essentially" correspond to multiplication by $\chi_{E/L}$ on $K/\ma$. 
%The frobenius endomorphism should be a Hecke character
\begin{thm}[Gr\"ossencharacter of an elliptic curve]\llabel{thm:grossen-ec}
Let $E/L$ be an elliptic curve with complex multiplication by $\sO_K$, and suppose $K\subeq L$. Let $\mathbf x\in \I_L$ and $\mathbf y=\nm_{L/K}(\mathbf x)\in \I_K$. Then there exists a unique $\al=\al_{E/L}(x)\in K^{\times}$ with the following properties.
\begin{enumerate}
\item
$\al\sO_K=(\mathbf y)$. %, where $p:\I_K\to I_K$ sends an idele to its corresponding ideal (see~(\ref{eq:idele-to-ideal})).
\item
For any fractional ideal $\ma\subeq K$ and any analytic isomorphism $f:\C/\ma\to E(\C)$, the following commutes.
\[
\xymatrix{
K/\ma \ar[r]^{\al \mathbf y^{-1}}\ar[d]^f & K/\ma \ar[d]^f\\
E(L\abe)\ar[r]^{\phi_L(\mathbf x)} & E(L\abe).
}
\]
\end{enumerate}

Moreover, defining $\chi_{E/L}:\I_L\to \C^{\times}$ by
\[
\chi_{E/L}(\mathbf x):=\al_{E/L}(\mathbf x)[\nm_{L/K}(\mx^{-1})]_{\iy},
\]
$\chi_{E/L}$ is a Gr\"ossencharacter of $K$, and $\chi_{E/L}$ is ramified at $\mP$ (i.e. $\chi_{E/L}(U_{\mP})$ is not identically 1) iff $E$ has bad reduction at $\mP$.
\end{thm}
%\begin{rem}
%Note that the image of $f$ is just the torsion points of the elliptic curve:
%\[
%f:K/\ma 
%\]
%
%We can think of this theorem as telling how to transfer the action of the Frobenius 
%\end{rem}
\begin{proof}
\noindent\underline{Part 1:} 
Since $f$ is an isomorphism, uniqueness is clear. To construct $\al$, choose any $\si\in \Aut(\C/L)$ such that $\si|_{L\abe}=\phi_L(\mx)$. We use Theorem~\ref{thm:mt-cm} with $\si$ and $\mathbf y\in \I_K$, noting the following points.
\begin{enumerate}
\item
$E^{\si}=E$ since $E$ is defined over $L$ and $\si$ fixes $L$.
\item
The image of $f$ is contained in $E(L\abe)$ as $E\tors\in E(L\abe)$ by Lemma~\ref{lem:K-ab-ext}.
\item
By compatibility of the Artin map,
$\phi_L(\mx)|_{K\abe}=\phi_K(\nm_{L/K}\mathbf x)=\phi_K(\mathbf y)$.
\end{enumerate}
%Note that , and note that  Thus Theorem~\ref{thm:mt-cm} gives
We obtain an analytic map $f'$ making the following commute.
\[
\xymatrix{
K/\ma \ar[r]^{\mathbf y^{-1}}\ar[d]^f & K/\mathbf y^{-1}\ma\ar@{->}[d]^{f'}\\
E(L\abe)\ar[r]^{\phi_L(\mx)} & E(L\abe).
}
\]
%Because $\C/\my^{-1}
Because
\[
\C/\mathbf y^{-1} \ma\cong E^{\si}(\C)\cong E(\C)\cong \C/\ma,
\]
we have that $\mathbf y^{-1}\ma$ is homothetic to $\ma$, i.e. there exists $\be$ so that $\be$ takes $K/\mathbf y^{-1}\ma$ back to $K/\ma$. Defining $f''(x)=f'(\be^{-1}x)$, we have that it differs from $f$ by some automorphism $[\ze]$: $f\circ [\ze]=f''$. Let $\al=\be\ze$. Then we can extend the above diagram as follows.
\[
\xymatrix{
K/\ma \ar[r]^{\mathbf y^{-1}}\ar[d]^f & K/\mathbf y^{-1}\ma\ar@{->}[d]^{f'}\ar[r]^{\al}  & K/\ma \ar[ld]^{f}\\
E(L\abe)\ar[r]^{\phi_L(\mx)} & E(L\abe) &
}
\]
As $\al\mathbf y^{-1}\ma=\ma$, we get $(\al)=(\mathbf y)$.

To see that $\al$ is independent of $f$ and the ideal $\ma$, let $f'$ be another analytic isomorphism $K/\ma'\to E(L\abe)$. Let the map $K/\ma'\to K/\ma$ be multiplication-by-$\ga$. Then $f(\ga x)$ is also an analytic isomorphism $K/\ma'\to E(L\abe)$. Hence $\ga^{-1} f^{-1}\circ f'$ is an automorphism $[\ze]$ of $K/\ma'$, i.e. $f'(x)=f([\ze]\ga x)$. Thus $\phi_L(\mathbf x)[f'(x)]=f'(\al \mathbf y^{-1} x)$ as well.\\
%Showing that the map is independent of $f$ is another exercise in drawing a commutative cube, and left to the reader.%. COMMUTATIVE CUBE!

\noindent\underline{Part 2:} 
$\al_{E/L}$ and hence $\chi_{E/L}$ is a homomorphism since it's clear that $\phi_L(\mathbf x\mathbf x')\circ f=f\circ \al\al'\mathbf y\mathbf y'^{-1}$, and $\phi_L(\mathbf x^{-1})\circ f=f\circ \al^{-1}\mathbf y$.

We need to check that $\chi_{E/L}(L^{\times})=1$ and that $\chi_{E/L}$ factors through a modulus. 

For the first point, note $\phi_L(L^{\times})=1$, the identity element of $G(L\abe/L)$. Let $i:K^{\times}\to \I_K$, $L^{\times}\to \I_L$ be the diagonal maps, and suppose $\mathbf x=i(x)$. We have $\mathbf y=\nm_{L/K}(i(x))=i(\nm_{L/K}(x))$. Then $\al$ is just the element such that $\al \nm_{L/K}(x)^{-1}$ induces the identity map, i.e. $\al=\nm_{L/K}(x)=[\nm_{L/K} \mathbf x]_{\iy}$, so $\chi_{E/L}(\mathbf x)=1$. 
%First, the fact that $\al\nm_{L/K}(\mx)^{-1}\ma=\ma$ gives $\al\nm_{L/K}(\mx^{-1})

For the second point, fix $m\ge 3$ ($m=3$ works fine). We'll show that for any idele $\mathbf x$ in a small enough open subset of finite index, $\phi_{L}(\mathbf x)$ acts just like multiplication by $\al_{E/L}(\mathbf x)$ {\it and} fixes $E[m]$, without the extra $\nm_{L/K}(\mathbf x)_{\iy}$ factor, so that $\al$ will actually be 1.

Let $B_m$ be the kernel of the Artin map $\I_L\to G(L(E[m])/L)$ (abelian by Lemma~\ref{lem:K-ab-ext}), so that it induces an isomorphism
\beq{eq:LEm}
\phi_{L(E[m])/L}:\I_L/B_m\xra{\cong} G(L(E[m])/L).
\eeq
We show that 
\[
U_m:=B_m \cap L^{\times}\pa{\nm_{L/K}^{-1}\mathbb U_K(1,m)
}\subeq \ker \chi_{E/L}.
\]
This is of finite index in $\I_L$ since $B_m$ is open of finite index in $\I_L$ and $K^{\times}\mathbb U_K(1,m)$ is open of finite index in $\I_K$.

Fixing an analytic isomorphism $f:\C/\ma\xra{\cong}E(\C)$, we get that for any $t\in m^{-1}\ma/\ma$ and any $\mathbf x\in U_m$, $f(t)\in E[m]$ so 
%RSince $\phi_L(\mathbf x)$ fixes $L$, we get
\bal
f(t)&=f(t)^{\phi_L(\mathbf x)}&\text{by~(\ref{eq:LEm}) and }\mathbf x\in B_m\\
&=f(\al\nm_{L/K}(\mathbf x)^{-1}t)&\text{by the Main Theorem~\ref{thm:mt-cm}}\\
&=f(\al t)&t\in m^{-1}\ma/\ma\text{ and }\nm_{L/K}(\mathbf x)_{\mfp}\equiv 1\pmod{m\sO_{K_{\mfp}}}\text{ for all }\mfp.
\end{align*}
Thus multiplication by $\al$ fixes $m^{-1}\ma/\ma$, i.e. $\al\equiv 1\pmod{m\sO_K}$. Note $\nm_{L/K}(\mathbf x)^{-1}\in \mathbb U_K(1,m)$, so
\[
(\al)=(\mathbf y)=(\nm_{L/K}(\mathbf x))=\sO_K
\]
and $\al$ is a unit. Together with $\al\equiv 1\pmod{m\sO_K}$, we get $\al=1$.\footnote{Any number in the form $m\tau+1$,  $\tau\in \sO_K$ with norm 1 has norm at least $(\nm_{K/\Q} (m)-1)^2-1$, by the triangle inequality. In order for it to have norm 1, $\tau=0$.}\\

\noindent{\underline{Part 3:}} 
The relationship between ramification and bad reduction hinges on the N\'eron-Ogg-Shafarevich Criterion. See~\cite[pg. 169-170]{Si94}.
%unfinished
%We have the following chain of equivalences.
%\begin{enumerate}
%\item
%$E$ has good reduction at a prime $\mP$.
%\item 
%The inertia group $I_{\mP}(K\abe/K)$ acts trivially on $E[m]$ for infinitely many $m\perp \mP$. ((1)$\iff$(2) is N\'eron-Ogg-Shafarevich.)
%\item For all $x\in U_{\mP}$ and $t\in m^{-1} \ma/\ma$,
%\[
%f(t)^{\phi_L(\mx)}=f(t).
%\]
%Note (2)$\iff$(3) follows from the fact from local class field theory that
%\[
%\phi_K(i_{\mP}(U_{\mP}))=\phi_{K_{\mP}}(U_{\mP})=I_{\mP}\abe
%\]
%\fixme{add REF.}
%Here we think of $I_{\mP}$ inside $G(\ol K/K)$ by the map $I_{\mP}\hra D_{\mP}\hra G(\ol K/K)$. %%%%write out
%\item For all $\mathbf x\in U_{\mP}$ and $t\in m^{-1}\ma/\ma$,
%\[
%f(\al_{E/L}(\mx) \nm_{L/K}(\mx)^{-1}t)=f(t).
%\]
%\item
%$\chi_{E/L}(\mathbf x)\equiv 1\pmod{m\sO_K}$ for all $\mathbf x\in U_{\mP}$. ((4)$\iff$(5)) is because \fixme{blah blah blah}.)
%\end{enumerate}
\end{proof}
Note that if $\chi_{E/L}$ is unramified at $\mP$, then $\chi_{E/L}(i_{\mP}(U_{\mP}))=1$, so it makes sense to talk about $\chi_{E/L}(\mP)$ (defined as $\chi_{E/L}(i_{\mP}(\pi))$ for any uniformizer $\pi$).
\begin{pr}
Let $E/L$ be an elliptic curve with CM by $\sO_K$, with $K\subeq L$. Let $\mP$ be a prime of $L$ of good reduction, let $\wt{E}$ be the reduction of $E$ modulo $\mP$. Let $\phi_{\mP}$ be the Frobenius on $\wt{E}$. 
Then the following commutes.
\[
\xymatrixcolsep{5pc}
\xymatrix{
E\ar[r]^{[\chi_{E/L}(\mP)]}\ar[d]
\ar[d] & E\ar[d]\\
\wt E\ar[r]^{\phi_{\mP}} & \wt E
}
\]
\end{pr}
%Compare to Proposition~\ref{pr:pi-is-frob}.
\begin{proof}
Let $\pi$ be a uniformizer of $L_{\mP}$, and let $\varpi
=i_{\mP}(\pi)$. 
Note that $\varpi_{\iy}=1$. Hence $\nm_{L/K}(\varpi)_{\iy}=1$, giving
\[
\chi_{E/L}(\mP)=\chi_{E/L}(\varpi)=\al_{E/L}(\varpi).
\]
If $m$ is an integer such that $\mP\nmid m$, then $\nm_{L/K}(\varpi)$ fixes $m^{-1}\ma/\ma$ (since it is 1 at all $\mQ$ with  $\mQ\mid m$). Then
\bal
f(t)^{\phi_L(\varpi)}&=f([\al_{E/L}(\varpi)]\nm_{L/K}(\varpi)^{-1}t)&\text{definition of }\al_{E/L}\\
&=f([\chi_{E/L}(\mP)]\nm_{L/K}(\varpi)^{-1}t)\\
&=[\chi_{E/L}(\mP)]f(\nm_{L/K}(\varpi)^{-1}t)&f\text{ preserves the action of }\sO_K\\
&=[\chi_{E/L}(\mP)]f(t)&\nm_{L/K}(\varpi) \text{ fixes } m^{-1}\ma/\ma.
\end{align*}
Modulo $\mP$, $\phi_L(\varpi)$ is just the $q$th power Frobenius map, so we get
\[
\phi_{\mP}|_{\wt E[m]}=\wt{[\chi_{E/L}(\mP)]}|_{E[m]}.
\]
Since an isogeny is determined by its action on $E[m]$ for $m\to \iy$ (the kernel of a nonzero isogeny is finite), we get that this is true for $E$, not just $E[m]$, as needed.
\end{proof}
%$E$ has good reduction at a prime $\mP$, iff the inertia group $I_{\mP}\abe$ acts trivially on $E[m]$ for infinitely many $m\perp \mP$. From local class field theory, we know
%\[
%\phi_K(i_{\mP}(U_{\mP}))=\phi_{K_{\mP}}(U_{\mP})=I_{\mP}\abe
%\]
%where we think of $I_{\mP}$ inside $G(\ol K/K)$ by the map $I_{\mP}\hra D_{\mP}\hra G(\ol K/K)$. Hence, $I_{\mP}\abe$ acts trivially on $E[m]$ iff
%\[
%f(t)^{\phi_L(\mx)}=f(t)\text{ for all }x\in U_{\mP},\, t\in m^{-1} \ma/\ma.
%\]
%This becomes
%\bal
%f(\al_{E/L}(\mx) \nm_{L/K} \mx^{-1}t)=f(t)\text{ for all }\mathbf x\in U_{\mP}, t\in m^{-1}\ma/\ma
%\eal
%which is true iff $\psi_{E/L}(x)\equiv 1\pmod{m\sO_K}$ for all $\mathbf x\in U_{\mP}$.

%$\chi_{E/L}(\mathbb U_K(1,\mm))=1$ (\fixme{change other notation!}). 

To study the Galois representation $G(\ol K/H_K)\to \Aut E\tors$ of $E$, we reduce modulo a prime $\mP$ of $L$, and show that on this reduced curve, the $q$th power Frobenius acts exactly as multiplication by the Gr\"ossencharacter. In particular, the $q$th power Frobenius is represented by multiplication by $\chi_{E/L}(\mP)$ when we think of $E\tors$ as $K/\ma$. Thinking of $E\tors$ as a 2-dimensional space $\Q^2$, this says exactly that the eigenvalues of the Frobenius acting on $E\tors$ is exactly $\chi_{E/L}(\mP)$ and $\ol{\chi}_{E/L}(\mP)$. Typically we just restrict our attention to $\ell$-power torsion points for some $\ell$.
%Thus thinking of this as giving a representation $G(\ol K/K)\to \Aut(E\tors)$, an element $\phi_L(\mP)$ gets sent to 

\section{$L$-series of CM elliptic curve}
\llabel{sec:l-series-cmec}
\subsection{Defining the $L$-function}
We define the $L$-series of an elliptic curve as the $L$-series of the corresponding Galois representation.
\index{L-series of elliptic curve}
\begin{df}%See def of GReps!
\llabel{df:E-L1}
Let $E$ be an elliptic curve defined over $K$, and $\rh_{\ell}$ the associated Galois representation $G(\ol K/K)\to \Aut V_{\ell}E\cong \GL_2(\Q_{\ell})$.

Define the \textbf{local $L$-factor} of $E$ at a prime $\mfp$ of $K$ as follows. Choose $\ell$ such that $\mfp\nmid \ell$, and let 
\[
L_{\mfp}(E,s):=L_{\mfp}(\rh_{\ell},s)=\det(1-q^{-s}\Frob(\mfp)| (V_{\ell}E)^{I_{\mfp}})^{-1},
\]
where $q=\fN\mfp$ and $I_{\mfp}$ is the inertia subgroup of $G(\ol K/K)$. (Choose an embedding $\Q_{\ell}\hra \C$.)
The \textbf{$L$-series} of $E$ is the product of local factors
\[
L_{\mfp}(E/K,s):=\prod_{\mfp} L_{\mfp}(E,s).
\]
\end{df}
\begin{rem}
This is (almost) the same as saying: fix a prime $\ell$ and let  $L(E/K,s):=L(\rho_{\ell},s)$. The only difference is that we run into trouble with the local factor $L_{\mfp}(\rh_{\ell},s)$ on the right hand side, so we have to choose a different $\ell'$ and let this local factor be $L_{\mfp}(\rh_{\ell'},s)$ instead.
\end{rem}
The following is an equivalent definition (that is more concrete).
\begin{df}\llabel{df:E-L2}
Let $N$ be the conductor\footnote{$N$ is divisible by exactly the primes of bad reduction} of the elliptic curve $E$. Define the local $L$-factor by
\[
L_{\mP}(E,s)=1-a_qq^{-s}+\chi(q)qq^{-2s},\quad a_q=q+1-|E(\F_q)|,\quad \chi(q)=\begin{cases}
1,&m\perp N\\
0,&\text{else}
\end{cases}
\]
where $q=\fN \mfp$. 
Thus
\[
L_v(E,s)=\begin{cases}
1-a_qq^{-s}+qq^{-2s},&\text{good reduction}\\
1-q^{-s},&\text{split multiplicative reduction}\\
1+q^{-s},&\text{non-split multiplicative reduction}\\
1,&\text{additive reduction.}
\end{cases}
\]
\end{df}
Note that $a_q$, the ``trace of Frobenius," is related to the number of points of $E$ over $\F_q$. Hence the $L$-function contains information about the number of points of $E$ over each $\F_q$.

Showing that these two definitions are equivalent requires us to show that $(V_{\ell}E)^{I_{\mfp}}$ is 2, 1, or 0-dimensional when $E$ has good, multiplicative, and additive reduction, respectively. The general idea is that the action of $I_{\mfp}$ on $V_{\ell}E$ contains exactly the information lost by looking at the reduced elliptic curve, since $I_{\mfp}$ is exactly the kernel of $D_{\mfp}(\ol K/K)\to G(\ol k/k)$, so nontrivial action of $I_{\mfp}$ corresponds to bad reduction.

In the CM case, we cannot have multiplicative reduction, so the $L$-series is particularly simple. We will show that the two definitions are equivalent in this case.
\begin{thm}\llabel{thm:cmec-no-mr}
Let $E/K$ be a CM elliptic curve. Then $E$ cannot have multiplicative reduction at any prime.
\end{thm}
\begin{proof}
An elliptic curve $E$ has potential good reduction iff its $j$-invariant is integral~\cite[VII.5.5]{Si86}. CM have integral $j$-invariants, so have potential good reduction, i.e. have good or multiplicative reduction.  
%is preserved by finite field extension, so $E$ must have good reduction over $K$ already.
\end{proof}
\begin{proof}[Proof that Definitions~\ref{df:E-L1} and~\ref{df:E-L2} are equivalent in the CM case]
Suppose $E$ has CM by an order $\sO$ in $K$, and $E$ is defined over $L$. 
By N\'eron-Ogg-Shafarevich, $I_{\mfp}$ acts trivially on $V_{\ell}E$ iff $E$ has good reduction at $\mfp$. Let $q=\fN\mfp$.

In the case of good reduction we need to show $\det(1-q^{-s}\Frob(\mfp)|V_{\ell}E)=1-a_qa^{-s}+qq^{-2s}$.
Every endomorphism $\phi$ on $E$ satisfies $\phi^2-\tr(\phi)\phi+\deg(\phi)=0$, where $\tr(\phi)=1+\deg(\phi)-\deg(1-\phi)$. Since $\Frob(\mfp)$ acts as the Frobenius morphism $\phi_{\mfp}$, its characteristic polynomial is
\[
\det(\la-\Frob(\mfp))= \la^2-\tr(\phi_{\mfp})\la+\deg(\phi_{\mfp}).
\]
But
\begin{align*}
\deg(\phi_{\mfp})&=q\\
\tr(\phi_{\mfp})&=1+\deg(\phi_{\mfp})-\deg(1-\phi_{\mfp})\\
&=q+1-\ker(1-\phi_{\mfp})\\
&=q+1-|E(\F_q)|.
\end{align*}
(This part of the proof doesn't use the fact that $E$ has CM.)

Since $E$ has no multiplicative reduction by Theorem~\ref{thm:cmec-no-mr}, it remains to prove that $W:=(V_{\ell}E)^{I_{\mfp}}=0$ when $E$ has multiplicative reduction. We know by N\'eron-Ogg-Shafarevich that $\dim(W)\le 1$. But because $E$ is CM, $V_{\ell}E\cong (\varprojlim_n \ell^{-n}\ma/\ma)\ot \Q$ has the structure of a $\sO_K\ot \Q_{\ell}$-vector space. If $a\in W$, then for any $\al\in K$, $\al a\in W$ because $[\al]$ commutes with the Galois action. Hence $W$ is not just a $\Q_{\ell}$-subspace of $V$, but also a $\sO_K\ot \Q_{\ell}$-subspace. Hence its dimension over $\Q_{\ell}$ is even, and must be 0.
\end{proof}
\subsection{Analytic continuation}
\begin{thm}[Deuring]\llabel{thm:cmec-l}
Let $E/L$ be an elliptic curve with CM by $\sO_K$ with $K\subeq L$.Then
\[
L(E/L,s)=L(s,\psi_{E/L})L(s,\ol{\psi_{E/L}}).
\]
\end{thm}
\begin{cor}[Analytic continuation of $L$-function for CM elliptic curves]\llabel{cor:cmec-l-ac}
Let $E/L$ be an elliptic curve with CM by $\sO_K$. Then $L$ admits an analytic continuation to $\C$ and satisfies a functional equation relating its values at $s$ and $2-s$.
\end{cor}
This theorem for general elliptic curves is very deep (it follows from the Modularity Theorem and the analytic properties of $L$-functions associated to modular forms).
\begin{proof}[Proof of Theorem~\ref{thm:cmec-l}]
By Theorem~\ref{thm:cmec-no-mr}, $E$ has no multiplicative reduction. Let $\mP$ be a prime, and consider 2 cases.
\begin{enumerate}
\item
$E$ has good reduction at $\mP$. Choose any $\ell$ not dividing $\mP$. The characteristic polynomial of the action of $\phi_{\mP}$ on $V_{\ell}E$ is 
$\det(\la- \phi_{\mP}|V_{\ell}E)$. However, if we make the identification $E\tors\cong K/\ma$, we have
\[
V_{\ell}E=\varprojlim \ell^{-n}\ma/\ma,
\]
and we know that $\phi_{\mP}$ acts on $E\tors\cong K/\ma$ as multiplication by $\chi_{E/L}(\mP)$. Therefore, the eigenvalues of the action of $\phi_{\mP}$ on $V_{\ell}E$ are just $\chi_{E/L}(\mP)$ and $\ol{\chi}_{E/L}(\mP)$, and 
\[
\det(\la- \phi_{\mP}|V_{\ell}E)=(\la-\chi_{E/L}(\mP))(\la-\ol{\chi}_{E/L}(\mP)).
\]
Taking $\la=p^{s}$ and dividing by $p^{2s}$ gives
\[
L_{\mP}(E/L,s)=\det(1- p^{-s}\phi_{\mP}|V_{\ell}E)
=L_{\mP}(s,\chi_{E/L})L(s,\ol{\chi}_{E/L}).
\]
\item
$E$ has bad reduction at $\mP$. Then $\chi_{E/L}(\mP)=0$ by definition, and $L_{\mP}(E/L,s)=1=(1-\chi_{E/L}(\mP))(1-\ol{\chi}_{E/L}(\mP))=L_{\mP}(s,\chi_{E/L})L(s,\ol{\chi}_{E/L})$.
\end{enumerate}
Multiplying together all the local factors gives the result.
\end{proof}
\begin{proof}[Proof of Corollary~\ref{cor:cmec-l-ac}]
The $L$-functions of Gr\"ossencharacters have analytic continuation (Theorem~\ref{ch:cft-app}.\ref{thm:l-analytic-cont}, which works for Gr\"ossencharacters as well). Thus the result follows directly from Theorem~\ref{thm:cmec-l}.
\end{proof}
Thus we have carried out the program in Section~\ref{ch:cft-app}.\ref{sec:intro-langlands} for CM elliptic curves, to get the correspondences.
\[
\text{(CM Elliptic curves)}\rightarrow \text{(Galois representation)}
\rightarrow \text{(2 Gr\"ossencharacters)}\\
\]
Remember Gr\"ossencharacters are 1-dimensional automorphic representations. If we wanted a modular form, we can use the technique of {\it automorphic induction} to construct a modular form from 2 Gr\"ossencharacters.