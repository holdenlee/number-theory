\chapter{Elliptic curves over local fields}
\fixme{
Summary.
\begin{enumerate}
\item
\item
\item
\item Define unramified for $G(\ol K/K)$-sets by saying that the action is entirely captured by $G(K\ur/K)$, i.e, $I_v$ acts trivially. $E[m], T_{\ell}(E)$ are unramified for $m,\ell\perp \chr(k)$ because of injectivity $E[m]\hra \wt E(k')$, which means it suffices to look at the action on $l/k$, i.e. $G(K\ur /K)\cong G(k^s/k)$.
\item Define good, multiplicative (split/nonsplit), additive reduction. 5.4
\begin{enumerate}
\item
Cannot improve from unramified extension (any change of coordinates over an unramified extension can be done over the base field, up to a unit, since the $u'$ doesn't have fractional valuation).
\item Good/multiplicative reduction preserved: we'll still have $v(\De), v(c_4)=/\ne0$ in extension.
\item Can improve from additive reduction: intuition---$v(c_4)>0$---by extending valuation can make CoC so that $v(c_4)=0$. Use Legendre form.
\item Good reduction iff $j$-invariant integral: if good reduction, then $\De\in (R')^{\times}$, need to show in $R$. But $j=\fc{c_4'}{\De'}$, and by good reduction $\De'$ is in $R'^{\times}$. If $j$-invariant integral---note that $j$-invariant is most closely linked to the Legendre form, so we look at that---$\la$ must be integral, and not ``almost" a repeated root, so $\De$ (think of it as the discriminant of the polynomial!) is good.  
\end{enumerate}
\item With NOS, we have a strong link between good reduction and non-ramification, in the sense that a weak result about non-ramification of torsion points (just $E[m]$ unramified at $v$ for infinitely many integers $\perp \chr(k)$) implies good reduction.

Pf. Choose $m$ super-large and relatively prime. The key is that $m$ is more than $E/E_0(K^{\text{nr}})$---because then $E_1$ can't capture everything and this forces the existence of $(\Z/\ell\Z)^2$ in the reduction part, which cannot happen for mult/add reduction.

7.2 Isogenous $\implies$ both have good reduction, or bad reduction. Look at $m$-torsion!

7.3 Pot good reduction iff $I_v$ acts on $T_{\ell}(E)$ through finite quotient for some/all primes $\ell\ne \chr(k)$. Forward direction easy---just extend field until good reduction. Backwards direction: look at fixed field, can find finite subfield giving good reduction (use giving compositum).
\end{enumerate}
}

\section{Introduction}
Let $K$ be a field. 
\begin{df}\llabel{df:cam8-1}
$v:\Kt\to \Z$ is a \textbf{discrete valuation} if
\begin{enumerate}
\item
$v(xy)=v(x)+x(y)$.
\item
$v(x+y)\ge \min(v(x),v(y))$.
\end{enumerate}
($v(0)$ is either 0 or undefined.)
\end{df}
\begin{ex}
\begin{enumerate}
\item
$K=\Q$ prime. $v=v_p$ where $v_p(p^r\fc{a}{b})=r$ with $a,b\in \Z$ coprime to $p$.
\item
$[K:\Q]<\iy$, ring of integers $\sO_K$ where $\mfp\subeq \sO_K$ is a prime ideal.
\end{enumerate}
We have $v=v_{\mfp}$ where $v_{\mfp}(x)$ is the power of $\mfp$ in the factorization of $x\sO_K$ into prime ideals. 
\end{ex}
\begin{rem}
Let $x,y\in K$. The definition gives 
\[
\begin{cases}
v(x+y)&\ge \min(v(x),v(y))\\
v(x)&\ge \min(v(x+y),v(y)=v(-y)).
\end{cases}
\]
So if $v(x)<x(y)$ then $v(x+y)=v(x)$. Hence if $v(x)\ne v(y)$, then $v(x+y)=\min(v(x),v(y))$. 
\end{rem}
\begin{lem}\llabel{lem:cam8-2}
Let $E/K$ be an elliptic curve, with Weierstrass equation 
\[
y^2+a_1xy+a_3y=x^3+a_2x^2+a_4x+a_6.
\]
Assume $v(a_i)\ge 0$ for all $i$. Let $O\ne P=(x,y)\in E(K)$. Then either
\begin{enumerate}
\item
$v(x),v(y)\ge 0$, or
\item
$v(x)=-2r, v(y)=-3r$ for some $r\ge 1$.
\end{enumerate}
\end{lem}
\begin{proof}
\ul{Case $v(x)\ge 0$:} Suppose $v(y)<0$. Then $v(y^2+a_1xy+a_3y)=2v(y)<0$ and $v(RHS)\ge 0$, contradiction. Thus $v(y)\ge 0$.\\

\ul{Case $v(x)<0$:} We have $v(LHS)\ge \min(2v(y),v(x)+v(y),v(y))$, $v(RHS)=3v(x)$, and therefore (checking a few cases) $v(y)<v(x)<0$. Then $v(LHS)=2v(y)$. Thus $v(x)=-2r$, $v(y)=-3r$ for some $r\ge 1$.
\end{proof}
Notation: Denote the valuation ring, unit group, maximal ideal, and residue field by 
\begin{align*}
\sO_K&=\set{x\in K^{\times}}{v(x)\ge 0}\cup \{0\}\\
\sO_K^{\times}&=\set{x\in K^{\times}}{v(x)=0}\cup \{0\}\\
\pi\sO_K&=\set{x\in K^{\times}}{v(x)\ge 1}&\text{picking }\pi\in K,v(\pi)=1.\\
%setdry
k&=\sO_K/\pi\sO_K.
\end{align*}
\begin{df}
A Weierstrass equation for $E$ with coefficients $a_1,\ldots, a_6\in K$ is
\begin{enumerate}
\item
\textbf{integral} if $a_1,\ldots,a_6\in \sO_K$.
\item
\textbf{minimal} if $v(\De)$ is minimal among all integral models for $E$ (among all the Weierstrass equations you can write down).
\end{enumerate}
\end{df}
%amongst all the Weierstrass equations you can write down, the discriminant is as small as possible.
\begin{rem}
\begin{enumerate}
\item
Putting $x=u^2x'$ and $y=u^3y'$ gives $a_i=u^ia_i'$, therefore integral models exist.
\item
$a_1,\ldots, a_6\in \sO_K$ gives $\De\in \sO_K$ and $v(\De)\in \N_0$.
\end{enumerate}
\end{rem}
We'd like to apply formal groups to elliptic curves over local fields.

Fixing any $0<c<1$ we define a metric on $K$ by
\[
\begin{cases}
c^{v(x-y)}&\text{if }x\ne y\\
0&\text{ if}x=y.
\end{cases}
\]
Definition~\ref{df:cam8-1} gives 
\[
d(x,z)\le \max(d(x,y),d(y,z)),
\]
the ultrametric inequality (much stronger than the triangle inequality). 
Let $\hat K$ be the completion of $K$ with respect to $d$. By continuity, $+,\times$ extend to $\hat K$, so $\hat K$ is a field. $v$ extends to $\hat K$ and is a discrete valuation. Note that the residue field does not change.

In examples 1 and 2, we write $\hat K=\Q_p$ and $K_{\mfp}$. For the rest of this section, $K$ is a field complete with respect to a discrete valuation $v:\Kt\to \Z$. $\sO_K,\pi,k$ are defined as before.

We further assume $\chr(K)=0$ and $\chr(k)=p>0$. (For example, $K=\Q_p$, $\sO_K=\Z_p$, $\pi\sO_K=p\Z_p$, $k=\F_p$.) $K$ is complete so $\sO_K$ is complete with respect to $\pi^r\sO_K$ for any $r\ge 1$. In \S7 we put $t=-\fc xy$ we put $t=-\fc xy$, $w=-\rc y$. Then 
\begin{align*}
\hat E(\pi^r\sO_K)&=
\set{(x,y)\in E(K)}{-\fc xy,-\rc y \in \pi^r\sO_K}\cup \{0\}\\
&=\set{(x,y)\in E(K)}{v\pf{x}{y},v\pf{1}{y} \ge r}\cup \{0\}\\
&\stackrel{\text{Lem.~\ref{lem:cam8-2}}}= 
\set{(x,y)\in E(K)}{v(x)\le -2r,v(y)\le -3r}\cup \{0\}.
\end{align*}
By Lemma~\ref{lem:cam7-2}, this is a subgroup of $E(K)$, say $E_r(K)$.

Thu. 7/11

We get a filtration
\[
\cdots \subeq E_3(K)\subeq E_2(K)\subeq E_1(K)=\hat E(\pi \sO_K).
\]
More generally, if $\cF$ is a formal group over $\sO_K$, 
\[
\cdots\subeq  \cF(\pi^3\sO_K)\subeq \cF(\pi^2\sO_K)\subeq \cF(\pi\sO_K).
\]
\begin{pr}\llabel{pr:cam8-3}
Let $\cF$ be a formal group over $\sO_K$. Let $e=v(p)$. If $r>\fc{e}{p-1}$, then
\begin{align*}
\log:\cF(\pi^r\sO_K)& \to \wh{\G}_a(\pi^r \sO_K)\\
\exp:\wh{\G}_a(\pi^r\sO_K)& \to\cF(\pi^r \sO_K)
\end{align*}
are inverse homomorphisms.
\end{pr}
\begin{proof}
%We need to show the power series converge; then the fact they are homomorphism and inverses will follow from what we've already shown.

Let $x\in \pi^r\sO_K$. We must show the power series $\exp$ and $\log$ of Theorem~\ref{thm:cam7-3} converge. Then the fact they are homomorphism and inverses will follow from what we've already shown. Recall 
\[
\exp(T)=T+\fc{b_2}{2!}T^2+\fc{b_3}{3!}T^3+\cdots
\]
for $b_2,b_3,\ldots, \in \sO_K$. We have
\bal
v(n!)&=ev_p(n!)\\
&=e\pa{\sum_{j=1}^{m} \ff{n}{p^j}},\qquad p^m\le n<p^{m+1}\\
&\le e\sum_{j=1}^m \fc{n}{p^j}\\
&=en \fc{\rc p-\rc{p^{m+1}}}{1-\rc p}\\
&=en\rc{p-1} (1-p^{-m})\le \fc{e}{p-1}(n-1).
\end{align*}
Therefore 
\begin{align*}
v\pa{\fc{b_n}{n!}x^n}&\ge nr-\fc{e}{p-1}(n-1)\\
&=(n-1)\ub{\pa{r-\fc{e}{p-1}}}{>0}+r.
\end{align*}
This is $\ge r$ and goes to $\iy$ as $n\to \iy$. 
Because we have the ultrametric law, convergence of terms in a sum implies convergence of the sum. Thus $\exp(x)$ converges and belongs to $\pi^r\sO_K$. 

For $\log$ the denominators are smaller, so a fortiori the series for $\log$ converges.\footnote{In complex analysis we're used to $\exp$ having better convergence, but here it's the other way around.}
\end{proof}
So for $r$ sufficiently large,
\[
\cF(\pi^r\sO_K)\cong \wh{\G}_a(\pi^r\sO_K)\cong (\pi^r\sO_K,+)\cong (\sO_K,+).
\]
\begin{lem}[Filtration of formal groups]\llabel{lem:cam8-4}
If $r\ge 1$ then $\fc{\cF(\pi^r\sO_K)}{\cF(\pi^{r+1}\sO_K)}\cong (k,+)$.
\end{lem}
\begin{proof}
note $F(X,Y)=X+Y+XY(\cdots)$. If $x,y\in \sO_K$ then $F(\pi^rx,\pi^ry)\equiv \pi^r(x+y)\pmod{\pi^{r+1}}$. We get a  group homomorphism
\bal
\cF(\pi^r\sO_K)&\to k\\
\pi^rx &\mapsto x\pmod \pi
\end{align*}
surjective (by definition of $k$ as a residue field) with kernel $\cF(\pi^{r+1}\sO_K)$. Hence
\[
\fc{\cF(\pi^r\sO_K)}{\cF(\pi^{r+1}\sO_K)}\xrc (k,+).
\]
\end{proof}
\begin{cor}
If $|k|<\iy$ then $\cF(\pi\sO_K)$ has a subgroup of finite index isomorphic to $(\sO_K,+)$. 
\end{cor}
We use the following notation for reduction modulo $\pi$:
\bal
\sO_K&\to \frac{\sO_K}{\pi \sO_K}=k\\
x&\mapsto \wt x.
\end{align*}

Let $E/K$ be an elliptic curve. 
\begin{pr}
The reductions modulo $\pi$ of two minimal Weierstrass equations for $E$ define isomorphic curves over $k$. 
\end{pr}
\begin{proof}
Say the Weierstrass equations are related by $[u;r,s,t]$, $u\in \Kt$ and $r,s,t\in K$. Then $\De_1=u^{12}\De_2$. If both equations are minimal, then $v(\De_1)=v(\De_2)$, then $v(u)=0$, i.e., $u\in \sO_K^{\times}$.

In the case where $\chr(k)\ne 2,3$, we can work with shorter Weierstrass forms, and we only need to worry about $u$.

In the general case, the transformation formulae for $a_i$'s and $b_i$'s (for example $u^2b_2'=b_2+12r$) gives $r,s,t\in \sO_K$. The Weierstrass equations for the reduction $\pi$ are now related by $[\wt u\ne 0;\wt r,\wt s,\wt t]$. 
\end{proof}
\begin{df}\llabel{df:ec-reduction}
The \textbf{reduction} $\wt E/k$ of $E/K$ is defined by the reduction of the reduction of a minimal Weierstrass equation modulo $pi$. We say $E$ has \textbf{good reduction} if $\wt E$ is non-singular (i.e., $\wt E$ is an elliptic curve). Otherwise, it has \textbf{bad reduction}.
\end{df}
For an integral Weierstrass equation, 
\begin{itemize}
\item
$v(\De)=0$ implies good reduction,
\item
$0<v(\De)<12$ implies bad reduction (we can only change $\De$ by powers of 12 so we can't get $v(\De)=0$; the discriminant vanishes when you reduce modulo $\pi$.), and 
\item
when $v(\De)\ge12$, the Weierstrass equation might not be minimal. \fixme{good or bad reduction?}
\end{itemize}

There is a well-defined map 
\begin{align*}
\Pj^2(K)&\to \Pj^2(k)\\
(x:y:z)&\mapsto (\wt x:\wt y:\wt z)
\end{align*}
where we choose the representative with $\min(v(x),v(y),v(z))=0$. 

We restrict this map to get 
\begin{align*}
E(K)&\to \wt E(k)\\
P&\mapsto \wt P.
\end{align*}
Lemma~\ref{lem:cam8-2} gives that $E_1(K)=\set{P\in E(K)}{\wt P=O}$.

Why should $\wt E(k)$ be a group? In the case of bad reduction we don't have a group law... but we can get one by deleting a singular point. Let 
\[
\wt E_{ns} =\begin{cases}
 \wt E &\text{ if $E$ has good reduction,}\\
\wt E\bs \{\text{singular point}\}&\text{ if $E$ has bad reduction.} 
\end{cases}
\]

The chord and tangent process defines a group law on $\wt E_{ns}$: If the line passes through the singular point, it passes through with multiplicity 2. Hence the third point of intersection will not be the singular point.

In the case of bad reduction, then over $\ol k$, $\wt E\cong \G_a$ or $\G_m$.

For simplicity, assume $\chr(k)\ne 2$. Note $\wt E:y^2-f(x)$ has $\deg(f)=3$. There are two possibilities.
\begin{enumerate}
\item
The singular point is a double root, for example $y^2=x^2(x+1)$. This is a curve with a node, and we get multiplicative reduction.
\item
The singular point is a triple root, for example, $y^2=x^3$. This is a curve with a cusp, and we get additive reduction.
\end{enumerate}

{\color{blue}Lecture 9/11}

(Picture)

We check that we have a group law on a singular curve with the singular point removed. We'll just do the special case of $y^2=x^3$.
We have 
\bal
\G_a&\xrc \wh E_{ns}\\
t&\mapsto (t^{-2},t^{-3})\\
O&\mapsto O \text{ (point at $\iy$)}\\
\fc xy& \mapsfrom (x,y).
\end{align*}
We check this is a group homomorphism. Let $P_1,P_2,P_3$ be on the line $ax+by=1$. Let $P_i=(x_i,y_i)$. Then  $x_i^3=y_i^2=y_i^2(ax_i+by_i)$, so $t_i=\fc{x_i}{y_i}$ is a root of $x^3-aX-b$, and $t_1+t_2+t_3$. 

This is the additive case; there is a similar calculation for the multiplicative case. The one thing to notice is that rational parametrizations aren't unique, because we can compose by a Mobius transformation. We want to make sure that the point at $\iy$ on $\Pj^1$ goes to the singular line. In the multiplicative case, 2 points on the projective line map to the singular point.
We get a rational map between the curve and $\Pj^1$ with $0,\iy$ deleted, the multiplicative group. We want 1 to be mapped to the identity; this helps us pick the right Mobius map.

\begin{df}
Define
\[
E_0(K)=\set{P\in E(K)}{\wt P\in \wt E_{ns}(k)}.
\]
\end{df}
\begin{pr}\llabel{pr:cam8-6}
$E_0(K)\subeq E(K)$ is a subgroup, and reduction modulo $\pi$ is a surjective group homomorphism, $E_0(K)\rra \wt E_{ns}(k)$. 
%good reduction: no singular point to delete.
\end{pr}
%chord and tangent process
\begin{proof}
(Group homomorphism) A line in $\Pj^2$ defined over $K$ is given by $\ell:aX+bY+cZ=0$ for some $a,b,c\in K$ not all 0. We may assume $\min(v(a),v(b),v(c))=0$. Reduction modulo $\pi$ gives a line $\wt{\ell}:\wt aX+\wt bY+\wt cZ$ where $\wt a,\wt b,\wt c$ re not all 0. 
Thinking of lines as points in dual projective space, our method of reducing lines is the same as our method of reducing points.

If $P_1,P_2,P_3\in E(k)$ with $P_1+P_2+P_3=O$, then they lie on a line $\ell$. Then $\wt P_1,\wt P_2,\wt P_3\in \wt E(k)$ lie on the line $\wt{\ell}$.
%original 3 points distinct
%intersect at line, look for roots of cubic polynomial

If $\wt P_1,\wt P_2\in \wt E_{ns}(k)$, then $\wt P_3\in \wt E_{ns}(k)$. So of $P_1,P_2\in E_0(K)$ then $P_3\in E_0(K)$ and $\wt P_1+\wt P_2+\wt P_3=0$.

(Surjective) Let $f(x,y)=y^2+a_1xy+a_3y-x^3-a_2x^2-a_4x-a_6$. Let $\wt P\in \wt E_{ns}(k)\bs \{0\}$, say $\wt P=(\wt x_0,\wt y_0)$ for some $x_0,y_0\in \sO_K$. Since $\wt P$ is nonsingular, either $\pd fx(x_0,y_0)\nequiv 0\pmod{\pi}$ or $\pd fy(x_0,y_0)\nequiv 0\pmod{\pi}$. If $\pd fx(x_0,y_0)\nequiv 0 \pmod{\pi}$ then let $g(t)=f(t,y_0)\in \sO_K[t]$. We will use Hensel's lemma~\ref{lem:hensel-ring} to turn our approximate root into an exact root.

Then
\[
\begin{cases}
g(x_0)\equiv 0\pmod{\pi}\\
g'(x_0)\in \sO_K^{\times}.
\end{cases}
\]
Hensel's lemma gives $b\in \sO_K$ such that $g(b)=0$, $b\equiv x_0\pmod{\pi}$. Then $P=(b,y_0)\in E(K)$ reduces to $\wt P=(\wt x_0,\wt y_0)$. (In fact, $P\in E_0(K)$.) The case where $\pd fy(x_0,y_0)\nequiv 0\pmod{\pi}$ works in the same way.
\end{proof}
This is a general argument to lift points on curves by Hensel's lemma;  the form of $f$ didn't really matter.

We have for $r>\fc{e}{p-1}$,
\[
\xymatrix{
\hat E(\pi^r\sO_K)\aq{d} &&&\hat E(\pi\sO_K)\aq{d}&&\\
E_r(K)\aq{d} \ha{r}_{(k,+)} & \cdots \ha{r}_{(k,+)} & E_2(K)\ha{r}_{(k,+)} & E_1(K)\ha{r}_{\wt E_{ns}(k)} & E_0(K) \ha{r} & E(K)\\
(\sO_K,+) &&&&&
}
\]
where each quotient before $E_1(K)$ is isomorphic to $(k,+)$ and $\fc{E_0(K)}{E_1(K)}\cong \wt E_{ns}(k)$. 
%keep passing to groups of finite index. 
Our goal is to show $E(K)$ contains a subgroup of finite index isomorphic to $(\sO_K,+)$; it remains to  study the top layer, how $E_0(K)$ sits inside $E(K)$. 
If good reduction we're already done; we just need to consider bad reduction. %Neron model, we will do a compactness model to show the index is finite

\begin{lem}
If $|k|<\iy$, then $\Pj^n(K)$ is compact with respect to the $\pi$-adic topology. 
\end{lem}
\begin{proof}
We have
\[
\fc{\pi^r \sO_K}{\pi^{r+1} \sO_K} \cong \fc{\sO_K}{\pi\sO_K}\cong k
\]
by $\pi^r x\bmod{\pi^{r+1}}\mapsto x\bmod \pi$. Because $k$ is finite, $\fc{\sO_K}{\pi^r\sO_K}$ is finite for all $r$. 

Let $(x_n)$ be a squence in $\sO_K$. $(x_n)$ has a subsequence $(x_n^{(1)})$ that is constant modulo $\pi$, and inductively $(x_n^{(i)})$ has a subsequence $(x_n^{(i+1)})$ that is constant modulo $\pi^{i+1}$. Then $(x_n^{(n)})$ is a Cauchy sequence, and hence converges. Thus $\sO_K$ is sequentially compact and hence compact. (In general, profinite groups are compact.)

$\Pj^n(K)$ is the union of compact sets $\set{[a_0:\cdots a_{i-1}:1:a_{i+1}:\cdots a_n]}{a_i\in \sO_K}$ and hence compact.
\end{proof}
\begin{lem}
Suppose $E_0(K)\subeq E(K)$ has finite index.
\end{lem}
\begin{proof}
$E(K)\subeq \Pj^2(K)$ is a closed subset and hence a compact topological group. If $\wt E$ has a singular point $(\wt x_0,\wt y_0)$, $E(K)\bs E_0(K)=\set{(x,y)\in E(K)}{v(x-x_0)\ge 1, v(y-y_0)\ge 1}$ is a closed subset (as $v$ is continuous). Thus $E_0(K)\subeq E(K)$ is an open subgroup. 
The cosets of $E_0(K)$ in $E(K)$ form an open cover of $E(K)$. 
%subgroup open iff closed of finite index?
Then $E(K)$ is compact, and $[E(K):E_0(K)]<\iy$. The index
\[
c_K(E)=[E(K):E_0(K)]<\iy
\]
is called the \textbf{Tamagawa number}.
\end{proof}
\begin{rem}
If $E$ has good reduction, then $c_K(E)=1$, but the converse is false. If you do more geometry, using Neron models, then you get a better understanding of $c_K$.

If you work in a more abstract setting, reduction consists of several components. When we look at the Weierstrass equation, we see only one component; the other ones get contracted to a singular point and we don't see them. ($c_K$ is the number of components.) One can prove various nice facts about $c_K$: either $c_K(E)=v(\De)$ or $c_K(E)\le 4$. We've insisted on the minimal Weierstrass equation, but only needed it in two places: the reduction is well-defined, and in the above fact on $c_K$.

%Note that we get additive or multiplicative reduction when we assume completeness
Split multiplicative reduction is where get multiplicative reduction without having to make a field extension.
\end{rem}
\begin{thm}\llabel{thm:cam8-9}
Let $K$ be a field complete with respect to a discrete valuation, $\chr(K)=0$ with finite residue field. Then $E(K)$ contains a subgroup of finite index isomorphic to $K^+$. 
In particular, $E(K)\tors$ is finite.
%whenever have abelian group, can look at torsion subgroup.
\end{thm}
\begin{rem}
The fields in Theorem~\ref{thm:cam8-9} are exactly the finite extensions of $\Q_p$. 
%The hypotheses for this theorem $K$ finite extension for $\Q_p$. 
\end{rem}
Next time we'll prove a result useful when we look at elliptic curves over global fields, about unramified extensions.

{\color{blue}{Lecture 12-11}}
We recall some facts on local fields. Let $K$ be a finite extension of $\Q_p$. Let $K$ be a finite extension of $\Q_p$. Note that $K$ is complete wrt $v_K$. Let $L/K$ be a finite extension. Then we have a commutative diagram
\[
\xymatrix{
\Kt\sj{r}^{v_K} \ha{d} & \Z\ar[d]^{\times e}\\
L^{\times} \sj{r}^{v_L} & \Z.
}
\]
where $[L:K]=ef, f=[k':k]$, and $k,k'$ are the residue fields of $K$ and $L$ (and have characteristic $p$). If $L/K$ is Galois then there is a natural group homomorphism 
\[G(L/K)\to G(k'/k)\] (since the Galois action preserves the valuation). This map is surjective with kernel of order $e$.
\begin{df}
$L/K$ is unramified if $e=1$.
\end{df}
\begin{pr}
For each integer $m\ge 1$,
\begin{enumerate}
\item
$k$ has a unique extension of degree $m$, and
\item
$K$ has a unique unramified extension of degree $m$.
\end{enumerate}
Moreover, these extensions are Galois with cyclic Galois group.
\end{pr}
The takeaway is that given any extension of residue fields, you can find an unramified extension with that residue field.
\begin{thm}\llabel{thm:cam8-10}
Let $[K:\Q_p]<\iy$. Suppose $E/K$ has good reduction and $p\nmid n$. If $P\in E(K)$ then $K([n]^{-1}P)/K$ is unramified.
\end{thm}
(We use the notation $[n]^{-1}P=\set{Q\in E(\ol K)}{nQ=P}$, $K(\{P_1,\ldots, P_r\})=K(x_1,\ldots, x_r,y_1,\ldots, y_r)$ where $P_i=(x_i,y_i)$.)
\begin{proof}
%size $n^2$ because the size of the fiber is the degree (we are over characteristic 0 so everything is separable).
$[n]:\wt E\to \wt E$ is a separable isogeny, since $p\nmid n$. Thus $|[n]^{-1}\wt P|=\deg[n]=n^2$. Here $[n]^{-1}\wt P=\set{Q'\in \wt E(\ol k)}{nQ'=\wt P}$. 
%
Consider the extension of residue fields $k'=k([n]^{-1}\wt P)$. 
Let $m=[k':k]$. let $L/K$ be the unramified extension of degree $m$, so $L$ has residue field $k'$. We claim that each $Q'\in \wt E(k')$ with $nQ'=\wt P$ is the reduction of some $Q\in E(L)$ with $nQ=P$. 
%really working with fact complete fields

By Proposition~\ref{pr:cam8-6}, there exists $Q_0\in E(L)$ reducing to $Q'$. Then $nQ_0-P\in E_1(L)$. 
Since $p\nmid n$, Corollary~\ref{cor:cam7-5} tells us multiplication by $n$ on $E_1(L)$ is an isomorphism, so there exists $Q_1\in E_1(L)$ such that $nQ_0-P=nQ_1$.
Then $P=n(Q_0+Q_1)$. Taking $Q=Q_0+Q_1$ proves the claim.

We found $n^2$ points just by finding points defined over $L$. Thus all $n^2$ points in $[n]^{-1}P$ are defined over $L$.

Thus $K([n]^{-1}P)= L$ and $K([n]^{-1}P)/K$ is unramified.
\end{proof}
%2:15 tues.
%visiting nottingham