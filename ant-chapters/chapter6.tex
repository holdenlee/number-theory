\chapter{Valuations and completions}\llabel{valuations-and-completions}
%Local and global fields chapter is in the same document.
%Take $\si\in G(K^{\text{gal}}/\Q)$. Note composition by $\si$ permutes the $\si_i$, say by $\nu$. Then 
%\begin{align*}
%P&=\sum_{\text{even }\pi\in S_m}\prod_{i=1}^m\si_i\al_{\nu^{-1}\pi(i)}\\
%N&=\sum_{\text{odd }\pi\in S_m}\prod_{i=1}^m\si_i\al_{\nu^{-1}\pi(i)}\\
%\end{align*}
%and hence $\si$ permutes $\{P,N\}$.
Here is some motivation for considering $\mfp$-adic fields.
\begin{enumerate}
\item One useful tool in arithmetic geometry is the {\it local to global} principle, which says that the existence of solutions modulo all primes tells us something about the existence of solutions in the original field or ring, such as $\Q$ or $\Z$. For example, the Hasse-Minkowski Theorem. However, it is not enough to check for solutions modulo all powers of $p$ --- because a solution modulo $p$ does not necessarily give a solution modulo powers of $p$. The solution is to look for solutions in a field which contains information {\it modulo all powers of $p$}, a $p$-adic field.
\item When we take a $\mfp$-adic fields, the only prime ideal remaining is $\mfp$; all others primes become units. This vastly simplifies algebraic number theory; we don't have to worry about primes that split. Then we can recover facts about the global field.
\end{enumerate}
\section{Case study: $p$-adic integers}\llabel{padic-exs}
We first examine how $p$-adic rationals are defined, before generalizing to other number fields.

Often we look at the integers modulo higher and higher powers of a prime $p$; for example,  when we were looking at the existence of primitive roots (Theorem~\ref{primitive-roots-exist}) or the structure of $\Z/p^n\Z$ (Theorem~\ref{mult-structure}). Hensel's lemma told us that under certain conditions we can lift solutions modulo higher and higher powers of $p$.

Rather than work with powers of $p$ piecemeal, we can devise a structure that holds information modulo all powers of $p$ at once. To do this, we define the ring $p$-adic integers $\Z_p$ and $p$-adic rationals $\Q_p$, which contain $\Z$ and $\Q$, respectively. We will do this in two ways:
\begin{enumerate}
\item Define $\Z_p$ as an {\it inverse limit} of the rings $\Z/p^n\Z$ and $\Q_p$ as the fraction field.
\item Give $\Q$ a topology (or even better, a metric) related to divisibility by $p$, and complete $\Q$ with respect to this topology.
\end{enumerate}
\subsection{$p$-adics as an inverse limit}
\begin{df}\llabel{p-adic-comp-seq}
A $p$-adic integer is a compatible sequence
\[
(x_n)_{n\ge 1}
\]
where $x_n\in \Z/p^n\Z$ and such that $x_{n+1}\equiv x_n\pmod{p^n}$ for all $n$, i.e. $x_{n+1}$ maps to $x_n$ under the projection map $\Z/p^{n+1}\Z\to \Z/p^n\Z$.

The ring structure is defined by componentwise addition and multiplication. The ring of $p$-adic integers is denoted by $\Z_p$ and its fraction field is denoted by
\[
\Q_p=\Frac(\Z_p).
\]
\end{df}
%We can phrase this in the following more abstract way.
%\begin{df}
%Let $\{A_i\}$ and $\{\ph_k\}$ be a set of rings and homomorphisms between them\footnote{or objects and morphisms in any other abelian category}.
%We say a sequence of maps $\al_i:A\to A_i$ is compatible if for every map $\ph^i_j:A_i\to A_j$ in our set of maps,
%\[
%\al_j=\ph^i_j\circ \al_i.
%\]
%%directed set = set + preorder + upper bound
%
%The inverse limit
%\[
%A=\varprojlim A_i
%\]
%is the unique ring with compatible maps $\al_i$, satisfying the following universal mapping property:
%For every ring $B$ with compatible maps $\be_i$, there is a map $\ph:A\to B$ such that $\al_i=\be_i\circ\ph$ for every $i$.
%\end{df}
%%ex. \Z,m\le n if m|n
%We say the $A_i$ form a {\it filtered} system if 
%\begin{enumerate}
%\item for every $\ph^j_l:A_j\to A_l$ and $\ph^k_l:A_k\to A_l$ there exist $\ph^i_j:A_i\to A_j$ and $\ph^i_k:A_i\to A_k$, and
%\item for every $\ph^j_k,{\ph^j_k}':A_j\to A_k$ there exists $\ph^i_j$ such that $\ph^j_k\circ \ph^i_j = {\ph^j_k}'\circ \ph^i_j$.
%\end{enumerate}
%\begin{thm}\llabel{inv-limit-comp-seq}
%If the $A_i$ are filtered, then $\varprojlim A_i$ can be realized as the set of all sequences
%\[\set{
%(a_i)}{ a_i\in A_i,\,\ph^i_j(a_i)=a_j \text{ for all $\ph^i_j$}
%},\]
%and ring structure defined by componentwise addition and multiplication.
%\end{thm}
%\begin{proof}
%See 18.705 notes.
%\end{proof}
In light of Theorem~\ref{inv-limit-comp-seq}, we can phrase this definition in a more abstract way:
\[
\Z_p=\varprojlim \Z/p^n\Z
\]
where there are maps $\ph^m_n: \Z/p^{m}\Z\to \Z/p^n\Z$ given by projection whenever $m\ge n$. %For all practical purposes, the definition in Definition~\ref{p-adic-comp-seq} suffices, and we can take the conclusion of Theorem~\ref{inv-limit-comp-seq} as the definition of inverse limit. The UMP of inverse limit is useful, though.
\subsection{$p$-adics as completions}
We can give define a topology on $\Z$ by decreeing that it be invariant under translation and that a neighborhood base of $0$ be $\{p^n\Z,n\ge 0\}$. This is the same as the topology induced by the norm
\[
|a|_p=p^{-v}\text{ when }a=\frac{p^vb}{c},\,p\nmid b,c.
\]
\begin{df}[Alternate definition of $p$-adics]
$\Q_p$ is the completion of $\Q$ with respect to the $p$-adic norm.
\end{df}
We show the equivalence more generally in ().
\subsection{Units in $\Z_p$}
\begin{pr}
The group of units in $\Z_p$ is
\[
\Z_p^{\times}\cong
\begin{cases}
\Z_p\times \Z/(p-1)\Z,&p\ne 2\\
\Z_2\times \Z/2\Z\times \Z/2\Z,&p=2.\end{cases}
\]
\end{pr}
\begin{proof}
Note that
\[
\Z_p^{\times}=\varprojlim_{n\ge 1} (\Z/p^n\Z)^{\times}
\]
because any inverse modulo $p^n$ can be lifted to an inverse modulo $p^{n+1}$.

The proposition follows from taking inverse limits in Theorem~\ref{mult-structure}.
\end{proof}
\subsection{Monsky's Theorem*}
We use the $2$-adic valuation to prove the following theorem from combinatorial geometry. Surprisingly, no proof is known that does not use $p$-adics.
\begin{thm}
A unit square cannot be cut into an odd number of triangles of equal area.
\end{thm}
The idea of the proof is as follows.
\begin{enumerate}
\item Extend the 2-adic valuation to a nonarchimedean valuation on the real numbers.
\item Color each point in the plane one of three colors, based on the 2-adic valuation of the coordinates. We show that the sides of the square only have two colors, with the vertices alternating colors, and that a triangle of area $\rc m$ where $m$ is odd, cannot contain vertices of al three colors. The last facts depends crucially on the fact that the area formula for a triangle has a factor of $\rc 2$ in it.
\item By Sperner's Lemma (from graph theory), the coloring in such a subdivision is inconsistent. %if there were such a subdivision, one ``rainbow" triangle would have to exist, a contradiction.
\end{enumerate}
\begin{proof}
We omit the proof of the first item.\footnote{There is a way around it; see Proofs from the Book.} \fixme{Add a proof.} Assuming it, 
color the points of the plane in three colors depending on which of the following conditions is satisfied.
\begin{enumerate}
\item[(A)] $|x|_2<1,|y|_2<1$
\item[(B)] $|x|_2\ge 1,|x|_2\ge |y|_2$
\item[(C)] $|y|_2\ge 1,|y|_2> |x|_2$
\end{enumerate}
First, we show that if $(\De x,\De y)$ has color A, then translating by $(\De x,\De y)$ does not change the color of $A$. Indeed, consider 3 cases.
\begin{enumerate}
\item
$(x,y)$ is of color A. By the nonarchimedean property, we have
\[
|x+\De x|_2\le \max(|x|_2,|\De x|_2)\le 1, \,
|y+\De y|_2\le \max(|y|_2,|\De y|_2)\le 1,
\]
so $(x+\De x,y+\De y)$ is again of color A.
\item
$(x,y)$ is of color $B$. Since $|x|_2\ge 1>|\De x|_2$, we gave
 %and $|x|_2\ge |y|_2$ we have
\[
|x+\De x|_2=|x|_2\ge 1.
\]
Since $|x|_2\stackrel{.}{\ge} |y|_2$ and $1>|\De_y|_2$ we have
\[
|y+\De y|_2\le \max(|y|_2,|\De_y|_2)\stackrel{.}{\le} |x|_2=|x+\De x|_2.
\]
Hence $(x+\De x,y+\De y)$ is again of color B.
\item
$(x,y)$ is of color $C$. The proof is the same as above except $x,y$ are interchanged and there is strict inequality in the dotted inequalities above.
\end{enumerate}
Now suppose that $A,B,C$ are three points of those respective colors. By translation we may assume that $A=O$. Let $B=(x,y)$ and $C=(x',y')$. We have
\begin{align*}
|x|_2&\ge |y|_2\\
|y'|_2&> |x|_2\\
\implies |xy'|_2&>|x'y|_2.
\end{align*}
\begin{enumerate}
\item
$A,B,C$ cannot be collinear, as that would imply $xy'=x'y$.
\item
We show $A,B,C$ cannot form a triangle of area $\rc m$ for $m$ odd. The area is $\pm \rc 2(xy'-x'y)$, and we have
\[
|\rc 2 (xy'-x'y)|=\ab{\rc 2}_2|x|_2|y'|_2>1,
\]
while $\ab{\rc m}=1$.
\end{enumerate}
Next we establish the following combinatorial lemma.
\begin{lem}[Sperner's lemma]
Suppose $\cal P$ is a polygon that has been subdivided into triangles.
Define a {\it vertex} or {\it segment} to be a vertex or edge of one of these triangles, and say a segment is of type $\cal C_1\cal C_2$ if the endpoints are colored $\cal C_1$ and $\cal C_2$. We say a triangle is {\it rainbow} if it has vertices of all 3 colors.

Suppose every vertex of the subdivision is colored with either $\cal A$, $\cal B$, or $\cal C$, such that the following hold.
\begin{enumerate}
\item No outer edge of $\cal P$ contains vertices of all three colors.
\item There are are an odd number of segments of type $\cal A\cal B$ on the outer edges.
\end{enumerate}
Then $\cal P$ contains a triangle whose vertices are all different colors.
\end{lem}
\begin{proof}
We count the number of segments of type $\cal A\cal B$. In a monochromatic triangle the count is 0, in a two-colored triangle the count is 0 or 2, and in a three-colored triangle the count is 1. Let $n$ be the sum of the counts over all triangle. Every interior segment of type $\cal A\cal B$ is counted twice, as it is part of two triangles, so
\[
n=2i+e,
\]
where $i$ and $e$ denote the number of interior and exterior segments of type $\cal A\cal B$. Since $e$ is odd by assumption, $n$ is also odd. But this can only happen if there is a three-colored triangle.
\end{proof}
Now the points $O=(0,0)$, $X=(1,0)$, $Y=(1,1)$, and $Z=(0,1)$ are colored with $\cal A$, $\cal B$, $\cal B$, $\cal C$, respectively. We've shown that each side contains segments of at most 2 colors; segments of type $\cal A\cal B$ can only appear on side $OX$ and $XY$; in the former there must be an odd number (since $O,X$ are different colors) and in the latter there must be an even number. Thus the conditions of Sperner's Lemma are satisfied, and any subdivision must contain a rainbow triangle, which cannot have area $\rc m$ for $m$ odd.
\end{proof}
\section{Valuations}
\begin{df}
A \textbf{valuation} on a field $K$ is a function $|\cdot |:K\to \R$ such that
\begin{enumerate}
\item
$|x|\ge 0$ with equality only when $x=0$.
\item
$|xy|=|x||y|$.
\item
$|x+y|\le |x|+|y|$.
\end{enumerate}
If the stronger condition $|x+y|\le \max(|x|,|y|)$ holds, then $|\cdot|$ is \textbf{nonarchimedean}.
\end{df}
\begin{ex}
For a number field $K$, any embedding $\si:K\hra \C$ gives a valuation on $K$:
\[|a|:=|\si a|.\]
\end{ex}
\begin{ex}\llabel{p-adic-val}
The \textbf{$\mfp$-adic valuation} is
\[
|a|_{\mfp}=\prc{\mathfrak N \mfp}^{v_{\mfp}(a)}.
\]
In the special case $K=\Q$, $\mfp=(p)$, we have
\[
|a|_p=\prc{p}^{v_{\mfp}(a)}.
\]
\end{ex}
\begin{pr}\llabel{nonarch-crit}
A valuation is nonarchimedean if and only if it is bounded on $\Z$ (in fact, bounded by 1). Hence if $\chr(K)\ne 0$, then $K$ only has nonarchimedean valuations.
\end{pr}
\begin{proof}
If $|\cdot|$ is archimedean, then $|1+\cdots +1|\le |1|=1$, so $|n|\le 1$ for any $n\in \Z$.

Conversely, suppose that $|\cdot |$ is bounded on $\Z$, say by $C$. We have
\begin{align*}
|(a+b)^n|&=\ab{\sum_{k=0}^n \binom nk a^kb^{n-k}}\\
&\le \sum_{k=0}^n C |a|^k|b|^{n-k}\\
&\le C(n+1)\max(|a|,|b|)^n.
\end{align*}
Hence for all $n\ge 1$, $|a+b|\le (C(n+1))^{\rc n}\max(|a|,|b|)$. Taking $n\to \iy$ gives the result.
\end{proof}
\begin{pr}[Relationship between additive and multiplicative valuations]
Fix a base $b$. There is a correspondence between additive and multiplicative valuations, given by
\begin{align*}
|x|&=b^{-v(x)}\\
v(x)&=-\log_b(x).
\end{align*}
Different values of $b$ give equivalent valuations. If $v(K^{\times})$ is discrete in $\R$, then it is a multiple of a discrete valuation.
\end{pr}
We say $|\cdot|$ is discrete when $|K^{\times}|$ is a discrete subgroup of $\R_{>0}$.

\begin{pr}[Valuation defines subrings]\llabel{pr:val-OK}
Suppose $K$ is equipped with a nonarchimedean valuation. Then
\begin{enumerate}
\item
$A:=\set{a\in K}{|a|\le1}$ is a subring of $K$, with 
\item
$U:=\set{a\in K}{|a|=1}$ is a group, and 
\item
$\mfp:=\set{a\in K}{|a|<1}$ is a prime ideal.
%its unique maximal ideal.
\end{enumerate}
%Suppose $K$ is complete. 
%The valuation is discrete if and only if $\mm$ is principal; then $A$ is a DVR. In that case, if $\pi$ is such that $|\pi|<1$ is as large as possible, then $\pi$ generates $\mm$, and is called a \vocab{uniformizer}.
\end{pr}
\begin{proof}
Straightforward.
\end{proof}
%Suppose $K$ is a field with a discrete nonarchimedean valuation $|\cdot|$. Let $\pi$ be a local uniformizing parameter, i.e. the largest element of $K$ with $|\pi|<1$. Equivalently, $\pi$ generates the maximal ideal $\mm$ in the subring of $\pi$-integers.

\begin{pr}[Elementary properties of discrete valuations]\llabel{elem-prop-dv}
$\,$
\begin{enumerate}
\item
$|a+b|\le \max(|a|,|b|)$ with equality if $|a|\ne |b|$.
\item (``All triangles are isosceles.") If $d(c,b)<d(c,a)$ then $d(a,c)=d(a,b)$. (The longer side is the repeated one.)
\item If $a_1+\cdots +a_n=0$, then the maximum valuation of the summands must be attained for at least two of them.
\end{enumerate}
\end{pr}
\begin{proof}
Straightforward.
\end{proof}
\subsection{Equivalent valuations}
A valuation on $K$ defines a metric (and hence a topology) on $K$ by
\[
d(a,b)=|a-b|.
\]
For example, high powers of $p$ have small $p$-adic valuation, so numbers differing by high powers of $p$ are close together in the $p$-adic valuation.

\begin{pr}\llabel{val-equiv-crit}
Let $|\cdot|_1,|\cdot|_2$ be valuations on $K$, with the first being nontrivial. Then the following are equivalent.
\begin{enumerate}
\item
$|\cdot|_1,|\cdot|_2$ determine the same topology on $K$.
\item
If $|\al|_1<1$, then $|\al|_2<1$.
\item $|\cdot|_1=|\cdot |_2^a$ for some $a>0$.
\end{enumerate}
We say that $|\cdot|_1$ and $|\cdot|_2$ are \textbf{equivalent} if the above conditions hold.
\end{pr}
\begin{proof}$\,$

\noindent \underline{$(1)\implies (2)$}: 
Note $|\al|_j<1$ if and only if $|\al^n|_j=|\al|_j^n\to 0$, i.e. $\al^n$ converges to 0 in the topology of $\ad_j$. Since the topologies are the same,
\[
|\al|_1<1\iff \al^n\text{ converges to }0\iff |\al|_2<1.
\]
%Suppose $|\al|_1<1$. Then the sequence $\al^n$ converges to 0, since $|\al^n|_1=|\al|_1^n\to 0$. Since the topologies are the same, $|\al^n|_2=|\al|_2^n\to 0$, Hence $|\al|_2<1$.

\noindent \underline{$(2)\implies (3)$}:
Take $y$ so that $|y|_1>1$, and let $a=\frac{|y|_2}{|y|_1}$, so that $|y|_2=|y|_1^a$. We show that $|x|_2=|x|_1^a$ for all $x\in K$.
%Suppose that $|x|_2=|x|_1^b$. We need to show $a=b$. 

Suppose $|x|_1=|y|_1^{b_1}$ and $|x|_2=|y|_2^{b_2}$. We need to show $b_1=b_2$, i.e. so the following commutes.
\[
\xymatrix{
|x|_1\ar@{.>}[r]^{{}^{\wedge} a} & |x|_2\\
|y|_1\ar[u]^{{}^{\wedge}b_1} \ar[r]^{{}^{\wedge}a} & |y|_2\ar[u]^{{}^{\wedge}b_2}
}
\]
We approximate $b_1$ with rational numbers $\frac mn$. First suppose $b_1>\frac mn$. Then
\[
\ab{\frac{y^m}{x^{n}}}_1=|y|^{m-b_1n}<1
\]
so by hypothesis 
\[
|y|_2^{m-b_2n}=\ab{\frac{y^m}{x^{n}}}_2<1
\]
giving $b_2>\frac mn$. Similarly, if $b_1<\frac mn$, then the above argument with $\frac{x^n}{y^m}$ shows $b_2<\frac mn$. Since $\Q$ is dense in $\R$, we have $b_1=b_2$.\\

\noindent\underline{$(3)\implies (1)$}: The open ball of radius $r$ with respect to $\ad_1$ is the same as the open ball of radius $r^a$ with respect to $\ad_2$.
\end{proof}

\section{Places}
\begin{df}
A \textbf{place} is an equivalence class of nontrivial valuations on $K$.\footnote{Some books use ``prime" instead of ``place." We use the latter term to avoid confusion.} We denote by $V_K$ the set of places of $K$, by $V_K^0$ the set of nonarchimedean places and $V_K^{\iy}$ the set of archimedean places.
\end{df}

We aim to classify all places in a number field $K$. %To do this, we first classify all places of $\Q$, and then consider how places in a field extension correspond to places in the base field.
\begin{pr}\llabel{nonarch-vals}
Let $K/\Q$ be an algebraic extension. Then the places on $K$ are exactly the $\mfp$-adic valuations $|\cdot|_{\mfp}$ for $\mfp$ a prime ideal of $\cO_K$. %for some prime ideal $\mfp$ of $A:=\cO_K$. 
\end{pr}
\begin{proof}
Since $K$ is algebraic over $\Q$, an element $\al\in \cO_K$ satisfies a monic polynomial equation with coefficients in $\Z$:
\[
x^n+a_{n-1}x^{n-1}+\cdots +a_0=0.
\]
By Proposition~\ref{nonarch-crit}, $a_j\in \Z$ gives $|a_j|\le 1$.
By the nonarchimedean property,
\[|\al|^n=|a_{n-1}\al^{n-1}+\cdots +a_0|\le \max_{0\le m\le n-1}|a_{m}||\al^{m}|\le \max_{0\le m\le n-1}|\al|^m.\]
Hence $|\al|<1$. This means if $A$ is as in Proposition~\ref{pr:val-OK}---the ring of integers of $|\cdot|$, then $\cO_K\subeq A$. Let $\mP$ be the prime ideal corresponding to $|\cdot|$. Since $\mP$ is prime in $A$, $\mfp:=\mP\cap \cO_K$ is prime in $\cO_K$. Note $\mfp\ne (0)$ because if so $|\cdot|$ is trivial.

Now suppose $v_{\mfp}(y)=n$. Let $\pi\in \mfp\bs \mfp^2$ be a uniformizer. Then $(y\pi^{-n})$ is a fractional ideal; suppose ideals $\mfp_1,\ldots, \mfp_m$ appear in its factorization with exponents at least $-k$. Take $b\in \bigcap_{j=1}^n \mfp_j^k$. Then $(y\pi^{-n}b)$ is an integral ideal $(c)$ not divisible by $\mfp$. We have $c\in A\bs \mfp$. Writing $|\pi|=\prc{\mathfrak N\mfp}^a$, we have
\[
|y|=\ab{\fc cb}\ab{\pi^n}=\ab{\rc{\mathfrak N \mfp}}^n=|y|_{\mfp}^a.
\]
Moreover, two equivalent nonarchimedean valuations would have the same maximal ideals and hence correspond to the same prime $\mfp$.
\end{proof}
\begin{thm}[Ostrowski]\llabel{ostrowski}
The following is a list of all places on $\Q$.
\begin{enumerate}
\item
Archimedean: $|\cdot |_{\iy}$.\footnote{A stronger version of part 1 is as follows. Let $K$ be complete with respect to an archmimedean norm. Then $K=\R$ or $\C$, and the norm is the normal absolute value raised to a power in $(0,1]$.}
\item
Nonarchimedean: $|\cdot|_p$, where $p$ ranges over all primes.
\end{enumerate}
\end{thm}
\begin{proof}
Let $\ad$ be a valuation on $\Q$ and $m,n$ be integers greater than 1. To compare $|m|$ and $|n|$, we write $m$ in base $n$:
\[
m=a_rn^r+\cdots +a_0, \quad 0\le a_k\le n-1, \,a_r>0.
\]
Let $N=\max\{1,|n|\}$. Then by the triangle inequality,
\[
|m|\le \sum_{k=0}^r a_kN^k.
\]
Since $r\le \frac{\ln m}{\ln n}$, we get
\[
%|m|\le \pa{1+\frac{\ln m}{\ln n}}nN^{\frac{\ln m}{\ln n}}.
|m|\le \pa{1+\rc{N}+\rc{N^2}+\cdots }nN^{\frac{\ln m}{\ln n}}
\le 2nN^{\frac{\ln m}{\ln n}}
\]
Replacing $m$ by $m^t$ and taking the $t$th root gives
\[
|m|\le %\pa{1+\frac{t\ln m}{\ln n}}
(2n)^{\rc t}N^{\frac{\ln m}{\ln n}}.
\]
Taking $t\to \iy$ %and noting the first two factors approach 1, we get
gives
\begin{equation}\llabel{ostrowski1}
|m|\le N^{\frac{\ln m}{\ln n}}.
\end{equation}
Consider two cases.
\begin{enumerate}
\item For all integers $n>1$, $|n|>1$. Then~(\ref{ostrowski1}) gives $|m|^{\rc{\ln m}}\le |n|^{\rc{\ln n}}$. By symmetry, we get $|m|^{\rc{\ln m}}= |n|^{\rc{\ln n}}$. Since this is true for all $m$ and $n$, $|n|^{\rc{\ln n}}=c$ is constant, i.e.
\[
|n|=c^{\ln n}=n^{\frac{\ln n}{\ln c}}
\]
for all $n\in \Z$. Since $\Z$ generates $\Q$ as a group, we get that $\ad$ is equivalent to the standard archimedean valuation.
\item For some $n>1$, $|n|\le 1$. Then~(\ref{ostrowski1}) shows that $|m|\le 1$ for all $m>1$. Thus by Proposition~\ref{nonarch-crit}, $\ad$ is nonarchimedean.
The nonarchimedean valuations are given by Proposition~\ref{nonarch-vals}.\qedhere
%Let $A$ be the ring of integers of $\ad$ and $\mm$ its maximal ideal. Since $\mm$ is prime in $A$, $\mm\cap \Z$ is prime in $\Z$. Suppose it equals $(p)$. Write $|p|=\prc{p}^a$. 
%Any $y\in \Q$ can be written as $np^e$ where $n$ has neither $p$ in the numerator or denominator. Note $n$ is a fraction of elements in $A\bs \mm=\set{x}{|x|=1}$ so has absolute value 1. Then
%\[
%|y|=|n||p^e|=\prc{p}^{ae}=|y|_p^a.
%\]
%Moreover, two equivalent valuations would have the same $A$ and hence correspond to the same prime $p$.\qedhere
\end{enumerate}
\end{proof}
Later on we will return to the question of finding all valuations on an extension of $\Q$ (Theorem~\ref{vals-on-number-fields}).
%Let $K$ be a number field. 
%We know that for each prime ideal $\mfp$ there corresponds a $\mfp$-adic valuation, and that these valuations are inequivalent by Proposition~\ref{val-equiv-crit}. However, there are more valuations given by real and complex embeddings. (WHY DO WE CARE?)
%We now classify the places of $K$.

%That silly thing with the normalized valuations for $\C$...

%Stuff got moved to next chapter b/c it requires completions.

%Lying over.
%Normalized valuation.
%Lemma on extension.
\fixme{Generalize the nonarchimedean stuff to number fields.}
%\begin{thm}[Ostrowski]
%The only fields (up to isomorphism) that are complete with respect to a nonarchimedean valuation are $\R$ and $\C$.
%\end{thm}
\subsection{Approximation}
\begin{thm}[Weak approximation theorem]%, Artin and Whaples]
\llabel{thm:weak-approx}
Let $v_1,\ldots,v_n$ be all the places of $K$, with valuations $\ad_1,\ldots, \ad_n$.  
The map
\[
\phi:K\to \prod_{j=1}^N K_{v_j}
\]
induced by the inclusions $K\hra K_{v_j}$ 
has dense image.

In other words, given $a_1,\ldots, a_n\in K$, for any $\ep>0$, there exists $a\in K$ such that 
\[
|a-a_j|_j<\ep\text{ for all $j$}.
\]
\end{thm}
\begin{proof}
\noindent{\underline{Step 1:}} We show that there exists $a$ such that 
\begin{align}\llabel{weak-approx-1}
\ab{a}_{1}&>1.\\
\nonumber\ab{a}_{j}&<1,\quad i=2,\ldots, n.
\end{align}

We induct on $n$. For $n=2$, note that by 
Proposition~\ref{val-equiv-crit}(2), we can find $b,c$ so that
\begin{align*}
|b|_{1}&< 1,&|b|_{2}&\ge1\\
|c|_{1}&\ge1,&|c|_{2}&<1.
\end{align*}
Now take $a=\fc cb$.

For the induction step, suppose we've found $b$ so that~(\ref{weak-approx-1}) holds for $n-1$. Choose $c$ so that 
\[
|c|_1>1,\quad |c|_n<1;
\]
we will use it to ``correct" $|b|_n$ as necessary. Consider three cases.
\begin{enumerate}
\item $|b|_n<1$: We can let $a=b$.
\item $|b|_n=1$: Let $a=b^r c$, for large enough $r$.
This works because
\[
\lim_{r\to \iy}\ab{b^rc}_j=\begin{cases}
\iy, &j=1\\
0, &2\le j\le n-1\\
|c|_n<1, &j=n.
\end{cases}
\]
%is large enough so that $|b^r|_1<\ab{\rc c}_1$ and $|b^r|_j>\ab{\rc c}_j$ for $2\le j\le n-1$. Note $|b^r c|_n=|c|_n<1$, as needed.
\item $|b|_n>1$: First note that from $1-|a^r|\le|1+a^r|\le1+|a^r|$ we get
\begin{equation}\llabel{weak-approx-expr}
\lim_{r\to \iy}\ab{\fc{x^r}{1+x^r}}=\begin{cases}
0, &|x|<1\\
1,&|x|>1.
\end{cases}
\end{equation}
Let $a=\fc{cb^r}{1+b^r}$, for large enough $r$. This works because the above gives
\[
\lim_{r\to \iy}\ab{\fc{cb^r}{1+b^r}}_j=\begin{cases}
|c|_1>1, &j=1\\
0, &2\le j\le n-1\\
|c|_n<1, &j=n.
\end{cases}
\]
\end{enumerate}
%$|\cdot|_{v_1}\sim|\cdot|_{v_2}$ iff for every $\be\in K$, $\ab{\be}_{v_1}<1$ implies $\ab{\be}_{v_2}<1$
\noindent\underline{Step 2:} Now we show that there are points in the image of $\phi$ arbitrarily close to $(1,0,\ldots,0)$. Indeed, choosing $a$ as in step 1, we have by~(\ref{weak-approx-expr}) that
\[
\lim_{r\to \iy}\ph\pf{a^r}{1+a^r}=(1,0,\ldots, 0).
\]
\noindent\underline{Step 3:} From step 2, choose $b_j$ sufficiently close to $(0,\ldots, 0, \underbrace{1}_j,0,\ldots, 0)$. Let
\[
a=\sum_{j=1}^n a_nb_n
\]
to find $\ph(a)$ can be arbitrarily close to $(a_1,\ldots, a_n)$.
\end{proof}
Note that if we include only the finite places, then this follows from the %CITE 
Chinese remainder theorem.
\fixme{Add a brief explanation of strong approximation.}
\section{Completion}
\begin{df}
Let $K$ be a field with valuation $|\cdot|$. The \textbf{completion} of $K$, denoted $\hat{K}$ is the field containing $K$ (i.e. there is a injection $K\hra \hat{K}$ preserving valuation) satisfying the following properties.
\begin{enumerate}
\item $\hat{K}$ is complete in its topology.
\item (UMP) For any homomorphism $\ph$ from $K$ to a complete field $L$, there exists a unique homomorphism $\hat{K}\to L$ making the following commute.
\[
\xymatrix{
\hat K\ar@{.>}[r] & L\\
 K\ar@{^(->}[u]\ar@{^(->}[ur]&.
}
\]
I.e., {\it $\hat{K}$ is the smallest complete field containing $K$.}
\end{enumerate}
\end{df}
\begin{proof}[Proof of existence]
For existence, let $\hat{K}$ be the set of equivalence classes of Cauchy sequences in $K$, and deem two sequences $\{a_n\}$ and $\{b_n\}$ equivalent if $\lim_{n\to \iy}|a_n-b_n|=0$. Define $K\hra \hat K$ by sending $a$ to $(a,a,\ldots)$. Extend the valuation by letting defining the norm of a $\{a_n\}$ to be $\lim_{n\to \iy} |a_n|$. 
See any book on real analysis for the details. 

For the second part, given a sequence $\{a_n\}\in \hat K$, map it to $\lim_{n\to \iy}\ph(a_n)\in L$. 
Uniqueness follows from the universal property.
\end{proof}
\subsection{Completions of archimedean fields; Classification of places}
\begin{thm}[Ostrowski]\llabel{ostrowski2}\llabel{vals-on-number-fields}
The only complete archimedean fields, up to isomorphism of valued fields and equivalence of valuation, are $\R$ and $\C$.
\end{thm}
\begin{proof}
\fixme{(See Neukirch, p. 124. I don't quite see the motivation for this proof.)}
Suppose $K$ is a complete archimedean field. Then on $\Q$, by the first Ostrowski Theorem~\ref{ostrowski}, $\ved$ is equivalent to the usual valuation $\ved_\iy$. By Proposition \ref{val-equiv-crit}, by raising to a power we may assume it is exactly the usual valuation. We must have $\wh Q=\R\subeq K$.

We show every element $\xi\in K$ satisfies a quadratic equation over $\R$. This is equivalent to saying that $\min_{z\in \C} f(z)=0$ where
\[
f(z)=|\xi^2-(z+\ol z) \xi +z \ol z|.
\]
Let $m=\min |f(z)|$; by way of contradiction suppose $m\ne 0$. Note $S=\set{z}{f(z)=m}$ is compact; let $z_0$ be an element of $S$ of maximal absolute value. 

The idea is the following: perturb (the $n$th power of) $\xi^2-(z_0+\ol{z_0}) \xi +z_0 \ol{z_0}$ so that the valuation is very close to $m^n$; taking $n\to \iy$ we force all roots to have absolute value $m$; however, we can ensure one of the roots has larger absolute value, a contradiction.

Consider the polynomial
\[
G(x) = (x^2-(z_0+\ol{z_0})x+z_0\ol{z_0})^n -(-\ep)^n.
\]
We have by the triangle inequality
\[
G(\xi) \le m^n +\ep^n \implies G(\xi)\le (m^n+\ep^n)^2.
\]
On the other hand, note $|G(x)|^2$ factors into quadratics so by definition of $m$, we have $|G(\xi)|^2\ge m^{2n}$. We can do better: let $z_1$ be the root of $x^2-(z_0+\ol{z_0})x+z_0\ol{z_0}+\ep=0$; we have $|z_1|>m=|z_0|$.  Now $z_1$ is a root of $G(x)$, so one of the quadratic factors of $|G(x)|^2$ is $x^2-(z_1+\ol{z_1})x+z_1\ol{z_1}$ which by the assumption on $z_0$, must be $>m$ at $x=\xi$. We get 
\[
m^{2n-1}(\xi^2-(z_1+\ol{z_1})\xi+z_1\ol{z_1})
\le |G(\xi)|^2
\le (m^n+\ep^n)^2
\]
which is a contradiction for $n$ large enough.
\end{proof}
We can now finish our classification of places on $K/\Q$.
\begin{thm}[Classification of places of $K$]
Let $K$ be a number field. There is exactly one place of $K$ for each
\begin{enumerate}
\item
prime ideal $\mfp$,
\item 
real imbedding, and
\item conjugate pair of complex embeddings.
\end{enumerate}
%We will often denote the places corresponding to primes with the primes themselves.
\end{thm}
The valuations corresponding to prime ideals, i.e. $\mfp$-adic valuations, are called \textbf{finite places}, while the those corresponding to real and complex embeddings are called \textbf{infinite (real or complex) places}.
\begin{proof}
The nonarchimedean valuations of $K$ are given by Proposition~\ref{nonarch-vals}, while each archimedean valuation $v$ corresponds to an embedding (respecting valuations)
\[
K\hra K_v\cong \R\text{ or }\C,
\]
the isomorphism coming from Theorem~\ref{ostrowski2}. Note that complex conjugate embeddings give the same valuation.
\end{proof}
\begin{cor}
Let $L/K$ be extensions of number fields. If $v$ is a place corresponding to a prime $\mfp$ of $K$, then the places $w\mid v$ in $L$ correspond to primes $\mP\mid \mfp$. If $v$ is a place of $K$ corresponding to an embedding $\si:K\to \R\text{ or }\C$, then the places $w\mid v$ correspond to of $\si$ to $L$.
\end{cor}
\subsection{Completions of nonarchimedean fields}
Suppose $K$ is a field with a discrete nonarchimedean valuation $|\cdot|$. Since $K$ is dense in $\hat K$ and 
\[|K\bs\{0\}|=\set{|\pi|^m}{m\in \Z}\]
is discrete in $\hat K$, we get $|K|=|\hat K|$.

Proposition~\ref{pr:val-OK} becomes much nicer when we look at the completion $\hat K$.
\begin{pr}[Valuation defines subrings]\llabel{pr:val-OK}
Let $\hat K$ be complete with a nonarchimedean valuation. Then
\begin{enumerate}
\item
$A = \cO_{\hat K}:=\set{a\in \hat K}{|a|\le1}$ is a subring of $\hat K$, with 
\item
$U = \cO_{\hat K}^{\times}:=\set{a\in \hat K}{|a|=1}$ the unit group, and 
\item
$\mm:=\set{a\in K}{|a|<1}$ its unique maximal ideal
%its unique maximal ideal.
\end{enumerate}
Let $\pi$ be a \vocab{uniformizer}, i.e. the largest element of $K$ with $|\pi|<1$. %(i.e., $\pi$ is a uniformizer in the completion $K_{\pi}$). Equivalently, $\pi$ generates the maximal ideal $\mm$ in the subring of $\pi$-integers.
Then $\pi$ generates $\mm$.
\end{pr}

\begin{pr}\llabel{pi-adic-expansion}
Suppose $K$ is a field with a discrete nonarchimedean valuation $|\cdot|$. 
Let $S$ be a set of representatives for $A/\mm$. Then every element of $\hat K$ has a unique expression in the form
\[
\sum_{n\ge N} a_n \pi^n.
\]
(More precisely, the sum represents $\lim_{m\to \iy}\sum_{n=N}^{m}a_n\pi^n$.) 
The norm is given by
\[
\ab{\sum_{n\ge N} a_n \pi^n}=|\pi|^{N},\quad a_{N}\ne0.
\]
\end{pr}
In other words, we can write elements of $\hat{K}$ as ``numbers with infinite $\pi$-expansions going off to the left," as we saw in section~\ref{padic-exs}.
\begin{proof}
Suppose $|x_n|=|\pi|^n$. Then $x\pi^{-n}\in A$, so there exists $s\in S$ with $x\pi^{-n}-a\in \mm$. Then 
\[
|x_n-a_n\pi^n|=|\pi^{n+1}|;
\]
set $x_{n+1}=x_n-a_n\pi^n$. Starting this process at $x=x_N$, we get 
\[
x=\sum_{n\ge N} a_n\pi^n.
\]
We have $x-x_N\pi^N\in \pi^{-N+1}$ so $|x|=|\pi|^{N}$.
\end{proof}
We have a surprising result that is not true in the archimedean case.
\begin{pr}\llabel{pr:sum-nonarch}
Let $K$ be a field complete with respect to a discrete nonarchimedean valuation. 
Suppose $x_n\to 0$. Then $\sum_{n=1}^{\iy} x_n$ converges.
\end{pr}
\begin{proof}
We show $s_n=\sum_{m=1}^n x_n$ be the sequence of partial sums. We show $s_n$ is Cauchy. For each $\ep>0$ there is $N$ such that for $n\ge N$, $|x_n|\le \ep$. Then by the nonarchimedean property, for all $n'>n\ge N$, $|s_{n'}-s_n|=|x_{n+1}\cdots +x_{n'}|\le \ep$. Thus $s_n$ is Cauchy and hence converges.
%Let $\{s_n\}_{n\ge 1}$ be a Cauchy sequence in $K$. Let
%\[
%s_n=\sum_{m\gg -\iy} a_n(m)\pi^j;
%\]
%where $a_n(m)\in S$; this sum is finite. We have
%\[
%|s_{n_1}-s_{n_2}|=p^{-\min\set{m}{a_{n_1}(m)\ne a_{n_2}(m)}}.
%\]
%Hence for each $m$, $a_n(m)$ eventually stabilizes, say at $a_n$. Then
%\[
%\lim_{n\to \iy} s_n=\sum_{n\gg -\iy} a_n\pi^n.
%\]
\end{proof}
Thus we have two ways to think of a $\mfp$-adic valuation.

We've thought of it analytically: give $K$ a metric and take the completion of $K$ as a metric space to obtain $\wh K$. 
The algebraic notion of completion is the following: the completion of a ring $A$ with respect to an ideal $\mfp$ is defined as $\hat{A}=\varprojlim_{n\ge 0} A/\mfp^n$.

To connect up the analytic and algebraic definitions of the completion, note that 
%The algebraic way is that the completion of a ring $A$ with respect to an ideal $\mm$ is defined as 
$\hat{A}=\varprojlim_{n\ge 0} A/\mfp^n$ is the completion of $\hat{A}$ with the topology given by the neighborhood base $\{\mfp^n\}_{n\ge 0}$; this gives $A$ the structure of a topological ring.

Now if $K$ is a field with a discrete nonarchimedean valuation, then we can define $A$ and $\mfp$ as in Proposition~\ref{pr:val-OK}. The topology given by the valuation is exactly the topology that has the neighborhood base of 0 being $\{\mfp^n\}_{n\ge 0}$. This shows the following.
\begin{pr}
Let $K$ be a field with a discrete nonarchimedean valuation. Then defining $A,\mfp$ as in Proposition~\ref{pr:val-OK}, 
%The following are equivalent characterizations of $\hat{K}$.
%\begin{enumerate}
%\item The completion of $K$ under $|\cdot|$, i.e. under the $\pi$-adic topology.
%\item The set of sums
%\[
%\sum_{n\ge N} a_n \pi^n
%\]
%where the $a_n$ are chosen from a set $S$ of representatives for $A/\mm$, and
%\[
%\ab{\sum_{n\ge N} a_n \pi^n}=|\pi|^{-n_0},\quad a_{N}\ne0.
%\]
%\item (*) The inverse limit $\varinjlim R/\mm^n$.
%\end{enumerate}
\[
\hat{K}=\Frac(\varprojlim A/\mfp^n).
\]
\end{pr}
%\begin{proof}
%Indeed, elements of $\cO_{\hat{K}}$ are given by $\sum_{n\ge 0}a_n\pi^n$; we can think of this as an element in $A/
%\end{proof}

\begin{df}
Define the exponential function as a power series
\[
e^x=\sum_{n=1}^{\iy} \frac{x^n}{n!}=1+x+\frac{x^2}{2!}+\cdots.
\]
\end{df}
We investigate the convergence of $e^x$. 
\begin{pr}
$e^x$ converges for $v_p(x)>\rc{p-1}$.
\end{pr}
\begin{proof}
Writing $a=a_rp^r+\cdots +a_0$ in base $p$, we find by Example~\ref{factorial-order} that
\[
v_p(n!)=\frac{n-\sum_{i=0}^r a_i}{p-1}.
\]
Hence 
\[v_p\pf{x^n}{n!}=nv_p(x)-\frac{n-\sum_{i=0}^r a_i}{p-1}=n\pa{v_p(x)-\rc{p-1}}+o(n).\]
By Proposition~\ref{pr:sum-nonarch}, $e^x$ converges if and only if $v_p\pf{x^n}{n!}\to -\iy$; this gives the result.
\end{proof}

\section{Hensel's lemma}
The following is the first version of Hensel's lemma for $\pi$-adics. 

\fixme{Also state the elementary version of Hensel's lemma for reference.}
\index{Hensel's lemma}
\begin{lem}[Hensel's lemma, I]\llabel{hensel1}
Let $A$ be a complete local ring, $f(X)\in A[X]$, and $a_0$ be a simple root of $f(X)$ modulo $\pi$, i.e. $f(a_0)\equiv 0\pmod{\pi}$ and $f'(a_0)\nequiv 0\pmod{\pi}$. Then there exists a unique root $a$ of $f(X)$ with $a\equiv a_0\pmod{\pi}$.
\end{lem}
Note this can be generalized as follows: Suppose $f(a_0)\equiv 0\pmod{\pi^n}$ and $v_{\pi}(f'(a_0))=k<n$. Then there is a unique root $a$ of $f(X)$ with $a\equiv a_0\pmod{\pi^{n-k}}$. The proof is the same, and is left to the reader!
\begin{proof}
%The proof is by induction.
We find zeros of $f(X)$ modulo higher and higher powers of $\pi$.

Using induction, we find %$b_0,\ldots, b_n$ such that $a_n=b_n\pi^n+\cdots +b_0$ and
$a_n$ satisfying 
\[
f(a_n)\equiv 0\pmod{\pi^{n+1}}.
\]
%For the base case, let $a_0=b_0$. 
The base case holds by hypothesis. For the induction step, note that by Taylor expansion of polynomials,
\begin{align*}
f(a_n+h\pi^{n+1})&=f(a_n)+h\pi^{n+1}f'(a_n)+\cdots\\
&\equiv f(a_n)+h\pi^{n+1}f'(a_n)\pmod{\pi^{n+2}}.
\end{align*}
Since $f'(a_n)\nequiv 0\pmod{\pi}$ and $f(a_n)\equiv 0\pmod{\pi^{n+1}}$, we can choose $h$ so that this is 0 modulo $\pi^{n+1}$. (Explicitly, $h=-\frac{f(a_n)}{\pi^{n+1}}\cdot \rc{f'(a_n)}$.) We let $a_{n+1}=a_n+h\pi^{n+1}$. By construction, the sequence $a_n$ converges; let $a$ be its limit. Since $a\equiv a_n\pmod{\pi^n}$, we get $f(a)\equiv f(a_n)\equiv 0\pmod{\pi^{n+1}}$ for all $n$, and therefore $f(a)=0$.
\end{proof}
The first form of Hensel's lemma tells us about lifting a root $a_0$ of $\ol f$ ($f$ modulo $\pi$) to a root $a$ of $f$ in $K$. We can think of this as lifting a linear factor $x-a_0$ of $\ol f$ to a linear factor $x-a$ of $f$. A stronger form of Hensel's lemma says that we can in fact lift any factor of $f$ to one of $\ol{f}$.
\begin{thm}[Hensel's lemma, II]\llabel{hensel2}
Let $k$ be the residue field of $A$ and $f$ be a monic polynomial. If $\ol f=g_0h_0$ where $g_0$ and $h_0$ are monic and relatively prime, then $f=gh$ for some $g$ and $h$ such that $\ol{g}=g_0$ and $\ol{h}=h_0$.

If $\ol f=g_1\cdots g_n$ is the complete factorization of $\ol f$ in $k[X]$, then the complete factorization of $f$ in $K[X]$ is $f=f_1\cdots f_n$ where $\ol{f_j}=g_j$.
\end{thm}
\fixme{Add a statement about uniqueness.}
\begin{proof}
First we need the following lemma, which tells us that if the reductions of polynomials are relatively prime, then so are the original polynomials.
\begin{lem}\llabel{lem:naka-bezout}
Let $A$ be a local ring with residue field $k$. If $g,h\in A[X]$ are such that $\ol{g}$ and $\ol{h}$ are relatively prime, then $g$ and $h$ are relatively prime in $A[X]$ and there exist polynomials $u, v$ with $\deg u<\deg h$ and $\deg v<\deg g$ such that 
\[
ug+vh=1.
\]
\end{lem}
\begin{proof}
Since $\ol g$ and $\ol h$ are relatively prime in $k[X]=(A/\mm)[X]$, $(\ol g, \ol h)=A[X]/\mm A[X]$ and $(g,h)+\mm A[X]=A[X]$. 
Nakayama's lemma says that if $A$ is a local ring with maximal ideal $\mm$ and $M\supeq N$ are $A$-modules with $M=N+\mm M$, then $M=N$. Apply this with $M=A[X]$, which is finitely generated over $A$ to get $A[X]=(g,h)$. 
%Since $A[X]/\mm A[X]$ is finitely generated (on account of $g,h$ being monic), %see ``$\implies$" in Lemma~\ref{criterion-for-integrality}
This means we can choose $u,v$ such that $ug+vh=1$. To satisfy the degree condition, drop all terms with higher degree.
\end{proof}
We proceed as in the proof of Theorem~\ref{hensel1}. Suppose we have found $g_n$ and $h_n$ such that
\[
f\equiv g_nh_n\pmod{\pi^{n+1}}.
\]
We have
\[
(g_n+v\pi^{n+1})(h_n+u\pi^{n+1})\equiv g_nh_n+(ug_n+vh_n)\pi^{n+1}\pmod{\pi^{n+2}}.
\]
By Lemma~\ref{lem:naka-bezout} we can choose $u$ and $v$ such that the above is congruent to $f$ modulo $\pi^{n+2}$. Again let $g_{n+1}=g_n+v\pi^{n+1}$, $h_{n+1}=h_n+u\pi^{n+1}$, and take the limit as $n\to \iy$.

The second part follows from induction. Note $f=f_1\cdots f_n$ is the complete factorization because any factorization of $f$ gives a factorization for $\ol f$.
\end{proof}
\begin{df}
A \textbf{henselian field} is a field with nonarchimedian valuation $v$ which satisfies Hensel's Lemma (with $\mfp$ the maximal ideal corresponding to $v$).
\end{df}
Hensel's lemma says that a field that is complete with respect to a discrete valuation is henselian.
\section{Extending valuations}
\begin{thm}[Extending discrete valuations]\llabel{extend-discrete-valuations}
Let $K$ be henselian %{\it complete} with respect to a discrete valuation $|\cdot |_K$ 
and let $L/K$ be finite separable of degree $n$. Then $|\cdot|_K$ extends uniquely to a discrete valuation $|\cdot|_L$ on $L$, given by
\[
|\be|_L=|\nm_{L/K}\be|_K^{\rc  n}.
\]
\end{thm}
\begin{proof}
Neukirch, pg. 131-132.
\end{proof}
\begin{df}
Let $K$ be henselian. Let $v:K^{\times}\to \Z$ be the corresponding additive valuation, extended to $K^{\text{al}\times}\to \Q$. Given a polynomial
\[
f(X)=X^n+a_{n-1}X^{n-1}+\cdots +a_0\in K[X]
\]
define the \textbf{Newton polygon} of $f(X)$ to be the lower convex hull\footnote{draw the convex hull, and remove the segments joining $(0,v(a_0))$ and $(n,0)$ from the top}
 of
\[
P_i:=(i,v(a_{i})).
\]
\end{df}
%[EXAMPLE HERE.]
\begin{pr}
Suppose the bottom of the Newton polygon has segments of $x$-length $n_i$ and slope $-s_i$. Then 
\begin{enumerate}
\item $f(X)$ has exactly $n_i$ roots $\al\in K^{\text{al}}$ with $v(\al)=s_i$, and
\item $f_i(X)=\prod_{v(\al_i)=s_i} (X-\al_i)$ has coefficients in $K$.
\end{enumerate}
\end{pr}
\begin{proof}
We prove the following statement by induction: if $f(X)=\prod (X-\al_j)\in \ol{K}[X]$ and exactly $n_i$ of the roots $\al_j$ have order equal to $s_i$, then the Newton's polygon of $f(X)$ has a segment of slope $-s_i$ and $x$-length $n_i$.

The case $n=1$ follows since the only line segment on the bottom joins $(0, v(\al_i))$ and $(1,0)$. Now suppose the claim proved for $n$. Consider
\[
g(X)=(X-\al)f(X)=\sum_{k=0}^{n+1} (a_{k-1}-\al a_{k})X^k
\]
(where nonexistent coefficients are set to 0). 
Let $t=v(\al)$. 
Let $k_0$ be the point such that the slopes of the line segments of Newton's polygon $N$ for $k<k_0$ are $s\le -t$, and such that the slopes of the line segments of $N$ for $k>k_0$ are greater than $s>-t$. Let
\begin{align*}
d_k&=v(a_k)\\
\ell_k&=\text{$y$-value of intersection of $N$ with $x=k$}\\
d_k'&=v(a_{k-1}-\al a_{k})\\
\ell_k'&=\begin{cases}
\ell_k+t,&0\le k\le k_0\\
\ell_{k-1}&k_0<k\le n.
\end{cases}
\end{align*}
Let $N'$ be the broken line formed by joining $(k,\ell_k')$. $N'$ consists of segments of the same slopes as $N$, plus one more segment of slope $-t$ and $x$-length 1, in increasing order. It suffices to show that $N'$ is the lower convex hull of the points $(k,d_k')$.

Here is an example with $p=5$, $f(X)=(X-5)(X-10)(X-15)(X-125)$ and $\al=25$.\footnote{Of course, $f$ does not have to split over $\Q[X]$ and the valuations don't have to be integers.}
\[
\setlength{\unitlength}{.5cm}
\begin{picture}(5,8)
\linethickness{0.5mm}
{\color{Gray}
\put(0,0){\line(1,0){5}}
\put(0,0){\line(0,1){8}}
\linethickness{0.15mm}
\multiput(1,0)(1,0){5}{\line(0,1){8}}
\multiput(0,1)(0,1){8}{\line(1,0){5}}}
\put(0,6){\line(1,-3){1}}
\put(1,3){\line(1,-1){3}}
\put(0,8){\line(1,-3){1}}
\put(1,5){\line(1,-2){1}}
\put(2,3){\line(1,-1){3}}
%\multiput(5,0)(-.5,.5){6}{\line(-1,1){.4}}
%This segment isn't showing up for some reason, I don't know why.
%\multiput(0,8)(.5,-1.5){2}{\line(1,-3){.4}}
%\multiput(1,5)(.5,-1){2}{\line(1,-2){.4}}
\end{picture}
%\setlength{\unitlength}{.25cm}
%\begin{picture}(10,16)
%\linethickness{0.5mm}
%{\color{Gray}
%\put(0,0){\line(1,0){10}}
%\put(0,0){\line(0,1){16}}
%\linethickness{0.15mm}
%\multiput(2,0)(2,0){5}{\line(0,1){16}}
%\multiput(0,2)(0,2){8}{\line(1,0){10}}}
%\multiput(4,6)(1,-1){6}{\line(1,-1){.8}}
%\put(0,12){\line(1,-3){2}}
%\put(2,6){\line(1,-1){6}}
%\multiput(0,16)(1,-3){2}{\line(1,-3){.8}}
%\multiput(2,10)(1,-2){2}{\line(1,-2){.8}}
%%\multiput(2,4)(.5,-.5){2}{\line(1,-1){.4}}
%\end{picture}
\]

Consider 2 cases. We will use 
\[
d_k'=v(a_{k-1}-\al a_{k})\ge 
\min(v(a_{k-1}),v(\al a_{k}))=\min(d_{k-1},d_k+t),
\]
with equality holding if $d_{k-1}\ne d_k+t$.
\begin{enumerate}
\item $k\le k_0$: We have
\begin{align*}
d_{k-1}& \ge \ell_{k-1}\stackrel{(*)}{\ge} \ell_k+t=\ell'_k,\\
d_k+t& \ge \ell_k+t= \ell'_k
\end{align*}
where in (*) we use the fact that the slope of the segment $(k-1,\ell_{k-1})(k,\ell_k)$ is at most $-t$. 
Hence $(k,d_k')$ lies above $N'$. Now suppose $(k,d_k)$ lies on a corner of $L$ (excluding $k=k_0$). Then $d_k=\ell_k$ and inequality holds in (*):
\[
d_{k-1}>\ell_k+t=\ell'_k=d_k+t
\]
so $d_k'=\ell'_k$ and $(k,d_k')$ lies on $N'$.
\item $k>k_0$: We have
\begin{align*}
d_{k-1}&\ge\ell_{k-1}=\ell_k'\\
d_{k}+t&\ge \ell_k+t\stackrel{(*)}{>} \ell_{k-1}=\ell_k'.
\end{align*}
where in (*) we use the fact that the slope of the segment $(k-1,\ell_{k-1})(k,\ell_k)$ is greater than $-t$. Hence $(k,d_k')$ lies above $(k,\ell_k')$. Now suppose $(k-1,d_{k-1})$ lies on a corner of $L$. Then $d_{k-1}=\ell_{k-1}$ so
\[
d_{k}+t\ge\ell_k+t>d_{k-1}=\ell_{k-1}=\ell_k',
\]
showing $d_k'=\ell_k'$ and $(k,d_k')$ lies on $N'$.\qedhere
\end{enumerate}
\end{proof}
%
\section{Places as Galois orbits}
Here is an alternate definition of a place.
\begin{df}
Let $(K,v)$ be a field with valuation and $L/K$ be an extension. A \textbf{place} on $L$ over $v$ is a $G(\ol K_v/K_v)$-orbit on $\Hom_K(L,\ol K_v)$.
\end{df}
\begin{ex}
Let $K=\R$, and $L$ a finite extension of $K$. Then the places of $L$ over $\R$ are just $\Hom_K(L,\R)$, the real embeddings of $L$, and the complex places are just $G(\C/\R)\bs\Hom_K(L,\C)$, i.e. pairs of complex conjugate embeddings.
\end{ex}
We show this is equivalent to our previous definition.
\begin{thm}
%Assume...
Let $K$ be a number field. 
There is a bijective correspondence between equivalence classes of valuations $w\mid v$, $v$ on $K$, and $G(\ol K_v/K_v)$-orbits on $\Hom_K(L,\ol K_v)$:
\[
\set{w\mid v}{w\in M_L}\xrightarrow{\cong} G(\ol K_v/K_v)\bs\Hom_K(L,\ol K_v).
\]

Letting $\ol v$ be the unique extension of $v$ to $\ol K_v$, 
the embedding $\tau:L\hra \ol K_v$ is associated to the valuation $\ad_{\ol{v}}$ restricted to $L$.
\end{thm}
\begin{proof}
\fixme{Add me.}
\end{proof}
%
\section{Krasner's lemma and consequences}
The following shows a surprising difference between local and global fields.
\index{Krasner's lemma}
\llabel{Krasner's lemma}
\begin{lem}[Krasner's lemma]\llabel{krasner}
Let $K$ be complete with respect to a nonarchimedean valuation $|\cdot|$, and extend $|\cdot|$ to an algebraic closure $K^{\text{al}}$. Let $\al,\be\in K^{\text{al}}$. If $\be$ is separable over $K(\al)$, and
\begin{equation}\llabel{belong-ineq}
|\be-\al|<|\be'-\be|
\end{equation}
for any conjugate $\be'\ne \be$ of $\be$ over $K$, then $\be\in K(\al)$.
\end{lem}
We say that $\al$ {\it belongs} to $\be$ if inequality~(\ref{belong-ineq}) holds.
\begin{proof}
By the fixed field theorem, it suffices to show that for all embeddings $\si:K(\al,\be)\hra K^{\text{al}}$ fixing $K(\al)$, that $\si(\be)=\be$. We have
\[
|\si(\be)-\al|=|\si(\be)-\si(\al)|=|\be-\al|
\]
since $|\bullet|=|\sigma \bullet|$ and $\si(\al)=\al$. Hence
\[
|\si(\be)-\be|=|(\si(\be)-\al)+(\al-\be)|\le |\be-\al|,
\]
the last following since $|\cdot|$ is nonarchimedean. By the minimality assumption we must have $\si(\be)=\be$.
\end{proof}
We define a norm on polynomials by setting
\[
\ve{\sum_{k=0}^n c_kX^k}=\max_{0\le k\le n} |c_k|.
\]
Using Krasner's Lemma, we show that polynomials that are close together have roots that are closely related.
%\begin{thm}\llabel{krasner-poly}
%%If $\al_1,\ldots, \al_n$ are the roots of $f$. If 
%%\[
%%\ve{f-g}< \frac{\min_{i\ne j}|\al_i-\al_j|^n}
%%{\max\{|\be_k|,1\}^n},
%%\]
%Given $f$, there exists $\ep>0$ such that if $\ve{f-g}<\ep$ and $\be$ is a root of $g$, then $K[\al]=K[\be]$ for some root $\al$ of $f$.
%%Letting $\al_1,\ldots, \al_n$ be the roots of $f$, we can take
%%\[
%%\ep=\frac{\min_{i\ne j}|\al_i-\al_j|^n}{\max()}
%%\]
%\end{thm}
%\begin{proof}
% Choose $\de$ so this last quantity is at most $\min_{i\ne j} |\al_i-\al_j|$. Then by Krasner's Lemma~\ref{krasner}, $\al\in K[\be]$. Since $\be$ and $\al$ both have degree $n$ over $K$, $K(\al)=K(\be)$.
%\end{proof}
In fact, we have the following stronger result.
Using Krasner's Lemma, we show that polynomials that are close together have roots generating the same extensions.
\begin{thm}\llabel{krasner-poly}
Given $f$, there exists $\ep>0$ such that if $\ve{f-g}<\ep$, then there is an ordering of roots $\al_1,\ldots, \al_n$ and $\be_1,\ldots, \be_n$ of $f$ and $g$, respectively, counting multiplicities, such that $K(\al_j)=K(\be_j)$.
\end{thm}
\begin{proof}
\noindent\underline{Step 1:} First we show that the roots of $g$ approach the roots of $f$, as $\ve{f-g}\to 0$.
\begin{lem}\llabel{root-continuity1}
Keep the hypothesis of the theorem. Suppose $\ep>0$. Then there exists $\de>0$ such that if $\ve{f-g}<\de$, then for every root $\be$ of $g$, there exists a root $\al$ of $f$ such that $|\be-\al|<\ep$.
\end{lem}
\begin{proof}
First note that the roots of a monic polynomial $h$ are bounded in terms of $\ve{h}$. Indeed, letting $h(X)=\sum_{k=0}^n c_kX^k$, if $\ga$ is a root of $h$, then by Proposition~\ref{elem-prop-dv}(3), we must have $c_k\ga^k \ge \ga^n$ for some $0\le k<n$, and hence
\[
\ga\le c_k^{\rc{n-k}}\le \max(1,\ve{h}).
\]

Suppose $\ve{f-g}\le\de$ is small (say, less than 1). Then $\ve{g}\le \ve{f}+\de$, which is bounded. Hence the roots of $\ve{g}$ are bounded, say by $C$. Let $\be$ be a root of $g$. On the one hand, we have
\begin{equation}\llabel{function-continuity}
(f-g)(\be)\le \ve{f-g}\max\{|\be|^n,1\}\le\de \max\{C^n,1\}
\end{equation}
and on the other,
\[
(f-g)(\be)=f(\be)=\prod_{k=1}^n (\be-\al_k).
\]
Hence $|\be-\al_k|\le (\de \max\{C^n,1\})^{\rc n}$ for some $n$. We can choose $\de$ so this is less than $\ep$.
\end{proof}
\noindent{\underline{Step 2:}} We strengthen the lemma to account for multiplicities. 
\begin{lem}\llabel{root-continuity2}
Keep the hypotheses of the theorem. For every $\ep>0$ there exists $\de>0$ such that whenever $\ve{f-g}<\de$, there exist orderings $\al_1,\ldots, \al_n$ and $\be_1,\ldots, \be_n$ such that $|\be_k-\al_k|<\ep$ for all $k$.
\end{lem}
\begin{proof}
%Choose $\ep<\rc{2}\min_{\al_i\ne \al_j} |\al_i-\al_j|$ in Lemma~\ref{root-continuity1}. Then for $\ve{f-g}<\de$, each root of $g$ is closest 
By Lemma~\ref{root-continuity1}, as $\ve{f-g}\to 0$, the  distance from the roots of $g$ to the closest roots of $f$ approaches 0. %Let $\al_1',\ldots, \al_m'$ be the distinct roots of $f$, and suppose they have multiplicities $r_1,\ldots, r_m$. 
%For each root $\be_k$ of $g$ let $\al_k$ be the closest root of $f$. Suppose by way of contradiction that 
%multiplicities $\al_1$ 
%%of elements $\al_k$ do not match $s_1,\ldots, s when we take $\ep$ small enough. Let $r_1,\ldots, r_m$ be the multiplicities of roots of $f$. Then we can find $(s_1,\ldots, s_m)$ and a sequence of polynomials $g_k$ such that the multiplicities of roots of 
%Suppose the distinct roots $\al_1',\ldots, \al_m'$ of $f$ have multiplicities $r_1,\ldots, r_m$. For $\be$ a root of $g$ (counting multiplicities), let $\al(\be)$ be the root of $f$ closest to $f$, and let $s_1(g),\ldots, s_m(g)$ be the multiplicites of roots of $f$ in the $\al(\be)$. Suppose by way of contradiction that $(s_1(g),\ldots, s_m(g))$ is not constantly $(r_1,\ldots, r_m)$ for $g$ close enough to $f$. Then we can find a sequence $g_j\to f$ such that $(s_1(g_j),\ldots, s_m(g_j))$ is constant and not equal to $(r_1,\ldots, r_m)$. We have
%\[
%
%\]
Let $\be_1(g),\ldots, \be_n(g)$ be the roots of $g$. For each $k$ let $\al_k(g)$ be the root of $f$ closest to $\be_k(g)$. We have $\max_k|\be_k(g)-\al_k(g)|\to 0$ as $g\to f$. Suppose the distinct roots $\al_1',\ldots, \al_m'$ of $f$ have multiplicities $r_1,\ldots, r_m$, and suppose that they occur with multiplicities $s_1,\ldots, s_m$ in the $\al_k(g)$. Suppose by way of contradiction that $(s_1(g),\ldots, s_m(g))$ is not constantly $(r_1,\ldots, r_m)$ for $g$ close enough to $f$. Then we can find a sequence $g_j\to f$ such that $(s_1(g_j),\ldots, s_m(g_j))$ is constant and not equal to $(r_1,\ldots, r_m)$. Then
\[
g_j(X)=\prod_{k=1}^n (X-\be_k(g_j))\to \prod_{k=1}^m(X-\al_k'(g_j))^{s_k}\ne \prod_{k=1}^m(X-\al_k')^{r_k}=f(X),
\]
contradiction.
\end{proof}
\noindent\underline{Step 3:} Take $\ep=\min_{i\ne j} |\al_i' - \al_j'|$ in Lemma~\ref{root-continuity2}. Then Krasner's Lemma~\ref{krasner} gives the conclusion.
\end{proof}
From this we get that every field extension of $\Q_p$ can be described by a field extension of $\Q$, by choosing a close enough approximation to a minimal polynomial.


\begin{cor}
Let $L/\Q_p$ be a finite extension. Then there is a finite extension $K/\Q$ such that $[K:\Q]=[L:\Q_p]=n$ and $K\cdot \Q_p=L$.
\end{cor}
\begin{proof}
Using the primitive element theorem, choose $\al$ so that $\Q_p(\al)=L$. Let $f\in \Q_p[X]$ be the minimal polynomial of $\al$. By Theorem~\ref{krasner-poly}, for $g$ close enough to $f$, there is a root $\be$ of $g$ such that $\Q_p(\al)=\Q_p(\be)$. Take $g\in \Q[X]$ sufficiently close, and $L=K(\be)$. Then 
\[K\cdot \Q_p=K(\al)=K(\be)=L.\qedhere \]
\end{proof}
%%%%%%%%%%%%%%%%%%%%5
\chapter{Local and global fields}
\llabel{lg-fields}
\section{Topology of local fields}
\begin{df}
A \textbf{local field} is a field $K$ with a nontrivial valuation $\ad$ such that $K$ is locally compact.
\end{df}
Note this requires that $K$ is complete.
\begin{pr}\llabel{ring-of-integers-compact}
Let $K$ be complete with respect to a discrete nonarchimedean valuation. (Then we can define $A:=\cO_K=\set{x\in K}{|x|\ge 0}$ and $\mm = \set{x\in K}{|x|>0}$.) Then $A$ is compact if and only if $k:=A/\mm$ is finite.
\end{pr}
\begin{proof}
Suppose $A$ is compact.
Note $\mm=\set{x}{|x|<1}$ is open, and any translate of it is open. Note $A=\bigsqcup_{a\in A/\mm} a+\mm$ where the union is over representatives in $A/\mm$. A finite number of these cover $A$, so $k$ is finite.

Conversely, suppose $k:=A/\mm$ is finite. It suffices to show that $A$ is closed and totally bounded\footnote{A set is {\it totally bounded} if for every $r$, $A$ can be covered by a finite number of sets with diameter at most $r$.}.
\begin{enumerate}
\item $A$ is closed since $A=\set{x}{|x|\le |\pi|}$.
\item $A$ is totally bounded: Given $\ep>0$, choose $r$ so that $|\pi|^{r+1}<\ep$. Now every element is in a ball of radius 1 centered at one of the finite number of points in the form $a_0+a_1\pi+\cdots+a_r\pi^r$. \qedhere
\end{enumerate}
\end{proof}
\begin{pr}\llabel{compact-sets-in-lf}
If $K$ has finite residue field then $\cO_K^{\times}$, $\mfp^n$, and $1+\mfp^n$ are all compact.
\end{pr}
\begin{proof}
From Proposition~\ref{ring-of-integers-compact}, $A$ is compact. The above are all closed subsets of $A$ so compact.
\end{proof}

\begin{thm}
The following is a complete classification of local fields, up to isomorphism.
\begin{enumerate}
\item $\R$ and $\C$ with the usual metric.
\item Finite extensions of $\Q_p$.
\item Field of formal Laurent series $k((T))$ over finite field.
\end{enumerate}
\end{thm}
All local fields are either $\R$, $\C$, or are complete with respect to a discrete nonarchidean valuation.
\begin{proof}
Neukirch, p. 135.
\end{proof}
\fixme{To add: Explanation of why local fields are much easier to study.}
%
\subsection{Open sets and continuity}
\begin{pr}\llabel{power-open}
For any local field $K$ and any $n$, the $n$th power map is open on $K^{\times}$, i.e. it takes open subsets of $K^{\times}$ to open sets.
\end{pr}
\begin{proof}
For $K=\R$ or $\C$, this is clear.

For $K$ $\pi$-adic, this is an easy consequence of Hensel's Lemma. Let $y\in K^{\times n}$. We may suppose $v(y)=0$. Suppose $x_0^p-y=0$. Let $k=v(p)$ and
let $\ep$ be such that $v(\ep-y)\ge 2k+1$. Consider the polynomial $f(x)=x^n-y$. Now $f(x_0)\equiv 0\pmod{\pi^{2k+1}}$ so by Hensel's Lemma $x_0$ lifts to a solution of $f$ in $K$. 
\fixme{This is the version of Hensel in ACIM, p. 14. Add this in.}
\end{proof}
\begin{pr}\llabel{pr:nm-cont}
For any extension of local fields $L/K$, any $\si\in G(L/K)$ acts as a homeomorphism, and the norm map $\nm_{L/K}$ is continuous on $K^{\times}$.
\end{pr}
\begin{proof}
In the case $\C/\R$, we can verify this directly. Otherwise, $L$ is complete with respect to a discrete nonarchimedean valuation. We show that $\si$ preserves distance. Because $\si$ is a field automorphism it suffices to show $|\si(x)|=|x|$. 
Let $\pi$ be a uniformizer. For $\si\in G(L/K)$, $\si \mm=\mm$, so it must take the uniformizer to another uniformizer, $\si(\pi)=\pi'=a_1\pi+a_2\pi^2+\cdots $ with $a_1\ne 0$. Then $\si(a_j\pi^j+a_{j+1}\pi^{j+1}+\cdots)=a_j{\pi'}^j+a_{j+1}{\pi'}^{j+1}$ have the same valuation, as needed. 
%\fixme{Add.}

$\nm_{L/K}(x) = \prod_{\si\in G(L/K)}\si(x)$ by Proposition~\ref{ntr-fext}. $\si$ and multiplication are continuous, so $\nm_{L/K}$ is continuous.
\end{proof}
\section{Unramified extensions}
\begin{df}
Let $K$ be a complete field with residue field $k$; let $L$ be a finite extension of $K$ with residue field $l$. We say $L/K$ is \textbf{unramified} if $l/k$ is separable and the prime ideal $\mfp$ in $\cO_K$ does not ramify in $L$.

$L/K$ is \textbf{totally ramified} if $\mfp$ ramifies completely; by the degree equation this is equivalent to $l=k$.
\end{df}
Note from the residue equation that 
\begin{equation}\llabel{doesnt-ramify}
\mfp\text{ does not ramify} \iff [L:K]=[l:k].
\end{equation}

Our main theorem of this section is Theorem~\ref{unramified-is-separable}. We will show that if $L/K$ is unramified, then $l/k$ is separable. If $l/k$ is separable, though, we need an extra condition to make sure $L/K$ is unramified; namely that a minimal polynomial for $L/K$ stays a minimal polynomial for $l/k$, so that~(\ref{doesnt-ramify}) holds.
%To prove this we first need a preliminary proposition. One might wish that unramified extensions of complete fields correspond exactly to those with separable extensions of residue fields. However, we also need that 
%this is not exactly true, unless we strengthen the separability condition slightly.
\begin{pr}\llabel{complete-unramified-criteria}
Let $K$ be a complete field with residue field $k$; let $L$ be a finite extension of $K$ with residue field $l$. Suppose $L=K(\al)$, and let $g(x)\in K[x]$.
The following are equivalent.
%This argument is from Lang.
\begin{enumerate}
\item $L/K$ is unramified, and $g$ is the minimal polynomial of $\al$.
\item $l/k$ is separable, with $l=k(\ol{\al})$, $g$ has $\al$ as a root, $\ol{g}$ is the minimal polynomial of $\ol{\al}$, and $\ol{g}$ has no repeated roots.
\end{enumerate}
\end{pr}
%(Note the last condition is stronger than saying $l/k$ is separable.)
\begin{proof}
Suppose (1) holds. Then $\ol{g}$ has $\ol{\al}$ as a root. Note $L=K(\al)$ gives $l=k(\ol{\al})$. By~(\ref{doesnt-ramify}), $\ol{\al}$ has degree $[l:k]=[L:K]$ over $k$. Since $\ol{g}$ has degree $[L:K]$, it must be the minimal polynomial of $\ol{\al}$, and have no repeated roots. This shows $l/k$ is separable.

Suppose (2) holds. We have
\[
[L:K]= \deg g=\deg \ol{g}= [l:k],
\]
the first equality is because $\al$ is a root of $g$ and the fact that $\ol g$ is irreducible means $g$ is irreducible (hence the minimal polynomial of $\al$); and 
the last equality is because $\ol{g}$ is the minimal polynomial of $\ol{\al}$. % and has no repeated roots.
%But $[L:K]\ge[l:k]$, so equality holds and 
Hence $\mfp$ (the prime ideal of $\cO_K$) is unramified by~(\ref{doesnt-ramify}). %Note $L/K$ is separable because $g$ has no repeated roots.
Thus $L/K$ is unramified.
\end{proof}
For local fields, the property of being unramified behaves well under extensions and products.
\begin{pr}\llabel{unramified-props}$\,$
\begin{enumerate}
\item
Suppose that $K\subeq L\subeq M$ are finite extensions. If $M/L$ and $L/K$ are unramified, then $M/K$ is unramified.
\item
Suppose that $K\subeq L,M$ are finite extensions. If $L/K$ is unramified, then $LM/M$ is unramified.
\item
Suppose that $K\subeq L,M$ are finite extensions. If $L/K$ and $M/K$ are unramified, then $LM/K$ is unramified.
\end{enumerate}
\[
\xymatrix{
M\ar@{-}[d]^{\text{unram}}\ar@/_1pc/@{.}[dd]_<<<<<<<{\text{unram}}&&&LM\ar@{-}[ld]\ar@{.}[rd]^{\text{unram}}&&&LM\ar@{.}[dd]^{\text{unram}}\ar@{-}[rd]\ar@{-}[ld]&\\
L\ar@{-}[d]^{\text{unram}}&&L\ar@{-}[rd]^{\text{unram}}&& M\ar@{-}[ld]&L\ar@{-}[rd]^{\text{unram}}&&M\ar@{-}[ld]^{\text{unram}}\\
K&&&K&&&K&
}
\]
\end{pr}
\begin{proof}
Let $k,l,m,n$ be the residue fields of $K,L,M,LM$, and $\mfp$, $\mP$, and $\mP'$ be the prime ideals of $\cO_K$, $\cO_L$, $\cO_M$, respectively.
\begin{enumerate}
\item We have $\mfp\cO_M=\mP\cO_L=\mP'$. Separability is transitive, so $M/K$ is unramified.
\item Write $L=K(\al)$. By Proposition~\ref{complete-unramified-criteria}, we can find $g$ with $\al$ as root such that $\ol{g}$ is the minimal polynomial of $l=k(\al)$ over $k$, and is separable. Then the minimal polynomial for $n=m(\al)$ over $m$ divides $\ol{g}$, hence is separable. By Proposition~\ref{complete-unramified-criteria} again, $LM/M$ is unramified.
\item By part 2, $LM/M$ is unramified. Since $M/K$ is unramified, by part 1 $LM/K$ is unramified.\qedhere
\end{enumerate}
\end{proof}
\begin{thm}\llabel{unramified-is-separable}
Let $K$ be a field; fix an algebraic closure.
There is an equivalence of categories between
\begin{itemize}
\item finite unramified extensions $L/K$, and  
\item finite separable extensions $l/k$,
\end{itemize}
where the morphisms are field inclusions. (All maps between fields are inclusions.)
%\[
%\xymatrix{
%L_1\ar[r]\ar[d] &L_2\ar[d]\\
%l_1=L_1/\mfp_1\ar[r]&l_2=L_2/\mfp_2.
%}
%\]
The equivalence is given by $L\mapsto \cO_L/\mP$ where $\mP$ is the prime in $\cO_L$.

Moreover,
\begin{enumerate}
\item $L\subeq M$ if and only if $l\subeq m$.\footnote{This is part of the statement of equivalence of categories.}
\item The residue field of $LM$ is $lm$.
\item $L/K$ is Galois if and only if $l/k$ is Galois, and
\[
G(L/K)\xra{\cong} G(l/k)
\]
by restricting $\si\in G(L/K)$ to $B=\cO_L$ and modding out by $\mP B$.
\end{enumerate}
\end{thm}
\begin{proof}
By Proposition~\ref{complete-unramified-criteria}, $L$ does get sent to a separable extension.

First we show the map is surjective. Given $l/k$ separable, choose $\be$ so that $l=k(\be)$ and choose $f$ so that $\ol{f}$ be the minimal polynomial of $\be$. Since $\be$ is a simple root of $\ol{f}$, by Hensel's Lemma~\ref{hensel1} we can lift it to a root $\al$ of $f$. Then $K(\al)$ is mapped to $k(\be)$.

Part (2) is clear. For (1), if $L\subeq M$ then clearly $l\subeq m$. Conversely, suppose $l\subeq m$. 
%Next we show it is injective. Suppose $L_1,L_2$ are unramified with residue field $l$.
Now $LM$ is also unramified (Proposition~\ref{unramified-props}) and has residue field $l\cdot m=m$. Hence,
\[
[M:K]=[m:k]=[lm:k]=[LM:K],
\] 
showing $L\subeq M$.

If $l=m$, then the above shows that $L=M$. Hence the map is injective. %The action on maps $L_1\to L_2$ is self-explanatory.

For (3), note an extension is Galois iff it is the (minimal) splitting field of a separable polynomial $f$. Take $g$ to be the minimal polynomial of a primitive element $\al$; note $\ol{\al}$ generates $l/k$. 
Note by Proposition~\ref{complete-unramified-criteria}, $\ol{g}$ is separable. 
%by Proposition~\ref{complete-unramified-criteria}, 
If $L/K$ is Galois, then $g$ splits over $L$ so %and that its roots generate $L/K$. Thus 
$\ol{g}$ splits over $l$. Combining the previous two statements, $l/k$ is Galois.
Conversely, suppose $l/k$ is Galois. Since $\ol{g}$ splits into nonrepeated linear factors, Hensel's Lemma~\ref{hensel2} lifts it to a factorization of $g$. Hence $g$ splits over $K$ into distinct linear factors, showing $L/K$ is Galois.
\end{proof}
Suppose $k$ is a finite field. In this case, the separable extensions $l/k$ are exactly the finite extensions. Moreover, we understand what these extensions are; there is one of each degree, and we can find the corresponding $L/K$ explicitly. Furthermore, by surjectivity in (3), $G(L/K)$ contains a unique element mapping to the Frobenius element in $G(l/k)$; see Definition~\ref{frobenius-element}.
%\begin{rem}\llabel{unram-finite-field}$\,$
%Suppose $k$ is a finite field (in particular, this is true if $K$ is the completion of a number field).
%\begin{enumerate}
%\item
%$k$ is perfect, so every finite extension $l/k$ is separable.
%\item
%There is exactly one unramified extension of $L/K$ of a given degree. \fixme{Comment: Lots of polynomials have roots generating the same extension!} 
%\item 
%By surjectivity in (3), $G(L/K)$ contains a unique element mapping to the Frobenius element. See Definition~\ref{intro-cft}.\ref{frobenius-element}.
%\end{enumerate}
%\end{rem}
\begin{lem}\llabel{nroot-unram}
Let $\al$ be a root of
\[
f(X):=X^n-a=0
\]
where $a$ is a unit and $p\nmid n$. Then $K(\al)/K$ is unramified.
\end{lem}
\begin{proof}
Let $g(X)\mid f(X)$ be the minimal polynomial of $\al$. Let $L=K(\al)$ and $l$ be its residue field.

Note that $f'(X)=nX^{n-1}\ne 0$ has no common factor with $f(X)=X^n-a$, even when reduced modulo $\mfp$, as $p\nmid n$ and $a\nin \mfp$. Hence $f(X)$, and {\it a fortiori} $g(X)$, has no repeated root in $k$. Any factorization of $g(X)$ in $k$ gives a factorization of $g(X)$ in $K$ by Hensel's Lemma. Hence $g$ remains irreducible in $k[X]$. This shows $[l:k]=[L:K]$. By the degree equation, $L/K$ must be unramified.
\end{proof}
\begin{thm}\llabel{unram-ram}
Let $L/K$ be an extension of complete fields with finite residue fields. Then %the compositum of all unramified extension of $K$ contained in $L$ is unramified over
there exists a field $K\subeq L_u\subeq L$ such that $L_u/K$ is unramified and every unramified extension of $K$ contained in $L$ is contained in $L_u$. Moreover,
\begin{enumerate}
\item
$L_u$ is obtained by adjoining to $K$ all roots of unity in $L$ whose order is relatively prime to $q:=\chr(K)$.
\item $L/L_u$ is totally ramified.
\end{enumerate}
\[
\xymatrix{
L\ar@{-}[d]^{\text{totally ramified}}\\
L_u\ar@{-}[d]^{\text{unramified}}\\
K
}
\]
\end{thm}
We call $L_u$ the \textbf{maximal unramified extension} of $K$ contained in $L$. %\fixme{This is useful...} 
\fixme{We can define this when $L/K$ is infinite (details).}
\begin{proof}
Let $L_u$ be the compositum of all unramified extensions of $K$ contained in $L$. Then $L_u$ is unramified by Proposition~\ref{unramified-props}, and it contains all unramified extensions of $K$ contained in $L$.

For each $n$ not a multiple of $p$, $K(\ze_n)/K$ is unramified by Lemma~\ref{nroot-unram}. Letting $q=|k|$, the corresponding extension of residue fields is $k(\ze_n)/k=\F_{q^{\ord_q(n)}}/\F_q$. We get all finite extensions $l/k$ in this way, thus all unramified extensions $L'/K$ in this way. Taking the roots of unity inside $L$ gives the result.
%Now the finite separable extensions are exactly the finite extensions of $l/k$, which are the splitting fields of the polynomials $X^{q^n-1}-1$. \fixme{reference} 
%a%Together, these fields contains all roots of unity of order relatively prime to $q$ (since if $m\perp q$ then $q$ would have finite order with respect to $m$).
%As $n$ ranges over $\N$, $q^n-1$ contain all factors relatively prime to $n$ (since if $m\perp q$ then $q$ would have finite order with respect to $m$). %By the construction in Theorem~(\ref{unramified-is-separable}), $X^{q^n-1}-1$
%By Lemma~\ref{nroot-unramified}, 
%then the corresponding fields $L/K$ are generated by roots of unity of order relatively prime to $L$. Taking the compositum gives the result.
\end{proof}
\section{Ramified extensions}
\begin{df}
Let $L/K$ be a ramified extension of local fields, with $q:=\chr(k)=p^n$. We say
\begin{enumerate}
\item
$L/K$ is \textbf{tamely ramified} if $p\nmid [l:k]$.
\item
$L/K$ is \textbf{wildly ramified} if $p\mid[l:k]$.
\end{enumerate}
\end{df}
We seek analogues of Lemma~\ref{nroot-unram} in for ramified extensions.

For a prime $\mfp$ of a Dedekind domain $A$ (not necessarily corresponding to a local field) let $v_{\mfp}$ denote the corresponding valuation. (That is, if $v_{\mfp}(a)$ is defined such that $\mfp^{v_{\mfp}(a)}$ is the highest power of $\mfp$ dividing $(a)$.) 
Note the following two facts.
\begin{enumerate}
\item If $\mfp B=\mP^e$, then
\[
v_{\mfp}(a)=v_{\mP}(a)^e.
\]
\item If $a_1+\cdots +a_n=0$, then the minimum value of $v_{\mfp}(a_i)$ is attained for two indices.
\end{enumerate}
\begin{df}\llabel{eisenstein-df}
An \textbf{Eisenstein extension} relative to $\mfp$ is an extension $K(\al)/K$ where the minimal polynomial of $\al$ is of the form
\[
f(x)=x^n+a_{n-1}x^{n-1}+\cdots +a_0
\]
where $v_{\mfp} a_i>0$ and $v_{\mfp} a_0=1$.
\end{df}
\begin{thm}\llabel{eisenstein-ramification}
The prime ideal $\mfp$ totally ramifies in any Eisenstein extension relative to $\mfp$:
\[
\mfp B=\mP^e, \quad \mP=(f(\al), \mP)^e.
\]
\end{thm}
\begin{proof}
Let $\mP^e|| \mfp$. Note $e\le n=[L:K]$.
We calculate the valuation of $f(\al)$ with respect to $\mP$.
\begin{align*}
v_{\mP}(\al^n)&= nv_{\mP}(\al)\\
v_{\mP}(a_k\al^k)&= e+kv_{\mP}(\al)>e,&1\le k\le n-1\\
v_{\mP}(a_0)&=e.
\end{align*}
Since $f(\al)=\al^n+\cdots +a_0=0$, the minimum valuation must be attained for two terms. The only way this is possible is if $nv_{\mP}(\al)=e$. Then  $v_{\mP}(\al)=1$ and $n=e$, as needed.
\end{proof}

\begin{thm}\llabel{thm:tot-ram-eis}
Let $K$ be complete with respect to a nonarchimedean valuation. 
The totally ramified extensions of $K$ are exactly those of the form $K(\al)$ where $\al$ is the root of an Eisenstein polynomial.
\end{thm}
%Recall that an Eisenstein polynomial is in the form $f(X)=a_nX^n+\cdots +a_0$ where $\ord a_n=0$, $\ord a_k>0$ for $1\le k<n$, and $\ord a_0=1$ (Definition~\ref{factorization}.\ref{eisenstein-df}). \fixme{Should I define Eisenstein polynomial or Eisenstein extension?}
\begin{proof}
%Suppose $\al$ is the root of an Eisenstein polynomial:
%\[
%a_n\al^n+\cdots +a_1\al+a_0=0.
%\]
%Extend the valuation $\ad$ to $L$. We have by~\ref{elem-prop-dv} that the maximum valuation is attained by two terms in the above sum. The 
%Suppose $\al$ is the root of an Eisenstein polynomial $f$. By Lemma~\ref{newton-lemma}, since the \fixme{lower?} Newton polygon consists of just the segment joining $(0,1)$ and $(n,0)$, we have that $\ord(\al)=\rc n$, i.e. the prime ideal $\mfp$ of $K$ equals $(\al)^n$, i.e. the extension is totally ramified of degree $n$.
The forward direction follows directly from Theorem~\ref{eisenstein-ramification}.

Conversely, let $L/K$ be a totally ramified extension. Take $\al$ to be a generator of the maximal ideal $\mP$ of $\cO_L$. Note $v(\al)=\rc n$ since $(\al)^n=\mfp$. Note that for any $a_{n-1},\ldots, a_0$, we have
\[
v(a_k\al^k)=v(a_k)+\frac{k}{n}\equiv \frac{k}{n}\pmod 1,
\]
since $v(a_k)$ is an integer. Thus, the nonzero terms $a_k\al^k$, $0\le k<n$, have different orders. Thus by Proposition~\ref{elem-prop-dv}, $a_{n-1}\al^{n-1}+\cdots +a_0\ne 0$ unless all coefficients are 0. This shows that $\al$ must have degree $n$; suppose $\al^n+a_{n-1}\al^{n-1}+\cdots +a_0=0$. Again by Proposition~\ref{elem-prop-dv}, the minimum order is attained for two terms. We have
\begin{align*}
v_{\mfp}(\al^n)&=nv_{\mfp}(\al)=1\\
v_{\mfp}(a_k\al^k)&=kv(a_k)+\frac{k}{n},&0\le k\le n-1.
\end{align*}
The only way this can happen 
% as observed above...
is if $\al^n$ and $v(a_0)$ are the nonzero terms with least order. This gives $v(a_0)=1$, and $v(a_k)>0$ for $1\le k\le n$, i.e. the polynomial is Eisenstein.
\end{proof}
\begin{thm}
Suppose $L/K$ is a totally and tamely ramified extension of degree $n$.
Then $L=K(\al)$ for some $\al$ a root of
\[
X^n-\pi=0
\]
for $\pi \in \mfp$.
%This doesn't make sense...
%
%Conversely, if $L=K(\al)$ where $\al$ is a root of
%\[
%X^n-a=0,\quad p\nmid n,
%\]
%then $L/K$ is tamely ramified. If $p\nmid \ord_{\mfp}(a)$, then $L/K$ is totally ramified.
\end{thm}
\begin{proof}
%Take $(\al)=\mP$. Since $L/K$ is totally ramified, $(\al)^n=\mfp$, so $\al^n=\pi\in \mfp$.
%
%\begin{enumerate}
%\item $L/K$ is tamely ramified: Write $a=\pi^ru$, where $\ord_{\mfp}(\pi)=1$ and $u$ is a unit. We exhibit a field containing $K(\al)$ that is tamely ramified over $K$. Consider 
%\[
%\xymatrix{
%K(\ze_n,\pi^{\rc n},u^{\rc n})\ar@{-}[d]\\
%K(\ze_n,u^{\rc n})\ar@{-}[d]\\
%K
%}
%\]
%By Theorem~\ref{unram-ram}, $K(\ze_n,u^{\rc n})$ is unramified over $K$. But $K($
%\end{enumerate}
Take $\be\in \mP$. Since $L/K$ is totally ramified, $v_{\mfp}(\be^n)=1$. Hence $\be^n=u\pi$ for some $u\in B^{\times}$, and $\be$ is a zero of
\[
g(X):=X^n-u\pi.
\]
Unfortunately, $u$ may not be in $A$. However, we show that this polynomial is close enough to 
\[
f(X):=X^n-u'\pi
\]
for some $u'\in A$ and proceed as in Theorem~\ref{krasner-poly} to show that the roots of these two polynomials generate the same extension.

Since $L/K$ is totally ramified, $l=k$, i.e. $A/\mfp A\xra{\cong} B/\mP B$. Thus there exists $u'\equiv u\pmod{\mP}$ with $u'\in A$. This means $|u'-u|<1$. Letting $\al_1,\ldots, \al_n$ be the roots of $f(X)=0$,
\[
|\be-\al_1|\cdots |\be-\al_n|=|f(\be)|=|u\pi-u'\pi|<|\pi|=|\al_1|\cdots |\al_n|
\]
so $|\be-\al_j|<|\al_j|$ for some $j$; \wog{} $j=1$.

Since $L/K$ is tamely ramified, $p\nmid n$ and  $f'(\al_1)=n\al_1^{n-1}$ has valuation $|\al_1|^{n-1}$. Hence
\begin{equation}\llabel{derivative-valuation}
|\al_1|^{n-1}=|f'(\al_1)|=|(\al_1-\al_2)\cdots (\al_1-\al_n)|.
\end{equation}
Note $|\al_j|=|u'\pi|^{\rc n}=|\al_1|$; hence $|\al_1-\al_j|\le |\al_1|$. By~(\ref{derivative-valuation}), equality must hold. Hence $|\be-\al_1|<|\al_1-\al_j|$ for all $j\ne 1$, and by Krasner's Lemma~\ref{krasner}, $K(\al_1)\subeq K(\be)$. Since both extensions are totally ramified of degree $n$, $L=K(\be)=K(\al_1)$.
%As in Theorem~\ref{krasner-poly}, we compute $
%Then
%\[
%\ve{(X^n-u\pi)-(X^n-u'\pi)}=|(u-u')\pi|<|\pi|.
%\]
%Now note 
%For the converse, 
\end{proof}
The analogues of Proposition~\ref{unramified-props} carry over exactly.
\begin{pr}\llabel{ramified-props}$\,$
\begin{enumerate}
\item
Suppose that $K\subeq L\subeq M$ are finite extensions. If $M/L$ and $L/K$ are tamely ramified, then $M/K$ is tamely ramified.
\item
Suppose that $K\subeq L,M$ are finite extensions. If $L/K$ is tamely ramified, then $LM/M$ is tamely ramified.
\item
Suppose that $K\subeq L,M$ are finite extensions. If $L/K$ and $M/K$ are tamely ramified, then $LM/K$ is tamely ramified.
\end{enumerate}
\end{pr}
\begin{thm}
Let $K$ be a field with characteristic 0 and finite residue field, and let $\mfp$ be a prime in $\cO_K$. 
Given $n$, there are only finitely many %totally ramified 
extensions of $K_{\mfp}$ with degree at most $n$.
\end{thm}
\begin{proof}
First we show that there are finitely many totally ramified extensions of degree $n$. By Theorem~\ref{thm:tot-ram-eis}, very such extension is realized by adjoining a root of an Eisenstein polynomial of degree $n$. By taking the coefficients, an Eisenstein polynomial can be identified with a point of 
\begin{equation}\llabel{poly-compact}
\underbrace{\mfp\times \cdots \times \mfp}_{n-1}\times A^{\times}\pi.
\end{equation}
The topology given by $\ve{\cdot}$ is exactly the product topology here; this is compact by Proposition~\ref{compact-sets-in-lf}. Now for each polynomial $f$, by Theorem~\ref{krasner-poly} there exists an open set $U_f$ such that any $g\in U_f$ has roots generating the same extensions as those of $f$. Since~(\ref{poly-compact}) is compact, a finite number of $U_f$ cover $f$. The roots corresponding to those $f$ generate all the totally ramified extensions of degree $n$.

By Theorem~\ref{unram-ram}. Any finite extension $L$ of degree $n$ is an totally ramified extension of degree $\frac{n}{m}$ of an unramified extension $L_u$ of degree $m$ for some $m$. By the remark after Theorem~\ref{unramified-is-separable}, %~\ref{unram-finite-field} \fixme{(Promote this to a corollary of something)}, 
there is exactly one unramified extension of degree $m$; for each $L_u$ by the above there are a finite number of possibilities for $L$.
\end{proof}
\section{Witt vectors*}
We know from Proposition~\ref{pi-adic-expansion} that every element of $\hat{K}$ can be written as $\sum_{n\ge N} a_n\pi^n$ where the $a_n$ come from a fixed set of representatives for $A/\mm$. Although this allows us to write down any element, unless the set of representatives is closed under addition and multiplication (i.e. form a copy of $k$ in $A$), we cannot simply add and multiply the coefficients. Instead, we find that addition and multiplication are governed by {\it Witt vectors}. We will actually develop this theory in a more general context.
\begin{df}
Let $p$ be a prime number. A ring $R$ is a \textbf{strict $p$-ring} if $R$ is complete and Hausdorff with respect to the $p$-adic topology, $p$ is not a zero-divisor in $R$, and the residue ring $R/(p)$ is perfect. (A ring of characteristic $p$ is \textbf{perfect} if the map $x\mapsto x^p$ is bijective.)
\end{df}
We will primarily be interested in the case where $R$ is an unramified extension of $\Z_p$. %There is a unique unramified extension of $\Z_p$ with residue degree $f$, namely $\Z_p[\ze_{p^f-1}]$.
\begin{thm}
Let $K$ be a perfect ring of characteristic $p$.
\begin{enumerate}
\item
There is a strict $p$-ring $R$ with residue ring $K$, unique up to canonical isomorphism.
\item
There is a unique system of representatives $\tau:K\to R$, called the \textbf{Teichm\"uller representatives}, such that 
\[
\tau(xy)=\tau(x)\tau(y)
\]
for all $x,y\in K$.
\end{enumerate}
\end{thm}
The main example of interest to us is the following.
\begin{ex}
Fix $f$; then there is a unique unramified extension of $\Z_p$ with residue field $\F_q$, $q=p^f$, namely $\Z_p[\ze_{p^f-1}]$. The Teichmuller representatives are the $(q-1)$th roots of unity $\mu_{q-1}$. They are multiplicative, but not additive. The following construction will tell us how to add them.
\end{ex}

\begin{lem}\llabel{witt-operations}
Given $X=(X_0,X_1,\ldots)$, define
\[
W_n(X)=X_0^{p^n} +pX_1^{p^{n-1}} +\cdots +p^nX_n,\quad n\ge 0.
\]
Then there exist polynomials
\[
S_0,S_1,\ldots;P_0,P_1,\ldots\in \Z[X_0,X_1,\ldots,Y_0,Y_1,\ldots]
\]
such that
\begin{align*}
W_n(S)&=W_n(X)+W_n(Y)\\
W_n(P)&=W_n(X)\cdot W_n(Y).
\end{align*}
where $X=(X_0,X_1,\ldots)$, $Y=(Y_0,Y_1,\ldots)$, $S=(S_0,S_1,\ldots)$, and $P=(P_0,P_1,\ldots)$.
\end{lem}
The motivation for defining these polynomials is that %Lemma~\ref{witt-operations} 
they tell us how to add in strict $p$-rings using the base-$p$ represenation with Teichm\"uller representatives as coefficients.
\begin{thm}
Let $R$ be a strict $p$-ring, $k$ its residue ring, and 
$\tau:k\to R$ be the system of Teichm\"uller representatives. Then
\[
\su\tau(x_n) p^n+\su \tau(y_n)p^n = \su \tau(S_n(x_0^{p^{-n}},x_1^{p^{-(n-1)}},\ldots, x_n; y_0^{p^{-n}},y_1^{p^{-(n-1)}},\ldots, y_n)p^n.
\]
\end{thm}

\begin{proof}[Proof of Lemma~\ref{witt-operations}]
We will abbreviate
\begin{align*}
%X&=(X_0,X_1,\ldots)\\
W(X)&=(W_0(X),W_1(X),\ldots)\\
%S(X,Y)&=(S_0(X,Y),S_1(X,Y),\ldots)\\
%P(X,Y)&=(P_0(X,Y),P_1(X,Y),\ldots)\\
R&=\Z[X_0,X_1,\ldots;Y_0,Y_1,\ldots].
\end{align*}
All comparisons between $X,Y$ will be done componentwise, and we define $X^n=(X_0^n,X_1^n,\ldots)$.

We find the $S_m,P_m$ inductively, with the additional condition that $S_m,P_m$ are polynomials in $X_0,\ldots, X_m,Y_0,\ldots, Y_m$. To begin, note $W_0(X)=X_0$ so we set
\begin{align*}
S_0(X,Y)&=X_0+Y_0\\
P_0(X,Y)&=X_0Y_0.
\end{align*}
\begin{lem}
If $F_m,G_m\in R$ and $F_m\equiv G_m\pmod{p}$ for every $m$, then
\[
W_n(F)\equiv W_n(G)\pmod{p^{n+1}}.
\]
\end{lem}
\begin{proof}
First note that for any $f,g\in R$ such that $f\equiv g\pmod{p}$,
\[
f^{p^j}\equiv g^{p^j}\pmod{p^{j+1}}.
\]
The proof is by induction, with the induction step following by the binomial theorem: if $f^{p^{j-1}}=g^{p^{j-1}}+p^{j}h$ then
\[
f^{p^j}=(g^{p^{j-1}}+p^jh)^p=g^{p^j}+\underbrace{\binom p1p^{j-1}}_{p^j}hg^{p^{j-1}(p-1)}+p^{j+1}k
\]
for some $k\in R$.

This claim gives $f_j^{p^{n-j}}\equiv g_j^{p^{n-j}}\pmod{p^{n-j+1}}$ and hence
\[
p^jf_j^{p^{n-j}}\equiv p^jg_j^{p^{n-j}}\pmod{p^{n+1}}.
\]
Summing these up give the result.
\end{proof}
Directly from the definitions, we have
\begin{align*}
W_n(X)&=W_{n-1}(X^p)+p^nX_n.
\end{align*}
Hence the equations
\begin{align*}
W_n(S)&=W_n(X)+W_n(Y)\\
W_n(P)&=W_n(X)W_n(Y)
\end{align*}
are equivalent to
\begin{align}
\nonumber
W_{n-1}(S^p)+p^nS_n&=W_{n-1}(X^p) +p^nX_n +W_{n-1}(Y^p)+p^nY_n\\
\llabel{witt-add}
&=W_{n-1}(S(X^p,Y^p))+p^n(X_n+Y_n)\\
\nonumber
W_{n-1}(P^p)+p^nP_n&=(W_{n-1}(X^p) +p^nX_n)(W_{n-1}(Y^p)+p^nY_n)\\
\llabel{witt-mult}
&=W_{n-1}(P(X^p,Y^p))+p^n(X_nW_{n-1}(Y^p)+Y_nW_{n-1}(X^p)+p^nX_nY_n)
\end{align}
where~(\ref{witt-add}) and~(\ref{witt-mult}) follow from the hypothesis for $n-1$. 
Solving for $S_n$ and $P_n$, these are equivalent to
\begin{align*}
S_n&=X_n+Y_n+\frac{W_{n-1}(S(X^p,Y^p))-W_{n-1}(S^p)}{p^n}\\
P_n&=X_nW_{n-1}(Y^p)+Y_nW_{n-1}(X^p)+p^nX_nY_n+\frac{W_{n-1}(P(X^p,Y^p))-W_{n-1}(P^p)}{p^n}.
\end{align*}
However, since taking $p$th powers is a homomorphism modulo $p$, for any $f\in R$ we have $f(X,Y)^p\equiv f(X^p,Y^p)\pmod p$.
Applying this to $f=S_j,P_j$, we see the conditions of the lemma are satisfied, so the numerators are divisible by $p^n$, and we can successfully define $S_n$ and $P_n$.
\end{proof}
\begin{thm}
Let $A$ be a commutative ring. For 
\[
a=(a_0,a_1,\ldots),\quad b=(b_0,b_1,\ldots),\quad a_i,b_i\in A_i,
\]
the operations
\[
a\stackrel{W}{+} b=S(a,b),\quad a\stackrel{W}{\cdot} b=P(a,b).
\]
turn the set $A^{\N_0}$ into a commutative ring $W(A)$.
\end{thm}
This is called the \textbf{ring of Witt vectors} over $A$.
\begin{proof}
We first prove that associativity, commutativity, and distributivity hold as polynomial identities in the $a_j,b_j$. 
The result then follows by considering the substitution homomorphism $\Z[a_0,\ldots;b_0,\ldots]\to A$.

\begin{lem}
The function $W:R^{\N}\to R^{\N}$, where $R:= \Z[a_0,\ldots;b_0,\ldots]$, is injective.
\end{lem}
\begin{proof}
Suppose $X=(X_0,X_1,\ldots)$ and $W(X)=(Y_0,Y_1,\ldots)$. We show the $X_j$ are determined by induction. We have $X_0=W_0(X)=Y_0$. For the induction step, note
\[
Y_n=W_{n}(X)=X_0^{p^n}+pX_1^{p^{n-1}}+\cdots +p^nX_n;
\]
since $X_0,\ldots, X_{n-1},Y_n$ are determined and multiplication by $p^n$ is injective in $R$, $X_n$ is determined.
\end{proof}
Lemma~\ref{witt-operations} gives
\begin{align*}
W(X\stackrel W+Y)&=W(S(X,Y))=W(X)+W(Y)\\
W(X\stackrel W{\cdot} Y)&=W(P(X,Y))=W(X)\cdot W(Y).
\end{align*}
Hence $W:W(A)\to R^{\N}$ is a map that preserves addition and multiplication; moreover, it is injective. Its image is a subalgebra of $R$, since it contains 0 and 1:
\begin{align*}
W(0,0,\ldots)&=(0,0,\ldots)\\
W(1,0,\ldots)&=(1,1,\ldots).
\end{align*}
Hence $\stackrel{W}{+}$ and $\stackrel{W}{\cdot}$ turn $W(A)$ into a commutative algebra with unit (we are basically ``pulling back" the algebra structure from $R^{\N}$ to $W(A)$ using $W$).
\end{proof}
\subsection{Frobenius and Transfer maps} %Verschiebung maps}

%\chapter{Global fields}\llabel{global-fields}
\section{Extending valuations on global fields}
\begin{thm}\llabel{ext-val-irr}
Let $|\cdot|$ be a valuation on $K$ and let $\hat{K}$ be the completion of $K$ with respect to $|\cdot|$. Let $L=K(\al)$ be a finite separable extension of $K$, and let $f$ be the minimal polynomial of $\al$.

The completions of $L$ with respect to the extensions $|\cdot|'$ of $|\cdot|$ are exactly $\hat{K}[X]/(h)$ as $h$ ranges over irreducible factors of $f$ in $\hat{K}$.
%There is a bijective correspondence between extensions $|\cdot |'$ of $|\cdot|$ to $L$ and irreducible factors of $f$ in $\hat{K}$.
\end{thm}
\begin{proof}
Suppose we are given an extension $|\cdot|'$. Let $\hat{L}$ be the completion of $L$ with respect to $|\cdot|'$.
\[
\xymatrix{
L=K(\al)\ar@{^(->}[r]\ar@{-}[d] & \hat L=\hat K(\al)\ar@{-}[d]\\
K\ar[r] & \hat{K}
}
\]
To see $\hat L = \hat K(\al)$, note we have $\hat L=\wh{K(\al)}\supeq \wh K(\al)$; for the reverse inclusion note $\hat K(\al)$ contains $\al$ and is complete (as it is a finite-dimensional vector space over a complete field). 

Now considering the extension $\hat L/\hat K$, $\al$ is the root of one of the irreducible factors of $f$ in $\hat K[X]$.

Conversely, given an irreducible factor $h$ of $f$ in $\hat K[X]$, consider $\hat K(\al')=\hat K[X]/\an{h}$. 
The valuation on $\hat K$ extends uniquely to $\hat K(\al')$ by Theorem~\ref{extend-discrete-valuations}. Then let $K(\al)\hra \hat K(\al')$ be the map sending $\al$ to $\al'$. (This makes sense as the minimal polynomials of $\al,\al'$ over $K$ are both $f$.) By the same argument as before, $\hat L=\wh{K(\al)}=\hat K(\al')$, as desired.
\[
\xymatrix{
L=K(\al)\ar@{^(->}[r]\ar@{-}[d] & \hat L=\hat K(\al')\ar@{-}[d]\\
K\ar[r] & \hat{K}
}
\]
\end{proof}
%%segway here
\begin{thm}\llabel{completion-tensor}
Let $\hat K$ be the completion of $K$ with respect to
a archimedean or discrete nonarchimedean valuation $|\cdot|$.
Let $L/K$ be a finite separable extension. There are finitely many extensions of $|\cdot|$ to $L$; denoting them by $|\cdot|_i$ and the respective completions of $L$ be $L_i$, we have the natural isomorphism
\[
\hat K\otimes_K L\cong\prod_{i} L_i.
\]
\end{thm}
\begin{proof}
By the primitive element theorem, %~\ref{pet}, 
we can write $L=K(\al)$. Let $f$ be the minimal polynomial of $\al$. Let $f$ factor into irreducibles in $\hat K[X]$ as
\[
f=f_1\cdots f_n.
\]
Then
\[
\hat K\otimes_K L\cong \hat K\otimes_K K[x]/(f)\cong \hat K[x]/(f)\stackrel{\text{CRT}}{\cong} \prod_{i=1}^n \hat K[x]/(f_i)
\stackrel{\text{Thm~\ref{ext-val-irr}}}{\cong} \prod_{i=1}^n L_i.\qedhere
\]
\end{proof}
%we can think of the $L_i$ are all possible fields making the diagram commute
Note the map in the theorem sends 
\[
a\ot b\mapsto (a_1b,\ldots, a_nb),
\]
where $a_i$ is the embedding of $a$ into $L_i$.
We now have a way to calculate norms and traces in terms of completed fields.
%\begin{cor}\llabel{local-global-degree}
%Keep the same notation as above. Then
%\[
%[L:K]=\sum_{i} [L_i:K].
%\]
%\end{cor}
%\begin{proof}
%These are the dimensions of $\hat{K}\ot_K L$ and $\prod_i L_i$ as $\hat{K}$-vector spaces.
%\end{proof}
\begin{cor}\llabel{complete-ntr}
Keep the same notation as above. Then
\begin{enumerate}
\item $\nm_{L/K}(\al)=\prod_{i=1}^n \nm_{L_i/\hat K}(\al)$.
\item $\tr_{L/K}(\al)=\prod_{i=1}^n \tr_{L_i/\hat K}(\al)$.
\end{enumerate}
\end{cor}
\begin{proof}
Using Proposition~\ref{ntr-fext}(1) and Theorem~\ref{ext-val-irr}, we see
\begin{align*}
\nm_{L/K}(\al)&=\prod_{\al'\text{ root of }f}\al'=\pa{\prod_{\al'\text{ root of }f_1}\al'}\cdots \pa{\prod_{\al'\text{ root of }f_n}\al'}
=\prod_{i=1}^n \nm_{L_i/\hat K}(\al)\\
\tr_{L/K}(\al)&=\sum_{\al'\text{ root of }f}\al'=\pa{\sum_{\al'\text{ root of }f_1}\al'}+\cdots+ \pa{\sum_{\al'\text{ root of }f_n}\al'}
=\sum_{i=1}^n \tr_{L_i/\hat K}(\al).
\end{align*}
\end{proof}

\section{Product formula}
\index{product formula}
\begin{lem}\llabel{compare-normed-val}
Let $L/K$ be a finite extension of number fields, with normalized nonarchimedean valuations $w\mid v$, as in Example~\ref{p-adic-val}. Let $\ad_w'$ be $w$ normalized so it extends $v$. Then
\[
\ad_w = \ad_w'^{[L_w:K_v]}.
\]
\end{lem}
\begin{proof}
We check equality on the completions. Let $p$ be the characteristic of the residue fields, and $\pi_v\in K_v, \pi_w\in L_w$ be prime. 
Since $\ved_w$ is a map $L^{\times}\to p^{\Z}$, it suffices to check that $\ved_w$ and ${\ved_w'}^{[L_w:K_v]}$ agree for one element that does not get sent to 1. We check them on $\pi_v$. 
Note that if $\mP$ is the prime corresponding to $w$, $\fN \mP = |\cO_L/\mP \cO_L| = |\cO_{L_w}/\pi_w \cO_{L_w}|=|l|$ where $l=\cO_{L_w}/\pi_w\cO_{L_w}$ and similarly for $K$. 
% and $k = \cO_{K_v}/\pi_v$. 
Then 
\[
|\pi_v|_w = \prc{|l|}^{e(\pi_w/\pi_v)} = \prc{|k|}^{e(L_w/K_v)f(L_w/K_v)} = |\pi_v|_v^{[L_w:K_v]}
\]
by the degree equation~\ref{deg-eq}. %\ref
%milne pg. 132.
\end{proof}

\begin{thm}[Product formula]\llabel{product-formula}
For any nonzero $\al\in K$,
\[
\prod_{v\in V_K} |\al|_v=1.
\]
\end{thm}
\begin{proof}$\,$\\
%\underline{Step 1:} We first show this for $K=\Q$. Given $n\in \Q$, factor it as $n=\pm \prod_{i=1}^{\iy}p_i^{a_i}$ where $p_i$ are all the prime numbers; note only a finite number of the $a_i$ are nonzero. Then 
%\[
%|\al|=\pa{\prod_i |a|_{p_i}}|a|_{\iy}=\pa{\prod_{i=1}^{\iy} p_i^{-a_i}}|a|=1.
%\]
%
%\noindent\underline{Step 2:} We will pass to field extensions of $\Q$ using the following lemma.
%\begin{lem}[Extension formula]\llabel{extension-formula}
%Let $K\subeq L$ be number fields and let $v$ be a place of $K$. Then
%\[
%\prod_{w\mid v} \ve{\al}_w =\ve{\nm_{L/K}\al}_v.
%\]
%\end{lem}
%\begin{proof}
%Let $L_w$ be the completion of $L$ with respect to $w$. 
%We pass to complete fields using Corollary~\ref{complete-ntr} and then use BLAH.
%\[
%\ve{\nm_{L/K}\al}\stackrel{\ref{complete-ntr}}{=} \prod_{w\mid v} \ve{\nm_{L_w/\hat K}(\al)}
%\stackrel{!}= \prod_{w\mid v}\ve{\al}^{[L_w:\hat K]}
%\stackrel{\ref{complete-ntr}}=\prod_{w\mid v} \ve{\al}_w.
%\]
%\end{proof}
%
%\noindent\underline{Step 3:} Compute using steps 1 and 2.
%\[
%\prod_{w\text{ place of }L} |\al|_w=\prod_{v\text{ place of }\Q} \prod_{w\mid v}|\al|_w=\prod_v|\al|_v=1.
%\]
%\end{proof}
\noindent\underline{Step 1:} We first show the result for $K=\Q$. Given $n\in \Q$, factor it as $n=\pm \prod_{i=1}^{\iy}p_i^{a_i}$ where $p_i$ are all the prime numbers; note only a finite number of the $a_i$ are nonzero. Then 
\[
|\al|=\pa{\prod_{i=1}^{\iy} |\al|_{p_i}}|a|_{\iy}=\pa{\prod_{i=1}^{\iy} p_i^{-a_i}}\pa{\prod_{i=1}^{\iy} p_i^{a_i}}=1.
\]

\noindent\underline{Step 2:} We pass to field extensions of $\Q$ using the following lemma.
\begin{lem}[Extension formula]\llabel{extension-formula}
Let $K\subeq L$ be number fields and let $v$ be a place of $K$.  Then
\[
\prod_{w\mid v} \ab{\al}_w
%\prod_{w\mid v}\ve{\al}_w
=\ab{\nm_{L/K}\al}_v.
\]
\end{lem}
\begin{proof}
For a place on $L$ let $|\cdot|_w'$ be the valuation normalized so that it extends $v$. 
We have
\begin{align*}
\ab{\nm_{L/K}\al}_v&= \prod_{w\mid v} \ab{\nm_{L_w/ K_v}(\al)}_v\\
&= \prod_{w\mid v} \ab{\nm_{L_w/ K_v}(\al)}_{w}'\\
&= 
\prod_{w\mid v} \ab{\nm_{L_w/ K_v}(\al)}_{w}^{\rc{[L:K]}}
&\text{by Lemma~\ref{compare-normed-val}}\\
&=\prod_{w\mid v}\ab{\al}_w
&\text{by Theorem~\ref{extend-discrete-valuations}}
%&\text{by Lemma~\ref{local-global-degree}}\\
%=\prod_{w\mid v}\ve{\al}_w.
%\stackrel{\ref{complete-ntr}}
%=\ab{\al}_v^{[L:K]}.
\qedhere
\end{align*}
\end{proof}

\noindent\underline{Step 3:} Since every place on $K$ restricts to a unique place on $\Q$,
\[
\prod_{w\in V_K} \ab{\al}_w
=\prod_{v\in V} \prod_{w\mid v}\ab{\al}_w
%=\prod_{v\in V} \prod_{w\mid v}|\al|_w^{[K_w:\Q_v]}
%\stackrel{~\ref{extension-formula}}\
=\prod_{v\in V}|\nm_{L/K}(\al)|_v\stackrel{\text{Step }1}=1,
\]
where we apply step 1 to $\nm_{L/K}(\al)$.
\end{proof}
The product formula will be useful when defining a measure of size independent of scaling (see Chapter~\ref{height-functions}).
%%%
%\section{?}
%\begin{lem}\llabel{norm-val-ext}
%Let $K$ be a complete field with normalized valuation $|\cdot|$, let $L/K$ be a finite separable extension, and extend $|\cdot|$ to $L$.
%
%If $\ve{\cdot}$ is the normalized valuation of $|\cdot|$, then $\ve{\cdot}=|\cdot|^{n}$, where $n=[L:K]$.
%\end{lem}
%\begin{proof}
%Let $\mfp=(\pi)$ and $\mP=(\Pi)$ be the prime ideals of $\cO_K$ and $\cO_L$, respectively. Then $\pi=c\Pi^e$ for some $c\in \cO_L^{\times}$. Letting $f$ be the residue degree,
%\[
%\ve{\pi}=\ve{\Pi}^e?=?\prc{\N\mP}^e=\prc{\N\mfp}^{ef}=\ab{\pi}^{n}.
%\]
%\end{proof}
\section{Problems}
\begin{enumerate}
\item
Let $K$ be a complete nonarchimedean field whose residue field has characteristic $p$. Prove that the maximal tamely ramified (separable) extension of $K$ is
\[
K_{\text{tr}}=K_u\pa{\set{\pi^{\rc m}}{p\nmid m}}.
\]
\end{enumerate}
%\index{ramification groups}
%\index{decomposition group}
%\index{inertia group}
%\section{Ramification groups}
%Let $L/K$ be finite Galois, and suppose $K$ has perfect residue field. Let $B$ be the ring of integers in $L$, $\mP=(\Pi)$ be the maximal ideal of $B$, and $c=|\Pi|$.
%\begin{df}
%The \textbf{decomposition group} of the extension $L/K$ is
%\[
%D(L/K)=\set{\si\in G{L/K}}{\si(\mP)=\mP}.
%\]
%The $n$th \textbf{ramification group} is
%\begin{align*}
%G_i &=\set{\si\in G}{|\si(\al)-\al|<c^i\text{ for all }\al\in B}\\
%&=\set{\si\in G}{\ord_{\Pi}(\si(\al)-\al)>i\text{ for all}\al\in B}.
%\end{align*}
%In particular, the \textbf{inertia group} is
%\begin{align*}
%%D_{\mP}(L/K)=decomp grp
%G_0&=\set{\si\in G(L/K)}{\si(\al)-\al\in \mP}\\
%&=\ker (\ep: D\to G(l/k))
%\end{align*}
%and the ramification group is
%\[
%G_1=\set{\si\in G}{\ord_{\Pi}(\si(\al)-\al)>1}.
%\]
%\end{df}
%\begin{pr}[Properties of ramification groups]$\,$
%We have the filtration
%\[G\supeq D\supeq G_0\supeq G_1\supeq \cdots.\]
%\begin{enumerate}
%\item $G_0/G_1\hra l^{\times}$ and $G_i/G_{i+1}\hra l$.
%\item The $G_i$ are normal subgroups of $G$.
%\item For sufficiently large $i$, $G_i=\{1\}$.
%\item For $i\ge 1$, to check an element is in $G_i$ it suffices to check that $|\si\Pi-\Pi|<|\Pi|^i$:
%\end{enumerate}
%\end{pr}
%The fixed field of $G_0$ is the largest unramified extension $K_0$ of $K$ in $L$, and 
%\[
%G/G_0=G(K_0/K)=G(l/k).
%\]
%BLAH
%\begin{pr}
%If $l/k$ is normal, then $\ep:D/T\xra{\cong} G(l/k)$.
%\end{pr}
%%%%%%%%%%
%%%%%%%%%%
%\begin{thm}
%Suppose $L/K$ is a Galois extension; suppose $\mfp$ in $K$ factorizes as
%\[
%\mfp=\prod_{j=1}^g \mP_j^e.
%\]
%Take any $\mP\mid \mfp$ and let $\mP_D=L^{D(\mP)}$. Then
%\begin{enumerate}
%\item
%$[L^{\mP}:K]=g$ and $\mP_{D}$ is non-split in $L$.
%\item
%$\mfp$ is totally split in $L^{D(\mP)}$ iff $L^{D(\mP)}/K$ is Galois.
%\end{enumerate}
%\end{thm}
%\begin{proof}
%\begin{enumerate}
%\item
%By the Fixed Field Theorem, $D(\mP)=G(L/L^{D(\mP)})$. 
%Since $L/L^{D(\mP)}$ is Galois, $D(\mP)$ acts transitively on the primes of $L$ above $\mP_D$. However, $D(\mP)$ stabilizes $\mP$; thus $\mP$ is the only prime above $\mP$.
%\item
%\end{enumerate}
%\end{proof}
%\begin{df}
%Define the inertia group to be
%\[
%I(\mP):=\ker(\phi:D(\mP)\to G(l/k)).
%\]
%\end{df}


