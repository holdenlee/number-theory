\chapter{Rings of integers}\llabel{ring-of-integers}
When we have a field extension $L$ of $\Q$, we would like to define a ring of integers for $L$, with properties similar to the ring $\Z\subeq \Q$. We will define this ring of integers in a slightly more general context.
\section{Integrality}
\index{integral}
\begin{df}
Let $A$ be an integral domain and $L$ a field containing $A$. An element of $x\in L$ is \textbf{integral} over $A$ if it is the zero of a monic polynomial with coefficients in $A$:
\[
x^n+a_{n-1}x^{n-1}+\cdots +a_1x+a_0=0,\, n\ge 1,\, a_0,\ldots, a_{n-1}\in A.
\]
%(``$P=0$" is called an equation of integral dependence for $\al$.) 
The \textbf{integral closure} of $A$ in $L$ is the set of elements of $L$ integral over $A$.
\end{df}
\begin{ex}
The integral closure of $\Z$ in $\Q$ is simply $\Z$ itself (we see this more generally in Proposition~\ref{rational-roots-thm}). Thus, integral closure generalizes the notion of what it means to be an ``integer" in other number fields. As we will see in Example~\ref{quadratic-extensions}, for $d$ squarefree, the integral closure of $\Q(\sqrt{d})$ is $\Z[\sqrt{d}]$ when $d\equiv 3\pmod 4$ and $\Z\ba{\fc{1+\sqrt{d}}2}$ when $d\equiv 1\pmod 4$. Algebra is much nicer in integral extensions---which is why, for instance, we would study $\Z\ba{\fc{1+\sqrt{-3}}2}$ rather than just $\Z[\sqrt{-3}]$.
\end{ex}
%\begin{rem}
%If $A$ is a PID (for example $A=\Z$), 
%then by Gauss's Lemma this is equivalent to saying that the {\it minimal polynomial} of $x$ has coefficients in $A$.
%\end{rem}
\begin{thm}
Let $L$ be a field containing the ring $A$. Then
the elements of $L$ integral over $A$ form a ring.
%The integral closure of $A$ is a ring.
\end{thm}
\begin{proof}
We give two proofs. We need to show that if $a,b$ are algebraic over $A$ then so are $a+b$ and $ab$. 

\noindent \underline{Proof 1}: Let $p,q$ be the minimal polynomials of $a,b$, let $a_1,\ldots, a_k$ be the conjugates of $a$ and $b_1,\ldots, b_l$ be the conjugates of $b$. %For $r\in A$, the t
The coefficients of
\[%\prod_i (x-ra_i),
\prr{1\le i\le m}{1\le j\le n} (x-(a_i+b_j)),\quad \prr{1\le i\le m}{1\le j\le n} (x-(a_ib_j))\]%,\prod_{i} (x-(1/a_i))\]
are symmetric in the $a_i$ and symmetric in the $b_j$ so by the Fundamental Theorem of Symmetric Polynomials can be written in terms of the elementary symmetric polynomials in the $a_i$ and in the $b_j$, with coefficients in $A$. By Vieta's Theorem these are expressible in terms of the coefficients of $p,q$, which are in $A$. Hence these polynomials have coefficients in $A$. They have $a+b,ab$ as roots, as desired.\\

\noindent \underline{Proof 2}:
We use the following lemma.
\begin{lem}[Criterion for integrality]\llabel{criterion-for-integrality}
An element $\al\in L$ is integral over $A$ if and only if there exists a nonzero finitely generated $A$-submodule of $L$ such that $\al M\subeq M$. If so, then we can take $M=A[\al]$.
\end{lem}
\begin{ex}
For example, $\frac{1}{\sqrt2}$ fails this criterion over $\Z$---multiplying by it has the effect of making $M$ ``finer." $\sqrt{2}$, however, is integral.

In the case $A=\Z$ and $B=\Q$, $a\in \Q$ is integral over $\Z$ iff $a\in \Z$. Indeed, $a\in \Z$ satisfies $x-a$, and if $a\nin \Z$, then powers of $a$ contain arbitrarily large denominators so $\Z[\al]$ is not finitely generated.
\end{ex}
\begin{proof}
$\Rightarrow$: If $\al$ satisfies a monic polynomial of degree $n$, then $A[\al]$ is generated by $1,\al,\ldots, \al^{n-1}$.
%\al^n,\ldots are linear comb of these by monic poly eq.
%Warning gen as A-module means not allowed to make products.

$\Leftarrow$: Suppose $M$ is generated by $v_1,\ldots, v_n$. Then we can find a matrix $T$ with coefficients in $A$ such that
\[\al \begin{bmatrix}v_1\\ \vdots \\ v_n\end{bmatrix}=
T\begin{bmatrix}v_1\\ \vdots \\ v_n\end{bmatrix}.\]
Since $v_1,\ldots, v_n\neq 0$, $\al I-T$ is singular, and $\det(\al I-T)=0$. This gives a monic polynomial equation satisfied by $\al$.
\end{proof}
Now for $\al, \be\in L$ and let $M=A[\al]$ and $N=A[\be]$. 
Note
\begin{enumerate}
\item
if $M,N$ are finitely generated by $\{\al_i\}$ and $\{\be_j\}$, then
$MN$ is finitely generated by $\{\al_i\be_j\}$.
\item 
$\al\be MN\subeq MN$ and $(\al+\be)MN\subeq MN$.
\end{enumerate}
Hence $\al\be$ and $\al+\be$ are integral over $A$ by Lemma~\ref{criterion-for-integrality} as needed.
\end{proof}
%\begin{df}
%The integral closure of $\Z$ in $L$ is called the \textbf{ring of integers} $\cO_L$.
%\end{df}
For the rest of this chapter, $A$ is an integral domain, $K$ is its fraction field, $L$ is an extension of $K$, and $B$ is the integral closure of $A$ in $L$. %This is the setup for much of algebraic number theory.
\begin{equation}\llabel{aklb}
\xymatrix{
L\ar@{-}[r]\ar@{-}[d]&B\ar@{-}[d]\\
K\ar@{-}[r]&A
}
\end{equation}
\index{normality}
\begin{df}
$A$ is \textbf{integrally closed} or \textbf{normal} if its integral closure in $K=\Frac(A)$ is itself.
\end{df}

\begin{pr}
If $L$ is algebraic over $K$ then every element of $L$ can be written as $\fc{b}{a}$ where $b\in B$ and $a\in A$. Thus  $L=\Frac(B)$.
In particular, for any extension $L/\Q$, $\Frac(\cO_L)=L$.
\end{pr}
\begin{proof}
Given $\al\in L$, suppose that it satisfies the equation
\[
P(x):=a_nx^n+a_{n-1}x^{n-1}+\cdots +a_0=0
\]
with $a_0,\ldots, a_n\in K$ and $a_n\ne 0$. Since $\Frac(A)=K$, by multiplying by an element of $A$ as necessary we may assume $a_0,\ldots, a_n\in A$. Then
\[
a_n^{n-1}P\pf xd:=x^n+a_{n-1}x^{n-1}+a_na_{n-2}x^{n-2}+\cdots +a_n^{n-1}a_0.
\]
Hence $a_n\al$ is integral over $A$, i.e. $a_n\al \in B$. This shows $\al$ is in the desired form.

For the last part, take $K=\Q$ and $A=\Z$.
\end{proof}
For short we call~(\ref{aklb}) the ``AKLB" setup if we further assume $A$ is integrally closed in $K$. In the usual case, $A$ is the integral closure of $\Z$ in $K$. in this case, we write $A=\cO_K$.\footnote{Later on, when we take $K$ to be an extension of the $p$-adic field $\Q_p$, we will use $\cO_K$ to denote the integral closure of $\Z_p$ in $K$.}

When $F=\ol{\Q}$, the algebraic closure of $\Q$, $a\in \ol{\Q}$ is called an algebraic number and $a\in \cO_{\Q}$ is an algebraic integer.
\index{rational roots theorem}
\begin{thm}[Rational Roots Theorem]\llabel{rational-roots-thm}
A UFD is integrally closed.
\end{thm}
\begin{proof}
Suppose $R$ is a UFD with field of fractions $K$. Let $x\in K$ be integral over $R$; suppose $x$ satisfies
\[
x^n+a_{n-1}x^{n-1}+\cdots +a_0=0
\]
where $a_0,\ldots, a_{n-1}\in R$. Write $x=\frac pq$ where $p,q\in R$ are relatively prime. Then multiplying the above by $q^n$ gives
\begin{align*}
p^n+a_{n-1}p^{n-1}q+\cdots +a_1pq^{n-1} +a_0q^n&=0\\
q(a_{n-1}p^{n-1}+\cdots +a_0q^{n-1})&=-p^n
\end{align*}
Thus $q\mid p$, possible only if $q=1$. This shows $x\in R$.
\end{proof}
Note that in the definition of integrality, an element is integral if it is the zero of {\it any} monic polynomial in $A[x]$. However, it suffices to check that its {\it minimal} polynomial is in $A[x]$.
\begin{pr}\llabel{integral-min-poly}
Let $L$ be an algebraic extension of $K$ and $A$ be integrally closed. Then $\al\in L$ is integral over $A$ iff its minimal polynomial $f$ over $K$ has coefficients in $A$.
\end{pr}
\begin{proof}
The reverse direction is clear. For the forward direction, note all zeros of $f$ are integral over $K$ since they satisfy the same polynomial equation that $\al$ satisfies. The coefficients of $f$ are polynomial expressions in the roots so are integral over $A$, and hence in $A$ (since they are already in $K$).
\end{proof}
\begin{pr}[Finite generation]\llabel{finite-generation}$\,$
\vspace{0cm}
\begin{enumerate}
\item Let $A\subeq B\subeq C$ be rings. If $B$ is finitely generated as an $A$-module and $C$ is finitely generated as a $B$-module, then $C$ is finitely generated as an $A$-module.
\item If $B$ is integral over $A$ and finitely generated as an $A$-algebra, then it is finitely generated as an $A$-module.\qedhere
\end{enumerate}
\end{pr}
\begin{proof}$\,$
\vspace{0cm}
\begin{enumerate}
\item Take products of generators.
\item 
Let algebra generators be $\be_1,\ldots, \be_m$. Then
\[A\subeq A[\be_1]\subeq \cdots \subeq A[\be_1,\ldots, \be_m]\]
is a chain of integral extensions, so item 2 follows from 1.\qedhere
\end{enumerate}
\end{proof}
Combining this proposition with Lemma~\ref{criterion-for-integrality} we get the following:
\begin{pr}[Transitivity of integrality]\llabel{integrality}
Let $A\subeq B\subeq C$ be integral domains and $K$, $L$, $M$ be their fraction fields.
\begin{enumerate}
\item If $B$ is integral over $A$ and $C$ is integral over $B$, then $C$ is integral over $A$.
\item Let $A'$ is the integral closure of $A$ over $B$ and $A''$ be the integral closure of $A'$ over $C$. Let $A'''$ be the integral closure of $A$ in $C$. %Then $A''=A'''$ and $A'=A''\cap L$.
Then $A''=A'''$.
\item The integral closure of $A$ is integrally closed.
\end{enumerate}
\end{pr}
\begin{proof}$\,$
\vspace{0cm}
\begin{enumerate}
\item For $\ga\in C$, let $b_i$ be the coefficients of the minimal polynomial of $C$ over $B$. Then $\ga$ is integral over $A[b_0,\ldots, b_m]$, so by Proposition~\ref{finite-generation}, item 2, $A[b_0,\ldots, b_m,\ga]$ is finitely generated over $A$. Since $\ga A[b_0,\ldots, b_m,\ga]\subeq A[b_1,\ldots, b_m,\ga]$, by Lemma~\ref{criterion-for-integrality}, $\ga$ is integral over $A$.
\item By item 1 applied to $A\subeq A'\subeq A''$, $A''$ is integral over $A$ so $A''\subeq A'''$. Conversely, any element $a\in A'''$ is integral over $A$ so {\it a fortiori} integral over $A''$; thus $A'''\subeq A''$.
\item Follows from item 2 applied to $A=B=C$.\qedhere
%Follows from item 1 applied to $A\subeq \bar{A}\subeq \bar{\bar{A}}$.
\end{enumerate}
\end{proof}

\section{Norms and Traces}
Let $B$ be a free $A$-module of rank $n$. Then any element $\be\in B$ defines an $A$-linear map $m_{\be}$ (or $[\be]$), multiplication by $\be$. It is helpful to think of $\be$ as a linear map because then we can apply results from linear algebra.
\begin{df}\llabel{trace-det-char}
The trace, determinant, and characteristic polynomial of $m_{\be}$ are called the {\textbf{trace}}, {\textbf{norm}}, and {\textbf{characteristic polynomial}} of $\be$.
\end{df}
These are computed by choosing any basis of $e_1,\ldots, e_n$ for $B$ over $A$, and then computing the action of $\be$ on this basis.
\begin{pr}[Elementary properties]\llabel{nm-elem-pr}
The following hold ($a\in A; \be,\be'\in B$):
\begin{enumerate}
\item $\tr(\be+\be')=\tr(\be)+\tr(\be')$
\item $\tr(a\be)=a\tr(\be)$
\item $\tr(a)=na$
\item $\nm(\be\be')=\nm(\be)\cdot \nm (\be')$
\item $\nm(a)=a^n$
\end{enumerate}
\end{pr}
\begin{pr}[Behavior with respect to field extensions]\llabel{ntr-fext}
Suppose $L/K$ is a degree $n$ field extension, $M$ is a finite extension of $L$, and $\be\in L$.
\begin{enumerate}
\item (Relationship with roots of minimal polynomial) If $f(X)$ is the minimal polynomial of $\be$ over $K$ and $\be_1,\ldots, \be_m$ are the roots of $f(X)=0$ in a Galois closure of $K$, then letting $r=[L:K(\be)]=\frac nm$, 
\begin{enumerate}
\item
$\chr_{L/K}(\be)=f(X)^r$
\item
$\tr_{L/K}(\be)=r(\be_1+\cdots +\be_m)$
\item
$\nm_{L/K}(\be)=(\be_1\cdots \be_m)^r$
\end{enumerate}
\item (Relationship with embeddings) Suppose $L$ is separable over $K$, $M$ is a Galois extension of $K$, and $\sigma_1,\ldots, \sigma_n$ are the set of distinct embeddings $L\to M$ fixing $K$. 
%(which can be considered as elements of $G(M/K)/G(M/L)$).
Then
\begin{enumerate}
\item
$\tr_{L/K}(\be)=\sigma_1(\be)+\cdots +\sigma_n(\be)$
\item
$\nm_{L/K}(\be)=\sigma_1(\be)\cdots \sigma_n(\be)$
\end{enumerate}
In particular, this is true when $L=M$ is a Galois extension of $K$, and we can think of the $\si_k$ as simply the elements of $G(L/K)$.
\item (Transitivity of trace and norm) Suppose $\be\in M$ and $M/K$ is separable.\footnote{The last condition is not necessary. TODO: Find a proof of the general case.} Then
\begin{enumerate}
\item
$\tr_{M/K}(\be)=\tr_{L/K}(\tr_{M/L}(\be))$
\item
$\nm_{M/K}(\be)=\nm_{L/K}(\nm_{M/L}(\be))$
\end{enumerate}
\item (Integrality) %Let $A$ be an integral domain with fraction field $K$, let $L$ be a finite %separable (not nec.)
%extension of $K$, and $B$ the integral closure of $A$ in $L$.
Assume AKLB. If $\be\in B$, then the coefficients of $\chr_{L/K}(\be)$, and hence $\tr_{L/K}(\be)$ and $\nm_{L/K}(\be)$, are integral over $A$. In particular, if $A$ is integrally closed in $L$ then they are in $A$.
\end{enumerate}
\end{pr}
\begin{proof}$\,$
\begin{enumerate}
\item If $r=1$, i.e. $K[\be]=L$, then by the Cayley-Hamilton Theorem, $f(m_{\be})=0$. Since $f(X)$ is irreducible, $f(X)\mid \chr_{L/K}(\be)$. However, these are monic polynomials of the same degree so they are equal. 

In the general case, take a basis $x_i$ of $K[\be]$ over $K$ and a basis $y_j$ of $L$ over $K[\be]$. Then $x_iy_j$ form a basis of $L$ over $K$, and the matrix of $m_{\be}$ with respect to this basis is $n$ copies of $A$. This proves (a), which implies the rest of the statements.
%
%Alternatively, consider the action of $\be$ on its conjugates.
\item
%All embeddings of $L$ reside in the Galois closure of $L$ so we may assume $M=L^{\text{gal}}$.
%Note the $\si_k$ are restrictions of the automorphisms of $M$ fixing $K$, so we may consider them as automorphisms of $M$ fixing $K$, modded out by those fixing $L$. Then $\{\si_1,\ldots, \si_n\}=G(M/K)/G(M/L)$. Now $G(M/K)$ acts transitively on the conjugates of $\be$, say $\be=\be_1,\ldots, \be_m$, and the stabilizer of any $\be_k$ has order $\frac nm$. Thus the result follows from item 1.
%Let $\be_1,\ldots, \be_m$ be the conjugates of $\be$. Since automorphisms of $M$ fixing $K$ operate transitively on the $\be_k$, we have $\{\si_1(\be),\ldots,\si_n(\be)\}=\{ \be_1,\ldots, \be_m\}$. Each value $\be_k$ must be attained the same number of times, i.e. $r:=\frac nm$ times (since if $\si_j$ takes $\be$ to $\be_k$, it can be written as $\si_{j'}\si_{i}$ for some fixed $j'$ and $\si_i$ taking $\be$ to itself). The result follows from item 1.
Let $\be_1,\ldots, \be_m$ be the conjugates of $\be$. There are $m$ distinct imbeddings $K(\be)\to M$; they each take $\be$ to a different $\be_k$. Each of these imbeddings extend to $r:=[L:K(\be)]=\frac{n}{m}$ imbeddings $L\to M$. %\footnote{If $K\subeq K'\subeq L\subeq M$ and $M$ is Galois over $K$, then each imbedding $K'\to M$ fixing $K$ extends to $[L:K']$ imbeddings $L\to M$ fixing $K$. Proof: Let $\pi$ be the quotient map $G(M/K)/G(M/L)\to G(M/K)/G(M/K')$. Then $\ker \pi=[L:K']$.}
Now use item 1.
\item Note that for any finite extensions $K\subeq L\subeq N$ with $N$ Galois, an imbedding $L\hookrightarrow N$ fixing $K$ can be extended to a $K$-automorphism on $N$, and so be considered an element of the set $G(N/K)/G(N/L)$.\footnote{Using the primitive element theorem, write $L=K(\be)$. The imbeddings $L\to N$ are those taking $\be$ to a conjugate; there are $[L:K]$ imbeddings. But we know $G(N/K)/G(N/L)=[L:K]$, so all of the imbeddings must be extendable. We also use this fact (in addition to a counting argument) in the proof of 2.}

Let $N$ be a Galois extension containing $M$. 
By item 2,
\begin{align*}
\tr_{M/K}(\be)&=\sum_{\si\in G(N/K)/G(N/M)}\si(\be)\\
\tr_{L/K}(\tr_{M/L}(\be))&=\tr_{L/K}\pa{
\sum_{\si\in G(N/L)/G(N/M)}\si(\be)
}\\
&=\sum_{\tau\in G(N/K)/G(N/L)}\sum_{\si\in G(N/L)/G(N/M)}\tau(\si(\be))
\end{align*}
where in the second sum we take arbitrary representatives $\tau\in G(N/K)$ and $\si\in G(N/L)$.
These are equal because for any choice of these representatives, 
\[\{\si\in G(N/K)/G(N/M)\}=\{\tau\si\mid \tau\in G(N/K)/G(N/L),\si\in G(N/L)/G(N/M)\}\]
when considered in $G(N/K)/G(N/M)$ (i.e. as imbeddings $M\hookrightarrow N$ fixing $K$). 
The same is true of the norm.
\item 
The minimal polynomial of $\al$ has coefficients in $A$, by Proposition~\ref{integral-min-poly}. Hence the result follows from item 1.
%Suppose $P(\be)=0$ where $P$ is monic with coefficients in $A$. 
%The conjugates of $\be$ with respect to $K$ are all integral over $A$ since they are also zeros of $P$.
%
%Let $f(X)$ be the minimal polynomial. (Warning: $f(X)$ may be different from $P(X)$.) The coefficients of $f(X)$ are symmetric polynomials of the conjugates, so are integral over $A$. The conclusion follows from item 1. For the second part, note that the coefficients of $f(X)$ are (by definition) in $K$.\qedhere
\end{enumerate}
\end{proof}
\section{Discriminant}
\begin{df}\llabel{disc-df}
%For $L$ a finite extension of $K$, $B(\al,\be):=\tr_{L/K}(\al \be)$ is a symmetric bilinear form on $L$ as a vector space over $K$. The \textbf{discriminant} of $\be$ is the discriminant of the bilinear form,
%\[D(
%
If $B$ is a ring and free $A$-module of rank $m$, and $\be_1,\ldots, \be_m\in B$, then their \textbf{discriminant} is
\[D(\be_1,\ldots, \be_m)=\det[\tr_{B/A}(\be_i\be_j)]_{1\leq i,j\leq m}.\]

\begin{pr}\llabel{disc-basis}
If the change of basis matrix from $\ga_i$ to $\be_i$ is $T$, then 
\[D(\ga_1,\ldots, \ga_m)=\det(T)^2\cdot D(\be_1,\ldots, \be_m).\] 
\end{pr}
\begin{proof}
Let $M_1$ and $M_2$ be the matrices of the bilinear form
\[
(\al,\al')=\tr_{B/A}(\al\al')
\]
with respect to the bases $(\be_1,\ldots, \be_m)$ and $(\ga_1,\ldots, \ga_m)$, respectively.
Then, using the change of basis formula for bilinear forms,
\begin{align*}
D(\be_1,\ldots, \be_m) &= \det(M_1)\\
D(\ga_1,\ldots, \ga_m) &=\det(M_2)\\
M_2&=T^tM_1T\\
\det(M_2)&=\det(T)^2\cdot \det(M_1)
\end{align*}
from which the result follows.
\end{proof}

Consider the discriminant of an arbitrary basis of $B$ over $A$. 
By the above fact, this is well-defined up to multiplication by the square of a unit. The residue in $A/(A^{\times})^2$ is called the discriminant $\disc(B/A)$. The discriminant also refers to the ideal of $A$ this element generates.

Note $\disc(B/A)$ can be thought of as the determinant of the matrix of the bilinear form $(\be,\be')=\tr_{B/A}(\be\be')$.
\end{df}
\begin{pr}[Criterion for integral basis]\llabel{crit-int-basis}
Let $A\subeq B$ be integral domains and $B$ be a free $A$-module of rank $m$ with $\disc(B/A)\neq 0$. Then $\ga_1,\ldots, \ga_m\in B$ form a basis for $B$ as an $A$-module iff
\[(D(\ga_1,\ldots, \ga_m))=(\disc(B/A))\]
as ideals.
\end{pr}
\begin{proof}
Let $\be_i$ be a basis. 
If the change of basis matrix from $\ga_i$ to $\be_i$ is $T$, then by Proposition~\ref{disc-basis},
\[
D(\ga_1,\ldots, \ga_m)=\det(T)^2\cdot D(\be_1,\ldots, \be_m)
=\det(T)^2\disc(B/A)
\]
%In the remark above,
Now $\ga_i$ is basis iff $T$ is invertible, iff $\det(T)$ is a unit, iff $(D(\ga_1,\ldots, \ga_m))=(\disc(B/A))$.
\end{proof}
\begin{pr}[Discriminants and Field Extensions]\llabel{disc-and-fe}$\,$
\begin{enumerate}
\item (Relationship with embeddings)
Let $L$ be separable finite over $K$ of degree $m$, and $\sigma_1,\ldots,\sigma_m$ be the embeddings of $L$ into a Galois extension $M$ fixing $K$. Then for any basis $\be_1,\ldots, \be_m$ of $L$ over $K$,
\[
D(\be_1,\ldots, \be_m)=\det(\sigma_i\be_j)^2\neq 0.
\]
%\item (from LF, Serre p.51; add proof) If $M/L$ is separable of degree $n$, $L/K$ is separable, and $B,C,A$ are the ring of integers, then
%\[
%\disc(C/A)=\disc(B/A)^n\nm(\disc(C/B)).
%\]
\item (Nondegeneracy of trace pairing) If $B$ is free of rank $m$ over $A$ (with fraction fields $K,L$ as above), then the pairing \[(\be,\be')\mapsto \tr(\be\be')\] is a perfect $K$-bilinear pairing, and $\disc(B/A)=\disc(K/L)\neq 0$.
\end{enumerate}
Here {\it perfect} means that the map $a\mapsto (b\mapsto (a,b))$ is an isomorphism $L\to L^*$, and similarly for $b\mapsto (a\mapsto(a,b))$. This is equivalent to saying that the bilinear form is nondegenerate. %compare to nondegenerate 
\end{pr}
\begin{proof} (Unfinished) 
Use Proposition~\ref{ntr-fext}(1b), and that $\si_k,\det$ are both multiplicative. Inequality follows from independence of characters:

Let $G$ be a group, $F$ a field. Then the homomorphisms $G\to F^{\times}$ are linearly independent. 
%(In fact, the same is true if $F^{\times}$ is replaced by $\End(V)$ where $V$ is a finite-dimensional vector space over $F$.)?
\end{proof}
Thus for $K$ of degree $m$ over $\Q$, we can talk of $\disc(\cO_K/\Z)$.

A closely related quantity to the discriminant is the {\it different}.
\begin{df}
Assume AKLB, and suppose $L/K$ is a finite separable extension. The \textbf{codifferent} of $B$ with respect to $A$ is
\[
B^*=\{y\in L\mid \tr(xy)\in A \text{ for all }x\in B\}.
\]
The \textbf{different} of $B$ with respect to $A$ is
\[
\mathfrak D_{B/A}=(B^*)^{-1}.
\]
In other words, it is the largest $B$-submodule satisfying $\tr(E)\subeq A$.
\end{df}
Note that $\mD_{B/A}=(B^*)^{-1}$.
\begin{rem}\llabel{two-def-disc}
We will define the discriminant in general, when $B$ is not necessarily a free $A$-module, in Chapter~\ref{ramification}.
The relationship between the two definitions is the following: Let $\mfp$ be an ideal in $A$. Then $A_{\mfp}$ is a principal ideal domain (in fact, a DVR). Let $S=A-\mfp$; then $S^{-1}B$ is free over $S^{-1}A$ by the structure theorem for modules. We have $(\disc(S^{-1}B/S^{-1}A))=(\mfp A_{\mfp})^{m(\mfp)}$ for some $m(\mfp)$. Then
\[
\disc(B/A)=\prod_{\mfp}\mfp^{m(\mfp)}.
\]
\end{rem}
%(TODO: Add Serre's alternative description of the discriminant (def'd in terms of $\chi$).)

%Note that $\mathfrak D_{B/A}\subeq B\subeq B^*$.
%\begin{pr}(Behavior with respect to field extensions)
%Suppose
%\[
%\xymatrix{
%M\ar@{-}^n[r]\ar@{-}[d]&C\ar@{-}[d]\\
%L\ar@{-}[r]\ar@{-}[d]&B\ar@{-}[d]\\
%K\ar@{-}[r]&A
%}.
%\]
%\begin{enumerate}
%\item (Transitivity) 
%$\mathfrak D_{C/A}=\mathfrak D_{C/B}\mathfrak D_{B/A}$.
%\item Blah.
%\end{enumerate}
%[TODO: Add differential characterization of different.]
%\end{pr}
\section{Integral bases}
\begin{pr}[Finite generation of integral extensions]\llabel{fgoie}
Let $A$ be integrally closed and $L$ separable of degree $m$ over $K$. There are free finite $A$-submodules $M$ and $M'$ of $L$ such that $M\subeq B\subeq M'$.
$B$ is a finitely generated $A$-module if $A$ is Noetherian, and free of rank $m$ if $A$ is a PID.\footnote{Alternative proof: proceed as in 5.8.}
\end{pr}
\begin{proof}
Let $\{\be_1,\ldots, \be_m\}\subeq B$ be a basis for $L$ over $K$. Take a basis $\be_i'$ so that $\tr(\be_i\be_j')=\delta_{ij}$. Then
\[A\be_1+\cdots +A\be_m\subeq B\subeq A\be_1'+\cdots +A\be_m'.\]
The second inclusion follows because if $\be\in B$, then writing $\be=\sum_j b_j\be_j'$, we have that $b_i=\tr(\be\be_i)\in A$. (In other words, the $\be_i'$ form a basis for the codifferent $B^*$, which contains $B$.)\fixme{Move some of the stuff on codiff?}

If $A$ is Noetherian, then $M'$ is finitely generated, so its submodule $B$ is finitely generated over $A$. If $A$ is a PID, then by the Structure Theorem for Modules (over PIDs), $M$ is a direct sum of cyclic modules and a free module. Since it is contained in a free module of rank $m$ and contains a free module of rank $m$, it must be free of rank $m$.
\end{proof}
The following is immediate:
\begin{thm}
If $K$ is finite over $\Q$ (i.e. a number field), then $\cO_K$ is a finitely generated $\Z$-module. It is the largest subring that is finitely generated over $\Z$.
\end{thm}
\begin{df}
A basis for $\cO_K$ as a $\Z$-module is called an \textbf{integral basis}.
\end{df}
\begin{pr}\llabel{disc-calc}
Suppose $K$ has characteristic 0 (so $L$ separable over $K$), $L=K[\be]$, and $f$ is the minimal polynomial of $\be$ over $K$. Let $f(X)=\prod (X-\be_i)$ in the Galois closure of $L$. Then 
\[D(1,\be,\ldots, \be^{m-1})=\prod_{1\leq i<j\leq m}(\be_i-\be_j)^2=(-1)^{m(m-1)/2}\cdot \nm_{L/K}(f'(\be)).\]
This is called the discriminant of $f$.\footnote{This gives an alternative proof of the perfect pairing. }
\end{pr}
\begin{proof}
Note the $\be_i$ are conjugates of $\be$; assume $\be=\be_1$.

By Proposition~\ref{disc-and-fe}, we have
\[
D(1,\be,\ldots, \be^{m-1})=
\left|\begin{array}{cccc}
1 & \be_{1} & \cdots & \be_1^{m-1}\\
1 & \be_{2} & \cdots & \be_{2}^{m-1}\\
\vdots & \vdots & \ddots & \vdots\\
1 & \be_{m} & \cdots & \be_{m}^{m-1}\end{array}\right|^{2}
=
\prod_{1\leq i<j\leq m}(\be_i-\be_j)^2,
\]
where the last statement follows by evaluating the Vandermonde determinant.

For the second equality, note by Proposition~\ref{ntr-fext}(1c) that
\begin{align*}
\nm_{L/K}(f'(\be))=\nm_{L/K}((\be_1-\be_2)\cdots (\be_1-\be_m))&=\prod_{1\le i\le m}\prod_{1\le j\le m,\,j\ne i}
(\be_i-\be_j)\\
&=(-1)^{\frac{m(m-1)}2}
\prod_{1\leq i<j\leq m}(\be_i-\be_j)^2.
\end{align*}
\end{proof}
\begin{pr}%\llabel{crit-int-basis}
If $K=\Q[\al]$, $\al\in \cO_K$, and $D(1,\al,\ldots, \al^{m-1})=\disc(\cO/\Z)$ then $\{1,\al,\ldots, \al^{m-1}\}$ is an integral basis.
\end{pr}
\begin{proof}
Using change-of-basis and the correspondence between index and determinant,
\[
D(1,\al,\ldots, \al^{m-1})=\disc(\cO_K/\Z)\cdot [\cO_K:\Z[\al]]^2.
\]
Now $\disc(\cO_K/\Z)\in \Z$ so $[\cO_K:\Z[\al]]=1$.
\end{proof}
\begin{thm}[Stickelberger's Theorem]\llabel{stickelberger}$\,$
\begin{enumerate}
\item Let $s$ is the number of %of injections $K\to \C$ whose image is not contained in $\R$. 
complex (nonreal) embeddings $K\to \C$. 
Then
\[\sign[\disc(K/\Q)]=(-1)^{s/2}.\] 
\item $\disc(\cO_K/\Z)\equiv 0\ore 1\pmod 4$.
\end{enumerate}
\end{thm}
\begin{proof}
\begin{enumerate}
\item
%
%Let $\al_1,\ldots, \al_m$ be an integral basis. Let 
%
%Then
%\begin{align*}
%\disc(\cO_K/\Z)&=\det(\si_i\al_j)^2\\
%&=(P-N)^2\\
%&=(P+N)^2-4PN.
%\end{align*}
%Take $\si\in G(K^{\text{gal}}/\Q)$. Note composition by $\si$ permutes the $\si_i$, say by $\nu$. Then 
%\begin{align*}
%P&=\sum_{\text{even }\pi\in S_m}\prod_{i=1}^m\si_i\al_{\nu^{-1}\pi(i)}\\
%N&=\sum_{\text{odd }\pi\in S_m}\prod_{i=1}^m\si_i\al_{\nu^{-1}\pi(i)}\\
%\end{align*}
%and hence $\si$ permutes $\{P,N\}$. We conclude $\si$ fixes $P+N,PN$ and these are rational. Since they are integral over $\Z$ they are integers. Thus $\disc(\cO_K/\Z)\equiv (P+N)^2\equiv 0 \text{ or } 1\pmod4$.
%
Write $K=\Q[\al]$ by the Primitive Element Theorem and $\al_1,\ldots, \al_r$ be the real conjugates and $\be_1,\ol{\be_1},\ldots, \be_s,\ol{\be_s}$ be the complex conjugates. By Proposition~\ref{disc-calc},
\[\sign(D(1,\al,\ldots, \al^{m-1}))=\sign\pa{\prod_{1\le j\le s} (\be_j-\ol{\be_j})^2}=\prod_{1\le j\le s}i^2=(-1)^{s/2}.\]
\item Let $\al_1,\ldots, \al_m$ be an integral basis. Let $P$ and $-N$ be the sum of the terms in the expansion of $\det(\si_i\al_j)$ corresponding to even and odd permutations, respectively:
\begin{align*}
P&=\sum_{\text{even }\pi\in S_m}\prod_{i=1}^m\si_i\al_{\pi(i)}\\
N&=\sum_{\text{odd }\pi\in S_m}\prod_{i=1}^m\si_i\al_{\pi(i)}.
\end{align*}
Then
\begin{align*}
\disc(\cO_K/\Z)&=\det(\si_i\al_j)^2\\
&=(P-N)^2\\
&=(P+N)^2-4PN.
\end{align*}
Take $\si\in G(K^{\text{gal}}/\Q)$. Note composition by $\si$ permutes the $\si_i$, say by $\nu$. Then 
\begin{align*}
P&=\sum_{\text{even }\pi\in S_m}\prod_{i=1}^m\si_i\al_{\nu^{-1}\pi(i)}\\
N&=\sum_{\text{odd }\pi\in S_m}\prod_{i=1}^m\si_i\al_{\nu^{-1}\pi(i)}\\
\end{align*}
and hence $\si$ permutes $\{P,N\}$.
%Note any $\si\in G(K^{\text{gal}}/\Q)$ permutes $\{P,N\}$, so 
Hence $\si$ fixes $P+N,PN$ and they are rational. Since they are integral over $\Z$ they are integers. Thus the above is congruent to 0 or 1 modulo 4.
\end{enumerate}
\end{proof}

\begin{ex}[Quadratic extensions]\llabel{quadratic-extensions}
Any quadratic extension of $\Q$ is in the form $\Q(\sqrt{m})$ for some squarefree integer $m$. We find the ring of integers of $\Q(\sqrt{m})$. Consider two cases.
\begin{enumerate}
\item $m\equiv 2,3\pmod 4$: The minimal polynomial of $\sqrt m$ is $X^2-m$, so
\[
\disc(1,\sqrt m)=(\sqrt m-(-\sqrt m))^2=4m.
\]
Note $\frac{\disc (1,\sqrt m)}{\disc(\Q(\sqrt m)/\Q)}$ must be a square by Proposition~\ref{disc-basis} so $\disc(\Q(\sqrt m)/\Q)$ equals $m$ or $4m$. 
However, by Stickelberger's Theorem, $\disc(\Q(\sqrt m)/\Q)\equiv 0,1\pmod 4$. Hence $\disc(\Q(\sqrt m)/\Q)\ne m$ and $\disc(\Q(\sqrt m)/\Q)=4m$. By Proposition~\ref{crit-int-basis}, $1,\sqrt m$ is an integral basis.
\item $m\equiv 1\pmod 4$: Note $\frac{1+\sqrt m}2$ is integral with minimal polynomial $X^2-X-\frac{m-1}{4}$, so
\[
\disc\pa{1,\frac{1+\sqrt m}2}=\pa{\frac{1+\sqrt m}2-\frac{1-\sqrt m}2}^2=m.
\]
Since $m$ is squarefree, $\disc(\Q(\sqrt m)/\Q)=m$ and Proposition~\ref{crit-int-basis} says $1,\frac{1+\sqrt m}2$ is an integral basis. 
\end{enumerate}
\end{ex}

The following tells us about integral bases for products of fields. \fixme{Can we generalize from $\Q$ to extensions of $\Q$?}
\begin{pr}\llabel{disc-compositum}
Suppose that $K,L$ are field extensions of $\Q$ such that
\[
[KL:\Q]=[K:\Q][L:\Q].
\]
Let $d=\gcd(\disc(K/\Q),\disc(L/\Q))$. 
Then 
\begin{enumerate}
\item $\cO_K\subeq d^{-1}\cO_K\cO_L$.
\item If $\cO_{KL}=\cO_K\cO_L$, then $\disc(KL/\Q)=\disc(K/\Q)^{[L:\Q]}\disc(L/\Q)^{[K:\Q]}$.
\end{enumerate}
\end{pr}
\fixme{In particular, COROLLARY.}
\begin{proof}
Let $\{\al_1,\ldots, \al_m\}$ be an integral basis for $K$ and $\{\be_1,\ldots, \be_n\}$ be an integral basis for $L$. By the degree assumption, we know that $\{\al_i\be_j\}$ is a basis for $KL$ over $\Q$. Any element of $KL$ integral over $\Q$ can be written as
\begin{equation}\llabel{ga-integral}
\ga=\sumr{1\le i\le m}{1\le j\le m}{} \frac{a_{ij}}{r}\al_i\be_j
\end{equation}
where $\gcd(r,\gcd(a_{ij}))=1$. 

We need to show that $r\mid d$. Let $x_i=\sum_{j=1}^n \frac{a_{ij}}{r}\be_j$. We will turn~(\ref{ga-integral}) into a system of equations by considering all embeddings $K\hra \C$, solve for the $x_i$ using Cramer's rule, and in this way show that each $x_i$ is an algebraic integer in $L$ divided by a bounded denominator.

Note given embeddings $\si_K:K\hra \C$ and $\si_L:L\hra \C$, there is exactly one embedding $\si_{KL}:KL\hra \C$ such that restricts to $\si_K$ and $\si_L$. It is clearly unique if it exists. To show existence, write $K=\Q(\al)\cong\Q(x)/(f(x))$ by PET, and note that the characteristic polynomial of $f$ does not change upon passing to $L$ because of the degree assumption. Hence $KL=L(\al)=L(x)/(f(x))$, and in extending $\si_L$ to $\si_{KL}$, we are allowed to send $\al=x$ to $\si_L(\al)$.

Fix an embedding $\si:L\hra \C$, and let $\si_1,\ldots, \si_m$ be all embeddings $K\hra \C$. Then applying $\si_k$ to~\ref{ga-integral} we obtain the system of equations
\[
\sum_{i=1}^m \si_k(\al_i) x_i =\si_k(\ga),\quad 1\le k\le m.
\]
By Cramer's rule, letting $D=\det[(\si_k(\al_i))_{k,i}]$ we get $Dx_i=D_i$ where $D_i$ has the $i$th column of $D$ replaced by $(\si_k(\al_i))_{k=1}^m$. Note that $D$ and $D_i$ are both algebraic integers.
Using $\disc(\cO_K/\Z)=D^2$ (Proposition~\ref{disc-and-fe}), we get
\[
\disc(\cO_K/\Z)x_i=DD_i.
\]
Hence $\disc(\cO_K/\Z)x_i$ is an algebraic integer (in $\cO_L$). Since the $\be_j$ are an integral basis for $\cO_L$, this forces $r\mid \disc(\cO_K/\Z)$. Similarly, $r\mid \disc(\cO_L/\Z)$, as needed.

Now we prove the second part. Choose $(\al_1,\ldots, \al_m)$ a basis for $K/\Q$ and $(\be_1,\ldots, \be_n)$ a basis for $L/\Q$. Then $(\al_j\be_k)_{1\le j\le m,1\le k\le n}$ is a basis for $KL/\Q$. For $\ga\in KL$, let $(\ga)_{j,k}$ denote the coordinate of $\al_j\be_k$ in $\ga$. Then the $mn\times mn$ matrix
\begin{align*}
[\tr(\al_{i_1}\be_{i_2}\al_{i_1'}\be_{i_2'})]
&=\ba{\sum_{1\le j\le m,1\le k\le n} (\al_{i_1}\be_{i_2}\al_{i_1'}\be_{i_2'}\al_j\be_k)_{j,k}}\\
&=\ba{\sum_{1\le j\le m,1\le k\le n} (\al_{i_1}\al_{i_1'}\al_j)_j(\be_{i_2}\be_{i_2'}\be_k)_{k}}\\
&=\ba{\sum_{1\le j\le m} \sum_{1\le k\le n} (\al_{i_1}\al_{i_1'}\al_j)_j(\be_{i_2}\be_{i_2'}\be_k)_{k}}\\
&=
[\tr(\al_{i_1}\al_{i_1'})]\otimes[\tr(\be_{i_2}\be_{i_2'})].
\end{align*}
Taking determinants and using 
\[
\det(A\otimes B)=\det(A)^n\det(B)^m,\quad A\in M_{m\times m},B\in M_{n\times n}
\]
we get
\[
\disc(KL/\Q)=\disc(K/\Q)^{[L:\Q]}\disc(L/\Q)^{[M:\Q]}.
\]
\end{proof}
\fixme{[ADD an algorithm for computing integral bases]}

\section{Problems}
\begin{enumerate}
\item Suppose that $f\in\mathbb{Z}[x]$ is irreducible and has a root of absolute value at least $\frac 32$. Prove that if $\alpha$ is a root of $f$ then $f(\alpha^3+1)\neq 0$.

\item Let $a_1,\ldots, a_n$ be algebraic integers with degrees $d_1,\ldots, d_n$. Let $a_1',\ldots, a_n'$ be the conjugates of $a_1,\ldots, a_n$ with greatest absolute value. Let $c_1,\ldots, c_n$ be integers. Prove that if the LHS of the following expression is not zero, then
\[|c_1a_1+\ldots+c_na_n|\geq\left(\frac{1}{|c_1a_1'|+\cdots+|c_na_n'|}\right)^{d_1d_2\cdots d_n-1}.\]
For example, 
\[|c_1+c_2\sqrt{2}+c_3\sqrt{3}|\geq \left(\frac{1}{|c_1|+|2c_2|+|2c_3|}\right)^{3}.\]

\item Let $p$ be a prime and consider $k$ $p$th roots of unity whose sum is not 0. Prove that the absolute value of their sum is at least $\frac{1}{k^{p-2}}$.
\end{enumerate}

