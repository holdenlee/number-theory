%%%This is a science homework template. Modify the preamble to suit your needs. 

\documentclass[12pt]{article}

\makeatother
%AMS-TeX packages
\usepackage{amsmath}
\usepackage{amssymb}
\usepackage{amsthm}
\usepackage{array}
\usepackage{amsfonts}
\usepackage{cancel}
\usepackage[all,cmtip]{xy}%Commutative Diagrams
\usepackage[pdftex]{graphicx}
\usepackage{float}
%geometry (sets margin) and other useful packages
\usepackage[margin=1in]{geometry}
\usepackage{sidecap}
\usepackage{wrapfig}
\usepackage{verbatim}
\usepackage{mathrsfs}
\usepackage{marvosym}
\usepackage{hyperref}
\usepackage{graphicx,ctable,booktabs}
\usepackage{stmaryrd}

\newtheoremstyle{norm}
{6pt}
{6pt}
{}
{}
{\bf}
{:}
{.5em}
{}

\theoremstyle{norm}
\newtheorem{thm}{Theorem}[section]
\newtheorem{lem}[thm]{Lemma}
\newtheorem{df}{Definition}
\newtheorem{rem}{Remark}
\newtheorem{st}{Step}
\newtheorem{pr}[thm]{Proposition}
\newtheorem{cor}[thm]{Corollary}
\newtheorem{clm}[thm]{Claim}

%Math blackboard, fraktur, and script commonly used letters
\newcommand{\A}[0]{\mathbb{A}}
\newcommand{\C}[0]{\mathbb{C}}
\newcommand{\sC}[0]{\mathcal{C}}
\newcommand{\cE}[0]{\mathscr{E}}
\newcommand{\F}[0]{\mathbb{F}}
\newcommand{\cF}[0]{\mathscr{F}}
\newcommand{\cG}[0]{\mathscr{G}}
\newcommand{\sH}[0]{\mathscr H}
\newcommand{\Hq}[0]{\mathbb{H}}
\newcommand{\cI}[0]{\mathscr{I}}%ideal sheaf
\newcommand{\N}[0]{\mathbb{N}}
\newcommand{\Pj}[0]{\mathbb{P}}
\newcommand{\sO}[0]{\mathcal{O}}
\newcommand{\cO}[0]{\mathscr{O}}
\newcommand{\Q}[0]{\mathbb{Q}}
\newcommand{\R}[0]{\mathbb{R}}
\newcommand{\Z}[0]{\mathbb{Z}}
%Lowercase
\newcommand{\ma}[0]{\mathfrak{a}}
\newcommand{\mb}[0]{\mathfrak{b}}
\newcommand{\fg}[0]{\mathfrak{g}}
\newcommand{\vi}[0]{\mathbf{i}}
\newcommand{\vj}[0]{\mathbf{j}}
\newcommand{\vk}[0]{\mathbf{k}}
\newcommand{\mm}[0]{\mathfrak{m}}
\newcommand{\mfp}[0]{\mathfrak{p}}
\newcommand{\mq}[0]{\mathfrak{q}}
\newcommand{\mr}[0]{\mathfrak{r}}
%Letter-related
\newcommand{\bb}[1]{\mathbb{#1}}
\providecommand{\cal}[1]{\mathcal{#1}}
\renewcommand{\cal}[1]{\mathcal{#1}}
%More sequences of letters
\newcommand{\chom}[0]{\mathscr{H}om}
\newcommand{\fq}[0]{\mathbb{F}_q}
\newcommand{\fqt}[0]{\mathbb{F}_q^{\times}}
\newcommand{\sll}[0]{\mathfrak{sl}}
%Shortcuts for symbols
\newcommand{\nin}[0]{\not\in}
\newcommand{\opl}[0]{\oplus}
\newcommand{\ot}[0]{\otimes}
\newcommand{\rc}[1]{\frac{1}{#1}}
\newcommand{\rra}[0]{\rightrightarrows}
\newcommand{\send}[0]{\mapsto}
\newcommand{\sub}[0]{\subset}
\newcommand{\subeq}[0]{\subseteq}
\newcommand{\supeq}[0]{\supseteq}
\newcommand{\nsubeq}[0]{\not\subseteq}
\newcommand{\nsupeq}[0]{\not\supseteq}
%Shortcuts for greek letters
\newcommand{\al}[0]{\alpha}
\newcommand{\be}[0]{\beta}
\newcommand{\ga}[0]{\gamma}
\newcommand{\Ga}[0]{\Gamma}
\newcommand{\de}[0]{\delta}
\newcommand{\De}[0]{\Delta}
\newcommand{\ep}[0]{\varepsilon}
\newcommand{\eph}[0]{\frac{\varepsilon}{2}}
\newcommand{\ept}[0]{\frac{\varepsilon}{3}}
\newcommand{\la}[0]{\lambda}
\newcommand{\La}[0]{\Lambda}
\newcommand{\ph}[0]{\varphi}
\newcommand{\rh}[0]{\rho}
\newcommand{\te}[0]{\theta}
\newcommand{\om}[0]{\omega}
\newcommand{\Om}[0]{\Omega}
\newcommand{\si}[0]{\sigma}
%Brackets
\newcommand{\ab}[1]{\left| {#1} \right|}
\newcommand{\ba}[1]{\left[ {#1} \right]}
\newcommand{\bc}[1]{\left\{ {#1} \right\}}
\newcommand{\pa}[1]{\left( {#1} \right)}
\newcommand{\an}[1]{\langle {#1}\rangle}
\newcommand{\fl}[1]{\left\lfloor {#1}\right\rfloor}
\newcommand{\ce}[1]{\left\lceil {#1}\right\rceil}
%Text
\newcommand{\btih}[1]{\text{ by the induction hypothesis{#1}}}
\newcommand{\bwoc}[0]{by way of contradiction}
\newcommand{\by}[1]{\text{by~(\ref{#1})}}
\newcommand{\ore}[0]{\text{ or }}
%Arrows
\newcommand{\hr}[0]{\hookrightarrow}
\newcommand{\xr}[1]{\xrightarrow{#1}}
%Formatting
\newcommand{\subprob}[1]{\noindent\textbf{#1}\\}
%Functions, etc.
\newcommand{\Ann}{\operatorname{Ann}}
\newcommand{\AP}{\operatorname{AP}}
\newcommand{\Ass}{\operatorname{Ass}}
\newcommand{\Aut}{\operatorname{Aut}}
\newcommand{\chr}{\operatorname{char}}
\newcommand{\cis}{\operatorname{cis}}
\newcommand{\Cl}{\operatorname{Cl}}
\newcommand{\Der}{\operatorname{Der}}
\newcommand{\End}{\operatorname{End}}
\newcommand{\Ext}{\operatorname{Ext}}
\newcommand{\Frac}{\operatorname{Frac}}
\newcommand{\FS}{\operatorname{FS}}
\newcommand{\GL}{\operatorname{GL}}
\newcommand{\Hom}{\operatorname{Hom}}
\newcommand{\Ind}[0]{\text{Ind}}
\newcommand{\im}[0]{\text{im}}
\newcommand{\nil}[0]{\operatorname{nil}}
\newcommand{\ord}[0]{\operatorname{ord}}
\newcommand{\Proj}{\operatorname{Proj}}
\newcommand{\PSL}{\operatorname{PSL}}
\newcommand{\Rad}{\operatorname{Rad}}
\newcommand{\rank}{\operatorname{rank}}
\newcommand{\Res}[0]{\text{Res}}
\newcommand{\sign}{\operatorname{sign}}
\newcommand{\SL}{\operatorname{SL}}
\newcommand{\Spec}{\operatorname{Spec}}
\newcommand{\Specf}[2]{\Spec\pa{\frac{k[{#1}]}{#2}}}
\newcommand{\spp}{\operatorname{sp}}
\newcommand{\spn}{\operatorname{span}}
\newcommand{\Supp}{\operatorname{Supp}}
\newcommand{\Tor}{\operatorname{Tor}}
\newcommand{\tr}[0]{\text{trace}}
%Commutative diagram shortcuts
\newcommand{\fiber}[3]{\xymatrix{#1\times_{#3} #2}\ar[r]\ar[d] #1\ar[d] \\ #2 \ar[r] & #3}
\newcommand{\commsq}[8]{\xymatrix{#1\ar[r]^{#6}\ar[d]^{#5} &#2\ar[d]^{#7} \\ #3 \ar[r]^{#8} & #4}}
%Makes a diagram like this
%1->2
%|    |
%3->4
%Arguments 5, 6, 7, 8 on arrows
%  6
%5  7
%  8
\newcommand{\pull}[9]{
#1\ar@/_/[ddr]_{#2} \ar@{.>}[rd]^{#3} \ar@/^/[rrd]^{#4} & &\\
& #5\ar[r]^{#6}\ar[d]^{#8} &#7\ar[d]^{#9} \\}
\newcommand{\back}[3]{& #1 \ar[r]^{#2} & #3}
%Syntax:\pull 123456789 \back ABC
%1=upper left-hand corner
%2,3,4=arrows from upper LH corner, going down, diagonal, right
%5,6,7=top row (6 on arrow)
%8,9=middle rows (on arrows)
%A,B,C=bottom row
%Other
%Other
\newcommand{\op}{^{\text{op}}}
\newcommand{\fp}[1]{^{\underline{#1}}}
\newcommand{\rp}[1]{^{\overline{#1}}}
\newcommand{\rd}[0]{_{\text{red}}}
\newcommand{\pre}[0]{^{\text{pre}}}
\newcommand{\pf}[2]{\pa{\frac{#1}{#2}}}
\newcommand{\pd}[2]{\frac{\partial #1}{\partial #2}}
\newcommand{\bs}[0]{\backslash}
\newcommand{\ol}[1]{\overline{#1}}
\newcommand{\mmod}[1]{\,(\text{mod}^{\times} #1)}
\newcommand{\nmod}[1]{\,(\text{mod}\, #1)}
%Matrices
\newcommand{\coltwo}[2]{
\left[
\begin{matrix}
{#1}\\
{#2} 
\end{matrix}
\right]}
\newcommand{\matt}[4]{
\left[
\begin{matrix}
{#1}&{#2}\\
{#3}&{#4}
\end{matrix}
\right]}
\newcommand{\smatt}[4]{
\left[
\begin{smallmatrix}
{#1}&{#2}\\
{#3}&{#4}
\end{smallmatrix}
\right]}
\newcommand{\colthree}[3]{
\left[
\begin{matrix}
{#1}\\
{#2}\\
{#3}
\end{matrix}
\right]}
\newcommand{\iy}[0]{\infty}
%
%Redefining sections as problems
%


%
%Fancy-header package to modify header/page numbering 
%
\usepackage{fancyhdr}
\pagestyle{fancy}
%\addtolength{\headwidth}{\marginparsep} %these change header-rule width
%\addtolength{\headwidth}{\marginparwidth}
\lhead{section}%Problem \thesection}
\chead{} 
\rhead{\thepage} 
\lfoot{\small\scshape subject} 
\cfoot{} 
\rfoot{\footnotesize topic} % !! Remember to change the problem set number
\renewcommand{\headrulewidth}{.3pt} 
\renewcommand{\footrulewidth}{.3pt}
\setlength\voffset{-0.25in}
\setlength\textheight{648pt}
\allowdisplaybreaks[1]

%%%%%%%%%%%%%%%%%%%%%%%%%%%%%%%%%%%%%%%%%%%%%%%
%
%Contents of problem set
%    
\begin{document}
\title{Kummer Theory}% !! Remember to change the problem set number
\author{Holden Lee}
\date{5/21/11}% !! Remember to change the date
\maketitle
\thispagestyle{empty}

In this post we will give another proof of the fact that $n$th roots of rationals not differing by perfect $n$th powers are linearly independent, that is, if we have positive rationals $p_0=1,p_2,\ldots, p_k$ such that $\frac{p_i}{p_j}$ is not a perfect $n$th power for any $i,j$, then there do not exist rationals $a_0,a_1,\ldots, a_k$ such that
\[
a_0+a_1\sqrt[n]{p_1}+\cdots +a_k\sqrt[n]{p_k}=0.
\]
We (sort-of) proved this using (semi-)elementary methods back in this post (PUT URL HERE). 
(There I only did it for $n$ prime, but I believe the argument extends with some more work...) 
Today we give a proof using algebraic number theory.

%
%First we explain how the result follows from Kummer Theory. 
%Given the theorem, and numbers $q_i, 1\le i\le k$ such that $\al_i$ is the least 
%$q_i^{\al_i}=1$
%$q_i\neq q_j^ra^n$ for any $

We will prove this as follows.
\begin{enumerate}
\item
Normal basis theorem: Every finite Galois extension $L/K$ has a normal basis, i.e. there exists $\alpha\in L$ such that $\sigma \alpha$ for $\sigma\in G(L/K)$ form a basis for $L$ over $K$.
\item
Use Galois cohomololgy to get an exact sequence
\[
\{1\}\to K^{\times}\cap L^{\times n}\to K^{\times}\xrightarrow{x\mapsto x^n} K^{\times n} \to \text{Hom}(G, \mu_n)\to \{1\}
\]
and thus get an isomorphism
\[
L\cap K^{\times}/K^{\times n}\cong \text{Hom}(G,\mu_n).
\]
\item Using the above isomorphism, establish the following theorem:

Kummer Theory: Suppose $K$ is a field containing a primitive $n$th root of 1. Then there is a bijection between

1. Finite abelian extensions of $K$ of exponent $n$ (i.e. for any $\si$ in the Galois group $G(L/K)$, $\si^n=1$).

2. Finite subgroups of $K^{\times}/K^{\times n}$.

This correspondence is given by

$
\displaystyle
L\mapsto K^{\times }\cap L^{\times n}/K^{\times}
$

$
\displaystyle
K[B^{\frac{1}{n}}]\mapsfrom B.
$
\item From this conclude that the degree of the extension $\mathbb Q[p_1,\ldots, p_k]$ is what we would ``expect", and conclude linear independence. We need the fact that $x^n-a$ is irreducible iff $a$ is not a perfect $p$th power for any $p\mid n$, in this step.
\end{enumerate}

\section{Step 1: Normal basis theorem}
\begin{thm}[Normal basis theorem] Every finite Galois extension $L/K$ has a normal basis.
\end{thm}


\section{Step 2: Galois cohomology}
First we introduce group cohomology. Given a group $G$ there exists a presentation by free groups
\[
\cdots \xrightarrow{d^1}G_1\xrightarrow{d^0} G_0\to G\to 0.
\]
Now apply the functor $\text{Hom}(\bullet, H)$ to get 
\[
1\to \text{Hom}(G,H)\to \text{Hom}(G_0,H)\xrightarrow{d_0}\text{Hom}(G_1,H)\xrightarrow{d_1}\cdots.
\]
Define the $n$th cohomology group of $H$ with respect to $G$ to be
\[
H^n(G,H)=\text{ker}(d^n)/\text{im}(d^{n-1}),
\]
that is, it measures the failure of the sequence to be exact.

Now given a short exact sequence of groups
\[
1\to N\to H\to Q\to 1
\]
by lots of diagram chasing arguments we get a long exact sequence
\[
1\to H^0(G,N)\to H^0(G,H)\to H^0(G,Q)\xrightarrow{\delta} H^1(G,N) \to H^1(G,H)\to \cdots 
\]
where $\delta$ is the connecting homomorphism given by the following diagram
[ADD]
Note in proving this it was essential that $\text{Hom}$ is a (left/right?) exact functor, so that the $1\to H^0(N)\to H^0(G)\to H^0(Q)$ part is exact.

Apply this to the exact sequence
\[
1\to \mu_n \to L^{\times}\xrightarrow{x\mapsto x^n} L^{\times n}\to 1 
\]
with $G=G(L/K)$ to get
\[
1\to H^0(G,\mu_n) \to H^0(G,L^{\times})\to H^0(G,L^{\times n})\to H^1(G, \mu_n)\to H^1(G,L^{\times})\to \cdots.
\]
We need not go further because Hilbert's Theorem 90 tells us
\[
H^1(G(K/L), L^{\times})=1.
\]
Next noting that $H^0(G, H)$ is simply the subgroup of $H$ fixed by $G$, and that the subfield of $L$ fixed by $G$ is $K$ (fixed field theorem), the sequence becomes
\[
1\to \mu_n\to K^{\times}\to K^{\times n}\cap L^{\times} \to \Hom(G, \mu_n)\to 1,
\]
giving an isomorphism
\[
K^{\times n}\cap L^{\times}/K^{\times}\cong \Hom(G, \mu_n).
\]
\section{Step 3: Kummer Theory}

\section{Step 4: Finish}
\end{document}