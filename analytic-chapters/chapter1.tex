\chapter{Elementary estimates for primes}
\section{Chebyshev's Theorem}
Today we prove some asymptotic results about the distribution of prime numbers. Specifically, we derive estimates for the {\it prime-counting functions}
\begin{align*}
\vartheta(x)&=\sum_{p\leq x}\ln(p)\\
\psi(x)&=\sum_{p^k\leq x}\ln(p)\\%=\sum_{p^k, p^k||x}\ln(p^k)\\
\pi(x)&=\sum_{p\leq x} 1
\end{align*}
Note that we will always use $p$ to denote a prime.

Lacking the tools of complex analysis, it is difficult to find the exact asymptotic formulas; however, our elementary methods suffice to determine the asymptotics up to a constant multiple. Our main result is Chebyshev's Theorem:
\begin{thm}\cite[Theorem 6.3]{nath}
\label{cheb}
There exist positive constants $c_1$ and $c_2$ such that
\begin{equation}\label{compare}
c_1x\leq \vartheta(x)\leq \psi(x)\leq \pi(x)\ln(x)\leq c_2 x.
\end{equation}
for all $x\geq 2$. Moreover,
\begin{align}
\liminf_{x\to\infty}\frac{\vartheta(x)}{x}= \liminf_{x\to \infty} \frac{\psi(x)}{x}= \liminf_{x\to \infty}\frac{\pi(x)\ln(x)}{x}&\geq \ln(2)\label{inf}\\
\limsup_{x\to\infty}\frac{\vartheta(x)}{x}= \limsup_{x\to \infty} \frac{\psi(x)}{x}=  \limsup_{x\to \infty}\frac{\pi(x)\ln(x)}{x}&\leq 2\ln(2) \label{sup}
\end{align}
\end{thm}
We will prove this in three steps.
%In particular, this gives a weak version of the Prime Number Theorem
\subsection{Comparing the three functions}
Since all terms in the sum defining $\vartheta(x)$ are included in the sum defining $\psi(x)$, $\vartheta(x)\leq \psi(x)$. %Since $\ln(p)\leq \ln(x)$ for each prime $p\leq x$, $\psi(x)\leq \pi(x)\ln(x)$. 
%Turning $\psi(x)$ into a sum over $p$, we get
For a given $p$ there are $\fl{\frac{\ln(x)}{\ln(p)}}$ choices for $k$ so that $p^k\leq x$, so
\[\psi(x)=\sum_{p^k\leq x} \ln(p)=\sum_{p\leq x} \fl{\frac{\ln(x)}{\ln(p)}}\ln(p)\leq \sum_{p\leq x} \ln(x)=\pi(x)\ln(x).\]
This shows the middle two inequalities in~(\ref{compare}).

Given $\vartheta(x)\leq \psi(x)\leq \pi(x)\ln(x)$, to show that the three quantities in~(\ref{inf}) and~(\ref{sup}) are equal it suffices to show that
\begin{equation}\label{lim}
\liminf_{x\to\infty}\frac{\vartheta(x)}{x}\geq \liminf_{x\to \infty}\frac{\pi(x)\ln(x)}{x},\qquad
\limsup_{x\to\infty}\frac{\vartheta(x)}{x}\geq \limsup_{x\to \infty}\frac{\pi(x)\ln(x)}{x}
%\liminf_{x\to \infty}\left.\frac{\vartheta(x)}{x}\right/ \frac{\pi(x)\ln(x)}{x}= 1.
\end{equation}
To compare $\vartheta(x)=\sum_{p\leq x}\ln(p)$ and $\pi(x)\ln(x)=\sum_{p\leq x}\ln(x)$, note that for $p$ ``close" to $x$, we have $\ln(p)$ ``close" to $\ln(x)$ and relatively large, while the terms for small $p$ will not contribute much to either sum. Thus we %divide the sum in $\vartheta$ into two parts, over $p\leq x^{1-\de}$ and $p>x^{1-\de}$ for $\delta\in (0,1)$ and estimate each separately. %Let $\delta\in (0,1)$; we have
can just consider the terms with $p>x^{1-\delta}$, where $\delta\in (0,1)$.
\begin{align*}
\vartheta(x)&\geq%=\sum_{p\leq x^{1-\de}} \ln(p)+ 
\sum_{x^{1-\de}<p\leq x} \ln(p)\\
&\geq %\sum_{p\leq x^{1-\de}} \ln(x) +
\sum_{x^{1-\de}<p\leq x} \ln(x^{1-\delta})\\
&=\ln(x^{1-\de})(\pi(x)-\pi(x^{1-\de}))\\
&=(1-\de)\ln(x)(\pi(x)-\pi(x^{1-\de}))\\
&\geq(1-\de)\ln(x)(\pi(x)-x^{1-\de})
\end{align*}
Hence
\[\frac{\vartheta(x)}{x}\geq\frac{(1-\delta)\pi(x)\ln(x)}{x}-\frac{(1-\de)\ln(x)}{x^{\de}}.\]
Letting $\de\to 0$ gives~(\ref{lim}).
\subsection{Upper Bound}
We show that $\vartheta(x)\leq 2x\ln(x)$. Instead of thinking about bounding $\vartheta(x)$, it is easier to think about bounding $e^{\vartheta(x)}=\prod_{p\leq x} p$.
\begin{lem}~\cite[3.?]{pftb}
For any $x\in \N$,
\begin{equation}
\label{erdos}
\prod_{p\leq x}p\leq 4^{x-1}\end{equation}
\end{lem}
\begin{proof}
Use strong induction on $x$. For $x=1,2$ the statement holds. The induction step from odd $x>1$ to $x+1$ is obvious, since $x+1$ is not a prime.

Consider the induction step from even $x$ to $x+1$. Let $x=2n$. 
The key idea is that there cannot be ``too many" primes between $n+2$ and $2n+1$, because...
\begin{enumerate}
\item These primes all divide $\binom{2n+1}{n}=\frac{(2n+1)!}{n!(n+1)!}$.
\item %Since $\binom{2k+1}{k}$ is the largest binomial coefficient with top equal to $2k+1$, and $\sum_{i=0}^{2k+1} \binom{2k+1}{i}=2^{2k+1}$,
%\[\binom{2k+1}{k}\geq \frac{2^{2k+1}}{2k+2}\geq 4^k.\]
$\binom{2n+1}{n}$ can easily be bounded from above:
\[\binom{2n+1}{n}=\frac{1}{2}\pa{\binom{2n+1}{n}+\binom{2n+1}{n+1}}\leq \rc{2}\sum_{i=0}^n \binom{2n+1}{i}=4^n.\]
\end{enumerate}
Then
\[\prod_{p\leq x+1} p=\prod_{p\leq n+1}p\prod_{n+2\leq p\leq 2n+1} p\leq 4^n\cdot \binom{2n+1}{n}\leq 4^{2n}.\]
%We used the following facts:
%\begin{enumerate}
%\item $\binom{2k+1}{k}=\frac{(2k+1)!}{k!(k+1)!}$ contains every prime between $k+2$ and $2k+1$ as a factor.
%\item Since $\binom{2k+1}{k}$ is the largest binomial coefficient with top equal to $2k+1$, and $\sum_{i=0}^{2k+1} \binom{2k+1}{i}=2^{2k+1}$,
%\[\binom{2k+1}{k}\geq \frac{2^{2k+1}}{2k+2}\geq 4^k.\]
%\end{enumerate}
%Note that binomial coefficients $\binom{2k+1}{k}$ or $\binom{2k}{k}
\end{proof}
Taking the logarithm of both sides of~(\ref{erdos}) gives $\vartheta(x)\leq (x-1)\ln(4)\leq 2x\ln(x)$.
\subsection{Lower Bound}
We show that $\liminf_{x\to \infty}\frac{\pi(x)\ln(x)}{x}\geq \ln(2)$.
First consider when $x$ is even, say equal to $2n$. Like in Section 3, we consider a binomial coefficient, this time $\binom{2n}{n}$. We show that each prime cannot appear as a factor in $\binom{2n}{n}$ ``too many" times, so it can be bounded above by $(2n)^{\pi(2n)}$. We can easily bound $\binom{2n}{n}$ below:
\[\binom{2n}{n}\geq \frac{2^{2n}}{2n}\]
since it is the largest among $2, \binom{2n}{1},\ldots, \binom{2n}{2n-1}$. Putting these two bounds together will give the desired bound for $\pi(2n)$.

We need the following to count the highest prime powers dividing $\binom{2n}{n}$:
\begin{lem}\label{fact}\cite[Lemma 6.3]{pftb}
For every positive integer $n$, 
\[v_p(n!)=\sum_{k=1}^{\fl{\frac{\ln(n)}{\ln(p)}}} \fl{\frac{n}{p^k}},\]
where $v_p(m)$ denotes the largest integer $i$ such that $p^i|m$.
\end{lem}
\begin{proof}
%To find the highest power of $p$ dividing $n!$, 
%we note that 
There are $\fl{\frac{n}{p^k}}$ multiples of $p^k$ less than or equal to $n$. In the sum $\sum_{k\geq 1} \fl{\frac{n}{p^k}}$, each multiple of $p^k$ less than $n$ is counted $k$ times, once each as a multiple of $p,p^2,\ldots, p^k$.
%We count the number of times $p$ 
\end{proof}
From Lemma~\ref{fact}, we get
\[
v_p\pa{\binom{2n}{n}} =v_p\pa{\frac{(2n)!}{n!^2}}=v_p((2n)!)-2v_p(n!)
=\sum_{k=1}^{\fl{\frac{\ln(2n)}{\ln(p)}}} \fl{\frac{2n}{p^k}}- 2\fl{\frac{n}{p^k}}
\]
Since each term of the sum is at most 1,
\[
v_p\pa{\binom{2n}{n}}\leq \fl{\frac{\ln(2n)}{\ln(p)}}\leq \frac{\ln(2n)}{\ln(p)}.
\]
Thus
\[\frac{2^{2n}}{2n}\leq \binom{2n}{n}= \prod_{p\leq 2n} p^{v_p\pa{\binom{2n}{n}}}\leq (2n)^{\pi(2n)}.\]
Taking logs and remembering $x=2n$ gives %the desired bound.
$x\ln(2)-\ln(x)\leq \pi(x)\ln(x)$, which gives the desired bound. For odd $x$, the value of $\frac{\pi(x)\ln(x)}{x}$ can be compared to the value for $x-1$.

Finally,~(\ref{inf}) and~(\ref{sup}), and the fact that all the prime-counting functions are positive for $x\geq 2$, show the existence of $c_1$ and $c_2$ in~(\ref{compare}).  
This finishes the proof of Theorem~\ref{cheb}.
%\begin{enumerate}
%\item 
%
%\end{enumerate}•
\subsection{The $n$th prime}
We found an estimate for the number of primes less than or equal to a given number; we can use this bound to find an estimate for the $n$th prime number.
\begin{thm}\
Let $p_n$ denote the $n$th prime number. Then there exist constants $c_3,c_4$ such that 
\[c_3n\ln(n)\leq p_n\leq c_4n\ln(n)\]
for all $n\geq 2$.
\end{thm}
\begin{proof}
From Theorem~\ref{cheb},
\begin{equation}
\label{blah}
\frac{c_1p_n}{\ln(p_n)}\leq \pi(p_n)=n\leq \frac{c_2p_n}{\ln(p_n)}, 
\end{equation}
so
\[\frac{n\ln(p_n)}{c_2}\leq p_n\leq \frac{n\ln(p_n)}{c_1}.\]
The LHS is at least $c_3n\ln(n)$ by the trivial bound $n\leq p_n$. On the RHS, use %$p_n\leq \frac{n\ln(p_n)}{c_1}$ again
the LHS of~(\ref{blah}) again to get $\ln(p_n)\leq \ln\pa{\frac{n\ln(p_n)}{c_1}}$, giving $\ln(p_n)\leq c\ln(n)$ for some $c$. 
\end{proof}