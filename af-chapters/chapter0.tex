\noindent{\Large \textbf{Theta and Elliptic Functions}}
\vskip0.15in

The study of theta and elliptic functions starts as a story in complex analysis---engineering functions that satisfy a certain relation---and connects to complex manifolds, elliptic curves, integration, and quadratic forms.

We'll start out with the question: what is the complex analog of a periodic function? First, we'll have to review a bit of complex analysis (Section~\ref{sec:complex-analysis}). First, as practice for what we will do with theta and elliptic functions, we will ``derive" expressions for the sine and cosine, using our complex analytic machinery (Section~\ref{sec:sine}).

Unfortunately, we find that the only entire functions that are {\it doubly} periodic are constant, so we have to relax one of these conditions.
\begin{itemize}
\item
Relaxing periodicity gives the theory of \textbf{theta functions} (Section~\ref{sec:theta}). We use complex analysis to construct a ``basic" theta function $\te$ in 3 different ways, and show that they are all equal up to a constant. Then we characterize all theta functions, by showing that {\it every} theta function can be written using a product of translates of $\te$, in the spirit of the Fundamental Theorem of Algebra.
\item
Relaxing entirety gives the theory of \textbf{elliptic functions} (Section~\ref{sec:ef}). We construct a ``basic" elliptic function, the Weierstrass $\wp$-function, in 2 ways and show all elliptic functions can be built out of $\wp$ (and $\wp'$). We then give a list of basic properties of elliptic functions.

Knowing the theory of theta functions makes our job easier, as elliptic functions can be written in terms of theta functions.
\end{itemize}

Now comes the applications. In much the same way that $(\sin x,\ddd x\sin x=\cos x)$ parameterize the circle, we find that $(\wp,\wp')$ also parameterize an algebraic variety---an {\it elliptic curve} (Section~\ref{sec:ef-ec}). This is the key application of elliptic functions to number theory. 
Following history, we'll use elliptic functions to calculate the arc length of an ellipse in Section~\ref{sec:arc-length-ellipse} (this is where the name came from).

(See Ahlfors pg. 238 for this?)

(I might put in the application to the 4-square problem... but this is more easily done with modular forms.)

It is important to keep in mind that the functions here depend on {\it two} parameters. For instance, $\te$ is a function in $z$ and $\tau$:
\[
\te({\color{red}z},{\color{blue}\tau})
\]
As we will see, $\tau$ specifies a lattice $\La$.
\begin{itemize}
\item
In this article, %chapter
we are mostly concerned with $\te$ as a function of $\color{red}z$. (But it will be helpful to change our view sometimes!) $\te$ is a {\color{red}theta function} in the parameter $z$ (with certain transformation properties).
\item
However, we can also think of $\te$ as a function in $\color{blue}\tau$. Then $\te$ is (sort-of) a {\color{blue}modular function} in the parameter $\tau$ (with certain transformation properties, coming from the fact that $\te$ is really a function of the lattice $\an{1,\tau}$). 
\end{itemize}
Thus, theta functions and modular functions are intricately related! But the story of modular functions is a story for another day and article.

%there are a lot of identities: like trig.

\section{Some complex analysis}
\llabel{sec:complex-analysis}
In this section we review some facts from complex analysis. (For a longer ``crash course," see \S30 in the number theory text.) 

A complex analytic function is a function that is complex differentiable. This is a much stronger condition than real differentiability, and gives complex analytic functions very nice properties---for instance, they are equal to their power series expansions (barring any poles). An entire function is a complex analytic function defined on all of $\C$. A meromorphic function is allowed to have poles.

\begin{thm}[Liouville]\llabel{thm:liouville}
A bounded entire function is constant.
\end{thm}

\subsection{Series and product developments}
We know that locally, we can write a meromorphic function $f$ as a Laurent series $\sum_{n=-\iy}^{\iy} a_nx^n$. There are two other representations that are useful, depending on what information we have about the function $f$.
\begin{enumerate}
\item If we know the {\it poles} of $f$, we can write $f$ as a sum of rational functions
\[
f(z)=\sum_{n=1}^{\iy}\ba{ P_n\prc{z-z_n} - Q_n(z)}+g(z).
\]
\item If $f$ is entire and we know the {\it zeros} of $f$, we can write $f$ as an infinite product
\[
f(z)=z^me^{g(z)}\prod_{n=1}^{\iy}\pa{1-\frac{z}{z_n}}e^{P_n\pf{z}{z_n}}.
\]
(Think of this as ``factoring" $f$, much like a polynomial can be factored as in the fundamental theorem of algebra.)
These representations come about from convergence properties of holomorphic functions (which tell us that infinite sums and products converge to holomorphic functions) and by Liouville's theorem---if we engineer a function that is close enough to $f$ then it must be equal to $f$.
\end{enumerate}
\index{Mittag-Leffler}
\begin{thm}[Mittag-Leffler]
Let $z_n$ be a sequence with $\lim_{n\to \iy} |z_n|=\iy$ (or a finite sequence), and $P_n$ polynomials without constant term.
\begin{enumerate}
\item (Existence)
There is a meromorphic function $f$ with poles exactly at $z_n$, with Laurent expansion $P_n\prc{z-z_n}+\cdots $ at $z_n$.
\item (Uniqueness)
All such functions $f$ are in the form
\[
\sum_{n=1}^{\iy} \pa{P_n\prc{z-z_n}-Q_n(z)}+g(z)
\]
where $Q_n$ is a polynomial and $g(z)$ is analytic.
\end{enumerate}
\end{thm}
\begin{proof}
See Ahlfors~\cite[p. 187]{Ah79}.
\end{proof}
Warning: this does not converge for all $P_n$. Typically we take $Q_n$ to the the first terms of the Laurent expansion of $P_n\prc{z-z_n}$, to ensure cancellation of high-order terms.
\begin{df}
The \textbf{order} of an entire function $f$ is the smallest $\al\in [0,\iy]$ such that
\[
|f(z)|\precsim_{\ep} e^{|z|^{\al+\ep}}
\]
for all $\ep>0$. 
\end{df}
\begin{thm}[Hadamard]\llabel{product-development}
Let $z_n$ be a sequence with $\lim_{n\to \iy} |z_n|=\iy$.
%\item
%(Existence) There is an entire function with zeros exactly the $z_n$.
%\item
%(Uniqueness) All such $f$ are in the form
%\[
%f(z)=z^me^{g(z)} \prod_{n=1}^{\iy} \pa{1-\frac{z}{a_n}}e^{
%\frac{z}{a_n}+\rc2 \pf{z}{a_n}^2+\cdots +\rc{m_n}\pf{z}{a_m}^{m_n}
%}.
%\]
%\end{enumerate}
If $f$ is entire with order $\al<\iy$ with zeros $z_1,z_2,\ldots$ (with multiplicity, not including 0), then it has a product formula
\begin{equation}\llabel{product-formula}
f(z)=z^re^{g(z)} \prod_{n=1}^{\iy} \pa{1-\frac{z}{z_n}}e^{
\frac{z}{z_n}+\rc2 \pf{z}{z_n}^2+\cdots +\rc{m}\pf{z}{z_m}^{m}
},
\end{equation}
where
\begin{itemize}
\item $m=\fl{\al}$,
\item $r$ is the order of vanishing of $f$ at 0, and
\item $g$ is a polynomial of degree at most $a$.
\end{itemize}
The product converges uniformly locally. Moreover, 
\begin{equation}\llabel{num-zeros}
|\set{k}{z_k<R}|\precsim_{\ep} R^{\al+\ep}.
\end{equation}
Conversely, if $a=\fl{\al}$ and $z_k$ is a sequence satisfying~(\ref{num-zeros}), then the RHS of~(\ref{product-formula}) defines an entire function of order at most $\al$.
\end{thm}
\begin{proof}
See Ahlfors~\cite[p. 195]{Ah79}.
\end{proof}
Hence the order of a entire function gives an asymptotic bound for the number of zeros.\footnote{A function which grows faster is allowed to have more zeros---much like a polynomial with lots of zeros grows fast simply because it has higher degree.}

\fixme{Note that the two theorems are basically related by taking a logarithmic derivative.}

\section{Periodic functions}

We now use the product expansion to come up with expressions for $\sin$ and $\cos$.

\begin{thm}
We have that
\[
\sin \pi z=ze^{g(z)} \prod_{n\ne 0}\pa{1-\fc zn}e^{z/n}.
\]
\end{thm}
\begin{proof}
We know that $\sin \pi z$ has zeros at the integers $\Z$. The number of integers with absolute value at most $R$ is $O(R)$, so by the product development~\ref{product-development}, we know that
\[
f(z)=z \prod_{n\ne 0} \pa{1-\fc zn}e^{\fc za}
\]
is an entire function with zeros at exactly the integers.

Now we just need to show $\sin \pi z=\pi f(z)$.

First note that $\Z$ are the {\it only} zeros for $\sin$: We have $\sin \pi z=\fc{e^{i\pi z}-e^{-i\pi z}}{2}$, and this is 0 iff $e^{i\pi z}=e^{-i\pi z}$, iff $i\pi z=2\pi in -i\pi z$ for some $n$, iff $z\in \Z$. The zeros of $\sin$ and $f(z)$ are both simple, so $\fc{\sin \pi z}{f(z)}$ is an entire function. 

Second, note that $\fc{\sin \pi z}{f(z)}$ is {\it bounded}. \fixme{take log derivative, see Ahlfors p. 197. Blah blah.}
\end{proof}
Now let's take a step back. Suppose we didn't know the existence of the function $\sin$, and someone told us to construct a nontrivial complex analytic function with period 1. What would we do? We might look for a function which has zeros at exactly the integers, and hope that it is periodic. In this way we could have ``come up with" $\sin$ using the product expansion.

Now let's take an algebraic viewpoint. Note that $(\sin z,\cos z),\,z\in \R$ parameterizes the circle $x^2+y^2=1$. If we were studying the curve $x^2+y^2=1$; we might think: how could we parameterize it using nice analytic functions? The answer is quite obvious, but if we didn't know, we might think: the circle is homeomorphic to $\R/\Z$ so we would try to pick functions that are periodic with period 1... Again, we're led back to $\sin$.

Another question we could ask is, what about other functions that are periodic with period 1? Can we always write them in terms of $\sin$ and $\cos$? Unfortunately, no. \fixme{Counterexample?} However, for many functions we can: for instance, $\sin(n\pi z)$ can be expanded in terms of $\sin(\pi z)$ and $\cos(\pi z)$. We have a lot of trig identities.

In a way, although having a single period was easy to analyze, we are somewhat out of things to do at this point. If we have a random function with period 1, to characterize it we just need to know how it behaves along a strip $a+bi$, $a\in [k,k+1)$. This strip is infinite, however---$\Z$ is a 1-D lattice in a 2-D space, $\C$.

Things are much nicer if we have a function that is periodic in {\it two} directions, i.e., is periodic with respect to a 2-D lattice in $\C$. Then this function is determined by its values on a parallelogram, which is bounded. Here we have a hope of a simple (non-infinite) representation for such a function---if we can get two periodic functions like this, $f$ and $g$, and know either $\fc fg$ or $f-g$ is bounded, then the values it attains are just those on a parallelogram, and by Liouville's constant we know $f$ and $g$ just differ by a constant (multiplied or added, respectively).

$x^2+y^2=1$ is a simple curve, though it's not very interesting. When we're working over $\C$ we want to consider the surface over $\C$ and our parametrization doesn't really make $x^2+y^2=1$ into anything compact (over $\C$). {\it Elliptic curves} are where the idea of using analytic functions to parameterize algebraic varieties really shines. Elliptic curves are the simplest kind of {\it abelian variety} (1-dimensional), and abelian varieties allow nice analytic parameterizations. ($x^2+y^2=1$ is not an abelian variety.)

\section{Theta functions}

We ask: what is the complex analytic analogue of a periodic function? Since $\C$ is 2-dimensional, a natural analogue would be a function that is
\begin{enumerate}
\item
entire, and
\item has 2 $\R$-independent periods, i.e., there are $\om_1$ and $\om_2$ with $\fc{\om_1}{\om_2}\nin \R$, such that 
\[
f(z+\om_1)=f(z+\om_2)=f(z).
\]
\end{enumerate}
For $f$ satisfying the second condition, we say that $f$ is doubly-periodic, or $f$ is periodic with lattice $\an{\om_1,\om_2}$. We call $\Pi=\set{a\om_1+b\om_2}{a,b\in [0,1)}$ the fundamental parallelogram.

The bad news is that any such function must be a constant.

\begin{pr}
An entire doubly-periodic function is constant.
\end{pr}
\begin{proof}
If $f$ is periodic with lattice $\an{\om_1,\om_2}$, it is determined by its values in the fundamental parallelogram $\Pi$. Now $\ol{\Pi}$ is closed and bounded, hence compact, so $f$ is bounded on $\ol{\Pi}$. But this means $f$ is bounded, and by Liouville's Theorem~\ref{thm:liouville}, $f$ is constant.
\end{proof}

Hence to develop an interesting theory of complex periodic functions, we have to relax one of our conditions:
\begin{enumerate}
\item
double periodicity,
\item
entirety.
\end{enumerate}
In this section we relax double periodicity.
\begin{df}
A \textbf{theta function} of degree $n$ with parameter $b\ne 0$ with respect to $(\om_1,\om_2)$ is an entire function $f(z)$ such that
\[
f(z+\om_1)=f(z),\qquad f(z+\om_2)=be^{-\fc{2\pi i nz}{\om_1}} f(z).
\]
\end{df}
The reason for the multiplier $be^{-\fc{2\pi i nz}{\om_1}} f(z)$ will be apparent during our attempt to construct a theta function.

\subsection{Take 1: The Weierstrass $\si$-function}
For simplicity, let's take $\om_1=1$ and $\om_2=\tau$, $\Im \tau>0$. This can always be achieved by rescaling $z$ and permuting $\om_1,\om_2$. 

In trying to make a doubly periodic function, our first try will be to emulate what we did for $\sin$: let's try to make a function with zeros at exactly the points of $\La=\an{1,\tau}$. We use the product development~\ref{product-development}. This time, the number of points of $\La$ inside a ball of radius $R$ is $O(R^2)$, so we construct the following function.
\begin{df}
Define the \textbf{Weierstrass $\si$-function} with respect to the lattice $\La=\an{1,\tau}$ by
\[
\si(z):=z\prod_{\la\in \La^*}\pa{1-\fc{z}{\la}}e^{\fc z{\la} +\rc 2\pf{z}{\la}^2}
\]
where $\La^*=\La\bs \{0\}$.
\end{df}
By the product development, $\si$ has zeros exactly at the points of $\La$. Note that $\si$ implicitly depends on $\tau$ as well, but we are thinking of $\tau$ as being fixed. If we want to think of it as a function of 2 variables we will write $\si(z,\tau)$.

In this form, it is not clear that $\si$ satisfies any periodicity condition. In fact, it will not be doubly periodic. To see how it behaves with respect to $z\mapsfrom z+\om_1,z+\om_2$, we take a page from our development with $\sin$ and take the derivative.
\begin{pr}
$\si$ is a theta function.
\end{pr}
\begin{proof}
To deal with this product we take the {\it logarithmic derivative}:
\begin{align*}
\ze(z):=\fc{\si'(z)}{\si(z)}&=\rc z+\sum_{\la\in \La^*}\pa{\fc{-\rc{\la}}{1-\fc{z}{\la}}+\rc{\la}+\fc{z}{\la^2}}\\
&=\fc{1}{z}+\sum_{\la\in \La^*}\pa{\fc{1}{z-\la}+\rc{\la}+\fc{z}{\la^2}}.
\end{align*}
We'd like to know how this transforms with respect to $z\mapsfrom z+\la$, but the fact that the sum is over $\La^*$ instead of $\La$ causes problems. We take another derivative:
\[
\wp(z):=-\ze'(z)=\rc{z^2}+\sum_{\la\in \La^*} \pa{\rc{(z-\la)^2}-\rc{\la^2}}.
\]
Taking yet another derivative kills the constant and gives us a sum over $\La$.
\[
\wp'(z)=-\ze''(z)=-\fc{2}{z^3}-\sum_{\la\in \La^*}\fc{2}{(z-\la)^3}=\sum_{\la\in \La}\fc{2}{(z-\la)^3}.
\]
Because $\wp'(z)$ is a sum over $\La$, it is periodic with respect to $\La$:
\[
\wp'(z+\la)=\wp'(z)\implies (\wp(z+\la)-\wp(z))'=0
\]
Now we start integrating. This gives
\[
\wp(z+\la)=\wp(z)+\eta(\la)
\]
for some constant $\eta(\la)$ depending only on $\la$. ...

\end{proof}

\subsection{Take 2: A $q$-product expansion}
\llabel{sec:theta-q-prod}
To write down a theta function using the product development, we took a product over {\it all} nonzero points in the lattice $\La$. However, there's something different and clever we can do: we can use the fact that $\La$ is a lattice to write down a theta function using an infinite product over integers insted.

We want our function to have zeros exactly when
\[
z=m+n\tau
\]
for some $m,n\in \Z$, i.e., exactly when
\[
2\pi iz =2\pi i m+2\pi i n\tau\text{ for some }m,n\in \Z\qquad \iff \qquad e^{2\pi i z}=e^{2\pi i n\tau}
\]
for some $n$. For ease of notation, we'll write from now on
\beq{eq:q}
q=e^{2\pi i \tau},
\eeq
so our condition is $q^n=e^{2\pi i z}$ for some $n\in \Z$. 
We now write down the following function:
\[
\phi(z):=\prod_{n=0}^{\iy}(1-q^n e^{2\pi i z})(1-q^{n+1}e^{-2\pi i z}).
\]
By our above considerations, $\phi$ has zeros at exactly the points of $\La$. \fixme{(We do have to check convergence.)} (Note our notation $\phi$ is not standard; this is actually ad-hoc because we will want to multiply $\phi(z)$ by a constant depending on $q$ later on.)

Let's verify that $\phi$ is elliptic. We have
\begin{align*}
\phi(z+\tau)&=\prod_{n=0}^{\iy} (1-q^ne^{2\pi i(z+\tau)})(1-q^{n+1}e^{-2\pi i (z+\tau)})\\
&=\prod_{n=0}^{\iy} (1-q^{n+1}e^{2\pi i})(1-q^n e^{2\pi iz})\\
&=\fc{1-e^{-2\pi i z}}{1-e^{2\pi i z}}\prod_{n\in \Z}(1-q^n e^{2\pi i z})(1-q^{n+1}e^{-2\pi i z})\\
&=-e^{-2\pi iz}\phi(z).
\end{align*}
Thus $\phi$ is elliptic with parameter $b=-1$.

The kinds of products in the definition of $\phi$ will come up so often that we will establish some shorthand notation.
\begin{df}
Define the \textbf{$q$-Pochhammer symbol} by
\[
(a;q)_n=\prod_{k=0}^{n-1} (1-aq^k)
\]
and let
\[
(q)_n=(q;q)_n=\prod_{k=1}^{n}(1-q^k).
\]
(We allow $n=\iy$.)
\end{df}
Thus $\phi$ can be written
\[
\phi(z)=(e^{2\pi i z};q)_{\iy}(e^{-2\pi i z};-q)_{\iy}.
\]
\subsection{Take 3: Fourier expansion}
\llabel{sec:theta-fourier-exp}
This time we forgo products entirely, and work with the Fourier expansion. A theta function is periodic with period 1, so it has a Fourier expansion by the following theorem.
\begin{pr}
Let $f$ be a complex analytic function defined on $\Im z\ge k$, and suppose $f$ is holomorphic at infinity, that is, $\ln f$ has a removable discontinuity at $0$. Then $f$ has a Fourier expansion
\[
f(z)=\sum_{k=0}^{\iy} a_k e^{2\pi i z}.
\]
\end{pr}
\fixme{picture here!}

We now characterize all theta functions by their power series expansions.
\begin{pr}
The theta functions with degree...
\begin{itemize}
\item
$n<0$ is 0.
\item
$n=0$ are the functions $Ce^{2\pi i kz}$. They form a 1-dimensional space.
\item
$n>0$ are the functions $\sum_{k=0}^{\iy} a_k e^{2\pi i z}$, where $a_0,\ldots, a_{n-1}$ can be freely chosen and the $a_k$ satisfy %and with coefficients satisfying
%the recurrence relation 
%\[
%a_{k+n}=a_k q^kb^{-1}.
%\]

%\section{Theta functions}
%\begin{df}
%A \textbf{theta function} of degree $n$ on $[\om_1,\om_2]$ 
%with parameter $b\ne0$ is an entire function $f(z)$ such that
%\[
%f(z+\om_1)=f(z),\quad f(z+\om_2)=be^{-\frac{2\pi inz}{\om_1}}f(z).
%\]
%\end{df}
%We aim to classify all such functions. For simplicity assume $\om_1=1$ and $\om_2=\tau$, with $\Im \tau>0$. (Rescale.)
%\begin{pr}
%The space of theta functions of degree $n$ and parameter $b$ forms a $n$-dimensional space. They are in the form
%\[
%\sum_{k=0}^{\iy} a_k q^k
%\]
%where $q=e^{2\pi iz}$, $a_0,\ldots, a_{n-1}$ can be freely chosen, and the coefficients satisfy 
the recursive relation
\[
a_{m+pn}=b^{-p}q_0^{mp+\frac{np(p-1)}2}a_m,\quad q_0=e^{-2\pi i \tau}.
\]
They form a $n$-dimensional space.
\end{itemize}
%\fixme{DARN there are so many different definitions of the theta function.} 
In particular, the following is a theta function of degree $1$ and parameter $b$:
\[
\theta(z)=\sum_{k\in \Z} (-1)^k q^{\frac{k(k-1)}2}e^{2\pi i kz}%=C(q_0)\prod_{n=0}^{\iy} (1-q_0^n q)(1-q_0^{n+1}q^{-1}). 
\]
\end{pr}
\begin{proof}
We consider the 3 cases separately.
\end{proof}
\subsection{Factoring theta functions}
We have the following analogue of the fundamental theorem of algebra.
\begin{thm}
Any theta function of degree $n$ is in the form
\[
f(z)=K\theta(z-z_1)\cdots \theta(z-z_n)q^r
\]
for some $z_1,\ldots, z_n\in \C$ and $r\in \Z$.
\end{thm}

\subsection{(2)$=$(3): Jacobi Triple Product Identity}

We now relate the formulas we derived in Sections~\ref{sec:theta-q-prod} and~\ref{sec:theta-fourier-exp}. Since the space of theta functions of degree 1 is 1-dimensional, and the functions $\phi$ from Section~\ref{sec:theta-q-prod} and $\te$ from Section~\ref{sec:theta-fourier-exp} are both theta functions of degree 1, they must be related by a constant independent of $z$.

The following tells us exactly what this constant is.
\begin{pr}
We have
\[
\te(z)=BLAH \phi(z)=BLAH (e^{2\pi iz}; q)_{\iy}(e^{-2\pi i z};q)_{\iy}.
\]
\end{pr}
This statement is equivalent to the Jacobi triple product identity, which appears in many other contexts because it relates $q$-products and Fourier series.
\begin{thm}[Jacobi triple product identity]
We have
\[
\sum_{n\in \Z}(-1)^nq^{n^2}\om^n=\prod_{n=1}^{\iy} (1-q^{2n})(1-q^{2n-1}\om)(1-q^{2n-1}\om^{-1})=(q^2;q^2)(q\om;q^2)(q\om^{-1};q^2).
\]
as an identity of formal power series, and as complex analytic functions, for $|q|<1$.
\end{thm}
\begin{rem}
There are several different forms of the triple product identity. Another common formulation replaces $\om$ by $\om^{\rc 2}$, and we get
\[
\sum_{n\in \Z}(-1)^nq^{n^2}\om^n=\prod_{n=1}^{\iy} (1-q^{2n})(1-q^{2n-1}\om)(1-q^{2n-1}\om^{-1})=(q^2;q^2)(q\om;q^2)(q\om^{-1};q^2).
\]
\end{rem}

%\subsection{Transformation law}
%
%
%\section{Elliptic functions}
%\begin{df}
%An \textbf{elliptic function} on the lattice $\La$ is a meromorphic function $f(z)$ on $\C$ such that
%\[
%f(z+\om)=f(z)\quad \text{for all }\om \in \La, z\in \C.
%\]
%%We call $\La$ the period.
%Denote the space of all such functions by $\C(\La)$.
%\end{df}
%
%There are nice relationships involving the zeroes and poles of elliptic functions.
%\begin{thm}Let $f$ be an elliptic function on $\La$.
%\begin{enumerate}
%\item $\sum_{w\in \C/\La} \Res_w(f)=0$.
%\item $\sum_{w\in \C/\La} \ord_w(f)=0$, i.e. in a fundamental parallelogram there are as many zeros as poles, counting multiplicities.
%\item $\sum_{w\in \C/\La} \ord_w(f)w\in \La$.
%\end{enumerate}
%\begin{proof}
%\begin{enumerate}
%\item
%\item 
%\item Label the edges of the fundamental parallelogram as follows.
%\[
%\xymatrix{
%& \al+\omega_2 \ar[rr]^{C_2} & & \al+\omega_1+\omega_2\ar[ld]^{C_3} \\
%\al\ar[ur]^{C_1} & &\al+\omega_1 \ar[ll]^{C_4}&
%}
%\]
%We calculate $\int_{\partial P} \frac{zf'(z)}{f(z)}\,dz$ in two ways.
%
%\textbf{Way 1:} 
%\[
%\int_{\partial P} \frac{zf'(z)}{f(z)} \,dz=\ba{
%\int_{C_1} \frac{zf'(z)}{f(z)}\,dz
%+\int_{C_3} \frac{zf'(z)}{f(z)}\,dz
%}
%+\ba{
%\int_{C_2} \frac{zf'(z)}{f(z)}\,dz
%+\int_{C_4} \frac{zf'(z)}{f(z)}\,dz
%}.
%\]
%Noting that $C_3$ is just $C_1$ shifted by $\omega_1$ and reversed, and that $C_2$ is just $C_4$ shifted by $\omega_2$ and reversed, this equals
%\[
%\int_{\partial P} \frac{zf'(z)}{f(z)} \,dz=
%\int_{C_1} \ba{\frac{zf'(z)}{f(z)}
%-\frac{(z+\omega_1)f'(z+\omega_1)}{f(z+\omega_1 )}
%}\,dz
%+
%\int_{C_4} \ba{\frac{zf'(z)}{f(z)}- \frac{(z+\omega_2)f'(z+\omega_2)}{f(z+\omega_2)}
%}\,dz.
%\]
%Since $f$ is elliptic, $f(z)=f(z+\omega_1)=f(z+\omega_2)$, giving
%\[
%\int_{\partial P} \frac{zf'(z)}{f(z)} \,dz=
%-\omega_1\int_{C_1} \frac{f'(z)}{f(z)} \,dz-\omega_2\int_{C_4} \frac{f'(z)}{f(z)}\,dz.
%\]
%Now $\ln(f(z))$ can be defined in a neighborhood around $C_1$ and $C_4$, since $f$ has no poles or zeros on $\partial P$. Since $f(\al)=f(\al+\omega_1)=f(\al+\omega_2)$, we have $\ln(f(\al+\omega_1))-\ln(f(\al))=2\pi i c_1$ and $\ln(f(\al))-\ln(f(\al+\omega_2))=2\pi i c_2$ for some integers $c_1$ and $c_2$. But these equal the above integrals by definition of $\ln f(z)$, so
%\begin{equation}\label{p1-1-1}
%\int_{\partial P} \frac{zf'(z)}{f(z)} \,dz=
%-2\pi i(\omega_1 c_1+\omega_2 c_2).
%\end{equation}
%
%\textbf{Way 2:}
%Note $\Res_a \frac{f'(z)}{f(z)}=\ord_a f$ so $\Res_a \frac{zf'(z)}{f(z)}=a \ord_a f$. Letting $a_k$ be the poles and zeros of $f$ in $P$, we get by Cauchy's Theorem that
%\begin{equation}\label{p1-1-2}
%\int_{\partial P} \frac{zf'(z)}{f(z)}=2\pi i\sum_{k} \Res_{a_k} \frac{f'(z)}{f(z)}=2\pi i \sum_{k}m_k a_k.
%\end{equation}
%Equating~(\ref{p1-1-1}) and~(\ref{p1-1-2}) give
%\[
%\sum_{k}m_ka_k=-\omega_1c_1-\omega_2c_2\equiv 0\pmod{\La}.
%\]
%\end{enumerate}
%\end{proof}
%\end{thm}
%%\begin
%\begin{df}
%The \textbf{order} of an elliptic function is the number of poles in a fundamental parallelogram.
%\end{df}
%It turns out that elliptic functions can be expressed as quotients of theta functions.
%\begin{thm}
%\[
%f(z)=K\frac{\theta(z-a_1)\cdots \theta(z-a_k)}{\theta(z-b_1)\cdots \theta(z-b_k)}, \quad \sum_{i=1}^k a_i=\sum_{i=1}^k b_i.
%\]
%\end{thm}
%\section{Weierstrass $\wp$-function}
%Our basic example of an elliptic function is the following.
%\begin{df}
%Define the Weierstrass $\wp$-function for the lattice $\La$ by
%\[
%\wp(z)=\rc{z^2}+\sum_{\la\in \La\bs\{0\}}\ba{
%\rc{(z-\la)^2}-\rc{\la^2}
%}.
%\]
%\end{df}
%\begin{pr}
%The series defining $\wp$ converges absolutely and locally uniformly on $\C-\{\La\}$. 
%$\wp$ is an even elliptic function with period $\La$, analytic except for a double pole at each point of $\La$, 
%\end{pr}
%In fact, we will see that it is the building block for all elliptic functions.
%\begin{proof}
%
%\end{proof}
%\begin{thm}
%Every even elliptic function can be written as a polynomial in $\wp$. 
%Every elliptic function can be written as a polynomial in $\wp$ and $\wp'$. 
%\end{thm}
%\begin{thm}
%\[
%\wp(z)-\wp(a)=\frac{\theta(z+a)\theta(z-a)}{\theta(z)^2} \cdot \frac{\theta'(0)^2}{\theta(a)\theta(-a)}.
%\]
%\end{thm}
%\begin{thm}[Weierstrass differential equation]
%\[
%\wp'(z)^2=4(\wp(z)-e_1)(\wp(z)-e_2)(\wp(z)-e_3)
%=4\wp(z)^3-\underbrace{60G_4}_{g_2}\wp(z)- \underbrace{140G_6}_{g_3}(z)
%\]
%\end{thm}
%This says that for every $z$, the point $(\wp(z),\wp'(z))$ lies on the elliptic curve $y^2=4x^3-60G_4-140G_6$. Together with surjectivity and the Uniformization Theorem~\ref{uniformization} this implies that all elliptic curves can be parameterized in this way. (NONZERO DISC.)
%\begin{thm}[Unifomization theorem]\llabel{uniformization}
%Let $A,B\in \C$ satisfy $A^3-27B^2\ne 0$. Then there exists a unique lattice $\La\sub \C$ such that $g_2(\La)=A$ and $g_3(\La)=B$.
%\end{thm}
%%%%%%%%%%
%\subsection{$\wp$ and lattices}
%\begin{thm}
%%cox, 10.14
%Let $L$ be the lattice corresponding to $\wp(z)$. For $\al\in \C\bs \Z$, the following are equivalent.
%\begin{enumerate}
%\item $\wp(\al z)$ is  rational function in $\wp(z)$.
%\item $\al L\subeq L$.
%\item There is an order $\sO$ in an imaginary quadratic field $K$ such that $\al\in \sO$ and $L$ is homothetic to a proper $\sO$-ideal.
%\end{enumerate}
%Then 
%\[
%\wp(\al z)=\frac{A(\wp(z))}{B(\wp(z))}
%\]
%for relatively prime polynomials $A$ and $B$ such that 
%\[
%\deg(A)=\deg(B+1)=[L:\al L]=\N\al.
%\]
%\end{thm}
%\begin{proof}
%$(1)\implies (2)$: 
%Suppose that $\wp(\al z)=\frac{A(\wp(z))}{B(\wp(z))}$ with $A$ and $B$ relatively prime. Then
%\begin{equation}
%B(\wp(z))\wp(\al z)=A(\wp(z)).
%\end{equation}
%%Note that each linear factor $\wp(z)+r$ of $A(\wp(z))$ and $B(\wp(z))$ has the same poles with the same orders as $\wp(z)$. 
%For any $\om\in L$, %both 
%$\wp(\om)$ %and $\wp(\al z)$ 
%has a pole of order 2, and each linear factor $\wp(z)+r$ of $A(\wp(z))$ and $B(\wp(z))$ has a pole of order 2. 
%In particular, for $\om=0$, we get that the order is
%\[
%2\deg(B)+2=2\deg(A)
%\]
%showing that $\deg(A)=\deg(B)+1$. Now take any $\om\in L$. Counting the order of $\om$ on both sides, we find that $\wp(\al z)$ has a pole of order 2 at $\om$. Thus $\al \om\in L$. This shows $\al L\subeq L$.
%
%$(2)\implies (1)$:
%For any $w\in L$, since $\al L\subeq L$we have
%\[
%\wp(\al (z+w))=\wp(\al z+\underbrace{\al w}_{\in L})=\wp(\al z).
%\] 
%Hence $\wp(z)$ is elliptic with $L$ as a lattice of periods. Since it is even, by (?) it is a rational function in $\wp$.
%
%$(2)\implies (3)$: %Assume $\al$ acts on $L=\an{1,\tau}$ by $\smatt abcd$. Then
%By a homothety we may suppose $L=\an{1,\tau}$. Since $L$ has rank 2 as a $\Z$-module, $\tau$ must be of degree 2 over $\Q$. Now take
%\[
%\sO=\set{\be \in \Q(\tau)}{\be L\subeq L},
%\]  
%i.e. the ``codifferent."
%
%$(3)\implies (2)$: Easy.
%
%Now, supposing (1) is true, rearrange $\wp(\al z)=\frac{A(\wp(z))}{B(\wp(z))}$ to get
%\begin{equation}\label{awpb}
%A(x)=\wp(\al z) B(x)=0.
%\end{equation}
%Fix $z$ so that $2z\nin \rc{\al}L$ and such that $A(x)-\wp(\al z)B(x)$ has distinct zeros. (Claim: Given polynomials $A$, $B$, there are only a finite nmber of values of $c$ so that $A-cB$ has multiple roots.) 
%Let $\{w_i\}$ be a set of coset representatives for $L$ in $\rc{\al}L$. We claim that the roots of~(\ref{awpb}) are exactly $z+w_i$.
%
%We have
%\[
%A(\wp(z+w_i))-\wp(\al z)B(\wp(z+w_i))=A(\wp(z+w_i))-\wp(\al (z+w_i))B(\wp(z+w_i))=0
%\]
%by blah, so $\wp(z+w_i)$ are roots of~(\ref{awpb}).
%
%Now if $\wp(z+w_i)=\wp(z+w_j)$ then by BLAH, $(z+w_i)=\pm(z+w_j)\pmod{L}$, giving either $2z\equiv w_i-w_j\in \rc{\al}L$ and $2z\in \rc{\al}$, or $w_i\equiv w_j\pmod L$. The first is impossible by assumption on $z$, so $i=j$. This shows the roots are distinct. 
%
%Finally, given any root of~(\ref{awpb}), by surjectivity of $\wp$ we can write it in the form $\wp(y)$. We have
%\[
%\wp(\al y)=\frac{A(\wp(y))}{B(\wp(y))}=\wp(\al z),
%\]
%where the first equality is by definition of $A$ and $B$ and the second is because $\wp(y)$ is a root of~(\ref{awpb}). Then by BLAH, $\al y\pm \al z\equiv 0\pmod{L}$. Since $\wp$ is even, we may replace $y$ by $-y$ as necessary, to get $\al(y-z)\equiv 0\pmod{\rc{\al} L}$. Thus  $y\in z+\rc{\al}L$ and $\wp(y)=\wp(z+w_i)$ for some $i$, as needed.
%
%Since~(\ref{awpb}) has $[L:\rc{\al}L]=[\al L:L]$ roots,~(\ref{awpb}) and hence $A$ has degree $[\al L:L]$.
%\end{proof}
%
%Note the equivalence $(2)\iff (3)$ (which incidentally has nothing to do with elliptic functions) gives that a lattice is a proper fractional ideal of $\sO$ iff it has $\sO$ as its ring of complex multiplication. Nonzero fractional ideals are homothetic iff they determine the same element in the ideal class group. Hence there is a correspondence between IDEAL CLASS GRP and homothety classes of lattices with $\sO$ as full ring of complex multiplication.
%%%%%%%%%%%%%%%%%%%%%%
