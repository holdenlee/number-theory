\chapter{Arithmetical Functions}
\llabel{arith-f}
\section{Arithmetical functions}
\begin{df}
An \textbf{arithmetical function} is a function $f$ defined on $\N$. \begin{enumerate}
\item
If
\begin{equation}\llabel{arith-f-mult}
f(mn)=f(m)f(n)
\end{equation}
for every $m$ and $n$ relatively prime, then $f$ is \textbf{multiplicative}.
\item 
If~(\ref{arith-f-mult}) holds for every $m,n\in \N$, then $f$ is \textbf{completely multiplicative}.
\end{enumerate}
\end{df}
Note that if $n=p_1^{a_1}\cdots p_m^{a_m}$ is the prime factorization of $n$, then
\[
f(n)=\begin{cases}
f(p_1^{a_1})\cdots f(p_m^{a_m})&\text{if }f \text{ is multiplicative,}\\
f(p_1)^a_1\cdots f(p_m)^{a_m}&\text{if }f\text{ is  completely multiplicative}.
\end{cases}
\]
\section{Number of divisors}
\section{Totient}
\section{Sum of divisors}
\section{M\"obius function}
\section{Sums of digits}
%For a positive integer $N = \overline{a_na_{n-1}\cdots a_0}$ write $s(N) = a_n+a_{n-1}+\cdots+a_0$ - the sum of the digits of $N$.  Then we claim that $s(N)$ obeys the following properties of no small interest:
%\begin{pr}
%\begin{enumerate}
%\item $9 \mid n-s(n)$;
%\item $s(n+m) \le s(n)+s(m)$;
%\item $s(nm)\le \min{\{n\cdot s(m),m\cdot s(n)\}}$;
%\item $s(nm)\le s(n)s(m)$.
%\end{enumerate}
%\begin{proof}
%\begin{enumerate}
%\item We have $$n-s(n) = a_n(10^n-1)+a_{n-1}(10^{n-1}-1)+\cdots+a_1(10^1-1).$$  Since $9\mid10^n-1$, we must have $9\mid n-s(n)$.
%\item Write $N = \overline{a_ma_{m-1}\cdots a_0}$.  Then we may write
%\begin{align*}
%9\sum_{k \ge 1}\left\lfloor\frac{N}{10^k}\right\rfloor&= 9\left(\left\lfloor\frac{N}{10^m}\right\rfloor+\left\lfloor\frac{N}{10^{m-1}}\right\rfloor+\cdots+\left\lfloor\frac{N}{10}\right\rfloor\right)\\
%&=9\left(a_m+(10a_m+a_{m-1})+\cdots+(10^{m-1}a_m+\cdots+10a_2+a_1)\right)\\
%&=9\left(a_m(1+10+\cdots+10^{m-1})+a_{m-1}(1+\cdots+10^{m-2})+\cdots+a_1\right)\\
%&=a_m(10^m-1)+a_{m-1}(10^{m-1}-1)+\cdots+a_1(10-1)\\
%&=N-s(N)
%\end{align*}
%Now we recall the inequality $\lfloor x\rfloor+\lfloor y\rfloor \le \lfloor x+y\rfloor$ and write
%\begin{align*}
%S(n+m)&=n+m-9\sum_{k \ge 1}\left\lfloor\frac{n+m}{10^k}\right\rfloor\\
%&\le n+m-9\sum_{k \ge 1}\left(\left\lfloor\frac{n}{10^k}\right\rfloor+\left\lfloor\frac{m}{10^k}\right\rfloor\right)\\
%&=s(n)+s(m)
%\end{align*}
%exactly as desired.
%\item We need only verify that $s(nm) \le n\cdot s(m)$, as then our result will follow by symmetry.  But this is in turn easily proved using induction on $n$ and our previous result.
%\item Write $m = \overline{b_lb_{l-1}\cdots b_0}$.  Then $$s(nm)=s(n\sum_{i=0}^l b_i10^i)\le\sum_{i=0}^l s(nb_i10^i)=\sum_{i=0}^l s(nb_i)\le \sum_{i=0}^l s(n)b_i=s(n)s(m)$$ exactly as desired.
%\end{enumerate}
%\end{proof}
%\end{pr}
%\begin{prb}
%In the decimal expansion of $N$, the digits occur in strictly increasing order from right to left.  Find $s(9N)$.
%\begin{proof}
%Write $N = \overline{a_ka_{k-1}\cdots a_1a_0}$.  Then
%\begin{align*}
%9N = 10N-N&=\overline{a_ka_{k-1}\cdots a_1a_00}\\
%&\quad-\overline{a_ka_{k-1}\cdots a_1a_0}
%&=\overline{a_k(a_{k-1}-a_k)\cdots(a_0-a_1-1)(10-a_0)}
%\end{align*}
%Hence $s(9n) = a_k+(a_{k-1}-a_k)+\cdots+(a_0-a_1-1)+(10-a_0) = 9$.
%\end{proof}
%\end{prb}
%\begin{prb}
%Determine all possible values of the sum of the digits of a perfect square.
%\begin{proof}
%We may check that a perfect square modulo 9 is congruent to one of 0, 1, 4, and 7; hence the sum of the digits of a perfect square is congruent to 0, 1, 4, or 7 modulo 9.  We shall show that if $n \ge 1$ with $n \equiv 0, 1, 4, 7\mod{9}$, then there exists $N$ with $s(N^2) = n$.
%
%If $n = 1,4$ we take $N = 1, 2$.  We proceed to the cases $n > 4$.
%If $n = 9m$ then we may take $N = 10^m - 1$.  Then $$N^2 =
%10^m(10^m-2)+1=\underbrace{99\cdots9}_{m-1}8\underbrace{00\cdots0}_{m-1}1,$$
%so $s(N^2) = 9m$ as desired.  If $n = 9m+1$ then we take $N =
%10^m-2$.  Then $$N^2 = 10^m(10^m-2)+2 =
%\underbrace{99\cdots9}_{m-1}6\underbrace{00\cdots0}_{m-1}4,$$ so
%$s(N^2) = 9m+1$ as desired.  If $n = 9m+4$ then we take $N =
%10^m-3$.  Then $$N^2 = 10^m(10^m-6)+9 =
%\underbrace{99\cdots9}_{m-1}4\underbrace{00\cdots0}_{m-1}9,$$ so
%$s(N^2) = 9m+4$ as desired.  If $n = 9m-2$ then we take $N =
%10^m-5$.  Then $$N^2 = 10^m(10^m-10)+25 =
%\underbrace{99\cdots9}_{m-1}\underbrace{00\cdots0}_{m-1}25,$$ so
%$s(N^2) = 9m-2$ as desired.  This completes our proof.
%\end{proof}
%\end{prb}
%\begin{prb}
%Find all positive integers $n$ with $n = s(n)+p(n)$, where $p(n)$ is the product of the digits of $n$.
%\begin{proof}
%Write $$n = \overline{a_1a_2\cdots a_k} = a_1+a_2+\cdots+a_k+a_1a_2\cdots a_k.$$  We may rewrite this as $$a_1(10^{k-1}-1)+a_2(10^{k-2}+1)+\cdots+a_k(10-1) = a_1a_2\cdots a_k,$$ which in turn we rewrite as $$a_2(10^{k-2}+1)+\cdots+a_k(10-1) = a_1(a_2\cdots a_k-10^{k-1}+1).$$  Suppose $k \ge 3$; then we have $a_2\cdots a_k-10^{k-1}+1 \le 9\cdot9\cdots9-10^{k-1}+1 = 9^{k-1}-10^{k-1}+1 < 0$.  This means that $$a_1(10^{k-1}-1)+a_2(10^{k-2}+1)+\cdots+a_k(10-1) < 0,$$ a contradiction.  We have obviously $k \neq 1$; hence $k = 2$ and we have $0 = a_1(a_2-9)$.  Hence $a_2 = 9$, so the only such numbers are $19, 29, \cdots, 99$.
%\end{proof}
%\end{prb}
%\begin{prb}
%Determine all positive integers $n$ such that there exist positive integers $a$ and $b$ with $s(a) = s(b) = s(a+b) = n$.
%\begin{proof}
%We must have $9 \mid a-s(a), b-s(b), a+b-s(a+b)$.  Hence $9 \mid s(a)+s(b)-s(a+b) = n$.  Now suppose that $n = 9k$.  We claim that there do exist such $a$ and $b$.  Indeed, set $$a = \underbrace{531531\cdots531}_{3k\text{ digits}}, \qquad\qquad b = \underbrace{171171\cdots 171}_{3k\text{ digits}}.$$  Then it is easily seen that $s(a) = s(b) = 9k = n$.  But we have $$a+b = \underbrace{702702\cdots702}_{3k\text{ digits}},$$ so that $s(a+b) = n$ as well.  This completes our proof.
%\end{proof}
%\end{prb}
%
%\newpage
%\begin{center}
%\textbf{Problems}
%\end{center}
%
%\begin{enumerate}
%%\item Find the last five digits of $5^{1981}$.
%%\item Find the last three digits of $2003^{2002^{2001}}$.
%\item Find a positive integer $n$ such that $s(n) = 1996s(3n)$.
%\item Is there a perfect square that ends in 10 distinct digits?
%%\item Prove that for every positive integer $n$ there exists an $n$-digit number divisible by $5^n$ all of whose digits are odd.
%\item Let $A = s(4444^{4444})$, and set $B = s(A)$.  Find $s(B)$.
%\item Prove that there are infinitely many numbers not containing the digit 0 that are divisible by the sum of their digits.
%\item Prove that any number consisting of $2^n$ identical digits has at least $n$ distinct prime factors.
%\end{enumerate}
\section{Finite calculus}
\begin{thm}[Summation by parts, Abel summation]\llabel{sum-parts}
Suppose that $u$ is an arithmetic function, and let
\[
U(x)=\sum_{n\le x} u(x).
\]
Then for $m,n\in \N$ %(or $\Z$, if $u$ is defined over $\Z$),
\[
\sum_{x=m}^n u(x)v(x)=U(n)v(n)-U(m-1)v(m-1) -\sum_{x=m}^n
U(x-1)(v(x)-v(x-1)).
\]
%\end{thm}
%\begin{thm}[Abel summation]\llabel{abel-sum}
%Keep the above notation and suppose 
If $0\le a<b$ and $v$ has continuous derivative on $a<x<b$, then
\[
\sum_{a\le x\le b} u(x)v(x)=U(b)v(b)-U(a)v(a) -\int_{a}^b U(x)v'(x).
\]
\end{thm}
\begin{proof}
We imitate the proof of integration by parts. For a function $f$ define the function
\[
\De_-(f)=f(x)-f(x-1).
\]
This is the discrete analogue of differentiation. It is the inverse of summation in the sense that by telescoping,
\begin{equation}\label{telescope}
\sum_{x=m}^n\De_-(f)=f(n)-f(m-1).
\end{equation}
Note that $\De_-(U)=u$. 
We have the ``product rule"
\begin{align*}
\De_-(uv)&=u(x)v(x)-u(x-1)v(x-1)\\
&=(u(x)-u(x-1))v(x)+u(x-1)(v(x)-v(x-1))\\
&=\De_-(u)v+E_-u\De_-(v)
\end{align*}
where $E_-$ is the left shift operator $(E_-f)(x)=f(x-1)$. 
Replacing $u$ by $U$ and rearranging gives
\[
uv=\De_-(Uv)-E_-U\De_-(v).
\]
Summing over $m\le x\le n$ and telescoping using~(\ref{telescope}) gives
\[
\sum_{x=m}^n u(x)v(x)=U(n)v(n)-U(m-1)v(m-1) -\sum_{x=m}^n
U(x-1)(v(x)-v(x-1)).
\]
When $v$ has continuous derivative, noting $U(t)=U(\fl{t})$, we have
\begin{align*}
\sum_{x=m}^n
U(x-1)(v(x)-v(x-1))
&=\sum_{x=m}^n \int_{x-1}^{x}U(t)v'(t)\,dt\\
&=\int_{m-1}^n U(t)v'(t)\,dt.
\end{align*}
For general $a,b$, since $U$ is constant on $(\fl{b},b)$ and $(a,\fl{a}+1)$,
\begin{align*}
\sum_{a<x\le b} u(x)v(x)
&=\sum_{x=\fl{a}+1}^{\fl{b}} u(x)v(x)\\
&=U(\fl{b})v(\fl{b}) - U(\fl{a}) v(\fl a) +\int_{\fl{a}}^{\fl{b}} U(t)v'(t)\,dt\\
&=U(b)v(b) - U(a) v(a) +\int_{a}^{b} U(t)v'(t)\,dt\qedhere
\end{align*}
\end{proof}

Interesting: (Putnam ??) Suppose that $a$ is a real number such that all numbers $1^a,2^a,3^a,\ldots$ are integers. Prove that $a$ is also an integer.