\chapter{Diophantine equations}
Stuff I want to include in this chapter
\begin{enumerate}
\item linear Diophantine equations
\item quadratic Diophantine equations 
\begin{enumerate}
\item Pell
\item root flipping IMO 89/6. TST 02/6. Find in explicit form all ordered pairs of positive integers $(m,n)$ such that $mn-1\mid m^2+n^2$.
\item sum of squares
\item sum of 4 squares
\end{enumerate}
\item techniques:
\begin{enumerate}
\item
size comparison, analytical methods
\item
taking modulo. enumerating solutions
\item factoring (SFFT)
\item infinite descent 
\item Iurie's ``parameterization" trick. (IMO ??/6: Let $a>b>c>d$ be positive integers and suppose 
\[
ac+bd=(b+d+a-c)(b+d-a+c).
\]
Prove that $ab+cd$ is not prime.
\item Constructing solutions
\item
geometric methods (Minkowski)
\end{enumerate}
\end{enumerate}

\section{Linear Diophantine Equations}
An equation of the form
\begin{equation}
a_1 x_1+ \dots + a_n x_n =b, \label{eq1}
\end{equation}
where $a_1,\dots,a_n,b \in \mathbb{Z}$ is called linear diophantine
equation.

\begin{thm}
The equation (\ref{eq1}) is solvable if and only if
$\gcd(a_1,\dots,a_n)\mid b$.
\end{thm}

\begin{proof}
Let $d=\gcd(a_1,\dots,a_n)$. If $d\nmid b$ the equation is not
solvable. If $d\mid b$ we denote $a_i'=\cfrac{a_i}{d}, \
b'=\cfrac{b}{d}$. Then $\gcd(a_1',\dots,a_n')=1$ and the generalized
B\'{e}zout Lemma says that there exist $x_i'$ such that
$a_1'x_1'+\dots +a_n' x_n'=1$, which implies $a_1x_1'+\dots
+a_nx_n'=d$. We obtain $a_1(b'x_1')+\dots +a_n(b'x_n')=b'd=b.$
\end{proof}

\begin{cor}
Let $a_1,a_2$ be relatively prime integers. If $(x_1^0,x_2^0)$ is a
solution to the equation
$$a_1x_1+a_2x_2=b, $$
then all its solutions are given by
\begin{equation*}
\begin{cases}
x_1=x_1^0+a_2 t \\
x_2=x_2^0-a_1 t
\end{cases}
,t\in \mathbb{Z}.
\end{equation*}
\end{cor}

\begin{prb}
Solve the equation
$$15x+84y=39. $$
\end{prb}

\begin{proof}
The equation is equivalent to $5x+28y=13$. A solution is $y=1, \
x=-3$. All solutions are of the form $x=-3+28t, \ y=1-5t, \ t\in
\mathbb{Z}$.
\end{proof}

\begin{prb}
Solve the equation
$$3x+4y+5z=6. $$
\end{prb}

\begin{proof}
The equation can be written as $3x+4y=6-5z$, $s\in \mathbb{Z}$. A
solution of $3x+4y=1$ is $x=-1$, $y=1$. So a solution of
$3x+4y=6-5s$ is $x_0=5s-6$, $y_0=6-5s$. Hence all solutions are
\begin{equation*}
\begin{cases}
x=5s-6+4t\\
y=6-5s-3t
\end{cases}
\end{equation*}
\end{proof}

For any positive integer $a_1,\dots,a_n$ with
$\gcd(a_1,\dots,a_n)=1$ denote $g(a_1,\dots,a_n)$ to be the greatest
positive integer $N$ for which the equation
$$a_1x_1+\dots+a_nx_n=N $$
is not solvable in nonnegative integers. The problem of determining
$g(a_1,\dots,a_n)$ is known as the Frobenius coin problem.

\begin{prb}[Sylvester, 1884]
Let $a,b\in \mathbb{N}$ and $\gcd(a,b)=1$. Then $g(a,b)=ab-a-b.$
\end{prb}

\begin{proof}
Suppose $N>ab-a-b$. The solutions to the equation $ax+by=N$ are of
the form $(x,y)=(x_0+bt,y_0-at)$, $t\in \mathbb{Z}$. Let $t$ be an
integer such that $0\leq y_0-at\leq a-1$. Then $(x_0+bt)a=N-
(y_0-at)b > ab-a-b-(a-1)b=-a.$ Hence $x_0+bt>-1$, i.e. $x_0+bt\geq
0$ and the equation has a nonnegative solution. Thus $g(a,b)\leq
ab-a-b$.

Now we shall show that the equation
$$ax+by=ab-a-b $$
is not solvable in nonnegative integers. Otherwise we have
$$ab=a(x+1)+b(y+1). $$
Since $\gcd(a,b)=1$, we get $a\mid y+1, \ b\mid x+1$, thus $y+1\geq
a$, $x+1\geq b$. We obtain $ab=a(x+1)+b(y+1)\geq 2ab$, a
contradiction.
\end{proof}
\section{Pythagorean Triples}
A triple $(x,y,z)$ of integers is called Pythagorean if
\begin{equation}
x^2+y^2=z^2. \label{eq2}
\end{equation}

\begin{thm}
Any solution in positive integers of (\ref{eq2}) has the form
$$x=(m^2-n^2)k , \ y=2mnk, \ z=(m^2+n^2)k$$
$$x=2mnk, \ y=(m^2-n^2)k,\ z=(m^2+n^2)k, $$
where
\begin{enumerate}
\item $\gcd(m,n)=1, \ \gcd(x,y)=k.$
\item $m,n$ are of different parity.
\item $m>n>0$, $k>0$.
\end{enumerate}
\end{thm}

\begin{proof}
Let $\gcd(x,y)=k$. Then $x=ka$, $y=kb$, $\gcd(a,b)=1$. Then
$k^2(a^2+b^2)=z^2$. We get $k\mid z$ and set $z=kc$. We obtain
$$a^2+b^2=c^2.$$

Suppose that $a$ is an odd number. Then $b$ is even since otherwise
$c^2=a^2+b^2 \equiv 2 \pmod{4}$, a contradiction.

Thus $c$ is odd. We have
$$b^2=(c-a)(c+a), $$
which is equivalent to
$$\left( \frac{b}{2}\right)^2=\frac{c-a}{2} \frac{c+a}{2} .$$
Note that $\gcd\left( \cfrac{c-a}{2},\cfrac{c+a}{2}\right) =1$.
Otherwise there exists prime $p$ such that $p\mid \cfrac{c-a}{2}$,
$p\mid \cfrac{c+a}{2}$. We get $p\mid \cfrac{c-a}{2}\pm
\cfrac{c+a}{2}=c,a$ which implies $p\mid b$, a contradiction. Hence
$$\frac{c-a}{2}=n^2, \ \frac{c+a}{2}=m^2,\ \frac{b}{2}=mn $$
and we obtain
$$c=m^2+n^2, \ a=m^2-n^2, \ b=2mn. $$
\end{proof}

\begin{prb}
Solve in positive integers the equation
$$\frac{1}{x^2}+\frac{1}{y^2}=\frac{1}{z^2}. $$
\end{prb}

\begin{proof}
The equation is equivalent to
$$x^2+y^2=\left( \frac{xy}{2}\right)^2. $$
We obtain that $z\mid xy$. Hence $x^2+y^2=t^2$, $t=\cfrac{xy}{2}$.

Let $d=\gcd(x,y,t)$. Therefore $x=ad$, $y=bd$, $t=cd$,
$\gcd(a,b,c)=1$. We get
$$a^2+b^2=c^2, \ z=\frac{abd}{c}. $$
Hence $a,b,c$ are pairwise relatively prime and we obtain that
$c\mid d$, which implies $d=kc$. Thus
$$x=kac, \ y=kbc, \ t=kc^2, \ z=kab. $$
We may assume that
$$a=m^2-n^2, \ b=2mn, \ c=m^2+n^2 $$
and we obtain
$$x=k(m^4-n^4), \ y=k2mn(m^2+n^2), \ z=k2mn(m^2-n^2). $$
\end{proof}
\section{Size comparison and analytical methods}
\section{Reducing modulo $n$}
\section{Factoring}
\begin{pr}[Simon's favorite factoring trick, SFFT]
\[
xy+bx+ay+ab=(x+a)(y+b).
\]
\end{pr}
Example.
\begin{ex}
%How many ways can a number $n$ be expressed as the difference of two squares?
Which numbers $n$ can be expressed as the difference of two squares?
\end{ex}
{\it Solution.}
We wish to solve
\[
x^2-y^2=n.
\]
Factor this equation as
\[
(x+y)(x-y)=n.
\]
Note that $x+y$ and $x-y=(x+y)-2y$ are of the same parity. If they are both odd, then $n$ is odd; if they are both even, then $n$ is divisible by 4.

Conversly, if $n$ is odd or $n$ is divisible by 4, then we can write $n=ab$ where $a,b$ are factors of $n$ having the same parity. We wish to have $a=x+y$ and $b=x-y$ so set
\begin{align*}
x&=\frac{a+b}{2}\\
y&=\frac{a-b}{2}.
\end{align*}
Note these are integers by the assumption on $a$ and $b$.
\section{Problems}
(Analysis)
(ISL 2004) Let $b\ge 5$ be an integer and define
\[
x_n=(\underbrace{11\ldots 1}_{n-1}\underbrace{22\ldots 2}_n5)_b.
\]
Prove that $x_n$ is a perfect square for all sufficiently large $n$ if and only if $b=10$.

Looking at prime divisors modulo stuff:
(ISL 2006/N5) Find all pairs $(x,y)$ of integers satisfying the equation
\[
\frac{x^7-1}{x-1}=y^5-1.
\]

%(AMTC 2010/N3) Find all solutions in positive integers to $a^2+2b^2=(x^5-x^3-1)(x^5-x^3-3)$. (Hint: You may use the fact that $-2$ is a quadratic residue modulo a prime $p$ if and only if $p\equiv 2, 1, $ or $3\pmod{8}$.)

(TST ??) Prove that for no integer $n$ is $n^7+7$ a perfect square.
%\section{Methods for solving diophantine equations}
%\textbf{The decomposition method}
%
%\begin{prb}
%Find the integer solutions to the equation
%$$(x^2+1)(y^2+1)+2(x-y)(1-xy)=4(1+xy). $$
%\end{prb}
%
%\begin{proof}
%Write the equation in the form
%$$x^2y^2-2xy+1+x^2+y^2-2xy+2(x-y)(1-xy)=4
%$$
%$$(xy-1)^2+(x-y)^2-2(x-y)(xy-1)=4. $$
%This is equivalent to
%$$(xy-1-x+y)^2=4 $$
%and we obtain
%$$xy-1-x+y=\pm 2 \Rightarrow (x+1)(y-1)=\pm 2. $$
%
%If $(x+1)(y-1)=2$, we obtain $(x,y)=(1,2), \ (-3,0), \ (0,3), \
%(-2,-1)$.
%
%If $(x+1)(y-1)=-2$, we obtain $(x,y)=(1,0), \ (-3,2), \ (0,-1), \
%(-2,3)$.
%\end{proof}
%
%\begin{prb}
%Let $p$ be a prime. Solve in positive integers the equation
%$$\frac{1}{x}+\frac{1}{y}=\frac{2}{p}. $$
%\end{prb}
%
%\begin{proof}
%We have
%$$2xy=px+py \Leftrightarrow 4xy-2px-2py=0 \Leftrightarrow (2x-p)(2y-p)=p^2.$$
%
%1) If $p=2$ we obtain $(x-1)(y-1)=2$, which implies $x=y=2.$
%
%2) If $p\geq 3$, we obtain $2x-p=1, \ p^2, \ -1, \ -p^2$ and
%$2y-p=p^2, \ 1, \ -p^2, \ -1$. We get
%$$(x,y)=\left( \frac{p+1}{2},\frac{p^2+p}{2}
%\right),\  \left( \frac{p^2+p}{2}, \frac{p+1}{2} \right), \ (p,p).$$
%\end{proof}
%
%\textbf{Using inequalities}
%\begin{prb}
%Find all pairs of integers $(x,y)$ such that $x^3+y^3 = (x+y)^2.$
%\end{prb}
%
%\begin{proof}
%$0=x^3+y^3-(x+y)^2=(x+y)(x^2-xy+y^2-x-y) \Leftrightarrow x+y=0$,
%i.e. $x=k$, $y=-k$ or $x^2-xy+y^2-x-y=0$, which is equivalent to
%$(x-y)^2+(x-1)^2+(y-1)^2=2.$
%
%We get $(x-1)^2\leq 1, \ (y-1)^2\leq 2$, which implies $x,y \in
%[0,2]$. We obtain $(x,y)=(0,1), \ (1,0), \ (1,2), \ (2,1), \ (2,2)$.
%\end{proof}
%
%\begin{prb}
%Find all triples of positive integers $(x,y,z)$ such that
%$$\left(1+\frac{1}{x} \right) \left( 1+\frac{1}{y} \right) \left( 1+\frac{1}{z}\right)=2. $$
%\end{prb}
%
%\begin{proof}
%We may assume that $x\geq y\geq z$. Hence $\left(
%1+\cfrac{1}{z}\right)^3 \geq 2$, which implies $z\leq 3.$
%
%If $z=1$, we obtain $\left(1+\cfrac{1}{x} \right)
%\left(1+\cfrac{1}{y} \right)=1$, not possible.
%
%If $z=2$, we obtain $\left(1+\cfrac{1}{x} \right)
%\left(1+\cfrac{1}{y} \right)=\cfrac{4}{3}$, which implies
%$\left(1+\cfrac{1}{y} \right)\geq \cfrac{4}{3} \Rightarrow y<7$.
%From $1+\cfrac{1}{x}>1$ we get $y>3$. Hence $y=4,5,6 \Rightarrow
%x=15,9,7.$
%
%If $z=3$ we get $\left(1+\cfrac{1}{x} \right) \left(1+\cfrac{1}{y}
%\right)=\cfrac{3}{2} \Rightarrow y<5$ and $y\geq z=3$, which implies
%$y=3,4 \Rightarrow x=8,5.$
%\end{proof}
%
%\begin{prb}
%Find all positive integers $(x,y,z)$ for which
%$$x^2+y^2+z^2+2xy+2x(z-1)+2y(z+1) $$
%is a perfect square.
%\end{prb}
%
%\begin{proof}
%Denote the given expression by $w^2$. Then it is easy to check that
%$$(x+y+z-1)^2<w^2<(x+y+z+1)^2. $$
%Hence
%$$x^2+y^2+z^2+2xy+2x(z-1)+2y(z+1)=(x+y+z)^2. $$
%We get $x=y$, hence the solutions are $(m,m,n)$.
%\end{proof}
%
%\begin{prb}
%Find all integers $(x,y)$ such that $(x^2-y^2)^2=1+16y.$
%\end{prb}
%
%\begin{proof}
%We must have $y\geq 0$. Then $|x|\neq |y|$ and we have $|x|\geq
%|y|+1$ or $|x|\leq |y|-1$. In either case, $(x^2-y^2)^2\geq
%(2y-1)^2$, so $16y+1\geq (2y-1)^2$, which implies $y\leq 5$. Now we
%find $(x,y)=(\pm 1,0), \ (\pm 4,3), \ (\pm 4,5)$.
%\end{proof}
%
%\begin{prb}
%Let $a$ and $b$ be positive integers such that $ab+1$ divides
%$a^2+b^2$. Show that $\cfrac{a^2+b^2}{ab+1}$ is a perfect square.
%\end{prb}
%
%\begin{proof}
%Let $(a,b)$ satisfy the condition. Then $(a,b)$ is a solution of the
%equation
%\begin{equation}
%a^2-kab+b^2=k \label{eq3}
%\end{equation}
%If $a=b$ from $(2-k)a^2=k>0$ we deduce $k=1$. Suppose that $a>b>0$
%and let $(a,b)$ be a solution of (\ref{eq3}) with $b$ minimal. It is
%easy to see that $(b,kb-a)$ is also a solution. If $kb=a$, we get
%$b^2=k$, i.e. $k$ is a perfect square. We have that $kb-a>0$, since
%otherwise
%$$(kb-a)^2-k(kb-a)b+b^2>k(a-kb)b>k. $$
%Finally, we shall prove that  $kb-a<b$. Indeed,
%$$kb-a<b \Leftrightarrow k<\frac{a+b}{b} \Leftrightarrow \frac{a^2+b^2}{ab+1}<\frac{a}{b}+1$$
%which follows from
%$$\frac{a^2+b^2}{ab+1}<\frac{a^2+ab}{ab+1}<\frac{a^2+ab}{ab}=\frac{a}{b}+1. $$
%Hence $(b,kb-a)$ is a solution which contradicts the minimality.
%\end{proof}
%
%\center{\textbf{The Modular Arithmetic Method}}
%\begin{prb}
%Show that the equation
%$$(x+1)^2 + (x+2)^2 + \dots + (x+2001)^2 = y^2$$
%is not solvable in integers.
%\end{prb}
%
%\begin{proof}
%Let $x=z-2001$. Then
%$$(z-1000)^2+\dots +(z+1000)^2=y^2 \Leftrightarrow 2001z^2+2(1^2+2^2+\dots +1000^2)=y^2 \Leftrightarrow$$ $$2001z^2+2\frac{1000 \cdot 1001 \cdot 2001}{6}=y^2 \Leftrightarrow
%2001z^2+1000 \cdot 1001 \cdot 667 = y^2. $$ Hence $y^2 \equiv 2
%\pmod{3}$, a contradiction.
%\end{proof}
%
%\begin{prb}
%Find all primes $p,q$ such that $$p^3-q^5=(p+q)^2 .$$
%\end{prb}
%
%\begin{proof}
%Suppose that $p,q \neq 3$. Then $p,q \equiv 1,2 \pmod{3}$. If
%$p\equiv q \pmod{3}$, then $p^3-q^5 \equiv 0 \pmod{3}$, but $(p+q)^2
%\not\equiv 0 \pmod{3} $.
%
%If $p \not\equiv q \pmod{3}$, then $(p+q)^2 \equiv 0 \pmod{3}$, but
%$p^3-q^5\not\equiv 0\pmod{3}$.
%
%If $p=3$, we get $q^5<27$, a contradiction.
%
%If $q=3$, we get $p^3-243=(p+3)^2$, which implies $p=7$.
%\end{proof}
%
%\begin{prb}
%Prove that the equation $x^5-y^2=4$ has no integer solutions.
%\end{prb}
%
%\begin{proof}
%We consider the equation modulo 11. Since $(x^5)^2=x^{10}\equiv 0,1
%\pmod{11}$, we get $x^5\equiv 0,\pm 1 \pmod{11}$, which implies
%$x^5-4\equiv 6,7,8 \pmod{11}$. However $y^2 \equiv 0,1,3,4,5,9
%\pmod{11}.$
%\end{proof}
%
%\begin{prb}
%Prove that the equation $x^{10}+y^{10}-z^{10}=2010$ has no integer
%solutions.
%\end{prb}
%
%\begin{proof}
%$x^{10}=11m+a$, $y^{10}=11n+b$, $z^{10}=11k+c$, where $a,b,c=0,1.$
%Then
%$$11(m+n-k)+a+b-c=182\cdot 11+8. $$
%Hence $11\mid 8-a-b+c$, a contradiction.
%\end{proof}
%
%\begin{prb}
%Prove that the equation $4xy-x-y=z^2$ has no solution in positive
%integers.
%\end{prb}
%
%\begin{proof}
%$$4xy-x-y=z^2 \Leftrightarrow 16xy-4x-4y=4z^2 \Leftrightarrow (4x-1)(4y-1)=(2z)^2+1.$$
%The number $4x-1$ has a prime divisor $p$ of the form $4k-1$. Since
%$p\mid (2z)^2+1$, $p\mid 1$, a contradiction.
%\end{proof}
%
%\newpage
%\begin{center}
%\textbf{Problems}
%\end{center}
%\begin{enumerate}
%\item Solve the equations:
%\begin{enumerate}
%\item $6x+10y+15z=7$.
%\item $2x+3y+4z+5t=6.$
%\end{enumerate}
%
%\item Solve the equation
%$$x^2+y^2 = 1997 (x-y) $$
%in positive integers.
%\item Find all Pythagorean triangles whose area is equal to the
%perimeter.
%
%\item Solve the following equations in positive integers:
%\begin{enumerate}
%\item $3(xy+yz+zx)=4xyz;$
%\item $xy+yz+zx-xyz=2.$
%\end{enumerate}
%\item Solve the equations:
%\begin{enumerate}
%\item $(x+y)^2+3x+y+1=z^2$ in positive integers
%\item $(x+1)^2-(x-1)^4=y^3$ in integers.
%\end{enumerate}
%\item Solve the equations:
%\begin{enumerate}
%\item $x^2y+y^2z+z^2x=3xyz$ in positive integers
%\item $(x^2-y^2)^2=1+16y$ in integers.
%\end{enumerate}
%\item Prove that the equation
%$$(x+1)^2+\dots + (x+99)^2=y^z $$
%is not solvable in positive integers, where $z\geq 2.$
%
%\item Find all positive integers $(x,y)$ with $$x^2-y!=2001. $$
%\item Find all positive integers $(x,y)$ such that
%$$3^x-2^y=7. $$
%\item Prove that the equation $x^3+y^4=7$ has no integer solutions.
%\end{enumerate}


%\section{Linear Diophantine equations}
%A linear Diophantine equation is in the form
%\[
%a_1x_1+a_2x_2+\cdots +a_nx_n=N.
%\]
%This has a solution in integers iff $d=\gcd(a_1,\ldots, a_n)\mid N$.
%
%\begin{thm}[Sylvester, 1886]
%Let $a,b\in \N$ with $\gcd(a,b)=1$. The largest $n$ for which
%\[
%ax+by=n
%\]
%has no solution for $x,y\in \N$ is 
%\[n=ab-a-b.\]
%\end{thm}
%\begin{proof}
%Suppose $ax+by=ab-a-b$. Adding $a+b$ to both sides gives
%\[
%a(x+1)+b(y+1)=ab.
%\] 
%We find that $b\mid a(x+1)$ since it divides the other two terms; since $\gcd(a,b)=1$ we get $b\mid x+1$. Similarly, $a\mid y+1$. Hence $x\ge b-1$ and $y\ge a-1$. But then
%\[
%a(x+1)+b(y+1)=2ab\ne ab,
%\]
%contradiction.
%
%Now we show that for any $n>ab-a-b$ there is a solution. We can find a solution $x_0,y_0$ by B\'ezout. Now we look for a solution of the form
%\begin{align*}
%x&=x_0+bt\\
%y&=y_0-at.
%\end{align*}
%By division, we can choose $t$ such that $0\le y\le a-1$. Then 
%\[
%a(x_0+bt)=N-b(y_0-at)\ge N-b(a-1)>ab-a-b-ab+b=-a.
%\]
%and hence $x_0+bt\ge 0$, as needed.
%\end{proof}
%\section{Pythagorean equation}
%\begin{thm}[Pythagorean triples]
%All integer solutions to $x^2+y^2=z^2$ are given by
%\begin{align*}
%x&=k(m^2-n^2),&y&=2kmn,&z&=k(m^2+n^2)\\
%x&=2kmn,&y&=k(m^2-n^2),&z&=k(m^2+n^2).
%\end{align*}
%\end{thm}
%\begin{proof}
%First note that if $k$ divides 2 of $x,y,z$, then it must divide the third, and we can divide $x,y,z$ by $k$. So we can assume $x,y,z$ are pairwise relatively prime.
%
%Note $x,y$ cannot both be odd; else the LHS would be $2$ modulo 4. Hence one of $x,y$ is odd, the other is even, and $z$ is even. Assume without loss of generality that
%$x$ is even. Then $\frac{z-x}2,\frac{z+x}2$ are in integers, and
%we can rewrite the equation in the form
%\begin{align*}
%y^2&=z^2-x^2\\
%\pf y2^2&=\pf{z-x}{2}\pf{z+x}{2}.
%\end{align*}
%Now $\gcd\pa{\frac{z-x}{2},\frac{z+x}{2}}=1$ since if $p\mid \frac{x-y}{2}$ and $p\mid \frac{z+x}{2}$ then $p\mid z,x$. Hence both $\frac{z-x}{2}$ and $\frac{z+x}2$ must be perfect squares. Let $\frac{z+x}{2}=m^2$ and $\frac{z-x}2=n^2$. Solve for $x,y,z$.
%\end{proof}
%Exercise: Find all integer solutions to $\rc{x^2}+\rc{y^2}=\rc{z^2}$.
%
%Rewrite as $y^2z^2+x^2z^2=x^2y^2$.
%\section{Root flipping}
%\begin{ex}[IMO 1989/6]
%Let $a$ and $b$ be positive integers such that
%\[
%ab+1\mid a^2+b^2.
%\]
%Prove that $\frac{a^2+b^2}{ab+1}$ is a perfect square.
%\end{ex}
%\begin{proof}
%Suppose $(a,b)$ satisfies $\frac{a^2+b^2}{ab+1}=k$. Then 
%\[
%P(x):=x^2-kxb+b^2-k=0 \text{ when }x=a.
%\]
%Treat this as a polynomial in $x$, we find that the other root is $kb-a$. Hence $(b,kb-a)$ is also a solution.
%
%If $a=b$ then $a^2(2-k)=k$ so $k=1$.
%
%Now suppose $a\ne b$; without loss of generality $a>b>0$. Consider the pair $(a,b)$ where $a>b>0$ and $b$ is minimal. We claim $kb-a>0$. Indeed, if $kb-a<0$ then $k=a(a-nb)+b^2>a$.... contradiction. Next note $b>nk-a$. 
%%k=a(a-kb)+b^2
%%a+b)/b>(a^2+b^2)/(ab+1)
%%a+ab^2+b>b^3
%%ab^2>b^3
%\end{proof}
%Exercise: $4ab-1\mid (4a^2-1)^2$. Prove that $a=b$.
%
%$4ab\mid (a-b)^2$. Let $k=\frac{(a-b)^2}{4ab-1}$. Take $a+b$ minimal.
%%%%%%%%
%We start with some problems in the field of
%Diophantine equations.
%\begin{prb}
%Prove that the equation $$x^2+2y^2+98z^2 =
%\underbrace{77\cdots7}_{2005}$$ has no integer solutions.
%\begin{proof}
%Suppose that $(x,y,z)$ is a solution to the given equation.  Then we
%must have $7\mid x^2+2y^2$.  Analysis of the possible remainders of
%squares modulo 7 reveals that we must have $7 \mid x, 7\mid y$.  As
%$7^2 \mid 98$, this means that we must have $$7^2 \mid
%\underbrace{77\cdots7}_{2005} =
%\frac{7}{9}\left(10^{2005}-1\right).$$  This is equivalent to having
%$10^{2005}\equiv1\mod{7}$.  As $10^6\equiv1\mod{7}$ by Euler's
%theorem, and as $6 \mid 2004$, this is equivalent to
%$10^1\equiv1\mod{7}$, which is obviously false.  Hence there are no
%integer solutions to the given equation.
%\end{proof}
%\end{prb}
%\begin{prb}
%Solve in integers the equation $$2^a + 8b^2 - 3^c = 283.$$
%\begin{proof}
%It is obvious that we must have $a, c \ge 0$.  As $283 \equiv
%3\mod{8}$, and as $3^c \equiv 1,3\mod{8}$, we must have $0 \le a \le
%2$.
%
%If $a = 0$ then we have $2 \mid 3^c$, a contradiction.  If $a = 1$,
%then we have $8\mid 3^c+1$, a contradiction.  Hence $a = 2$.  Thus
%$8b^2 - 3^c = 279$.  Hence $3 \mid b$; write $b = 3d$.  Then $8d^2 -
%3^{c-2} = 31$.  If $c \ge 3$, then $3 \mid d^2+1$, a contradiction.
%We must have $c \ge 2$, hence $c = 2$ or $c = 3$.  If we have $c =
%2$, then we have $d = \pm 2$.  If we have $c = 3$, then we have
%$8d^2 = 34$, a contradiction.  Hence our only solutions are $(a,b,c)
%= (2,-6,2), (2,6,2)$.
%\end{proof}
%\end{prb}
%\begin{prb}
%Prove that if $a,b,c$ are integers such that
%$$\frac{a(a-b)+b(b-c)+c(c-a)}{2}$$ is a perfect square, then $a = b
%= c$.
%\begin{proof}
%Write $(a(a-b)+b(b-c)+c(c-a))/2 = d^2$.  Let $x = a-b, y = b-c, z =
%c-a$.  Then we may compute $x+y+z = 0$ and $x^2+y^2+z^2 = 4d^2$.
%Then we may write $x = 2x_1, y = 2y_1, z = 2z_1$, giving
%$x_1^2+y_1^2+z_1^2 = d^2$.  But $d^2$ may be either 0 or 1 modulo 4,
%hence at least two of $x,y,z$ are even.  But if at least two of
%$x,y,z$ are even, then all three must be even, and we may write $x_1
%= 2x_2, y_1 = 2x_2, z_1 = 2z_2, d = 2d_1$, with $x_2^2 + y_2^2 +
%z_2^2 = d_1^2$.  We may proceed as above, obtaining that $2^n$
%divides $x,y,z$ for any natural number $n$.  Hence we must have $x =
%y = z = d = 0$ is the only possible solution.
%\end{proof}
%\end{prb}
%\begin{prb}
%Show that if $a,b$ are positive integers such that $4ab-1 \mid
%(4a^2-1)^2$, then $a = b$.
%\begin{proof}
%If $4ab-1\mid (4a^2-1)^2$, then we have $$4ab-1 \mid
%2b^2(16a^4-8a^2+1) = (2a^2-1)(16a^2b^2-1)+2a^2+2b^2+1.$$  Hence
%$4ab-1 \mid 2a^2+2b^2 - 1$.  We may also note that $4ab-1\mid
%(a-b)^2$.
%
%Assume for a contradiction that $a \neq b$; then we have $2(a^2+b^2)
%- 1 > 4ab - 1$.  As $2a^2+2b^2-1$ is odd, there must exist some
%natural number $t$ with $2a^2+2b^2-1 = (2t+1)(4ab-1)$.  Hence $a^2 -
%(2t+1)(2ab)+b^2 + t = 0$.  Consider this equation as a quadratic in
%$a$.  If $b > 0$ is an integer, then the sum of the two roots of the
%equation is an integer, and their product is an integer; hence if
%one of the roots is an integer, both are integers.  The same is true
%if we consider the equation as a quadratic in $b$.
%
%Consider all possible solution pairs $(a,b)$ to this equation, and
%choose a pair such that $a$ is minimal.  We may do this as $a > 0$
%is an integer.  By symmetry, $(a,b)$ is a solution if and only if
%$(b,a)$ is a solution; hence $b \ge a$.  Write $c = b-a$; then $c$
%takes on two (not necessarily distinct) non-negative integer values
%over the solution pairs for our fixed $a$.  Then we may manipulate
%our equation to obtain $c^2 - (4at)c - t(4a^2-1) = 0$.  As the
%product of the roots is thus $-t(4a^2-1)$ and must be non-negative,
%we have $-t(4a^2-1) \ge 0$.  As $4a^2-1 \ge 0$, we have $-t \ge 0$,
%so $t \le 0$.  Hence $t$ is not a natural number.  Contradiction;
%hence $a = b$ as desired.
%\end{proof}
%\end{prb}
%\begin{prb}
%Find the least positive integer $a$ such that the system
%\begin{align*}x+y+z&=a\\x^3+y^3+z^2&=a\end{align*} has no integer
%solutions.
%\begin{proof}
%If $a = 1,2,3$, then we have solutions $(x,y,z) = (1,0,0), (1,1,0),
%(1,1,1)$ as solutions.  Now suppose that there exist $x,y,z$
%satisfying the system when $a = 4$.  Then we have $4-z^2 = (x^3+y^3)
%= (x+y)(x^2+y^2-xy) = (4-z)(x^2-xy+y^2)$ giving that $(4-z^2)(4-z) =
%4+z+12/(z-4)$ is an integer.  Hence $z-4 = \pm1, \pm2, \pm3, \pm4,
%\pm6, \pm12$.  But we have $(x+y)^2-3xy = (4-z^2)/(4-z)$, so $3xy =
%(4-z)^2-(4-z^2)/(4-z)$, so therefore we have the system
%\begin{align*}x+y&=4-z\\xy&=\frac{(4-z)^3+z^2-4}{3(4-z)}\end{align*}
%for various values of our parameter $z$.  Then we may check all
%possible such systems and verify that they have no integer
%solutions.
%\end{proof}
%\end{prb}
%\begin{prb}
%Solve in integers the equation $$x^3+10x-1 = y^3+6y^2.$$
%\begin{proof}
%It is clear that $x$ and $y$ have opposite parity.  Then $k = x-y$
%is an odd number satisfying $(k+y)^3+10(k+y)-1 = y^3+6y^2.$  This
%becomes $$(3k-6)y^2+(3k^2+10)y+(k^3+10k-1) = 0.$$ The discriminant
%of this equation is given by $$D = -3k^4+24k^3-60k^2+252k+76$$ and
%must be a perfect square.  Since $D = -3k^2(k^2-8k+20)+252k+76$, and
%since $k^2-8k+20 = (k-4)^2+4$ is always positive, we have $D < 0$
%for $k \le -1$.  On the other hand, we have $D =
%3k^3(8-k)+2(38-k^2)+2k(126-29k)$.  Hence $D < 0$ for $k \ge 8$.  It
%follows that the only valid $k$ which give $D \ge 0$ are $k =
%1,3,5,7$.  Of these, $D$ is a perfect square only when $k = 1$ and
%$D = 289$ and when $k = 5$ and $D = 961$.  From here we find that
%the only possible solutions are $(x,y) = (6,5)$ and $(x,y) =
%(2,-3)$.
%\end{proof}
%\end{prb}
%\begin{prb}
%The positive integers $M$ and $n$ are such that $M$ is divisible by
%all positive integers from $1$ to $n$ inclusive, but that $M$ is not
%divisible by $n+1$, $n+2$, or $n+3$.  Find all possible values of
%$n$.
%\begin{proof}
%We claim that $n+1, n+2, n+3$ are prime powers.  Assume that this is
%not the case; then one of them has the form $ab$, where $a, b \ge 2$
%and $\gcd{(a,b)} = 1$.  If we have $a,b \le n$, then we have $ab
%\mid M$.  Suppose without loss of generality that $a > b$; then we
%have $a \ge n+1$, hence $ab-a = a(b-1) \le 2$.  As $a, b \ge 2$, we
%have $a(b-1) \ge 2$.  Hence $a = b = 2$, a contradiction; it follows
%that $n+1, n+2, n+3$ are prime powers.  As these are three
%consecutive integers, at least one of them is divisible by 2.  If
%two of $n+1,n+2,n+3$ are divisible by 2, then we must have $n+1 = 2$
%and $n+3 = 4$, giving $n = 1$.  Otherwise we have $n + 2 = 2^x$.
%Similarly, one of $n+1, n+3$ must then be divisible by 3, so we have
%$2^x = 3^y \pm 1$.  We split into two cases.
%
%Case 1 is the case where $2^x = 3^y + 1$ holds.  Then $2^x \equiv
%1\mod{3}$, so $x$ is even.  Writing $x = 2z$, we have
%$(2^z-1)(2^z+1) = 3^y$.  Then $2^z-1 = 3^a$ and $2^z+1 = 3^b$.
%Hence $a = 0, b = 1$, so $y = a+b = 1$.  Hence $x = 2$.  These
%values for $x$ and $y$ give $n = 1$, $n = 2$ as possible solutions.
%
%Case 2 is the case where $2^x = 3^y - 1$ holds.  If $x = 1$, then we
%have $y = 1$, obtaining $n = 1, 2$.  If $x \ge 2$, then we have $3^y
%\equiv 1\mod{4}$, so $y = 2z$.  Hence $2^x = (3^y+1)(3^y-1)$, so we
%get $x = 3, y = 2$.  Hence we get $n = 6$ or $n = 7$.  If $n = 7$
%then we have $n+3 = 10$, which is not a prime power.  Hence our only
%possible values of $n$ are $n = 1,2,6$.
%\end{proof}
%\end{prb}
%\begin{prb}
%Let $n$ be a positive integer, and denote $$u_n =
%\underbrace{11\cdots1}_{n}.$$  Furthermore, denote by $s(n)$ the sum
%of the digits of $n$.  Show that there does not exist a positive
%integer $m$ with $u_n \mid m$ and $s(m) < n$.
%\begin{proof}
%Assume the contrary; denote by $m$ the least integer with $u_n \mid
%m$ and $s(m) < n$.  We must have $m > 10^n$; write $m = 10^a+b$ with
%$b < 10^n$.  Then we have $m = (10^n-1)a+(a+b) = 9u_n+(a+b)$.  Hence
%$u_n \mid a+b$.  But then we have $s(a+b) \le s(a)+s(b) = s(m) < n$.
%As $a+b$ satisfies the conditions of the problem, and as $a+b < m$,
%this contradicts the minimality of $m$.  Hence no such $m$ exists.
%\end{proof}
%\end{prb}
%\begin{prb}
%Find all triples $(x,y,z)$ of positive integers such that
%$$\sqrt{\frac{2005}{x+y}}+\sqrt{\frac{2005}{y+z}}+\sqrt{\frac{2005}{z+x}}$$
%is a positive integer.
%\begin{proof}
%We start our proof with the following lemma.
%\begin{lem}
%If $p,q,r$ are rational numbers with $\sqrt{p}+\sqrt{q}+\sqrt{r}$
%rational, then $\sqrt{p}, \sqrt{q}, \sqrt{r}$ are rational.
%\begin{proof}
%Write $s = \sqrt{p}+\sqrt{q}+\sqrt{r}$, and for convenience set $t =
%s^2+r-p-q$.  We know that $s,t$ are rational.  Then we may write
%\begin{align*}
%(\sqrt{p}+\sqrt{q})^2&= (s-\sqrt{r})^2\\
%2\sqrt{pq}&=s^2+r-p-q-2s\sqrt{r} = t-2s\sqrt{r}\\
%4pq&= t^2+4s^2r-4st\sqrt{r}
%\end{align*}
%Hence $\sqrt{r}$ is rational.  In a similar manner we may show that
%$\sqrt{p}$, $\sqrt{q}$ are rational as desired.  This completes the
%proof of our lemma.
%\end{proof}
%\end{lem}
%Now let $x,y,z$ be positive integers with the desired property.
%Then let $a,b$ be positive integers with $\gcd{(a,b)} = 1$ such that
%$\sqrt{2005/(x+y)} = a/b$.  Hence $2005b^2 = (x+y)a^2$, so $a^2 \mid
%2005b^2$.  As $\gcd{(a^2,b^2)} = \gcd{(a,b)} = 1$, we have thus $a^2
%\mid 2005$.  But $2005 = 5\cdot401$, hence $a = 1$, so $x+y =
%2005b^2$.  In an analogous manner, there exist positive integers
%$c,d$ with $y+z = 2005c^2$ and $z+x = 2005d^2$.  Hence we have
%$$\sqrt{\frac{2005}{x+y}}+\sqrt{\frac{2005}{y+z}}+\sqrt{\frac{2005}{z+x}}
%= \frac{1}{b}+\frac{1}{c}+\frac{1}{d}.$$  We have $1\le
%\frac{1}{b}+\frac{1}{c}+\frac{1}{d} \le 3$ and
%$\frac{1}{b}+\frac{1}{c}+\frac{1}{d}$ an integer.  If
%$\frac{1}{b}+\frac{1}{c}+\frac{1}{d} = 3$, then we have $b = c = d =
%1$.  But the system $x+y = y+z = z+x = 2005$ gives $2(x+y+z) =
%3\cdot2005$, which has no solutions in integers $x,y,z$.  If
%$\frac{1}{b}+\frac{1}{c}+\frac{1}{d} = 2$, then we have $(b,c,d)$
%some permutation of $1,2,2$.  Without loss of generality, let $b =
%1$; then we have $x+y = 2005$, $y + z =2005\cdot4$, $z + x =
%2005\cdot4$.  We may write $2x = (x+y)-(y+z)+(z+x) = 2005$, hence
%$x$ is not an integer.  Contradiction; hence
%$\frac{1}{b}+\frac{1}{c}+\frac{1}{d} \neq 2$.
%
%Hence we must have $\frac{1}{b}+\frac{1}{c}+\frac{1}{d} = 1$.
%Assume without loss of generality that $b \ge c \ge d \ge 1$; then
%$d = 2$ or $d = 3$.  If $d = 3$ then we have $b = c = d = 3$.  But
%the system $x+y = y+z = z+x = 2005\cdot9$ gives $2(x+y+z) =
%27\cdot2005$, which has no solutions in integers $x,y,z$.  If $d =
%2$, then we have $b = c = 4$ or $b = 6, c = 3$.  If $b = 6, c = 3$,
%then we have the system of equations $x+y = 36\cdot2005, y+z =
%9\cdot 2005, z+x = 4\cdot2005$.  We write $2x = (x+y)-(y+z)+(z+x) =
%31\cdot2005$, hence $x$ is not an integer.  Contradiction; hence it
%remains only to check the case $b = c = 4$.  In this case we have $x
%+ y = 16\cdot 2005, y+z = 16\cdot 2005, z+x = 4\cdot 2005$.  Then we
%have $x = ((x+y)-(y+z)+(z+x))/2 = 2\cdot2005$.  Hence $y = 14\cdot
%2005$ and $z = 2\cdot 2005$.  Thus our only solutions are the
%permutations of $(14\cdot2005, 2\cdot2005, 2\cdot2005)$.
%\end{proof}
%\end{prb}
%\begin{prb}
%Let $a,b,c$ be rational numbers such that $a+b+c$ and $a^2+b^2+c^2$
%are equal integers.  Prove that the number $abc$ can be written as a
%ratio of a perfect cube and a perfect square that are coprime.
%\begin{proof}
%Set $t = a+b+c = a^2+b^2+c^2$.  Then the RMS-AM inequality gives
%$$\frac{(a^2+b^2+c^2)}{3} \ge \left(\frac{a+b+c}{3}\right)^2,$$
%hence $3t\ge t^2$, so $t \le 3$.  If $t = 0$ then we have $a=b=c=0$,
%and if $t = 3$ we have equality in RMS-HM, hence $a = b = c = 1$.
%Both of these have $abc$ satisfying the desired property.
%
%Now suppose that $t = 1$.  Let $d$ be the product of the
%denominators of $|a|,|b|,|c|$.  Then $x = ad, y = bd, z = cd$ are
%integers with $x+y+z = d$ and $x^2+y^2+z^2 = d^2$.  Without loss of
%generality, let $z \ge 0$.  Then we have $(x+y+z)^2 = x^2+y^2+z^2$
%implies that $xy+yz+zx = 0$, which means that $(x+z)(y+z) = z^2$.
%Then $x+z$ and $y+z$ have the same sign, so there exist $p,q,r$ with
%$p,q \ge 0$, $\gcd{(p,q)} = 1$, $x+z = rp^2$, $y+z = rq^2$, and $z =
%|r|pq$.  Hence $d = x+y+z = r(p^2+q^2)-|r|pq > 0$, so $r > 0$.  We
%may write thus $x = rp(p-q)$, $y = rq(q-p)$, $z = rpq$.  Hence we
%may write $$a = \frac{x}{d} = \frac{p(p-q)}{p^2+q^2-pq}, \qquad
%b=\frac{y}{d} = \frac{q(q-p)}{p^2+q^2-pq}, \qquad c =
%\frac{pq}{p^2+q^2-pq}.$$  Thus $$abc =
%\frac{\left(pq(p-q)\right)^2}{\left(pq-p^2-q^2\right)^2}.$$  We now
%need only show that $\gcd{(pq(p-q),p^2+q^2-pq)} = 1$.  Suppose
%otherwise; let $s$ be a prime dividing both $pq(p-q)$ and
%$p^2+q^2-pq$.  If $s \mid p$, then $s \mid p^2-pq$, hence $s \mid
%q^2$, so $s \mid q$.  This is impossible; hence $s \nmid p,q$.  But
%$s \mid pq(p-q)$, hence $s \mid p-q$.  Thus $s \mid
%(p^2+q^2-pq)-(p-q)^2 = pq$, a contradiction.  Our desired conclusion
%follows.
%
%The case $t = 2$ is similar and is left to the reader.
%\end{proof}
%\end{prb}
%
%\newpage
%\begin{center}
%\textbf{Problems}
%\end{center}
%
%\begin{enumerate}
%\item Find the least positive integer $n$ such that $2^{2000}$ divides $2003^n-1$.
%\item Prove that for any integer $a \ge 4$ there exist infinitely many square free positive integers $n$ that divide $a^n-1$.
%\item Find all positive integers $x,y,z,t$ such that $3^x5^y+7^z = 2^t$.
%\item Let $p$ be a prime such that $p^2$ divides $2^{p-1}-1$.  Prove that for any positive integer $n$ the number $(p-1)(p!+2^n)$ has at least three distinct prime divisors.
%\item Find all primes $p$ and $q$ such that $pq$ divides the product
%$(5^p-2^p)(5^q-2^q)$.
%\item Let $a,b,c$ be positive integers such that $ab$ divides $c(c^2-c+1)$ and $a+b$ divides $c^2+1$.  Show that the sets $\{a,b\}$ and $\{c,c^2-c+1\}$ coincide.
%\end{enumerate}