\chapter{Roth's Theorem by Fourier analysis}

In this chapter we prove Roth's Theorem using Fourier analysis. (In the preliminary version of this chapter, the reader will be referred to Ben Green's notes for specific parts.)

The main ideas are the following. 
\begin{enumerate}
\item
Instead of studying subsets of $[N]$, we study functions on $[N]$ because the space of functions forms a vector space. We associate with each $A\subeq [N]$,
\[
1_A\in [N]^*.
\]
The advantage of writing as functions is that we can break them up into a ``pseudorandom" part and an ``orderly" part more easily: write a function $f\in [N]^*$ as $f=f_1+f_2$ and deal with the two parts separately. The size of a subset corresponds very well to the $L^2$ norm of a function.
\item
Develop Fourier analysis on finite abelian groups. Why is Fourier analysis useful? It detects linear structure and the condition for three numbers to be in arithmetic progression is given by a single linear equation. The number of 3 term AP's will be given by a trilinear form evaluated at $1_A$,
\[
T(1_A,1_A,1_A).
\]
\item (Small Fourier coefficient means function looks random)
We can decompose $1_A$ as a constant function $\al 1_[N]$, and a noisy part, $f_A=1_A-\al 1_[N]$. (We take out the ``orderliness" that we know is guaranteed to be there.) 
The trilinear form above then decomposes as $\al^3$ plus 7 terms each of which has $f_A$ in it. 
We consider this dichotomy between order and randomness.
\begin{enumerate}
\item
Orderly: $f_A$ has a large Fourier coefficient (We've already removed the obvious large Fourier coefficient).
\item
Random: $f_A$ has no large Fourier coefficient.
\end{enumerate}
We show that ``one random function makes $T_A$ small," so ``in the second part, the absence of a large Fourier coefficient makes all 7 terms small, and we have the density is approximately what we want it to be.
\item A large Fourier coefficient means the function concentrated on subprogression. (We'll give more intuition in the relevant section.)
\item Use the ``density increment argument": We can iterate the last step, but note that the density can never be greater than 1, so we must stop after a calculable number of steps. If there is no 3-AP, then we must have ended in an impasse, and $N$ cannot be too large. (Be more precise here.)
\end{enumerate}
\section{Intuition}

\prbox{
Let $A$ be a subset of $[N]$. (In the below, an approximate answer is good enough.)
\begin{enumerate}
\item
What is the total number of 3-AP's in $[N]$?
\item
Suppose $A$ were randomly chosen: each element has probability $\al$ of being in $A$, independently. What is the expected number of 3-term AP's in $A$ (including the trivial ones, and accounting for increasing/decreasing order)? What is the ratio between this and the answer to (1)?
\item 
Suppose $A$ were arbitrary. Does $A$ have ``close" to the number of AP's you found in item 1?
\end{enumerate}
}
\vskip0.15in
{\it Solution.}
\begin{enumerate}
\item
%{\it How many 3-AP's are there in $[N]$?} 
Given the first term $i$, there are approximately $\fc{N}2$ possible common differences (including negative ones), so there are $\fc{N^2}{2}$ 3-AP's.
\item Except for the constant 3-AP's, each 3-AP has probability $\al^3$ of being contained in $[N]$. By linearity of expectation, the expected number of 3-AP's is
$
\al^3\fc{N^2}{2}.
$
\item No. Take $A=[\al N]$. Then the number of AP's is, by (1), approximately $\al^2\fc{N^2}{2}$.
\end{enumerate}
We see that if the set $A$ is very ordered as in (3), then it can contain significantly more 3-AP's. We expect that an arbitrary subset $A\subeq [N]$ of size $\al N$ 
satisfies one of the following.
\begin{enumerate}
\item
$A$ has close to the expected number of 3-AP's, $\al^3\fc {N^2}{2}$, or
\item
$A$ has {\it structure}.
\end{enumerate}
We'll have to quantify what we mean by ``structure" (large Fourier coefficient).

\section{Fourier analysis on finite abelian groups}
%
%Note that ``Fourier analysis" and ``representation theory" on finite abelian groups are basically the same thing, although they suggest different kinds of generalizations.\footnote{Being more specific? Is there ever a time when we care about representations in combinatorial number theory?}
%
%\subsection{Representation theory}
%[Inserted from my notes on representation theory of abelian groups]
%
%\begin{thm}[Structure Theorem for Abelian Groups]
%Let $G$ be a finite abelian group. Then there exist positive integers $m_1,\ldots, m_k$ so that
%\[G\cong \zmod{m_1}\times \cdots \times \zmod{m_k}.\]
%\end{thm}
%\begin{thm}\llabel{char}
%The group $G=\zmod{m_1}\times \cdots \times \zmod{m_k}$ has $|G|$ characters and each is given by an element $(r_1,\ldots, r_k)\in \zmod{m_1}\times \cdots \times \zmod{m_k}$:
%\[\chi_{r_1,\ldots, r_k}(n_1,\ldots, n_k)=\prod_{j=1}^k e^{\frac{2\pi ir_jn_j}{m_j}}.\]
%Moreover the set of characters $\widehat{G}$ form a multiplicative group isomorphic to $G$.\footnote{This is a noncanonical isomorphism.}
%\end{thm}
%\begin{proof}
%It is easy to check that $\chi=\chi_{r_1,\ldots, r_k}$ is a homomorphism. Let $e_j$ be the element in $G$ with 1 in the $j$th coordinate and 0's elsewhere. Since $\chi(e_j)^{m_j}=1$, we must have $\chi(e_j)=e^{\frac{2\pi ir_j}{m_j}}$ for some $r_j$. Each element of $G$ can be expressed as a combination of the $e_j$, so this shows all characters are in the above form.
%
%This shows that $(r_1,\ldots, r_k)\send \chi_{r_1,\ldots, r_k}$ is surjective and hence an isomorphism.
%\end{proof}
%\begin{cor}\llabel{numchar}
%Every finite abelian group $G$ has $|G|$ characters.
%\end{cor}
%%Let $\hat{G}$ denote the set of irreducible characters of
%\begin{thm}[Orthogonality relations]\llabel{orth}
%Let $G$ be a finite abelian group and $\chi_j$, $1\leq j \leq n$ be all characters of $G$. Then
%\begin{enumerate}
%\item (Row orthogonality) $\an{\chi_j,\chi_k}:=\displaystyle\rc{|G|}\sum_{g\in G} \chi_j(g)\overline{\chi_k(g)}=\begin{cases} 0,&j\neq k\\ 1,&j=k\end{cases}$.
%\item (Column orthogonality) $\displaystyle\sum_{j=1}^{n} \chi_j(g)\overline{\chi_j(h)}=\begin{cases} 0,&g\neq h\\ |G| ,&g=h\end{cases}$.
%\end{enumerate}
%\begin{proof}
%Write $G$ as $\zmod{m_1}\times \cdots \times \zmod{m_k}$. Let $(r_1,\ldots, r_k)$ and $(s_1,\ldots,s_k)$ be in $G$. Then
%\begin{align}
%\an{\chi_{r_1,\ldots, r_k},\chi_{s_1,\ldots, s_k}}
%&=\sum_{(p_1,\ldots, p_k)\in G}\prod_{j=1}^k e^{\frac{2\pi i(r_j-s_j)p_j} {m_j}}
%\llabel{yayy}
%\\
%&=\sum_{(p_1,\ldots, p_{k-1})\in G}\ba{\pa{\prod_{j=1}^{k-1} e^{\frac{2\pi i(r_j-s_j)p_j} {m_j}}}\sum_{p_k=0}^{m_k-1} e^{2\pi i(r_k-s_k)p_k}}.
%\llabel{insum}
%\end{align}
%If $(r_1,\ldots, r_k)=(s_1,\ldots, s_k)$ then~(\ref{yayy}) evaluates to $|G|$. Otherwise, we may assume without loss of generality that $r_k\neq s_k$; then the inner sum in~(\ref{insum}) evaluates to 0 by writing it as a geometric series.
%
%The proof for column orthogonality is similar.
%\end{proof}
%\end{thm}
%The most useful case of row orthogonality is when we set $\chi_k=\chi_0$:
%\begin{cor}\llabel{sum0}
%If $\chi$ is a character of $G$ and $\chi\neq \chi_0$ then
%\[\sum_{g\in G} \chi(g)=0.\]
%\end{cor}

%\subsection{Fourier analysis}
\begin{df}
Let $G=\Z/N\Z$ or $G=\Z$ and let $H=\rc N\Z/\Z$ or $H=\R/\Z$, respectively.

Define the \textbf{Fourier transform} on a function $G\to \C$ where $G=\Z/N\Z$ or $G=\Z$ by
\[
\hat{f}(\te)=\sum_{x\in G}f(x)e(-\te x), \qquad \te\in H
\]
where $e(t):=e^{2\pi i t}$.
\end{df}
%In the case $\Z/N\Z$, we think of $\te\in \rc N\Z/\Z$, and in the case $\Z$, we think of $\te\in \R/\Z$.
Note $\te$ is really parametrizing the dual space to $G$, the set of all homomorphism $G\to \C$.\footnote{More abstractly, the Fourier transform is defined on $G^*$, setting $\hat{f}(\chi)=\sum_{x\in G} f(x)\ol{\chi}(x)$. Here we've made the identifications $G^*\cong \rc{N}\Z/\Z$, $\R/\Z$, respectively.}
 
We have two options:
\begin{enumerate}
\item
Use Fourier analysis on $\Z/N\Z$: prove Roth's Theorem for $\Z/N\Z$ and use the fact this implies it for $\Z$.
\item
Use Fourier analysis on $\Z$ and prove Roth's Theorem directly.
\end{enumerate}
%(Several normalizations are sometimes used: for example, 

Here we will take the second approach. The first approach works with little change, other than the fact that the Fourier inversion formula is a sum instead of an integral.

Define 
\begin{align*}
\ve{f}=\ve{f}_{2}&=\sum_{n\in [N]}|f(n)|^2\\
%\ve{f}_{\iy}&=\sup_{n\in [N]}|f(n)|^2\\
\ve{\hat f}=\ve{\hat f}_2&=\int_0^1|\hat f(\te)|^2\,d\te\\
\ve{\hat f}_1&=\sup_{\te\in [0,1]}|\hat f(\te)|
\end{align*}
The $\ved{}_2$ norm controls the density, and the $\ved{}_{\iy}$ norm controls the largest coefficient.
\begin{pr}
\begin{enumerate}
\item
(Parseval's formula) We have $\ve{f}_{L^2(\Z)}=\ve{f}_{L^2(\R/\Z)}$, i.e., 
\[
\sum_{n\in [N]} |f(n)|^2=\int_{0}^1 |\hat f(\te)|^2\,d\te
\]
\item (Convolutions become addition under the Fourier transform)
For $f,g\in [N]^*$,
\[
\wh{f*g}=\hat{f}\ol{\hat{g}}.
\]
\end{enumerate}
\end{pr}
A note about the convolution: this is similar to convolutions of generating functions. Given a subset $A$, we could loading the characteristic function $1_A$ into the coefficients of a generating function, where instead of the familiar $x^n$, we have $\chi(n)=e(-n\te)$: the analogy is that $x^ax^b=x^{a+b}$ and $\chi(a)\chi(b)=\chi(a+b)$.

Thinking of the generating function analogy and property of the convolution, we come up with the following properties.
\begin{pr}[Additive quantities expressed in terms of Fourier coefficients]
\begin{enumerate}
\item 
We have
\[
\sum_{n\in \Z}
|\set{(a,b)\in A\times B}{a+b=n}|e(-\te n)|
=
(1_{A}*1_B)(n)
\]
\item 
We have
\[
\sum_{x,d}f(x)g(x+d)h(x+2d)
=
[f(x)*g(-2x)*h(x)](0)=\int_0^1 \hat f(\te)\hat g(-2\te)\hat h(\te)\,d\te
\]
so in particular
\[
|\set{(a,b,c)\in A\times B\times C}{a,b,c\text{ 3-AP}}|
=
1_A*1_{-B/2}*1_C(0)=\int_0^1 \hat1_A(\te)\hat 1_{B}(-2\te)\hat 1_C(\te)\,d\te.
\]
\end{enumerate}
\end{pr}
\begin{df}
Define the trilinear form
\[
T(f,g,h):=\int_0^1 \hat f(\te)\hat g(-2\te)\hat h(\te)\,d\te.
\]
\end{df}


\section{Dichotomy between order and randomness}

\begin{pr}[One pseudorandom function makes $T$ small]
Suppose that 
\begin{enumerate}
\item
for $i=1,2,3$, $\ve{f_i}^2_2\le \al N$ (the density of the $f_i$ are bounded)
\item
for {\it one }of $i=1,2,3$, $\ve{f_i}_{\iy}\le \eta N$ (the largest coefficient of $f_i$ is small).
\end{enumerate}
Then 
\[
T(f_1,f_2,f_3)\le \eta\al N^2,
\]
``$(f_1,f_2,f_3)$ does not concentrate on 3-AP's."
\end{pr}
\begin{proof}
We'll assume $\ve{f_3}_{1}\le \eta N$; the other cases are analogous. 
Using H\"older's inequality with $(2,2,\iy)$\footnote{This is just bounding $f_3$ by its $L^1$ norm and using Cauchy-Schwarz on the other terms},
\bal
T(f_1,f_2,f_3)&\stackrel{\text{Pr.}}{=}\int_0^1 \hat f_0(\te)\hat f_1(-2\te)\hat f_2(\te)\,d\te\\
&\le \ve{\hat f_1}_2 \ve{\hat f_2}_2\ve{\hat f_3}_{\iy}\\
&\le (\al N)^{\rc 2}(\al N)^{\rc 2} \eta N=\eta\al N^2.
\end{align*}
\end{proof}
How do we use this fact? We don't use it on $1_A$, which we {\it know} has a large Fourier coefficient, namely, the zeroth Fourier coefficient. Instead, we really take advantage of the fact we're dealing with functions, not just sets.

Note that dichotomies are by nature somewhat arbitrary: what matters is not the constant we use to divide them, but rather the criteria by which we are making the distinction: here, the criteria is {\it size of the largest Fourier coefficient }$\eta$.
\begin{thm}
Let $A\subeq [N]$ of density $\al$. Then either
\begin{enumerate}
\item
($A$ looks random) $A$ has at least $\al^3\fc{N^2}{2}-7\eta \al N^2-2\al N$ 3-AP's.
\item
($A$ has structure) $1_A$ has a Fourier coefficient $\ge \eta N$ (excluding the zeroth coefficient).
\end{enumerate}
\end{thm}
\begin{proof}
Suppose $1_A$ does not have a large Fourier coefficient.

Expand the trilinear form
\[
T(1_A,1_A,1_A)=T(\al 1_{[N]},\al 1_{[N]}, \al 1_{[N]})+\ub{T(\ldots f_A \ldots)}{\text{7 terms}}.
\]
The first is approximately $\al^3\fc{N^2}{2}$. The last 7 terms are at most $7\eta \al N$ by Lemma. Now subtract the constant AP's.
\end{proof}

\section{Large Fourier coefficient implies concentration}
We would like a large Fourier coefficient for $f_A$ to imply that the subset is structured in some sense and hence has 3-term AP's. (Blah blah), it's more natural to look for a subset on which $f_A$ is {\it denser}.

\begin{ex}
\begin{enumerate}
\item (A function that has only one Fourier mode) Let $N=3k+1$.  Consider the function
\[
f(n)=e\pa{\fc{N-1}{3}n}\in [N]^*.
\]
($N=3k$ would be more boring to analyze.)
It has only one Fourier coefficient, namely,
\[
\hat f\pf{p-1}{3}=1,\qquad \hat f(n)=0\text{ else}.
\]
In what sense if $f$ very orderly? Note that we can divide $[N]$ progressions on which $f$ is approximately the same, i.e., $e\pa{\fc{N-1}{3}n}$ points in approximately the same direction. For example, $\{3,6,9,\cdots\}$, as $f(3)=e\pf{p-1}{p}\approx 1$. We don't want to continue this too long though---the length should be $o(N)$, for instance $\sqrt N$. 

In general, for any function with large Fourier coefficient, we can use Dirichlet's Theorem to find $e(N\te)$ close to 1. We then split into AP's where $e(N\te)$ is approximately the same within each AP. Some inequalities will allow us to conclude that $f$ is large on one AP.
\item (An almost AP) Consider the set $\set{3j}{j\in [k]}\in [N]$. Then $\hat f(\xi)=\sum_{n=1}^{\fc{p-1}{3}} e^{-\fc{2\pi i}{p}3n\xi}$ is small when $\xi=\fc{j(p-1)}{3}$, $j=0,1,2$. So in this example, a set with many AP's gives a function with large Fourier coefficients.
\end{enumerate}
\end{ex}


