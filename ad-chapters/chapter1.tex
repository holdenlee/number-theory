\chapter{Local dynamics: Good reduction}
In order to study the dynamics of rational maps on $\Q$ or a global number field, we reduce it modulo various primes to rational maps on local fields, and then piece the information together to get information about our original system. 
\index{chordal metric}
\section{Nonarchimedean chordal metric}
Inspired by the chordal metric on the projective line $\Pj^1(\C)$
\[
\rho([x_1:y_1],[x_2:y_2])=\frac{|x_1y_2-x_2y_1|}{\sqrt{|x_1|^2+|y_1|^2}\sqrt{|x_2|^2+|y_2|^2}}
\]
giving $\Pj^1(\C)$ the topology of the Riemann sphere, we define for a nonarchimedean valuation $v$ the following:
\[
\rh_v([x_1:y_1],[x_2:y_2])=\frac{|x_1y_2-x_2y_1|_v}{\max(|x_1|_v,|y_1|_v)\max(|x_2|_v,|y_2|_v)}.
\]
It is clear that scaling the coordinates for the two points does not change this value. For convenience, we will often ``normalize" coordinates so that $x_1,y_1,x_2,y_2\in R$ and $\max(|x_1|_v,|y_1|_v)=\max(|x_2|_v,|y_2|_v)=1$ (i.e. at least one of $x_1,y_1$ and at least one of $x_2,y_2$ is a unit). Then the formula becomes
\[
\rh_v([x_1:y_1],[x_2:y_2])=|x_1y_2-x_2y_1|_v=\ab{\smatt{x_1}{x_2}{y_1}{y_2}}_v.
\]
In particular,
\[
\rh_v([x_1:y_1],[0:1])=|x_1|_v.
\]
\begin{pr}\label{chordal-metric-basic}
$\rh_v$ is a nonarchimedean metric satisfying $\rh_v(P_1,P_2)\le 1$ for any $P_1,P_2$.
\end{pr}
\begin{pr}\label{chordal-metric-invar}
The metric $\rh_v$ is invariant under fractional linear transformations. That is, letting 
\[
f(x,y)=\frac{ax+by}{cx+dy}, \quad \matt abcd\in \PGL_2(R),
\]
we have that
\[
\rh_v(f(P_1),f(P_2))=\rh_v(P_1,P_2).
\]
\end{pr}
Note: for convenience, we will sometimes write $f\in \PGL_2(R)$.
\begin{proof}
Normalize coordinates. Note that $[ax_i+by_i:cx_i+dy_i]$ are normalized coordinates for $f(P_i)$ because multiplying the coordinates by the inverse matrix of $\smatt abcd$ gives
\[
\matt d{-b}{-c}{a}\coltwo{ax_i+by_i}{cx_i+dy_i}=\coltwo{x_i}{y_i};
\]
since $\max(|x_i|_v,|y_i|_v)=1$ and $x_i,y_i$ are $R$-linear combinations of $ax_i+by_i$ and $cx_i+dy_i$, we must have $\max(|ax_i+by_i|_v)=\max(|cx_i+dy_i|_v)$.

Hence
\[
\rh_v(f(P_1),f(P_2))=\ab{\det\ba{\matt abcd\matt{x_1}{x_2}{y_1}{y_2}}}_v=\ab{\det\matt{x_1}{x_2}{y_1}{y_2}}_v=\rh_v(P_1,P_2).\qedhere
\]
\end{proof}
\begin{proof}[Proof of~\ref{chordal-metric-basic}(2)]
We may operate by linear fractional transformations on the points $P_1,P_2,P_3$ without changing the values on either side. Hence we make the following reductions.
\begin{enumerate}
\item
Applying $f=\frac{Y}{X}$ as necessary, we can assume $|x_2|_v\le |y_2|_v=1$.
\item
Apply $f=\frac{y_2X-x_2Y}{Y}$ so that $|P_2|=[0:1]$. (Note $\smatt{y_2}{-x_2}{0}{1}\in \PGL_2(R)$ since $|y_2|_v=1$.)
\end{enumerate}
The inequality now follows from
\begin{align*}
\rh_v(P_1,P_3)&=
|x_1y_3-x_3y_1|_v\\
&\le \max \{|x_1y_3|_v,|x_3y_1|_v\}\\
&=\max\{|x_1|_v,|x_3|_v\}\\
&=\max\{\rh_v(P_1,P_2),\rh_v(P_2,P_3)\}&\text{since }P_2=[0:1].\qedhere
\end{align*}
\end{proof}
\begin{df}
Let $(K,\ad)$ be a field with valuation, and $\phi(z)\in K(z)$ be a nonconstant rational map. The \textbf{multiplier} of $\phi$ at a fixed point $\al\in K$ is
\[
\la_{\al}(\phi)=\phi'(\al).
\]
If $\al$ has exact period $n$ for $\phi$, then we define
\[
\la_{\al}(\phi)=(\phi^n)'(\al)=\phi'(\al)\phi'(\phi(\al))\cdots \phi'(\phi^{n-1}(\al)).
\]
(The latter follows by the chain rule.)
We say that
\[
\al\text{ is}
\begin{cases}
\text{superattracting,}&\text{if }\la_{\al}(\phi)=0\\
\text{attracting,}&\text{if }\la_{\al}(\phi)<1\\
\text{neutral,}&\text{if }\la_{\al}(\phi)=1\\
\text{repelling,}&\text{if }\la_{\al}(\phi)>1.
\end{cases}
\]
If $\la_{\al}(\phi)=1$, we say that $\phi$ is rationally or irrationally neutral according to whether or not $\la_{\al}(\phi)$ is a root of unity.
[Analogy with $\C$ case?]
\end{df}
\section{Reduction of maps}
Let $K$ be a field with normalized discrete valuation $v$, let $R$ be the ring of integers, $\mfp$ the maximal ideal, and $k=R/\mfp$ the residue field. Given a point $P\in\Pj^N(K)$,  %\in [x_0:\ldots:x_n]\in \Pj^N(K)$ we define its reduction as follows: Let $x_j$ be the coordinate with minimum valuation, and define
%\[
%\widetilde{P}=\ba{\widetilde{x_0^{-1}x_j}:\ldots:\widetilde{x_0^{-1}x_n}}\in\Pj^1(k).
%\]
choose coordinates $[x_0:\ldots:x_n]$ so that $x_j\in R$ for all $j$ and at least one $x_i$ has valuation 0, and define $\wt P=[\wt{x_0},\ldots, \wt{x_n}]$.

We similarly define the reduction of a rational map $\phi$ as follows: 
First write $\phi(X,Y)=[F(X,Y):G(X,Y)]$ in normalized form, i.e. $F,G\in R[X,Y]$ and at least one coefficient of $F$ or $G$ is in $R^{\times}$. Then we let $\wt{\phi}=[\wt F:\wt G]$.
\begin{pr}[Basic properties of reduction]$\,$
\begin{enumerate}
\item
$\widetilde{P_1}=\widetilde{P_2}$ if and only if $\rh_v(P_1,P_2)<1$.
\item
For $P,Q\in \Pj^1(K)$ and $f\in \PGL_2(R)$, $\widetilde{P}=\widetilde{Q}$ if and only if $\wt{f(P)}=\wt{f(Q)}$.
\item Let $P_1,P_2,P_3$ be points with distinct reductions. There exists a fractional linear transformation $f\in \PGL_2(R)$ such that 
\[
f(P_1)=0,\qquad f(P_2)=1, \text{ and}\qquad f(P_3)=\iy.
\]
(Note that we can always find $f\in \PGL_2(K)$.)
\end{enumerate}
\end{pr}
\begin{proof}Normalize coordinates. 
\begin{enumerate}
\item
Suppose $x_1y_2\equiv x_2y_1\pmod{\mfp}$. If $x_1x_2\nequiv 0\pmod{\mfp}$, then 
\[
\wt{P_2}=[\wt{x_1}\wt{x_2}:\wt{x_1}\wt{y_2}]
=[\wt{x_1}\wt{x_2}:\wt{y_1}\wt{x_2}]=\wt{P_1}.
\]
If $x_1x_2\equiv 0\pmod{\mfp}$, then $P_1=P_2=[0:1]$.
\item Combine part 1 with Proposition~\ref{chordal-metric-invar}.
\item We build $f$ as a composition of the following.
\begin{enumerate}
\item
Applying $f_1=\frac YX=\smatt 0110$ as necessary, we may assume $v(x_1)\ge v(y_1)$. Normalize again so $v(y_1)=0$.
\item
Apply $f_2=\frac{y_1X-x_1Y}{Y}=\smatt{y_1}{-x_1}{0}{1}$ so $P_1=[0:1]=0$.
\item
Apply $f_3=\frac{X}{y_3X-x_3Y}=\smatt 10{y_3}{-x_3}$, which fixes $P_1$ and send $P_3$ to $[1:0]=\iy$.
\item
Apply $f_4=\frac{y_2X}{x_2Y}=\smatt{y_2}00{x_2}$ to scale $P_2$. (Map in PGL(R)?)\qedhere
\end{enumerate}
\end{enumerate}
\end{proof}
Define the \textbf{resultant} of $\phi=[F:G]$ to be $\Res(F,G)$ (see Section~\ref{polynomials}.\ref{resultant}). Note $\phi$ is defined up to $(R^{\times})^{2d}$ where $d=\deg \phi$.
\begin{thm}[Upper bound on expansion in chordal metric]\label{bd-exp-chordal}
Let $\phi:\Pj^1(K)\to \Pj^1(K)$. Then
\[
\rh_v(\phi(P_1),\phi(P_2))\le |\Res(\phi)|_v^{-2}\rh_v(P_1,P_2).
\] 
\end{thm}
\begin{proof}
Let $[x:y]$ be normalized. 
By Proposition~\ref{resultant-properties}(2) (suitably homogenized), there exist $F_1, G_1,F_2,G_2$ such that 
\begin{align*}
F_1F+G_1G&=\Res(\phi)X^{2d-1}\\
F_2F+G_2G&=\Res(\phi)Y^{2d-1}.
\end{align*}
By the triangle inequality,
\begin{align*}
|\Res(\phi)X^{2d-1}|_v&\le \max(|F(x,y)|_v,|G(x,y)|_v)\\
|\Res(\phi)Y^{2d-1}|_v&\le \max(|F(x,y)|_v,|G(x,y)|_v)
\end{align*}
Since $\max\{|x|_v,|y|_v\}=1$, we get
\begin{equation}\label{res-bd1}
|\Res(\phi)|_v \le \max(|F(x,y)|_v,|G(x,y)|_v)
\end{equation}
which bounds the extent to which $F(x,y), G(x,y)$ can both be divisible by high powers of $\mfp$.

Take $P_1,P_2$ to be normalized. We have the factorization
\begin{equation}\label{exp-chordal-factor}
F(X_1,Y_1)G(X_2,Y_2)-F(X_2,Y_2)G(X_1,Y_1)=(X_1Y_2-X_2Y_1)\underbrace{H(X_1,Y_1,X_2,Y_2)}_{\in R[X_1,Y_1,X_2,Y_2]}.
\end{equation}
Hence
\begin{align*}
\rh_v(\phi(P_1),\phi(P_2))&=\frac{|F(X_1,Y_1)G(X_2,Y_2)-F(X_2,Y_2)G(X_1,Y_1)|_v}{\max\{|F(X_1,Y_1)|_v,|G(X_1,Y_1)|_v\}\max\{|F(X_2,Y_2)|_v,|G(X_2,Y_2)|_v\} }\\
&\stackrel{\eqref{res-bd1}}{\le}
\frac{|F(X_1,Y_1)G(X_2,Y_2)-F(X_2,Y_2)G(X_1,Y_1)|_v}{|\Res(\phi)|_v^2}\\
&\stackrel{\eqref{exp-chordal-factor}}=\frac{|(X_1Y_2-X_2Y_1)H(X_1,X_2,Y_1,Y_2|_v}{|\Res(\phi)|_v^2}\\
&\le \frac{\rh_v(P_1,P_2)}{|\Res(\phi)|_v^2}.
\end{align*}
\end{proof}
\begin{pr}\label{good-reduction-prop}
Let $\phi:\Pj^1\to \Pj^1$ be defined over $K$, and write $\phi=[F:G]$ in normalized form. The following are equivalent.
\begin{enumerate}
\item
$\deg\phi=\deg\wt{\phi}$.
\item
$\wt F(X,Y)=\wt G(X,Y)=0$ has no solutions $[\al:\be]\in \Pj^1(\ol k)$.
\item $\Res(\phi)\in R^{\times}$.
\item $\Res(\wt F,\wt G)\ne 0$.
\end{enumerate}
We say that $\phi$ has \textbf{good reduction} if the above are satisfied.
\begin{proof}
Note that $\deg\phi-\deg\wt{\phi}$ equals the number of common roots of $\wt F=\wt G$. This shows $(1)\iff (2)$. Now (2), (3), and (4) are equivalent by applying Proposition~\ref{polynomials}.\ref{resultant-properties} to $\wt F$ and $\wt G$.
\end{proof}
\end{pr}
\begin{pr}[Basic facts about reduction]
Let $\phi,\psi:\Pj^1\to \Pj^1$ be rational maps with good reduction.
\begin{enumerate}
\item $\wt{\phi}(\wt P)=\wt{\phi(P)}$ for all $P\in \Pj^1(K)$.
\item $\phi\circ\psi$ has good reduction and $\tilde{\phi\circ \psi}=\tilde{\phi}\circ \tilde{\psi}$.
\item Reduction sends $\Per(\phi)\to \Per(\wt{\phi})$ and $\PrePer(\phi)\to \PrePer(\wt{\phi})$. Moreover it preserves exact periods.
\end{enumerate}
\end{pr}
\begin{proof}
Use the characterization of good reduction given by Proposition~\ref{good-reduction-prop}(2).
\end{proof}
\begin{df}
The \textbf{Fatou set} of $\phi$ is the maximal {\it open} set on which $\set{\phi^n}{n\in \N}$ is equicontinuous. The \textbf{Julia set} is the complement of the Fatou set.
\end{df}
\begin{thm}
Let $\phi:\Pj^1\to \Pj^1$ be a rational map with good reduction. Then
\begin{enumerate}
\item
$\phi$ is everywhere nonexpanding:
\[
\rh_v(\phi(P_1),\phi(P_2))\le \rh_v(P_1,P_2).
\]
\item $\phi$ has empty Julia set.
\end{enumerate}
\end{thm}
\begin{proof}
\begin{enumerate}
\item
Use Theorem~\ref{bd-exp-chordal} and the fact that $\rh_v(P_1,P_2)<1$ when $\wt{P_1}=\wt{P_2}$ (Proposition~\ref{reduction-properties}(1)).
\item
A nonexpanding map is equicontinuous with constant 1.
\end{enumerate}•
\end{proof}
\section{Periodic points}
We now characterize periodic points of $\phi$.
\begin{thm}\label{possible-periods}
Let $(K,\ad_v)$ be a nonarchimedean local field, $k$ be its residue field, and $\phi:\Pj^1(K)\to \Pj^1(K)$ be a rational function of degree $d\ge 2$ with good reduction. Let $P\in \Pj^1(K)$ be a periodic point of $\phi$. Let 
\begin{align*}
n&=\text{period of }P\text{ for }\phi\\
m&=\text{period of }\wt{P}\text{ for }\wt{\phi}\\
r&=\text{order of }\la_{\wt{\phi}}(\wt{P})=(\wt{\phi}^m)'(P) \text{ in }k^{\times}\\
p&=|k|.
\end{align*}
Then $n=m, \text{ or }mrp^e$ for some $e\in \N_0$.
\end{thm}
\begin{proof}
Replacing $\phi$ by $\phi^m$ and $m$ by 1, we may assume $m=1$, i.e. $\wt P$ is a fixed point of $\wt \phi$. If $\phi(P)=P$ we are in the first case, so assume this does not happen. We may further assume $P=[0:1]$, by taking $f$ sending $[0,1]$ to $P$ and replacing $\phi$ with $f^{-1}\circ \phi \circ f$. 

Our main technique is to write the iterates $\phi^i(0)$ in terms of $\phi'(0)$ by considering the Taylor expansion.
%Write the Taylor expansion of $\phi(z)$: 
Write
\[\phi(z)
=\frac{a_dX^d+\cdots +a_0}{b_dz^d+\cdots + b_0}
=\mu+\la z+\cdots\]
where $\mu=\frac{a_0}{b_0}\in \mfp$ (because $\wr{\phi}(0)=0$) and $\la=\phi'(0)$. By induction, we find that
\[
\phi^i(z)= \underbrace{\mu(1+\la+\cdots +\la^{i-1})}_{\phi^i(0)}+\underbrace{\la^i}_{(\phi^i)'(0)}z+\cdots 
%\phi^i(0)\equiv \mu(1+\la+\cdots +\la^{i-1})\pmod{\mu^2}.
\]
Since $\phi^n(0)=0$, this gives
\begin{equation}
\label{la-sum}
1+\la+\cdots +\la^{n-1}\equiv 0\pmod{\mfp}.
\end{equation}
Consider two cases.
\begin{enumerate}
\item
$\la\nequiv 1\pmod{\mfp}$. Then $r\ge 2$.  Multiplying~(\ref{la-sum}) by $\la-1$ gives $\la^n\equiv 1\pmod{\mfp}$. This shows $r\mid n$. If $n\neq r$, then replace $\phi$ with $\phi^r$. Then $\la$ is replaced with $\la^r$, so we are in the second case.
\item
$\la\equiv 1\pmod{\mfp}$. Then~(\ref{la-sum}) gives us $n\equiv 0\pmod{\mfp}$; hence $p\mid n$ and we can replace $\phi$ with $\phi^p$ and $n$ by $\frac np$. Then we are in this case again, and we repeat until $n=1$.\qedhere
\end{enumerate}
\end{proof}
\begin{cor}
Let $\phi:\Pj^1\to \Pj^1$ be a rational map with good reduction.
\begin{enumerate}
\item
Every periodic point of $\phi$ is nonrepelling.
\item
If $\wt{\phi}$ is separable, then $\phi$ has finitely many attracting periodic points.
\end{enumerate}
\end{cor}
\begin{thm}
Let $K$ be  number field, and $\phi:\Pj^1\to \Pj^1$ be a rational map over $K$. Suppose $\phi$ has good reduction at $\mfp$ and $\mq$, with different residue characteristics. Let $P$ be a periodic point with period $n$. Then
\[
n\le (\N\mfp^2-1)(\N\mq^2-1).
\]
In particular, $\Per(\phi,K)$ is finite for any $\phi$. %, and any polynomial over $K$ has finitely many rational periodic points.
\end{thm}
\begin{proof}
We have that 
\begin{align*}
m_{\mfp}&\le |\Pj^1(\F_p)|=\N\mfp+1\\
r_{\mfp}&\le |\F_p^{\times}|=\N\mfp-1
\end{align*}
and similarly for $\mq$. By Theorem~\ref{possible-periods}, we get
\[
n=m_{\mfp}r_{\mfp}^{e}p^{e'}=m_{\mq}r_{\mq}^fq^{f'}
\]
for some $e,f\in \{0,1\}$ and $e',f'\in \N_0$. Since $p,q$ are relatively prime, $n\le  m_{\mfp}r_{\mfp}m_{\mq}r_{\mq}$, giving the desired bound.

The second part now follows from the fact that the coefficients of $\phi$ can have nonzero valuation only for a finite number of primes, and the fact that $\phi^n(P)=P$ can only have finitely many solutions for a fixed $n$.
\end{proof}